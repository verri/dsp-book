\documentclass[twoside,a5paper,10pt,openright]{memoir}
% \setstocksize{230mm}{160mm}
\setlrmarginsandblock{23mm}{20mm}{*}
\setulmarginsandblock{20mm}{20mm}{*}
\checkandfixthelayout

\usepackage[T1]{fontenc}
\usepackage[final]{microtype}

\usepackage[english]{babel}
\usepackage[style=authoryear,maxbibnames=99,giveninits=true,uniquename=init,backref=true]{biblatex}
\addbibresource{references.bib}

\usepackage{amsmath}
\usepackage{mathtools}
\usepackage{blkarray}
\usepackage[colorlinks]{hyperref}
\usepackage{fontspec}
\usepackage{xltxtra}
\usepackage{sourcesanspro}
\usepackage[warnings-off={mathtools-colon,mathtools-overbracket}]{unicode-math}
\usepackage{graphicx}
\usepackage[super]{nth}
\usepackage{booktabs}
\usepackage{csquotes}
\usepackage{multirow}
\usepackage[table]{xcolor}
\usepackage{titletoc}
\usepackage[noend,noline,linesnumbered]{algorithm2e}
\usepackage[automake]{glossaries-extra}
\usepackage{orcidlink}
\usepackage{fontawesome5}
\usepackage{nicefrac}
\usepackage{cancel}

\DeclareMathOperator*{\argmax}{arg\,max}
\DeclareMathOperator*{\argmin}{arg\,min}
\DeclareMathOperator{\Prob}{P}
\DeclareMathOperator{\E}{E}
\DeclareMathOperator{\Var}{Var}
\DeclareMathOperator{\sign}{sign}
\DeclareMathOperator{\clamp}{clamp}

\usepackage[Bjornstrup]{fncychap} % must come before cleveref to avoid break appendix name
\usepackage{cleveref}

\usepackage{tikz}
\usetikzlibrary{shapes, arrows.meta, positioning, shapes.geometric}
\usetikzlibrary{fit}
\usetikzlibrary{datavisualization.formats.functions}
\usepackage{pgfplots}
\pgfplotsset{compat=1.18}

\tikzset{%
  decision/.style={draw, diamond, text centered, minimum height=0.5cm, minimum width=1cm},
  block/.style={rectangle, draw, text width=6em, text centered, rounded corners, minimum height=3em},
  mediumblock/.style={rectangle, draw, text width=3em, text centered, rounded corners, minimum height=2em},
  darkblock/.style={block, fill=gray, text=white},
  smallblock/.style={rectangle, rounded corners, draw, font=\tiny, minimum height=1em, inner sep=2pt},
  smalldarkblock/.style={smallblock, fill=gray, text=white},
  darkcircle/.style={draw, circle, fill=gray, text centered, text=white},
  smallcircle/.style={draw, circle, text centered, font=\tiny},
  smalldarkcircle/.style={smallcircle, fill=gray, text=white},
  line/.style={draw, -latex},
  dline/.style={draw, latex-latex},
  bigarrow/.style={draw, -latex, line width=3pt, gray},
}

\usepackage{titling}
\usepackage[most]{tcolorbox}
\usepackage{xhfill}

\usepackage{qrcode}
\newcommand{\linkwithqr}[1]{%
  \begin{minipage}{0.8\textwidth}
    \url{#1}
  \end{minipage}
  \hfill
  \begin{minipage}{0.15\textwidth}
    \qrcode[height=\textwidth]{#1}
  \end{minipage}
}

\newcommand{\aurl}{google.com}
\newenvironment{parwithqr}[1]{%
  \renewcommand\aurl{#1}%
  \noindent%
  \begin{minipage}{0.75\textwidth}%
}{%
  \end{minipage}%
  \hfill%
  \begin{minipage}{0.20\textwidth}%
    \qrcode[height=\textwidth]{\aurl}%
  \end{minipage}%
}
\newenvironment{lparwithqr}[1]{%
  \renewcommand\aurl{#1}%
  \noindent%
  \begin{minipage}{0.20\textwidth}%
    \qrcode[height=\textwidth]{\aurl}%
  \end{minipage}%
  \hfill%
  \begin{minipage}{0.75\textwidth}%
}{%
  \end{minipage}%
}

\nonfrenchspacing

\setsecnumdepth{subsection}
\maxtocdepth{subsection}

\babelhyphenation[english]{par-a-digm pipe-line pa-ram-e-ter-ize ro-bot-ics pen-al-ty
  quan-ti-fi-ca-tion}

\newtcolorbox[auto counter, number within=chapter, crefname={algorithm}{algorithms}]{algobox}[2][]{%
  float=thbp, base={Algorithm \thetcbcounter: #2},
  colframe=black!20!white, #1,
  before upper={\begin{algorithm}[H]},
  after upper={\end{algorithm}}}

\newabbreviation{erm}{ERM}{empirical risk minimization}
\newabbreviation{svm}{SVM}{support vector machine}
\newabbreviation{bnf}{BNF}{Backus–Naur form}
\newabbreviation{sql}{SQL}{structured query language}
\newabbreviation{ibm}{IBM}{International Business Machines Corporation}
\newabbreviation{rdbms}{RDBMS}{relational database management system}
\newabbreviation{etl}{ETL}{extract, transform, load}
\newabbreviation{bi}{BI}{business intelligence}
\newabbreviation{hdfs}{HDFS}{hadoop distributed file system}
\newabbreviation{lusi}{LUSI}{learning using statistical inference}
\newabbreviation{iot}{IoT}{internet of things}
\newabbreviation{lifo}{LIFO}{last-in-first-out}
\newabbreviation{fifo}{FIFO}{first-in-first-out}
\newabbreviation{pmf}{PMF}{probability mass function}
\newabbreviation{pdf}{PDF}{probability density function}
\newabbreviation{cdf}{CDF}{cumulative distribution function}
\newabbreviation{cicd}{CI/CD}{continuous integration/continuous deployment}
\newabbreviation{slt}{SLT}{statistical learning theory}
\newabbreviation{ai}{AI}{artificial intelligence}
\newabbreviation{ml}{ML}{machine learning}
\newabbreviation{vc}{VC}{Vapnik-Chervonenkis}
\newabbreviation{srm}{SRM}{structural risk minimization}
\newabbreviation{mlp}{MLP}{multilayer perceptron}
\newabbreviation{iqr}{IQR}{interquartile range}
\newabbreviation{cnn}{CNN}{convolutional neural network}
\newabbreviation{pca}{PCA}{principal component analysis}

\newglossaryentry{ontology}{%
  name=ontology,
  description={%
    Ontology is the study of being, existence and reality. In computer science and
    information science, an ontology is a formal naming and definition of the types,
    properties, and interrelationships of the entities that really or fundamentally exist
    for a particular domain.}
}

\newglossaryentry{leakage}{%
  name=data leakage,
  description={%
    Situation where information from the test set is used to transform the training
    set in any way or to train the model.}
}

\newglossaryentry{model}{%
  name=model,
  description={%
    A general function that can be used to estimate the relationship between the
    input and output variables in a dataset.}
}

\newglossaryentry{preprocessor}{%
  name=preprocessor,
  description={%
    A chain of data handling operations that transforms the input data into a format that
    is suitable for the model.}
}

\makeglossaries


\begin{document}

\setmathfont{STIXTwoMath}[
  Extension={.otf},
  Path={./fonts/},
  Scale=1]

\setmainfont{STIXTwoText}[
  Extension={.otf},
  Path={./fonts/},
  UprightFont={*-Regular},
  BoldFont={*-Bold},
  ItalicFont={*-Italic},
  BoldItalicFont={*-BoldItalic}]

\setmonofont{CourierPrime}[
  Extension={.ttf},
  Path={./fonts/},
  UprightFont={*-Regular},
  BoldFont={*-Bold},
  ItalicFont={*-Italic},
  BoldItalicFont={*-BoldItalic},
  Scale=0.9]

\newcommand{\code}[1]{\colorbox{black!10!white}{\texttt{#1}}}

%\setsansfont{SourceSansPro}

\urlstyle{rm}

\tcbset {
  base/.style={
    arc=0mm,
    bottomtitle=0.5mm,
    boxrule=0mm,
    colbacktitle=black!20!white,
    coltitle=black,
    left=2.5mm,
    leftrule=1mm,
    right=3.5mm,
    title={#1},
    toptitle=0.75mm,
    lower separated=false,
  }
}

\newtcbtheorem[auto counter, number within=chapter]{defbox}{Definition}{
  float=h!,
  colframe=black!20!white,
  description delimiters parenthesis,
  label type=definition,
  base
}{def}

\newtcolorbox{mainbox}[1]{
  float=ht,
  spread outwards=-2mm,
  spread inwards=-2mm,
  %spread downwards,
  colframe=black!20!white,
  fonttitle=\bfseries,
  base={#1},
  left=16mm,
  right=19mm,
}

\newtcolorbox{hlbox}[1]{
  float=h,
  colframe=black!20!white,
  fonttitle=\bfseries,
  base={#1},
}

\newtcolorbox[blend into=figures]{figurebox}[2][]{%
  float=thbp,
  base={#2},
  colframe=black!20!white,
  every float=\centering, #1}

\newtcolorbox[blend into=tables]{tablebox}[2][]{%
  float=thbp, base={#2},
  colframe=black!20!white,
  every float=\centering, #1}

\newcommand\boxsubtitle[1]{%
  \vspace{0.5em}
  \noindent\xrfill[0.5ex]{1pt}[black!20]\phantom{x}\textbf{#1}\phantom{x}\xrfill[0.5ex]{1pt}[black!20]%
  \vspace{0.5em}
}

\renewcommand{\vec}[1]{\mathbf{#1}}

\title{Data Science Project: An Inductive Learning Approach}
\author{Filipe A. N. Verri}

\hypersetup{%
  pdftitle={\thetitle},
  pdfsubject={Data Science},
  pdfauthor={\theauthor},
  pdfkeywords={data science, statistics, machine learning, databases},
  urlcolor=black!60,
  linkcolor=black,
  citecolor=black,}

\begin{titlingpage}
  \begin{tikzpicture}[remember picture, overlay]
    \fill[black!80] (current page.south west) rectangle (current page.north east);
    \node[anchor=south east, inner sep=0mm] at (current page.south east) {%
      \reflectbox{\includegraphics[width=\paperwidth]{images/bard2.jpg}}%
    };
    \node[white,anchor=north, yshift=-25mm](title) at (current page.north) {\HUGE\sffamily\uppercase{Data Science Project}};
    \node[white,anchor=north, inner sep=5mm] at (title.south) {\LARGE\sffamily\uppercase{An Inductive Learning Approach}};
    \node[anchor=south west, inner sep=11mm] at (current page.south west) {\HUGE\sffamily\uppercase{F.A.N. Verri}};
  \end{tikzpicture}
\end{titlingpage}

\newpage
\clearpage
\newpage
\frontmatter
\thispagestyle{empty}
\begin{tikzpicture}[remember picture, overlay]
  \node[anchor=south east, inner sep=8mm] at (current page.south east) {%
    \reflectbox{\includegraphics[height=0.3\paperheight]{images/toucan_bw.png}}%
  };
  \node[anchor=south west, inner sep=10mm] at (current page.south west)
    {\large\sffamily Version 0.1 ``\textit{Audacious Hatchling}''};
    % {\large\sffamily Version 1.0 ``\textit{Tropical Toucan}''};
  \node[anchor=south east, inner sep=10mm] at (current page.south east) {\large\sffamily\today};
  \node[anchor=north, yshift=-25mm](title) at (current page.north) {\HUGE\sffamily\uppercase{Data Science Project}};
  \node[anchor=north, inner sep=5mm](subtitle) at (title.south) {\LARGE\sffamily\uppercase{An Inductive Learning Approach}};
  \node[anchor=north, inner sep=5mm] at (subtitle.south) {\Large\sffamily\uppercase{Filipe A. N. Verri}};
\end{tikzpicture}

\newpage

{
  \footnotesize\noindent Cite this book as:
  \begin{verbatim}
  @misc{verri2024datascienceproject,
    author = {Verri, Filipe Alves Neto},
    title = {Data Science Project: An Inductive Learning Approach},
    year = 2024,
    publisher = {Leanpub},
    version = {v0.1.0},
    doi = {10.5281/zenodo.14498011},
    url = {https://leanpub.com/dsp}
  }
  \end{verbatim}

  \noindent\fullcite{verri2024datascienceproject}.
}

\vfill

{
  \footnotesize\noindent\color{red}
  \textbf{Disclaimer:} This version is a work in progress.  Proofreading and professional editing are
  still pending.
}

\vspace{0.5cm}
{
\footnotesize\noindent
The book is typeset with \XeTeX{} using the Memoir class.  All figures are
original and created with Ti\textit{k}Z.  Proudly written in
\href{https://neovim.io/}{Neovim}.  \LaTeX{} code written with the assistance of
\href{https://github.com/features/copilot}{GitHub Copilot}.
All the text is original and not AI-generated.
The book cover image was created with the assistance of
\href{https://gemini.google.com}{Gemini} and \href{https://openai.com/dall-e-2}{DALL·E 2}.
We use the beautiful \href{https://www.stixfonts.org/}{STIX fonts} for text and math.
Some icons are from \href{https://fontawesome.com/}{Font Awesome 5} by Dave Gandy.
}

\vspace{0.5cm}
{
\footnotesize\noindent
Scripture quotations are from The ESV® Bible (The Holy Bible, English Standard Version®),
copyright © 2001 by Crossway, a publishing ministry of Good News Publishers. Used by
permission. All rights reserved.
}

\vspace{0.5cm}
{
\footnotesize\noindent
\thetitle{} © 2023--\the\year{} by \theauthor{}~\orcidlink{0000-0002-8240-5129} is licensed under
Attribution-NonCommercial-NoDerivatives 4.0 International. To view a copy of this license,
visit
\href{http://creativecommons.org/licenses/by-nc-nd/4.0/}{creativecommons.org/licenses/by-nc-nd/4.0}.
}

\thispagestyle{empty}
\newpage

\thispagestyle{empty}

% Dedication page
\begin{center}
  \vspace{0.2\textheight}
  \itshape
  To my wife, for inspiring and supporting me in writing this book.

  \vfill
  \raggedleft Above all, God be praised.
\end{center}

\newpage

\thispagestyle{empty}
\phantom{foo}
\newpage

\tableofcontents
\thispagestyle{empty}

\chapter{Foreword}

\chapterprecishere{\raggedleft\textup{by} \textsc{Ana Carolina Lorena}}

Data is now a ubiquitous presence and is collected every time and everywhere. However, the
real challenge lies in harnessing this data to generate actionable insights that guide
decision-making and drive innovation. This is the essence of data science, a
multidisciplinary field that leverages mathematical, statistical, and computational
techniques to analyse data and solve complex problems.

The book ``Data Science Project: An Inductive Learning Approach'' by F.A.N. Verri provides
readers with a structured and insightful exploration of the entire data science pipeline,
from the initial stages of data collection to the final step of model deployment. The book
effectively balances theory and practice, focusing on the inductive principles
underpinning predictive analytics and machine learning.

While other texts focus solely on machine learning algorithms or delve deeply into
tool-specific details, this book provides a holistic view of every stage of a data science
project. It emphasises the importance of robust data handling, sound statistical learning
principles, and meticulous model evaluation. The author thoughtfully integrates the
mathematical foundations and practical considerations needed to design and execute
successful data science projects.

Beyond the technical mechanics, this book challenges
readers to critically evaluate their models' strengths and limitations. It underscores the
importance of semantics in data handling, equipping readers with the skills to interpret
and transform data meaningfully.

\begin{sloppypar}
\emergencystretch=1em
Whether you are a student embarking on your first data science project or a data scientist
professional seeking to expand and refine your skills, this book's clarity, rigour, and
practical focus make it a guide that will serve you well in tackling the complex
challenges of data-driven decision-making. The book will expand your understanding and
inspire you to approach data science projects with a commitment to creating responsible
and impactful solutions.
\end{sloppypar}

% vim: spell spelllang=en

\chapter{Preface}

\noindent Dear reader, \vspace{1em}

This book has started from the lectures notes of the course PO-235 Data Science Project
that I teach for graduate students at the Aeronautics Institute of Technology (ITA) in
Brazil.  I have been teaching this course since 2023, and I have been improving the
material every year.

Before that, I was the coordinator of the Data Science Specialization Program (CEDS) at
ITA.  That experience, which included lots of administrative work, but also teaching and
supervising professionals in the course, has helped me to understand the needs of the
market and the students.

Literature provides us a lot of great theoretical books on machine learning and
statistics, and excellent practical books on data science tools.  However, I missed
something that could provide a solid foundation on data science, covering all steps in a
data science project, including its software engineering aspects.

My goal is to provide a book that can be used as a textbook for a course on data science
projects or as a reference for professionals working in the field.  I try to be as
formal as possible, without losing the practical aspects of the subject.  Also, I do not
focus on a specific tool or programming language, but I try to explain the semantics of
data science tasks that can be implemented in any programming language.

Also, instead of teaching specific machine learning algorithms, I try to explain why
machine learning works, increasing awareness of its pitfalls and limitations.
For that, I assume you have a strong mathematical and statistics background.

One important artificial constraint I have imposed in the material (for the sake of the
course) is that I only consider \emph{predictive methods}, more specifically inductive
ones.  I do not address topics such as clustering, association-rules mining, transductive
learning, anomaly detection, time series forecasting, reinforced learning, etc.

I expect my approach on the subject to provide understanding of all steps in a data science
project including a deeper focus on correct evaluation and validation of data science
solutions.

Note that, in this book, I openly express my opinions and beliefs. Many times it might sound
controvertial.  I am not trying to be rude or to demean any researcher or practitioner in the
field.  I am just trying to be honest and transparent.

\vspace{1em}
\emph{I'd rather be bold and straightforward than cower about my beliefs.}
\vspace{1em}

Hope you enjoy the reading.

\newpage

\begin{parwithqr}{https://github.com/verri/dsp-book}
  I intend to make this book forever free and open-source.  You can find the source code at
  \href{\aurl}{github.com/verri/dsp-book}.  Derivatives are not allowed, but you can
  contribute to the book.  Contributors will be acknowledged here.
\end{parwithqr}

\vfill

\begin{lparwithqr}{https://leanpub.com/dsp}
  However, if you like this book, consider buying the e-book version at
  \href{\aurl}{leanpub.com/dsp}.  Any amount you pay will help me to keep this book
  updated and to write new books.
\end{lparwithqr}

\vfill

\begin{parwithqr}{https://github.com/verri/dsp-book/discussions}
  If you have suggestions or questions, please open or join a discussion at
  \href{\aurl}{github.com/verri/dsp-book/discussions}.  Feel free to ask anything.
  Theoretical discussions and practical advices are also welcome.
\end{parwithqr}

\vfill

% TODO: link to GitHub
\begin{lparwithqr}{https://comp.ita.br/\~verri/dsp-book-print}
  Students can find a printable version (A4 paper, double-sided, short-edge spiral
  binding) of this book at \href{\aurl}{comp.ita.br/\textasciitilde{}verri/dsp-book-print}.
\end{lparwithqr}

\vfill

\begin{parwithqr}{https://www.buymeacoffee.com/verri}
  If you want to support me, you can \emph{buy me a coffee} at
  \href{\aurl}{buymeacoffee.com/verri}.  I will appreciate it a lot.
  I have a lot of ideas for new books and courses, and financial support will help me to
  make them real.
\end{parwithqr}

\newpage

\section*{Contributors}

I would like to thank the following contributors for their help in improving this book:

\begin{itemize}
  \itemsep0em
  \item Johnny C. Marques
  \item Manoel V. Machado (aka \emph{ryukinix})
  \item Vitor V. Curtis
\end{itemize}


\mainmatter
\chapter{Brief history of data science}
\label{chap:history}

\chapterprecishere{``Begin at the beginning,'' the King said gravely, ``and
go on till you come to the end: then stop.''\par\raggedleft--- \textup{Lewis
Carroll}, Alice in Wonderland}

There are many points-of-view about the beginning of data science.  For the sake of
contextualization, I separate the topic in two approaches: the history of the term itself
and a broad timeline of data-driven sciences highlighting the important figures in each
age.

I believe that the history of the term is important to understand the context of the
discipline.  Along the years, the term has been used to label rather different fields of
study.  Before I present my view on the term, I present the views of two important
figures in the history of data science: Peter Naur and William Cleveland.

Moreover, studying the main facts and figures in the history of data-driven sciences
enables us to understand the evolution of the field and hopefully to guide us to evolve it
further.  Most of the important theories and methods in data science have been developed
simultaneously in different fields, such as statistics, computer science, and engineering.
The history of data-driven sciences is long and rich. I present a timeline of the ages of
data handling and the most important markers of data analysis.

I do not intend to provide a comprehensive history of data science.  I aim to provide
enough context to support the development of the material in the following chapters,
sometimes avoiding directions that are not relevant in the inductive learning context.

\begin{mainbox}{Chapter remarks}

  \boxsubtitle{Contents}

  \startcontents[chapters]
  \printcontents[chapters]{}{1}{}
  \vspace{1em}

  \boxsubtitle{Context}

  \begin{itemize}
    \item The term ``data science'' is recent and has been used to label rather different
      fields.
    \item The history of data-driven sciences is long and rich.
    \item Many important theories and methods in data science have been developed
      simultaneously in different fields.
    \item The history of data-driven sciences is important to understand the evolution of
      the field.
  \end{itemize}

  \boxsubtitle{Objectives}

  \begin{itemize}
    \item Understand the history of the term ``data science.''
    \item Understand the major milestones in the history of data-driven sciences.
    \item Identify important figures in the history of data-driven sciences.
  \end{itemize}

  \boxsubtitle{Takeways}

  \begin{itemize}
    \item We have evolved both in terms of theory and application of data-driven sciences.
    \item There is no consensus on the definition of data science (including which fields
      it encompasses).
    \item However, there is enough evidence to support data science as a new science.
  \end{itemize}
\end{mainbox}

{}
\clearpage

\section{The term ``data science''}

The term data science is recent and has been used to label rather different fields of
study.  In the following, I emphasize the history of a few notable usage of the term.

\def\naurds{(0,0) circle (20mm)}
\def\naurcs{(0:5mm) circle (15mm)}
\def\naurde{(0:40mm) circle (15mm)}

\colorlet{circle edge}{black!50}
\colorlet{circle area}{black!20}

\tikzset{filled/.style={fill=circle area, draw=circle edge, thick},
    outline/.style={draw=circle edge, thick}}

\paragraph{Peter Naur (1928 -- 2016)}

The term ``data science'' itself was coined in the 1960s by Peter Naur (/naʊə/). Naur was
a Danish computer scientist and mathematician who made significant contributions to the
field of computer science, including his work on the development of programming
languages\footnote{He is best remembered as a contributor, with John Backus, to the
\gls{bnf} notation used in describing the syntax for most programming
languages.}.
His ideas and concepts laid the groundwork for the way we think about programming and data
processing today.

% \begin{slidebox}{Peter Naur}{}
%   \begin{itemize}
%     \item Danish computer scientist and mathematician.
%     \item Coined the term ``data science'' in the 1960s.
%     \item Proposed the term ``datalogy'' as an alternative to computer science.
%   \end{itemize}
% \end{slidebox}

Naur disliked the term computer science and suggested it be called datalogy or data
science.  In the 1960s, the subject was practised in Denmark under Peter
Naur's term datalogy, which means the science of data and data processes.

He coined this term to emphasize the importance of data as a fundamental component of
computer science and to encourage a broader perspective on the field that included
data-related aspects. At that time, the field was primarily centered on programming
techniques, but Naur's concept broadened the scope to recognize the intrinsic role of data
in computation.

In his book\footfullcite{Naur1974}, ``Concise Survey of Computer Methods'', he
parts from the concept that \emph{data} is ``a representation of facts or ideas in a
formalised manner capable of being communicated or manipulated by some
process.''\footnote{I. H. Gould (ed.): ‘IFIP guide to concepts and terms in data
processing’, North-Holland Publ. Co., Amsterdam, 1971.} Note however that his view of the
science only ``deals with data [\dots] while the relation of data to what they represent
is delegated to other fields and sciences.''

\begin{figurebox}[label=fig:naur]{Naur's view of data science.}
  \centering
  \begin{tikzpicture}
    \begin{scope}
      \clip \naurds;
      \fill[filled] \naurcs;
    \end{scope}
    \draw[outline] \naurds node(ds) {};
    \draw[outline] \naurcs node {Computer science};
    \draw[outline] \naurde node {Domain expertise};
    \node[anchor=north,above] at (0,2) {Data science};
  \end{tikzpicture}
  \tcblower
    For Naur, data science studies the techniques to deal
    with data, but he delegates the meaning of data to other fields.
\end{figurebox}

It is interesting to see the central role he gave to data in the field of computer
science. His view that the relation of data to what they represent is delegated to other
fields and sciences is still debatable today.  Some scientists argue that data science
should focus on the techniques to deal with data, while others argue that data science
should encompass the whole business domain.  A depiction of Naur's view of data science is
shown in \cref{fig:naur}.

\def\clevelandds{(0,0) circle (20mm)}
\def\clevelandst{(0:-5mm) circle (15mm)}
\def\clevelandde {(2,1) circle (15mm)}
\def\clevelandcs {(2,-1) circle (15mm)}

\paragraph{William Cleveland (born 1943)}

In 2001, a prominent statistician used the term ``data science'' in his work to describe a
new discipline that comes from his ``plan to enlarge the major areas of technical work of
the field of statistics\footfullcite{Cleveland2001}.''
In 2014, that work was republished\footnote{W. S. Cleveland.
Data Science: An Action Plan for the Field of Statistics. Statistical Analysis and Data
Mining, 7:414–417, 2014. reprinting of 2001 article in ISI Review, Vol 69.}.
He advocates the expansion of statistics beyond theory into technical areas, significantly
changing statistics.  Thus, it warranted a new name.

% \begin{slidebox}{William Cleveland}{}
%   \begin{itemize}
%     \item American statistician.
%     \item Proposed the discipline ``data science'' in 2001.
%     \item Proposed the term ``data science'' as the new name for expansion of statistics.
%   \end{itemize}
% \end{slidebox}

As a result, William Swain Cleveland II is credited to define data science as it is most
used today. He is a highly influential figure in the fields of statistics, machine
learning, data visualization, data analysis for multidisciplinary studies, and high
performance computing for deep data analysis.

\begin{figurebox}[label=fig:cleveland]{Cleveland's view of data science.}
  \centering
  \begin{tikzpicture}
    \begin{scope}
      \clip \clevelandds;
      \fill[filled] \clevelandst;
      \fill[filled] \clevelandde;
      \fill[filled] \clevelandcs;
    \end{scope}
    \draw[outline] \clevelandds node(ds) {};
    \draw[outline] \clevelandst node {Statistics};
    \draw[outline] \clevelandde node {Domain expertise};
    \draw[outline] \clevelandcs node {Computer science};
    \node[anchor=north,above] at (0,2) {Data science};
  \end{tikzpicture}
  \tcblower
    For Cleveland, data science is the ``modern'' statistics,
    where it is enlarged by computer science and domain expertise.
\end{figurebox}

In his view, data science is the ``modern'' statistics, where it is enlarged by computer
science methods and domain expertise.  An illustration of Cleveland's view of data science
is shown in \cref{fig:cleveland}.  It is important to note that Cleveland never defined a
explicit list of computer science fields and business domains that should be included in
the new discipline.  The main idea is that statistics should rely on computational methods
and that the domain expertise should be considered in the analysis.

\paragraph{Buzzword or a new science?}

Be aware that scientific literature has no consensus on the definition of data science, and it is still considered
by some to be a buzzword\footnote{Press, Gil. ``Data Science: What's The Half-Life of a
Buzzword?''. Forbes. Available at
\url{https://www.forbes.com/sites/gilpress/2013/08/19/data-science-whats-the-half-life-of-a-buzzword/}}.

Most of the usages of the term in literature and in the media are either a rough
reference to a set of data-driven techniques or a marketing strategy.  Naur
(\cref{fig:naur}) and Cleveland (\cref{fig:cleveland}) are among the few that try to
carefully define the term.  However, both of them do not see data science as an
independent field of study, but an enlarged scope of an existing science.  I disagree;
the social and economical demand for data-driven solutions led to an evolution in our
understanding of the challenges we are facing.  As a result, we see many ``data
scientist'' being hired and many ``data science degrees'' programs emerging.

In \cref{chap:data}, I dare to provide a (yet another) definition for the term.  I
argue that its object of study can be precisely established to support it as a new
science.

% \begin{slidebox}{A new science}{}
%   \begin{itemize}
%     \item Both Naur and Cleveland do not see data science as an independent field of study.
%     \item I argue that data science is not a buzzword.
%     \item Our social and economical reality demands a new science.
%   \end{itemize}
% \end{slidebox}

\section{Timeline and historical markers}

\textcite{Kelleher2018}\footfullcite{Kelleher2018} provides an interesting timeline of data-driven methods and
influential figures in the field.  I reproduce it here with some changes, including
some omissions and additions.  On the subject of data analysis, I include some of the
exceptional remarks
from \textcite{Vapnik1999b}\footfullcite{Vapnik1999b}.

I first address data handling --- which I include data sources, collection, organization,
storage, and transformation ---, and then data analysis and knowledge extraction.

\subsection{Timeline of data handling}
\label{sub:time-handling}

The importance of collecting and organizing data goes without saying.  Data fuels analysis and
decision making.  In the following, I present some of the most important milestones in the history
of data handling.

\begin{figurebox}[label=fig:data-handling-history]{Timeline of the ages of data handling.}
  \centering
  \begin{tikzpicture}
    \draw (0,0) -- (8,0);
    \foreach \x in {0,1,...,8} {
      \draw (\x,-0.1) -- (\x,0.1);
    }
    \foreach \x/\y/\z in {%
        0/Pre-digital Age/{3,800 BC -- \nth{18} c.},
        2/Digital Age/{1890 -- 1960},
        4/Formal Age/{1970s},
        6/Integrated Age/{1980 -- 1990},
        8/Ubiquitous Age/{2000 -- present}} {
      \node[anchor=north] at (\x,-0.1) {\footnotesize\y};
      \node[anchor=south] at (\x,0.1) {\footnotesize\z};
    }
  \end{tikzpicture}
\end{figurebox}

\Cref{fig:data-handling-history} illutrates the proposed timeline.  Ages are not strict
boundaries, but rather periods where some important events happened.  Also, observe that
the timescale is not linear.  The Pre-digital Age is the longest period, and one could
divide it into smaller periods.  My choices of ages and their boundaries are motivated by
didactic reasons.

\subsubsection{Pre-digital age}

We can consider the earliest records of data collection to be the notches on sticks and
bones to keep tracking of passing of time.  The Lebombo bone, a baboon fibula with
notches, is probably the earliest known mathematical object.  It was found in the Lebombo
Mountains located between South Africa and Eswatini.
They estimate it is more
than 40,000 years old. It is conjectured to be a tally stick, but its exact purpose is
unknown. Its 29 notches suggests that may have been used as a lunar phase counter.
However, since it is broken at one end, the 29 notches may or may not be the total
number\footfullcite{Beaumont2013}.

Another important milestone in the history of data collection is the record of
demographic data.  One of first known census was conducted in 3,800 BC in the Babylonian
Empire.  It was ordered to assess the population and resources of
his empire.  Records were stored on  clay  tiles\footfullcite{Grajalez2013}.

Since the early forms of writing, humanity abilities to record events and information
increased significantly.  The first known written records date back to around 3,500 BC, the
Sumerian archaic (pre-cuneiform) writing.  This writing system was used to represent
commodities using clay tokens and to record transactions\footfullcite{Ifrah1998}.

``Data storage'' was also a challenge.  Some important devices that increased our capacity
to register textual information are the Sumerian clay tablets (3,500 BC), the Egyptian
papyrus (3,000 BC), the Roman wax tablets (100 BC), the codex
(100 AD), the Chinese paper (200 AD), the printing press (1440), and the typewriter (1868).

% Talvez citar na parte de análise de dados
% Other mechanisms were also developed to store information in a more structured way.  Some
% important devices are
% the abacus (2,700 BC), the Antikythera mechanism (150 -- 100 BC), the
% Chinese South Pointing Chariot (260 AD), the Pascaline (1642), the Jacquard loom (1801),
% the Babbage Difference Engine (1822), the Babbage Analytical Engine (1837).

Besides those improvements in unstructured data storage, at least in the Western and
Middle Eastern world, there are no significant advances in structured data collection
until the \nth{19} century.  (A Eastern timeline research seems much hard to perform.
Unfortunally, I left it out in this book.)

I consider a major influential figure in the history of data
collection to be Florence Nightingale (1820 -- 1910).  She was a passionate statistician
and probably the first person to use statistics to influence public and official
opinion.  The meticulous records she kept during the Crimean War
(1853 -- 1856) were the evidence that saved lives --- part of the mortality came from lack
of sanitation.  She was also the first to use
statistical graphics to present data in a way that was easy to understand.  She is
credited with developing a form of the pie chart now known as the polar area
diagram.  She also reformed healthcare in the United Kingdom and
is considered the founder of modern nursing; where great part of the work was to collect
data in a standardized way to quickly draw conclusions\footfullcite{Grajalez2013}.

% \begin{slidebox}{Pre-digital age}{}
%   \begin{itemize}
%     \item Babylonian census (3,800 BC)
%     \item Sumerian archaic (pre-cuneiform) writing (3,500 BC)
%     \item Egyptian papyrus (3,000 BC)
%     \item Phoenician alphabet (1,000 BC)
%     \item Greek alphabet (800 BC)
%     \item Roman wax tablets (100 BC)
%     \item Codex (100 AD)
%     \item Chinese paper (200 AD)
%     \item Printing press (1440)
%     \item Typewriter (1868)
%     \item Florence Nightingale (1820 -- 1910)
%   \end{itemize}
% \end{slidebox}
%
% \begin{slidebox}{Florence Nightingale}{}
%   \begin{itemize}
%     \item Passionate statistician.
%     \item First person to use statistics to influence public and official opinion.
%     \item Organized data from garden fruits and vegetables into numerical tables at the age of 9.
%     \item At 20 she was receiving two-hour lessons from a Cambridge-trained mathematician.
%     \item She found the sight of a long column of figures ``perfectly reviving.''
%     \item She went out to the Crimean War, to Scutari in Turkey, in 1854.
%     \item She found that not even the numbers of soldiers entering the hospitals, or leaving them – alive or dead – was known.
%     \item From the first she kept meticulous records.
%     \item The data she collected was the evidence that saved lives.
%     \item She was the first to use statistical graphics to present data in a way that was easy to understand.
%     \item She is credited with developing a form of the pie chart now known as the polar area diagram.
%     \item She reformed healthcare in the United Kingdom and is considered the founder of modern nursing.
%   \end{itemize}
% \end{slidebox}

\subsubsection{Digital age}

In the modern period, several devices were developed to store digital\footnote{Digital
means the representation of information in (finite) discrete form.  The term comes from the Latin
digitus, meaning finger, because it is the natural way to count using fingers.  Digital
here do not mean electronic.}
information.  One particular device that is important for data collection is the punched
card.  It is a piece of stiff paper that contains digital information represented by the
presence or absence of holes in predefined positions.  The information can be read by a
mechanical or electrical device called a card reader.  The earliest famous usage of
punched cards was by Basile Bouchon (1725) to control looms.  Most of the advances until
the 1880s were about the automation of machines (data processing) using hand-punched cards, and not
particularly specialized for data collection.

However, the 1890 census in the United States was the first to use machine-readable
punched cards to tabulate data. Processing 1880 census data took eight years, so the
Census Bureau contracted Herman Hollerith (1860 -- 1929) to design and build a tabulating
machine.  He founded the Tabulating Machine Company in 1896, which later merged with other
companies to become \gls{ibm} in 1924. Later
models of the tabulating machine were widely used for business applications such as
accounting and inventory control. Punched card technology remained a prevalent method of
data processing for several decades until more advanced electronic computers were
developed in the mid-\nth{20} century.

The invention of the digital computer is responsible for a revolution in data handling.
The amount of information we can capture and store increased exponentially.  ENIAC (1945) was
the first electronic general-purpose computer.  It was Turing-complete, digital, and
capable of being reprogrammed to solve a full range of computing problems.
It had 20 words of internal memory, each capable of storing a 10-digit decimal integer number.
Programs and data were entered by setting switches and inserting punched cards.

For the 1950 census, the United States Census Bureau used the
UNIVAC I (Universal Automatic Computer I), the first commercially produced computer in the
United States\footnote{Read more in \url{https://www.census.gov/history/www/innovations/}.}.

It goes without saying that digital computers have become much more powerful and
sophisticated since then.  The data collection process has been easily automated and
scaled to a level that was unimaginable before.  However, ``where'' storing data is
not the only challenge.  ``How'' to store data is also a challenge.  The next period of
history addresses this issue.

% \begin{slidebox}{Digital age}{}
%   \begin{itemize}
%     \item Punched card (1725)
%     \item 1890 census and Hollerith's tabulating machine (1890)
%     \item ENIAC (1945)
%     \item UNIVAC I used by the United Census Bureau (1950)
%   \end{itemize}
% \end{slidebox}

\subsubsection{Formal age}

In 1970, Edgar Frank Codd (1923 -- 2003), a British computer scientist,
published a paper entitled ``A Relational Model
of Data for Large Shared Data Banks''\footfullcite{Codd1970}.  In this paper, he introduced
the concept of a relational model for database management.

A relational model organizes data in tables (relations) where each row represents a record
and each column represents an attribute of the record.  The tables are related by common
fields.  Codd showed means to organize the tables of a relational database to minimize
data redundancy and improve data integrity.  \Cref{sec:normalization} provides more details
on the topic.

His work was a breakthrough in the field of data management.  The standardization of
relational databases led to the development of \gls{sql} in 1974.
SQL is a domain-specific language used in programming and designed for managing data held
in a \gls{rdbms}.

As a result, a new challenge rapidly emerged: how to aggregate data from different
sources. Once data is stored in a relational database, it is easy to query and manage
it. However, data is usually stored in different databases, and it is not always possible
to directly combine them.

% \begin{slidebox}{Edgar Frank Codd}{}
%   \begin{itemize}
%     \item British computer scientist
%     \item Introduced the concept of a relational model for database management
%     \item Standardized relational databases
%     \item Led the development of Structured Query Language (SQL)
%   \end{itemize}
% \end{slidebox}

\subsubsection{Integrated age}

The solution to this problem was the development of the \gls{etl}
process.  \gls{etl} is a process in data warehousing responsible for extracting data from
several sources, transforming it into a format that can be analyzed, and loading it into a
data warehouse.

The concept of data warehousing dates back to the late 1980s when IBM researchers Barry
Devlin and Paul Murphy developed the ``business data warehouse.''

Two major figures in the history of \gls{etl} are Ralph Kimball (born 1944) and Bill Inmon (born
1945), both American computer scientists.  Although they
differ in their approaches, they both agree that data warehousing is the foundation for
\gls{bi} and analytics, and that data warehouses should be designed to
be easy to understand and fast to query for business users.

A famous debate between Kimball and Inmon is the top-down versus bottom-up approach to
data warehousing.  Inmon's approach is top-down, where the data warehouse is designed
first and then the data marts\footnote{A data mart is a specialized subset of a data
warehouse that is designed to serve the needs of a specific business unit, department, or
functional area within an organization.} are created from the data warehouse.  Kimball's
approach is bottom-up, where the data marts are created first and then the data warehouse
is created from the data marts.

% \begin{slidebox}{Ralph Kimball}{}
%   \begin{itemize}
%     \item American computer scientist.
%     \item Developed the bottom-up approach to data warehousing.
%     \item From available data marts, the data warehouse is created.
%   \end{itemize}
% \end{slidebox}
%
% \begin{slidebox}{Bill Inmon}{}
%   \begin{itemize}
%     \item American computer scientist.
%     \item Developed the top-down approach to data warehousing.
%     \item Design the data warehouse first and then the data marts are created from the data warehouse.
%   \end{itemize}
% \end{slidebox}

One of the earliest and most famous case studies of the implementation of a data warehouse
is that of Walmart. In the early 1990s, Walmart faced the challenge of managing and
analyzing vast amounts of data from its stores across the United States. The company
needed a solution that would enable comprehensive reporting and analysis to support
decision-making processes.  The solution was to implement a data warehouse that would
integrate data from various sources and provide a single source of truth for the
organization.

\subsubsection{Ubiquitous age}

The last and current period of history is the ubiquitous age.  It is characterized by the
proliferation of data sources.

The ubiquity of data generation and the evolution of data-centric technologies have been
made possible by a multitude of figures across various domains.

\begin{itemize}
  \item Vinton Gray Cerf (born 1943) and Robert Elliot Kahn (born 1938), often referred to
    as the ``Fathers of the Internet,'' developed the TCP/IP protocols, which are
    fundamental to internet communication.
  \item Tim Berners-Lee (born 1955), credited with inventing the World Wide Web, laid the
    foundation for the massive data flow on the internet.
  \item Steven Paul Jobs (1955 -- 2011) and Stephen Wozniak (born 1950), from Apple Inc.,
    and William Henry Gates III (born 1955), from Microsoft Corporation, were responsible
    for the introduction of personal computers, leading to the democratization of data
    generation.
  \item Lawrence Edward Page (born 1973) and Sergey Mikhailovich Brin (born 1973), the
    founders of Google, transformed how we access and search for information.
  \item Mark Elliot Zuckerberg (born 1984), the co-founder of Facebook, played a crucial
    role in the rise of social media and the generation of vast amounts of user-generated
    content.
\end{itemize}

In terms of data handling, this change of landscape has brought about the
development of new technologies and techniques for data storage and processing.  Especially
the development of NoSQL databases and distributed computing frameworks.

NoSQL databases are non-relational databases that can store and process large volumes of
unstructured, semi-structured, and structured data.  They are highly scalable and
flexible, making them ideal for big data applications.

Some authors argue that the rise of big data is characterized by the five V's of big data:
Volume, Velocity, Variety, Veracity, and Value.  The amount of data generated is massive,
the speed at which data is generated is high, the types of data generated are diverse, the
quality of data generated is questionable, and the value of data generated is high.

Once massive amounts of unstructured data became available, the need for new data
processing techniques arose.  The development of distributed computing frameworks such as
Apache Hadoop and Apache Spark enabled the processing of massive amounts of data in a
distributed manner.

Douglass Read Cutting and Michael Cafarella, the developers of Apache Hadoop,
proposed the \gls{hdfs} and MapReduce, which are the
cornerstones of the Hadoop framework, in 2006.  Hadoop's distributed storage and
processing capabilities enabled organizations to handle and analyze massive volumes of
data.

Currently, Google holds a patent for
MapReduce\footnote{\url{http://static.googleusercontent.com/media/research.google.com/es/us/archive/mapreduce-osdi04.pdf}}.
However, their framework inherits from the architeture proposed in
\textcite{Hillis1985}\footfullcite{Hillis1985} thesis.
MapReduce is not particularly novel, but its simplicity and scalability made it popular.

Nowadays, another important topic is \gls{iot}.  IoT is a system of
interrelated computing devices that communicate with each other over the internet.
The devices can be anything from cellphones, coffee makers, washing machines, headphones,
lamps, wearable devices, and almost anything else you can think of.  The reality of IoT increased the
challenges of data handling, especially in terms of data storage and processing.

In summary, we currently live in a world where data is ubiquitous and comes in many
different forms.  The challenge is to collect, store, and process this data in a way that
is meaningful and useful, also respecting privacy and security.

\subsection{Timeline of data analysis}
\label{sub:time-analysis}

The way we think about data and knowledge extraction has evolved significantly over the
years.  In the following, I present some of the most important milestones in the history
of data analysis and knowledge extraction.

\begin{figurebox}[label=fig:data-analysis-history]{Timeline of the ages of data analysis.}
  \centering
  \begin{tikzpicture}
    \draw (0,0) -- (8,0);
    \foreach \x in {0,1,...,8} {
      \draw (\x,-0.1) -- (\x,0.1);
    }
    \foreach \x/\y/\z in {%
        0/Summary statistics/{3,800 BC -- 16th c.},
        4/Probability advent/{17th c. -- 19th c.},
        7/Learning from data/{20th c. -- present}} {
      \node[anchor=north] at (\x,-0.1) {\footnotesize\y};
      \node[anchor=south] at (\x,0.1) {\footnotesize\z};
    }
  \end{tikzpicture}
\end{figurebox}

\Cref{fig:data-analysis-history} illutrates the proposed timeline.  I consider changes of
ages to be smooth transitions, and not strict boundaries.  The theoretical advances are
slower than the technological ones --- the latter influences more data handling than data
analysis ---, so not much has changed since the beginning of the \nth{20} century.

\subsubsection{Summary statistics}

The earliest known records of systematic data analysis date back to the first censuses.
The term \emph{statistics} itself refer to the analysis of data \emph{about the state},
including demographics and economics.  That early (and simplest) form of statistical
analysis is called \emph{summary statistics}, which consists of describing data in terms
of its central tendencies (e.g. arithmetic mean) e variability (e.g. range).

\subsubsection{Probability advent}

However, after the seventeenth century, the foundations of modern probability theory were
laid out.  Important figures for developing the probability theory are Blaise Pascal (1623
-- 1662), Pierre de Fermat (1607 -- 1665), Christiaan Huygens (1629 -- 1695), and Jacob
Bernoulli (1655 -- 1705).

The foundation methods brought to life the field of statistical inference. In the
following years, important results were achieved.

\paragraph{Bayes' rule}

Reverend Thomas Bayes (1701 -- 1761) was an English statistician, philosopher, and
presbyterian minister.  He is known for formulating a specific case of the theorem that
bears his name: Bayes' theorem.  The theorem is used to calculate conditional
probabilities using an algorithm (his Proposition 9, published in 1763) that uses evidence to calculate
limits on an unknown parameter.

The Bayes' rule is the foundation of learning from evidence, once it allows us to
calculate the probability of an event based on prior knowledge of conditions that might be
related to the event.  Classifiers based on Naïve Bayes --- the application of Bayes'
theorem with strong independence assumptions between known variables --- is likely to have
been used since the second half of the eighteenth century.

\paragraph{Gauss' method of least squares}

Johann Carl Friedrich Gauss (1777 -- 1855) was a German mathematician and physicist who made
significant contributions to many fields in mathematics and sciences.  Circa 1794, he
developed the method of least squares for calculating the orbit of Ceres to minimize the
impact of measurement error\footnote{The method was first published by Adrien-Marie
Legendre (1752 -- 1833) in 1805, but Gauss claimed in 1809 that he
had been using it since circa 1794.}.

\paragraph{Playfair's data visualization}

William Playfair (1759 -- 1823) was a secret agent on behalf of Great Britain during its
war with France in the 1790s.  He invented several types of diagram between 1780s and
1800s, such as the line, area and bar chart of economic data, and the pie chart and circle
graph to show proportions.

% \begin{slidebox}{Probability advent}{}
%   \begin{itemize}
%     \item Foundations by Blaise Pascal, Pierre de Fermat, Christiaan Huygens, Jacob
%       Bernoulli, and others (\nth{17} century)
%     \item Bayes' rule by Thomas Bayes (1763)
%     \item Gauss' method of least squares by Johann Carl Friedrich Gauss (1794)
%     \item Playfair's data visualization by William Playfair (1780s -- 1800s)
%   \end{itemize}
% \end{slidebox}

\subsubsection{Learning from data}

In the twentieth century and beyond, new advances were made in the field of statistics.
The development of learning machines enabled the development of new techniques for data
analysis.

The recent advances in computation and data storage are crucial for the large-scale
application of these techniques.

\paragraph{Fisher's discriminant analysis}

In the 1930s, Ronald Fisher (1890 -- 1962) developed discriminant analysis, which was
considered a problem of constructing a decision rule to assign a vector to one of two
categories using given probability distribution functions\footnote{After, Rosenblatt's work,
however, it was used to solve inductive inference as well.}.

% See \cref{sub:fisher} for more details about the technique.

\paragraph{Shannon's information theory}

The field, that studies quantification, storage and communication of information, was
originally established by the works of Harry Nyquist (1889 -- 1976) and Ralph Hartley
(1888 -- 1970) in the 1920s, and Claude Shannon (1916 -- 2001) in the 1940s.
Information theory brought many important concepts to the field of data analysis, such as
entropy, mutual information, and information gain.  This theory is the foundation of
several machine learning algorithms.

\paragraph{K-Nearest Neighbors}

In 1951, Evelyn Fix (1904 -- 1965) and Joseph Lawson Hodges Jr. (1922 -- 2000) wrote a
technical report entitled ``Discriminatory Analysis, Nonparametric Discrimination:
Consistency Properties.''  In this paper, they proposed the k-nearest neighbors algorithm,
which is a non-parametric method used for classification and regression.  The algorithm
marks a shift from the traditional parametric methods --- and purely statistical ---
to non-parametric methods.

% See \cref{sub:knn} for more details about the technique.

\paragraph{Rosenblatt's perceptron}

In the 1960s, Frank Rosenblatt (1928 -- 1971) developed the perceptron, the first model of
a learning machine.  While the idea of a mathematical neuron was not new, he was the first
to describe the model as a program, showing the ability of the perceptron to learn simple
tasks such as the logical operations AND and OR.

% See \cref{sub:perceptron} for more details about the technique.

\paragraph{Hunt inducing trees}

In 1966, \citeauthor{Hunt1966}'s book\footfullcite{Hunt1966} described a way to induce decision trees from
data.  The algorithm is based on the concept of information entropy and is a precursor of
the \citeauthor{Quinlan1986}'s ID3 algorithm\footfullcite{Quinlan1986} and its variations.
These algorithm gave rise to the field of decision trees, which is a popular method for
classification and regression.

% See \cref{sub:tree} for more details about the technique.

% \begin{slidebox}{Learning from data (part I)}{}
%   \begin{itemize}
%     \item Fisher's discriminant analysis by Ronald Fisher (1930s)
%     \item Shannon's information theory by Claude Shannon (1940s)
%     \item K-Nearest Neighbors by Evelyn Fix and Joseph Lawson Hodges Jr. (1951)
%     \item Rosenblatt's perceptron by Frank Rosenblatt (1960s)
%     \item Hunt inducing trees by John Ross Quinlan (1966)
%   \end{itemize}
% \end{slidebox}

\paragraph{Empirical risk minimization principle}

Although many learning machines where developed until the 1960s, they did not advanced
significantly the understanding of the general problem of learning from data.  Between
1960s and 1986 --- before the backpropagation algorithm was proposed ---, the field of practical
data analysis was basically stagnant.  The main reason for that was the lack of a
theoretical framework to support the development of new learning machines.

However, these years were not completely unfruitful.  As early as 1968, Vladimir Vapnik
(born 1936)
and Alexey Chervonenkis (1938 -- 2014) developed the foundamental concepts of VC entropy
and VC dimension for the data classification problems.  As a result, a novel inductive
principle was proposed: the \gls{erm} principle.
This principle is the foundation of statistical learning theory.

% \paragraph{Algorithmic complexity}
%
% Solomonoff, Kolmogorov, and Chaitin proposed the first learning model based on
% algorithmic complexity.

\paragraph{Resurgence of neural networks}

In 1986, researchers developed independently a method to optimize coefficients of a neural
network\footfullcite{LeCun1986,Rumelhart1986}.  The method is called backpropagation and
is the foundation of the resurgence of neural networks.

This rebirth of neural networks happened in a scenario very different from the 1960s.
The availability of data and computational power fueled a new approach to the problem of
learning from data.  The new approach preferred the use of simple algorithms and
intuitive models over theoretical models, fueling areas such as bioinspired computing and
evolutionary computation.

\paragraph{Ensembles}

Following the new approach, ensemble methods were developed.  Based on ideas of
boosting\footfullcite{Schapire1990} and bagging\footfullcite{Breiman1996}, ensemble
methods combine multiple learning machines to improve the performance of the individual
machines.

The difference between boosting and bagging is the way the ensemble is built.  In
boosting, the ensemble is built sequentially, where each new model tries to correct the
errors of the previous models.  In bagging, the ensemble is built in parallel, where each
model is trained independently with small changes in the data.  The most famous bagging
ensemble methods are random forests\footfullcite{Ho1995}, while XGBoost, a gradient
boosting method\footfullcite{Friedman2001}, has been extensively used in machine learning
competitions.

\paragraph{Support vector machines}

In 1995, \citeauthor{Cortes1995}\footfullcite{Cortes1995} proposed the \gls{svm} algorithm, a
learning machine based on the VC theory and the \gls{erm} principle.  Based on Cover's
theorem\footfullcite{Cover1965}, they developed a method that finds the optimal hyperplane
that separates two classes of data in a high-dimensional space with the maximum margins.
The resulting method led to practical and efficient learning machines.

\paragraph{Deep learning revolution}

Although the ideia of neural networks with multiple layers were around since the 1960s,
only in the late 2000s the field of deep learning caught the attention of the scientific
community by achieving state-of-the-art results in computer vision and natural language
processing.  Yoshua Bengio, Geoffrey
Hinton and Yann LeCun are recognized for their for conceptual and engineering
breakthroughs in the field, winning the 2018 Turing Award\footnote{\url{https://awards.acm.org/about/2018-turing}}.

% \paragraph{Knowledge discovery in databases}
%
% 1990s, Fayyad, Piatesky-Shapiro, and Smyth.
% Developed the KDD process, which is the foundation of data mining.
% Data mining is the process of discovering patterns in large data sets involving methods at
% the intersection of machine learning, statistics, and database systems.

\paragraph{Generative deep models}

Nowadays, generative deep models are a hot topic in machine learning. They are a class of
statistical models that can generate new data instances. They are used in unsupervised
learning to discover hidden structures in unlabeled data (e.g. clustering), and in
supervised learning to generate new synthetic data instances.  The most famous generative
models are the generative transformers and generative adversarial networks.

\paragraph{LUSI learning theory}

In 2010s, \citeauthor{Vapnik2015}\footfullcite{Vapnik2015} proposed the \gls{lusi}
principle, which is an extension of the statistical learning theory.  The \gls{lusi}
theory is based on the concept of statistical invariants, which are properties of
the data that are preserved under transformations.  The theory is the foundation of the
learning from intelligent teachers paradigm.  They regard the \gls{lusi} theory as the
next step in the evolution of learning theory, calling it the ``complete statistical
theory of learning''.

% \begin{slidebox}{Learning from data (part II)}{}
%   \begin{itemize}
%     \item Empirical risk minimization principle by Vladimir Vapnik and Alexey Chervonenkis (1968)
%     \item Resurgence of neural networks (1986)
%     \item Ensembles (1990s)
%     \item Support vector machines by Vladimir Vapnik and Corinna Cortes (1995)
%     \item Deep learning revolution (2000s)
%     \item Generative models (2010s)
%     \item LUSI learning theory by Vladimir Vapnik and Rauf Izmailov (2010s)
%   \end{itemize}
% \end{slidebox}

\chapter{Fundamental data concepts}
\label{chap:data}

\chapterprecishere{The simple believes everything,
  \par\raggedleft but the prudent gives thought to his steps.
  \par\raggedleft--- \textup{Proverbs 14:15} (ESV)}

% \begin{itemize}
%   \item Variables, types etc
%   \item Records
%   \item Tidy data
% \end{itemize}

A useful start point for someone studying data science is the definition of the term
itself.

\begin{mainbox}{Chapter remarks}

  \boxsubtitle{Contents}

  \startcontents[chapters]
  \printcontents[chapters]{}{1}{}
  \vspace{1em}

  \boxsubtitle{Context}

  \begin{itemize}
    \item \dots
  \end{itemize}

  \boxsubtitle{Objectives}

  \begin{itemize}
    \item \dots
  \end{itemize}

  \boxsubtitle{Takeways}

  \begin{itemize}
    \item \dots
  \end{itemize}
\end{mainbox}

{}
\clearpage

For \textcite{Zumel2019}, \emph{``data science is a cross-disciplinary practice that draws
on methods from data engineering, descriptive statistics, data mining, machine learning,
and predictive analytics.''}  They compare the area with the operations research, stating
that data science focuses on implementing data-driven decisions and managing their
consequences.

\begin{slidebox}{Zumel and Mount's definition}{}
  \begin{itemize}
    \item Cross-disciplinary practice that draws on methods from data
    engineering, descriptive statistics, data mining, machine learning, and predictive
    analytics.
    \item Focuses on implementing data-driven decisions and managing their consequences.
  \end{itemize}
\end{slidebox}

\textcite{Wickham2023} state that \emph{``data science is an exciting discipline that
allows you to transform raw data into understanding, insight, and knowledge.''}

\begin{slidebox}{Wickham's definition}{}
  \begin{itemize}
    \item Transform raw data into understanding, insight, and knowledge.
    \item Not necessarily a definition; describes the purpose of data science.
  \end{itemize}
\end{slidebox}

I find the first definition too restrictive once new methods and techniques are always
under development.  We never know when new ``data-related'' methods will become obsolete
or a trend.  Also, \citeauthor{Zumel2019}'s view gives the impression that data science is a
operations research subfield.  Although I do not try to prove otherwise, I think it
is much more useful to see it as an independent field of study.  Obviously, there are
many intersections between both areas (and many other areas as well).  Because of such
intersections, I try my best to keep definitions and
terms standardized throughout chapters, sometimes avoiding popular terms that may generate
ambiguities or confusion.

The second one is not really a definition.  However, it states clearly \emph{what} data
science enables us to do.  From these thoughts, let's define the term.

\begin{defbox}{}{ds}
  Data science is the study of computational methods to extract knowledge from
  measurable phenomena.
\end{defbox}

I want to highlight the meaning of some terms in this definition.  \emph{Computational methods} means
that data science methods use computers to handle data and perform the calculations.
\emph{Knowledge} means information that humans can easily understand or apply to solve
problems.  \emph{Measurable phenomena} are events or processes where raw data can be
quantified in some way\footnote{TODO: talk about non-measurable phenomena}.  \emph{Raw data} are data collected directly from some source and
that have not been subject to any other transformation by a software program or a human
expert.  \emph{Data} is any piece of information that can be digitally stored.

\begin{slidebox}{My definition}{}
  \begin{itemize}
    \item Data science is the study of computational methods to extract knowledge from
      measurable phenomena.
    \item Computational methods use computers to handle data and perform the calculations.
    \item Knowledge is information that humans can easily understand or apply to solve
      problems.
    \item Measurable phenomena are events or processes where raw data can be quantified
      in some way.
    \item Raw data are data collected directly from some source and that have not been
      subject to any other transformation by a software program or a human expert.
    \item Data is any piece of information that can be digitally stored.
  \end{itemize}
\end{slidebox}

\textcite{Kelleher2018} summarize very well the challenges data science takes up:
``extracting non-obvious and useful patterns from large data sets [\dots]; capturing,
cleaning, and transforming [\dots] data; [storing and processing] big [\dots] data sets;
and questions related to data ethics and regulation.''

\begin{slidebox}{Kelleher and Tierney's challenges}{}
  \begin{itemize}
    \item Extracting non-obvious and useful patterns from large data sets.
    \item Capturing, cleaning, and transforming data.
    \item Storing and processing big data sets.
    \item Questions related to data ethics and regulation.
  \end{itemize}
\end{slidebox}

Data science contrasts with conventional sciences.  Usually, a ``science'' is named after
its object of study.  Biology is the study of the life, Earth science studies the planet
Earth, and so on.  I argue that data science does not study data itself, but how we can
use them to understand a phenomenon.

Besides, the conventional scientific paradigm is
essentially model-driven: we observe a phenomenon related to the object of study, we
reason the possible explanation (the model or hypothesis), and we validate our hypothesis
(most of the time using data, though).  In data science, however, we extract the knowledge
directly and primarily from the data.  The expert knowledge and reasoning may be taken
into account, but we give data the opportunity to surprise us.

Thus, the objects of the
study in data science are the computational methods and processes that can extract
reliable and ethical knowledge from huge amounts of data.

\def\verrids{(0,0) circle (20mm)}
\def\verrist{(-2.5,0) circle (15mm)}
\def\verride {(2.5,0) circle (15mm)}
\def\verrics {(0,-2.5) circle (15mm)}

\begin{figurebox}[label=fig:myview]{My view of data science.}
  \centering
  \begin{tikzpicture}
    \begin{scope}
      \clip \verrids;
      \fill[filled] \verrist;
      \fill[filled] \verride;
      \fill[filled] \verrics;
    \end{scope}
    \draw[outline] \verrids node(ds) {};
    \draw[outline] \verrist node {statistics};
    \draw[outline, text width=27mm, text centered] \verride node {domain expertise philosophy};
    \draw[outline] \verrics node {computer science};
    \node[anchor=north,above] at (0, 1) {data science};
  \end{tikzpicture}
  \tcblower
    Data science is an entire new science.  Being a new science
    does not mean that its basis is built from the ground up.  Most of the subjects in
    data science come from other sciences, but its object of study (computational methods
    to extract knowledge from measurable phenomena) is particular enough to unfold
    new scientific questions -- such as data ethics, data collection, etc.
\end{figurebox}

\begin{slidebox}{Data science vs conventional sciences}{}
  \begin{itemize}
    \item Conventional sciences are model-driven: observation, hypothesis, and validation.
    \item In data science, we extract the knowledge directly and primarily from the data.
    \item Data science studies the computational methods and processes that can extract
      reliable and ethical knowledge from huge amounts of data.
  \end{itemize}
\end{slidebox}

\section{Fundamental data theory}

As expected, data science is not a isolated science.  It incorporates several concepts
from other fields and sciences.  In this section, I explain the basis of each component of
the provided definition.

\subsection{Phenomena}

Phenomenon is a term used to describe any observable event or process.  They are the
source we use to understand the world around us.  In general, we use our senses to
perceive phenomena.  To make sense of them, we use our knowledge and reasoning.

Philosophy is the study of knowledge and reasoning.  It is a very broad field of study
that has been divided into many subfields.  One of them is epistemology, which is the
study of knowledge.  Epistemology is the field of philosophy that studies how we can
acquire knowledge and how we can distinguish between knowledge and opinion.  In
particular, epistemology studies the nature of knowledge, justification, and the
rationality of belief.

Another important subfield in philosophy is ontology, which is the study of being.  It
studies the nature of being, existence, or reality.  Ontology is the field of philosophy
that studies what exists and how we can classify it.  In particular, ontology studies the
nature of categories, properties, and relations.

Finally, logic is the study of reasoning.  It studies the nature of reasoning and
argumentation.  In particular, logic studies the nature of inference, validity, and
fallacies.

\begin{slidebox}{Philosophy}{}
  \begin{itemize}
    \item Epistemology: the study of knowledge.
    \item Ontology: the study of being.
    \item Logic: the study of reasoning.
  \end{itemize}
\end{slidebox}

In the context of data science, we usually focus on phenomena from particular domain of
expertise.  For example, we may be interested in the phenomena related to the stock
market, the phenomena related to the weather, or the phenomena related to the human
health.  Thus, we need to understand the nature of the phenomena we are studying.

Thus, fully understading the phenomena we are tackling requires both a general knowledge
of epistemology, ontology, and logic, and a particular knowledge of the domain of
expertise.

Observe as well that we do not restrict ourselves to the ``qualitative'' understanding of
philosophy.  There are several computational methods that implements the concepts of
epistemology, ontology, and logic.  For example, we can use a computer to perform
deductive reasoning, to classify objects, or to validate an argument.  Also, we have
strong mathematical foundations and computational tools to organize categories, relations, and
properties.

The reason we need to understand the nature of the phenomena we are studying is that we
need to guarantee that the data we are collecting are relevant to the problem we are
trying to solve.  Incorrectly perception of the phenomena may lead to incorrect data
collection, which may lead to incorrect conclusions.

\begin{slidebox}{Phenomena}{}
  \begin{itemize}
    \item Phenomena are the source we use to understand the world around us.
    \item We use our senses to perceive phenomena.
    \item We use our knowledge and reasoning to make sense of them.
    \item Computational methods can be used to implement knowledge and reasoning.
    \item Phenomena are the source of data.
    \item We need to understand the nature of the phenomena we are studying.
    \item Incorrectly perception of the phenomena may lead to incorrect data collection,
      which may lead to incorrect conclusions.
  \end{itemize}
\end{slidebox}

\subsection{Measurements}

In data science, we are interested in measurable phenomena.  Measurable phenomena are
those that we can quantify in some way.  For example, the temperature of a room is a
measurable phenomenon because we can measure it using a thermometer.  The number of
people in a room is also a measurable phenomenon because we can count them.

When we quantify a phenomenon, we perform data collection.  Data collection is the process
of gathering data on targeted phenomenon in an established systematic way.
Systematic means that we have a plan to collect the data and we understand the
consequences of the plan, including the sampling bias.  Sampling bias is the influence
that the method of collecting the data has on the conclusions we can draw from them.
Once we have collected the data, we need to store them.  Data storage is the process of
storing data in a computer.

To perform those tasks, we need to understand the nature of data.  Data are any piece of
information that can be digitally stored.  Data can be stored in many different formats.
For example, we can store data in a spreadsheet, in a database, or in a text file.  We can
also store data in many different types.  For example, we can store data as numbers,
strings, or dates.

In data science, studying data types is important because they need to correctly reflect
the nature of the source phenomenon and be compatible with the computational methods we
are using.  Data types also restrict the operations we can perform on the data.

The foundation and tools to understand data types come from computer science.  Among the
subfields, I highlight:
\begin{itemize}
  \item Algorithms and data structures: the study of data types and the computational
    methods to manipulate them.
  \item Databases: the study of storing and retrieving data.
\end{itemize}

\begin{slidebox}{Measurable phenomena}{}
  \begin{itemize}
    \item Measurable phenomena are those that we can quantify in some way.
    \item Data collection is the process of gathering data on targeted phenomenon in an
      established systematic way.
    \item The collection/sampling bias influence our conclusions.
    \item Data storage is the process of storing data in a computer.
    \item Data are any piece of information that can be digitally stored.
    \item Data can be stored in many different formats.
    \item Data can be stored in many different types.
    \item Data types need to correctly reflect the nature of the source phenomenon and be
      compatible with the computational methods we are using.
    \item Algorithms, data structures, and databases are important subfields of computer
      science when studying data collection, data storage, and data types.
  \end{itemize}
\end{slidebox}

\subsection{Knowledge extraction}

Once we have collected and stored the data, we need to extract knowledge from them.  In
data science, we use computational methods to extract knowledge from data.  These
computational methods may come from many different fields.  In particular, I highlight:
\begin{itemize}
  \item Statistics: the study of data collection, organization, analysis, interpretation,
    and presentation.
  \item Machine learning: the study of computational methods that can automatically learn from data.
  \item Artificial intelligence: the study of computational methods that can mimic human
    intelligence.
\end{itemize}

Also, many other fields contribute to the development of domain-specific computational
methods to extract knowledge from data.  For example, in the field of biology, we have
bioinformatics, which is the study of computational methods to analyze biological data.
Earth sciences have geoinformatics, which is the study of computational methods to
analyze geographical data.  And so on.

Each method has its own assumptions and limitations.  Thus, we need to understand the
nature of the methods we are using.  In particular, we need to understand the
expected input and output of them.  Whenever the available data do not match the
requirements of the method, we may perform data handling\footnote{%
  It is important to highlight that it is expected that some of the methods assumptions
  are not fully met.  These methods are usually robust enough to extract valuable
  knowledge even when data contain imperfections, errors and noise.  However, it is still
  useful to perform data handling to adjust data as much as possible.%
}.

Data handling mainly includes data cleaning, data transformation, and data
integration. Data cleaning is the process of detecting and correcting (or removing)
corrupt or inaccurate pieces of data.  Data transformation is the process of converting
data from one format or type to another.  Data integration is the process of combining
data from different sources into a single, unified view.

\begin{slidebox}{Knowledge extraction}{}
  \begin{itemize}
    \item We use computational methods to extract knowledge from data.
    \item Statistics, machine learning, and artificial intelligence are important
      sciences when studying knowledge extraction.
    \item Computational methods always have their own assumptions and limitations.
    \item Data handling is the process of adjusting data to the requirements of the
      computational methods, which includes:
    \begin{itemize}
    \item Data cleaning is the process of detecting and correcting (or removing) corrupt
      or inaccurate pieces of data.
    \item Data transformation is the process of converting data from one format or type
      to another.
    \item Data integration is the process of combining data from different sources into
      a single, unified view.
    \end{itemize}
  \end{itemize}
\end{slidebox}

\section{Structured data}

As one expects, when we measure a phenomenon, the resulting data come in many different
formats.  For example, we can measure the temperature of a room using a thermometer.  The
resulting data is a number.  We can assess English proficiency using an essay test.  The
resulting data is a text.  We can register relationships between proteins and
their functions.  The resulting data is a graph.

Thus, it is important to understand the nature of the data we are working with.

The most common data format is the \emph{structured data}.  Structured data are data that
are organized in a tabular format.  Each row in the table represents a single observation
and each column represents a variable that describes the observation.

We restrict the kind of information we store in each cell, i.e. the data type of each
measurement.  Each column has a data type.  The data type restrict the operations we can
perform on the data.  For example, we can perform arithmetic operations on numbers, but
not on text.

The most common classification of data types is Stevens’s types: nominal, ordinal,
interval, and ratio.  Nominal data are data that can be classified into categories.
Ordinal data are data that can be classified into categories and ordered.  Interval data
are data that can be classified into categories, ordered, and measured in fixed units.
Ratio data are data that can be classified into categories, ordered, measured in fixed
units, and have a true zero.  In practice, they differ on the logical and arithmetic operations
we can perform on them.

\begin{tablebox}{Stevens’s types.}
  \centering
  \rowcolors{2}{black!10!white}{}
  \begin{tabular}{cc}
    \toprule
    \textbf{Data type} & \textbf{Operations} \\
    \midrule
    Nominal & $=$ \\
    Ordinal & $=, <$ \\
    Interval & $=, <, +, - $ \\
    Ratio & $=, <, +, -, \times, \div$ \\
    \bottomrule
  \end{tabular}
\end{tablebox}

However, Stevens’s types do not exhaust all possibilities for data types.  For example,
probabilities are bounded at both ends, and thus do not tolerate arbitrary scale shifts.
\textcite{Paul1993} provide interesting insights about data types.  Although I do not
agree with all his points, I think it is a good reading.  In particular, I agree with
his criticism of statements that data types are evident from the data independent of the
questions asked.  The same data can be interpreted in different ways depending on the
context and the goals of the analysis.

However, I do not agree with the idea that good data analysis does not assume data types.
I think that data scientists should be aware of the data types they are working with and
how they affect the analysis.  With no bias, there is no learning.  There is no such a
thing as a ``bias-free'' analysis, the amount of possible combinations of assumptions
easily grows out of control.  The data scientist must take responsibility for the
consequences of their assumptions.  Good assumptions and hypothesis are a key part of the
data science methodology.

\begin{slidebox}{Structured data}{}
  \begin{itemize}
    \item Structured data are data that are organized in a tabular format.
    \item Each row in the table represents a single observation.
    \item Each column represents a variable that describes the observation.
    \item Each column has a data type.
    \item The data type restrict the operations we can perform on the data.
    \item The most common classification of data types is Stevens’s types: nominal,
      ordinal, interval, and ratio.
    \item Stevens’s types do not exhaust all possibilities for data types.
    \item Data scientists should be aware of the data types they are working with and
      how they affect the analysis.
    \item Inevitably, data scientists make assumptions about the data types.
  \end{itemize}
\end{slidebox}

When we work with structured data, two concepts are very important: database normalization
and tidy data.  Database normalization is mainly focused on the data storage.  Tidy data is
mainly focused on the requirements of data for analysis.  Both concepts have their
mathematical foundations and tools for data handling.

\subsection{Database normalization}
\label{sec:normalization}

Database normalization is the process of organizing the columns and tables of a relational
database to minimize data redundancy and improve data integrity.

Normal form is a state of a database that is free of certain types of data redundancy.
Before studying normal forms, we need to understand basic concepts in the database theory
and the basic operations in relational algebra.

\subsubsection{Relational database theory}

\paragraph{Projection}  The projection of a relation is the operation that returns a
relation with only the columns specified in the projection.  For example, if we have a
relation $X[A, B, C]$ and we perform the projection $\pi_{A, C}(X)$, we get a
relation with only the columns $A$ and $C$, i.e. $X[A, C]$.

\paragraph{Join}  The (natural) join of two relations is the operation that returns a
relation with the columns of both relations.  For example, if we have two relations $S[U
\cup V]$ and $T[U \cup W]$, where $U$ is the common set of attributes, join $S \bowtie T$
of $S$ and $T$ is the relation with tuples $(u, v, w)$ such that $(u, v) \in S$ and $(u,
w) \in T$.  The generalized join is built up out of binary joins:  $\bowtie \left\{ R_1,
R_2, \dots, R_n \right\} = R_1 \bowtie R_2 \bowtie \dots \bowtie R_n$. Since the join
operation is associative and commutative, we can parenthesize however we want.

\paragraph{Functional dependency}  A functional dependency is a constraint between two
sets of attributes in a relation.  It is a statement that if two tuples agree on certain
attributes, then they must agree on another attribute.  Specifically, the \emph{functional
dependency} $U \to V$ holds in $R$ if and only if for every pair of tuples $t_1$ and $t_2$
in $R$ such that $t_1[U] = t_2[U]$, it is also true that $t_1[V] = t_2[V]$.

\paragraph{Multi-valued dependency}  A multi-valued dependency is a constraint between
two sets of attributes in a relation.  It is a statement that if two tuples agree on
certain attributes, then they must agree on another set of attributes.  Specifically, the
\emph{multi-valued dependency} $U \twoheadrightarrow V$ holds in $R$ if and only if $R =
R[UV] \bowtie R[UW]$, where $W$ are the remaining attributes.

\paragraph{Join dependency}  A join dependency is a constraint between subsets of
attributes (not necessarily disjoint) in a relation.  $R$ obeys the join dependency $*
\left\{ X_1, X_2, \dots, X_n \right\}$ if $R = \bowtie \left\{ R[X_1], R[X_2], \dots,
R[X_n] \right\}$.

\subsubsection{Normal forms}

\paragraph{First normal form (1NF)}  A relation is in 1NF if and only if all attributes
are atomic.  An attribute is atomic if it is not a set of attributes.  For example, the
relation $R[A, B, C]$ is in 1NF if and only if $A$, $B$, and $C$ are atomic.

\paragraph{Second normal form (2NF)}  A relation is in 2NF if and only if it is in 1NF
and every non-prime attribute is fully functionally dependent on the primary key.  A
non-prime attribute is an attribute that is not part of the primary key.  A primary key
is a set of attributes that uniquely identifies a tuple.  A non-prime attribute is fully
functionally dependent on the primary key if it is functionally dependent on the primary
key and not on any subset of the primary key.  For example, the relation $R[U \cup V]$ is
in 2NF if and only if $U \to X,~\forall X \in V$ and there is no $W \subset U$ such that
$W \to X,~\forall X \in V$.

\paragraph{Third normal form (3NF)}  A relation is in 3NF if and only if it is in 2NF
and every non-prime attribute is non-transitively dependent on the primary key.  A
non-prime attribute is non-transitively dependent on the primary key if it is not
functionally dependent on another non-prime attribute.  For example, the relation $R[U
\cup V]$ is in 3NF if and only if $U$ is the primary key and there is no $X \in V$ such
that $X \to Y,~\forall Y \in V$.

\paragraph{Boyce-Codd normal form (BCNF)}  A relation $R$ with attributes $X$ is in BCNF
if and only if it is in 2NF and for each nontrivial functional dependency $U \to V$ in
$R$, the functional dependency $U \to X$ is in $R$.  In other words, a relation is in BCNF
if and only if every functional dependency is the result of keys.

\paragraph{Fourth normal form (4NF)}  A relation $R$ with attributes $X$ is in 4NF if
and only if it is in 2NF and for each nontrivial multi-valued dependency $U \twoheadrightarrow
V$ in $R$, the functional dependency $U \to X$ is in $R$.  In other words, a relation is
in 4NF if and only if every multi-valued dependency is the result of keys.

\paragraph{Projection join normal form (PJNF)} A relation $R$ with attributes $X$ is in
PJNF\footnote{Also known as fifth normal form (5NF).} if and only if it is in 2NF and
the set of key dependencies\footnote{Key dependency is a functional dependency in the form
$K \to X$.} of $R$ impllies each join dependency of $R$.  The PJNF guarantees that the
table cannot be decomposed without losing information (except by decompositions based on
keys).

Note that the ideia behind the definition of BCNF and 4NF are slightly different from the
PJNF.  In fact, if we consider that for each key dependency implies a join dependency, the
relation is in the so-called overstrong projection-join normal
form\footfullcite{Fagin1979}.  Such a level of normalization does not improve data storage
or eliminate inconsistencies.  In practice, it means that if a relation is in PJNF,
careless joins --- i.e. those that violate a join dependency --- produce
inconsistent results.

\paragraph{Example 1}  Consider the 2NF relation $R[A, B, C, D]$ with the functional
dependencies $A \to B,~B \to C,~C \to D$.  The relation is not in 3NF because $C$ is
transitively dependent on $A$.  To normalize it, we can decompose it into the
relations $R_1[A, B, C]$ and $R_2[C, D]$.  Now, $R_2$ is in 3NF and $R_1$ is in 2NF, but
not in 3NF.  We can decompose $R_1$ into the relations $R_3[A, B]$ and $R_4[B, C]$.
The original relation can be reconstructed by $\bowtie \left\{ R_2, R_3, R_4 \right\}$.

\paragraph{Example 2} Consider the 2NF relation $R[ABC]$\footnote{Here we abreviate ${A,
B, C}$ as $ABC$.} such that the primary key is the composite of $A$, $B$, and $C$.  The
relation is thus in the 4NF, as no column is a determinant of another column.  Suppose,
however, the following constraint: if $(a, b, c')$, $(a, b', c)$, and $(a', b, c)$ are in
$R$, then $(a, b, c)$ is also in $R$.  This can be illustrated if we consider $A$ as a
agent, $B$ as a product, and $C$ as a company.  If an agent $a$ represents companies $c$ and
$c'$, and product $b$ is in his portfolio, then assuming both companies make $b$, $a$
must offer $b$ from both companies.

The relation is not in PJNF, as the join dependency $* \left\{ AB, AC, BC \right\}$ is not
implied by the primary key.  (In fact, the only functional dependency is the trivial $ABC
\to ABC$.)  In this case, to avoid redundancies and inconsistencies, we must split the
relation into the relations $R_1[AB]$, $R_2[AC]$, and $R_3[BC]$.

It is iteresting to notice that in this case, the relation $R_1 \bowtie R_2$ might
contain tuples that do not make sense in the context of the original relation.  For
example, if $R_1$ contains $(a, b)$ and $R_2$ contains $(a, c')$, the join contains
$(a, b, c')$, which might not be a valid tuple in the original relation if $(b, c')$ is
not in $R_3$.  \emph{This is very important to notice, as it is a common mistake to assume
that the join of the decomposed relations always contains valid tuples.}

\paragraph{Example 3}  Consider the 2NF relation $R[A, B, C, D, E]$ with the functional
dependencies $A \to D$, $AB \to C$, and $B \to E$.  To make it PJNF, we can decompose it
into the relations $R_1[A, D]$, $R_2[A, B, C]$, and $R_3[B, E]$.  The original relation can
be reconstructed by $\bowtie \left\{ R_1, R_2, R_3 \right\}$.  However, unlike the
previous example, the join of the decomposed relations always contains valid tuples
--- excluding degenerate joins, where there are no common attributes.
The reason is that all join dependencies implied by the key dependencies are trivial when
reduced\footnote{\color{red}A proof in under development based on \fullcite{Vincent1997}.}.

\begin{slidebox}{Database normalization}{}
  \begin{itemize}
    \item Minimizes data redundancy.
    \item Improves data integrity.
    \item Semantics is expressed in terms of dependencies (functional, multivalued, join),
      which is usually not clear.
    \item Appropriate to store data.
  \end{itemize}
\end{slidebox}

\subsection{Tidy data}

It is estimated that 80\% of the time spent on data analysis is spent on data preparation.
Usually, the same process is repeated many times in different datasets. The ideia is that
organized data carries the meaning of the data, reducing the time spent on handling
the data to get it into the right format for analysis.

Tidy data is a data format that provides a standardized way to organize data values within
a dataset.  The main advantage of tidy data is that it provides clear semantics with focus
on only one view of the data.

Many data formats might be ideal for particular tasks, such as raw data, dense tensors, or
normalized databases.  However, most of the statitiscal and machine learning methods
require a particular data format.  Tidy data is a data format that is appropriate to those
tasks.

\begin{mainbox}{Wickham's thoughts on tidy data}
  \em
  Like families, tidy datasets are all alike but every messy dataset is messy in its own
  way.
\end{mainbox}

In an unrestricted table, the meaning of rows and columns are not fixed.  In a tidy table,
the meaning of rows and columns are fixed.

It is based on the idea that a dataset is a collection of values, where:
\begin{itemize}
  \item Each \emph{value} belongs to a variable and an observation.
  \item Each \emph{variable}, represented by a column, contains all values that measure
    the same attribute across (observational) units.
  \item Each \emph{observation}, represented by a row, contains all values measured on the
    same unit across attributes.
  \item \emph{Attributes} are the characteristics of the units, e.g. height, temperature,
    duration.
  \item \emph{Observational units} are the individual entities being measured, e.g. a
    person, a day, an experiment.
\end{itemize}
Table \ref{tab:tidy} summarizes the main concepts.

\begin{tablebox}[label=tab:tidy]{Tidy data concepts.}
  \centering
  \rowcolors{2}{black!10!white}{}
  \begin{tabular}{cccc}
    \toprule
    \textbf{Concept} & \textbf{Structure} & \textbf{Contains} & \textbf{Across} \\
    \midrule
    Variable & Column & Same attribute & Units \\
    Observation & Row & Same unit & Attributes \\
    \bottomrule
  \end{tabular}
\end{tablebox}

If we follow this structure, the meaning of data is implicit in the table itself.
However, it is not always trivial to organize data in a tidy format.  Usually, we have
more than one level of observational units, each one represented by a table.  Moreover,
there might exist more than one way to define what are the observational units in a
dataset.

To organize data in a tidy format, one can consider that:
\begin{itemize}
  \item Attributes are functionally related among themselves --- e.g. Z is a linear
    combination of X and Y, or X and Y are correlated, or $P(X, Y)$ follows some joint distribution.
  \item Units can be grouped or compared --- e.g. person A is taller than person B, or
    the temperature in day 1 is higher than in day 2.
\end{itemize}

A particular point that tidy data do not address is that values in a column might not be
in the same scale or unit of measurement\footnote{Attention: observational unit is not
unit of measurement.}.  For example, a column might contain the
temperature in an experiment, and another column might contain the unit of measurement
that was used to measure the temperature.  This is a common problem in databases, and it
must be addressed for machine learning and statistical methods to work properly.

Note that the order of the rows and columns is not important.  However, it might be
convenient to sort data in a particular way to facilitate the understanding.  For
instance, one usually expects that the first columns are \emph{fixed
variables}\footnote{Closely related (and potentially the same as) key in database
theory.}, i.e. variables that are not the result of a measurement, and the last columns
are \emph{measured variables}.  Also, arranging rows by some variable might highlight some
pattern in the data.

Usually, columns are named --- the collection of all column names is called the
header, while rows are numerically indexed.

\subsubsection{Common messy datasets}

\textcite{Wickham2014} lists some common problems with messy datasets and how to tidy
them\footnote{Operations are presented in \cref{chap:handling}.}.  The problems are
summarized below.

\paragraph{Headers are values, not variable names}  For example, consider
\cref{tab:messy1}.  This table is not tidy because the column headers are values, not
variable names.  This format is frequently used in presentations since it is more compact.
It is also useful to perform matrix operations. However, it is not appropriate for general
analysis.

\begin{tablebox}[label=tab:messy1]{Messy table, from Pew Forum dataset, where headers are values, not variable names.}
  \centering
  \rowcolors{2}{black!10!white}{}
  \begin{tabular}{l r r r c}
    \toprule
    Religion & <\$10k & \$10-20k & \$20-30k & \dots \\
    \midrule
    Agnostic & 27 & 34 & 60 & \dots \\
    Atheist & 12 & 27 & 37 & \dots \\
    Buddhist & 27 & 21 & 30 & \dots \\
    \dots & \dots & \dots & \dots & \dots \\
    \bottomrule
  \end{tabular}
\end{tablebox}

To make it tidy, we can transform it into the \cref{tab:tidy1} by explicitly introducing
variables \emph{Income} and \emph{Frequency}.
Note that the table is now longer, but it is also narrower.  This is a common pattern when
fixing this kind of issue.  The table is now tidy because the column headers are variable
names, not values.

\begin{tablebox}[label=tab:tidy1]{Tidy version of \cref{tab:messy1} where values are correctly moved.}
  \centering
  \rowcolors{2}{black!10!white}{}
  \begin{tabular}{l l r}
    \toprule
    Religion & Income & Frequency \\
    \midrule
    Agnostic & <\$10k & 27 \\
    Agnostic & \$10-20k & 34 \\
    Agnostic & \$20-30k & 60 \\
    \dots & \dots & \dots \\
    Atheist & <\$10k & 12 \\
    Atheist & \$10-20k & 27 \\
    Atheist & \$20-30k & 37 \\
    \dots & \dots & \dots \\
    \bottomrule
  \end{tabular}
\end{tablebox}

\paragraph{Multiple variables are stored in one column}  For example, consider the
\cref{tab:messy2}.  This table is not tidy because the column --- interestly called
\emph{column} ---, contains multiple variables.  This format is frequent, and sometimes the
column name contains the names of the variables.  Sometimes it is very hard to separate
the variables.

\begin{tablebox}[label=tab:messy2]{Messy table, from TB dataset, where multiple variables are stored in one column.}
  \centering
  \rowcolors{2}{black!10!white}{}
  \begin{tabular}{l l l r c}
    \toprule
    country & year & column & cases & \dots \\
    \midrule
    AD & 2000 & m014 & 0 & \dots \\
    AD & 2000 & m1524 & 0 & \dots \\
    AD & 2000 & m2534 & 1 & \dots \\
    AD & 2000 & m3544 & 0 & \dots \\
    \dots & \dots & \dots & \dots \\
    \bottomrule
  \end{tabular}
\end{tablebox}

To make it tidy, we can transform it into the \cref{tab:tidy2}.  Two columns are created
to contain the variables \emph{Sex} and \emph{Age}, and the old column is removed.  The
table keeps the same number of rows, but it is now wider.  This is a common pattern when
fixing this kind of issue.  The new version usually fixes the issue of correctly
calculating ratios and frequency.

\begin{tablebox}[label=tab:tidy2]{Tidy version of \cref{tab:messy2} where values are correctly moved.}
  \centering
  \rowcolors{2}{black!10!white}{}
  \begin{tabular}{l l l l r c}
    \toprule
    country & year & sex & age & cases & \dots \\
    \midrule
    AD & 2000 & m & 0--14 & 0 & \dots \\
    AD & 2000 & m & 15--24 & 0 & \dots \\
    AD & 2000 & m & 25--34 & 1 & \dots \\
    AD & 2000 & m & 35--44 & 0 & \dots \\
    \dots & \dots & \dots & \dots & \dots \\
    \bottomrule
  \end{tabular}
\end{tablebox}


\paragraph{Variables are stored in both rows and columns}  For example, consider the
\cref{tab:messy3}.  This is the most complicated case of messy data.  Usually, one of the
columns contains the names of the variables, in this case the column \emph{element}.

\begin{tablebox}[label=tab:messy3]{Messy table, adapted from airquality dataset, where variables are stored in both rows and columns.}
  \centering
  \rowcolors{2}{black!10!white}{}
  \begin{tabular}{llllcccc}
    \toprule
    id & year & month & element & d1 & d2 & \dots & d31 \\
    \midrule
    MX17004 & 2010 & 1 & tmax &    & 24 & \dots & 27 \\
    MX17004 & 2010 & 1 & tmin & 14 &    & \dots &    \\
    MX17004 & 2010 & 2 & tmax & 27 & 24 & \dots & 27 \\
    MX17004 & 2010 & 2 & tmin & 14 &    & \dots & 13 \\
    \dots & \dots & \dots & \dots & \dots & \dots & \dots & \dots \\
    \bottomrule
  \end{tabular}
\end{tablebox}

To fix this issue, we must first decide which column contains the names of the variables.
Then, we must lengthen the table in function of the variables (and potentially their
names), as seen in \cref{tab:tidy3a}.  Aftwards, we widen the table in function of their names.  Finally, we remove
implicit information, as seen in \cref{tab:tidy3b}.

\begin{tablebox}[label=tab:tidy3a]{Partial solution to tidy \cref{tab:messy3}. Note that
  the table is now longer.}
  \centering
  \rowcolors{2}{black!10!white}{}
  \begin{tabular}{lllc}
    \toprule
    id & date & element & value \\
    \midrule
    MX17004 & 2010-01-01 & tmax &    \\
    MX17004 & 2010-01-01 & tmin & 14 \\
    MX17004 & 2010-01-02 & tmax & 24 \\
    MX17004 & 2010-01-02 & tmin &    \\
    \dots & \dots & \dots & \dots \\
    \bottomrule
  \end{tabular}
\end{tablebox}

\begin{tablebox}[label=tab:tidy3b]{Tidy version of \cref{tab:messy3} where values are correctly moved.}
  \centering
  \rowcolors{2}{black!10!white}{}
  \begin{tabular}{llcc}
    \toprule
    id & date & tmin & tmax \\
    \midrule
    MX17004 & 2010-01-01 & 14 &    \\
    MX17004 & 2010-01-02 &    & 24 \\
    \dots & \dots & \dots & \dots \\
    \bottomrule
  \end{tabular}
\end{tablebox}

% É comum que, durante a coleta de dados, várias unidades observacionais sejam registradas
% numa mesma tabela Para arrumar estes dados, cada uma delas deve ser transportada para uma
% tabela diferente A separação evita diversos tipos de potenciais inconsistências (processo
% associado à normalização de bases de dados) No entanto, durante a análise é possível que
% tenhamos que desnormalizá-las

\paragraph{Multiple types of observational units are stored in the same table}  For
example, consider the \cref{tab:messy4}.  It is very common during data collection that
many observational units are registered in the same table.

\begin{tablebox}[label=tab:messy4]{Messy table, adapted from billboard dataset, where multiple types of observational units are stored in the same table.}
  \centering
  \rowcolors{2}{black!10!white}{}
  \begin{tabular}{lllll}
    \toprule
    year & artist & track & date & rank \\
    \midrule
    2000 & 2 Pac & Baby Don't Cry & 2000-02-26 & 87 \\
    2000 & 2 Pac & Baby Don't Cry & 2000-03-04 & 82 \\
    2000 & 2 Pac & Baby Don't Cry & 2000-03-11 & 72 \\
    2000 & 2 Pac & Baby Don't Cry & 2000-03-18 & 77 \\
    \dots & \dots & \dots & \dots & \dots \\
    2000 & 2Ge+her & The Hardest\dots & 2000-09-02 & 91 \\
    2000 & 2Ge+her & The Hardest\dots & 2000-09-09 & 87 \\
    2000 & 2Ge+her & The Hardest\dots & 2000-09-16 & 92 \\
    \dots & \dots & \dots & \dots & \dots \\
    \bottomrule
  \end{tabular}
\end{tablebox}

To fix this issue, we must each observation unit must be moved to a different table.
Sometimes, it is useful to create unique identifiers for each observation.
The separation avoids several types of potential inconsistencies.  However, take into
account that during data analysis, it is possible that we have to denormalize them.  The
two resulting tables are shown in \cref{tab:tidy4a} and \cref{tab:tidy4b}.

\begin{tablebox}[label=tab:tidy4a]{Tidy version of \cref{tab:messy4} containing the observational unit \emph{track}.}
  \centering
  \rowcolors{2}{black!10!white}{}
  \begin{tabular}{lll}
    \toprule
    track id & artist & track \\
    \midrule
    1 & 2 Pac & Baby Don't Cry \\
    2 & 2Ge+her & The Hardest Part Of Breaking Up \\
    \dots & \dots & \dots \\
    \bottomrule
  \end{tabular}
\end{tablebox}

\begin{tablebox}[label=tab:tidy4b]{Tidy version of \cref{tab:messy4} containing the observational unit \emph{rank of the track in certain week}.}
  \centering
  \rowcolors{2}{black!10!white}{}
  \begin{tabular}{lll}
    \toprule
    track id & date & rank \\
    \midrule
    1 & 2000-02-26 & 87 \\
    1 & 2000-03-04 & 82 \\
    1 & 2000-03-11 & 72 \\
    1 & 2000-03-18 & 77 \\
    \dots & \dots & \dots \\
    2 & 2000-09-02 & 91 \\
    2 & 2000-09-09 & 87 \\
    2 & 2000-09-16 & 92 \\
    \dots & \dots & \dots \\
    \bottomrule
  \end{tabular}
\end{tablebox}

\paragraph{A single observational unit is stored in multiple tables}  For example, consider
\cref{tab:messy5a,tab:messy5b}.  It is very common during data
collection that a single observational unit is stored in multiple tables.  Usually, the
table (or file) itself represents the value of a variable.  When columns are compatible,
it is straightforward to combine the tables.

\begin{tablebox}[label=tab:messy5a]{Messy tables, adapted from nycflights13 dataset, where
  a single observational unit is stored in multiple tables.  Assume that the origin
  filename is called \texttt{2013.csv}.}
  \centering
  \rowcolors{2}{black!10!white}{}
  \begin{tabular}{llll}
    \toprule
    month & day & time & \dots \\
    \midrule
    1 & 1 & 517 & \dots \\
    1 & 1 & 533 & \dots \\
    1 & 1 & 542 & \dots \\
    1 & 1 & 544 & \dots \\
    \dots & \dots & \dots & \dots \\
    \bottomrule
  \end{tabular}
\end{tablebox}

\begin{tablebox}[label=tab:messy5b]{Messy tables, adapted from nycflights13 dataset, where
  a single observational unit is stored in multiple tables.  Assume that the origin
  filename is called \texttt{2014.csv}.}
  \centering
  \rowcolors{2}{black!10!white}{}
  \begin{tabular}{llll}
    \toprule
    month & day & time & \dots \\
    \midrule
    1 & 1 & 830 & \dots \\
    1 & 1 & 850 & \dots \\
    1 & 1 & 923 & \dots \\
    1 & 1 & 1004 & \dots \\
    \dots & \dots & \dots & \dots \\
    \bottomrule
  \end{tabular}
\end{tablebox}

To fix this issue, we must first make the columns compatible.  Then, we can combine the
tables adding a new column that identifies the origin of the data.  The resulting table is
shown in \cref{tab:tidy5}.

\begin{tablebox}[label=tab:tidy5]{Tidy data where \cref{tab:messy5a,tab:messy5b} are combined.}
  \centering
  \rowcolors{2}{black!10!white}{}
  \begin{tabular}{lllll}
    \toprule
    year & month & day & time & \dots \\
    \midrule
    2013 & 1 & 1 & 517 & \dots \\
    2013 & 1 & 1 & 533 & \dots \\
    2013 & 1 & 1 & 542 & \dots \\
    2013 & 1 & 1 & 544 & \dots \\
    \dots & \dots & \dots & \dots & \dots \\
    2014 & 1 & 1 & 830 & \dots \\
    2014 & 1 & 1 & 850 & \dots \\
    2014 & 1 & 1 & 923 & \dots \\
    2014 & 1 & 1 & 1004 & \dots \\
    \dots & \dots & \dots & \dots & \dots \\
    \bottomrule
  \end{tabular}
\end{tablebox}

\subsection{Bridging normalization, tidyness, and data theory}
\label{sub:bridge}

First and foremost, both concepts, normalization and tidy data, are not in conflict.

In data normalization, given a set of functional, multivalued and join dependencies, there
exists a normal form that is free of redundancy.  In tidy data,
\citeauthor{Wickham2023} also state that there is only one way to organize the given data.

\textcite{Wickham2014} states that tidy data is 3NF.  However, he does not provide a
formal proof.  Since tidy data focuses on data analysis and not on data storage, I argue
that there is more than one way to organize the data in a tidy format.  It actually
depends on what you define as the observational unit.

Consider the following example.  We want to study the \emph{phenomenon} temperature in a
certain city.  We fix three sensors in different locations to measure the temperature.  We
collect data three times a day.  If we consider as the observational unit the
event of measuring the temperature, we can organize the data in a tidy format as shown in
\cref{tab:temp1}.

\begin{tablebox}[label=tab:temp1]{Tidy data where the observational unit is the event of measuring the temperature.}
  \centering
  \rowcolors{2}{black!10!white}{}
  \begin{tabular}{lllc}
    \toprule
    date & time & sensor & temperature \\
    \midrule
    2023-01-01 & 00:00 & 1 & 20 \\
    2023-01-01 & 00:00 & 2 & 21 \\
    2023-01-01 & 00:00 & 3 & 22 \\
    2023-01-01 & 08:00 & 1 & 21 \\
    2023-01-01 & 08:00 & 2 & 22 \\
    2023-01-01 & 08:00 & 3 & 23 \\
    \dots & \dots & \dots & \dots \\
    \bottomrule
  \end{tabular}
\end{tablebox}

However, since the sensors are fixed, we can consider the observational unit as the
\emph{temperature at some time}.  In this case, we can organize the data in a tidy format
as shown in \cref{tab:temp2}.

\begin{tablebox}[label=tab:temp2]{Tidy data where the observational unit is the temperature at some time.}
  \centering
  \rowcolors{2}{black!10!white}{}
  \begin{tabular}{llccc}
    \toprule
    date & time & temp. 1 & temp. 2 & temp. 3 \\
    \midrule
    2023-01-01 & 00:00 & 20 & 21 & 22 \\
    2023-01-01 & 08:00 & 21 & 22 & 23 \\
    \dots & \dots & \dots & \dots & \dots \\
    \bottomrule
  \end{tabular}
\end{tablebox}

In both cases, one can argue that the data is also normalized.  In the first case, the
primary key is the composite of the columns \emph{date}, \emph{time}, and \emph{sensor}.
In the second case, the primary key is the composite of the columns \emph{date} and
\emph{time}.

One can state that the first form is more appropriate, since it is flexible to add more
sensors.  However, the second form is very natural for machine learning and statistical
methods.  Given the definition of tidy data, I believe both forms are correct.

Another very interesting conjecture is whether we can formalize the eventual \emph{change
of observational unit} in terms of the order that joins and grouping operations are
performed.

\paragraph{Example}
Consider the following example: the relation $R[A, B, C, D, E]$ and the functional
dependencies $A \to D$, $B \to E$, and $AB \to C$.  The relation can be normalized up to
3NF by following one of the decomposition trees shown in \cref{fig:decomp}.
Every decomposition tree must take into account that the join of the projections are
lossless and dependency preserving.

\begin{figurebox}[label=fig:decomp]{Decomposition trees for the relation $R[ABCDE]$ and
  the functional dependencies $A \to D$, $B \to E$, and $AB \to C$ to reach 3NF.}
  \centering
  \begin{tikzpicture}
    \node (root1) at (0, 0) {ABCDE}
      child {node {AD}}
      child {node {ABCE}
        child {node {BE}}
        child {node {ABC}}};
    \node (root2) at (3, 0) {ABCDE}
      child {node {BE}}
      child {node {ABCD}
        child {node {AD}}
        child {node {ABC}}};
  \end{tikzpicture}
\end{figurebox}

Note that the decomposition that splits first $R[ABC]$ is not valid, since the resulting
relation $R[AB]$ is not a consequences of a functional dependency, see
\cref{fig:wrongdecomp}.

\begin{figurebox}[label=fig:wrongdecomp]{Invalid decomposition trees for the relation $R[ABCDE]$.}
  \centering
  \begin{tikzpicture}
    \node (root1) at (0, 0) {ABCDE}
      child {node {ABC}}
      child {node {ABDE}
        child {node {AD}}
        child {node {ABE}
          child {node {BE}}
          child {node[gray] {AB}}}};
    \node (root2) at (3, 0) {ABCDE}
      child {node {ABC}}
      child {node {ABDE}
        child {node {BE}}
        child {node {ABD}
          child {node {AD}}
          child {node[gray] {AB}}}};
  \end{tikzpicture}
  \tcblower
  We consider the functional dependencies $A \to D$, $B \to E$, and $AB \to C$.
  Note that $R[AB]$ is not a consequence of a functional dependency.
\end{figurebox}

In this kind of relation schema, we have a set of key attributes, here $\mathcal{K} = AB$,
and a set of non-prime attributes, here $\mathcal{N} = CDE$.  Note that the case
$\mathcal{K} \cap \mathcal{N} = \emptyset$ is the simplest we can have.

Observe, however, that transitive dependencies\footnote{Actually, when an attribute is
both key and non-prime, some joins may generate invalid tables.} and complex join
dependencies restrict even further the joins we are allowed to perform.
\textcolor{red}{Further formalization and study is under progress.}

Now, consider a very common case: in our dataset, keys are unknown.  Let $A$ be a student
id, $B$ be the course id, $D$ be the student age, $E$ be the course load, and $C$ be the
student grade at the course.  If only $CDE$ is known, the table $R[CDE]$ is already tidy
--- and the observational unit is the enrollment --- once there is no key to perform any
kind of normalization.  This happens in many cases where privacy is a concern.

But we can also consider that the observational unit is the student.  In this case, we
must perform joins traversing the leftmost decomposition tree in \cref{fig:decomp} from
bottom to top.  After each join, a summarization operation is performed on the relation
considering the student as the observational unit, i.e. over attribute $A$.  The first
join results in relation $R[ABCE]$ and the summarization operation results in a new
relation $R[AFG]$ where $F$ is the average grade and $G$ is the total course load taken by
the student.  They all calculated based on the rows that are grouped in function of $A$.
It is important to notice that, after the summarization operation, all observations must
contain a different value of $A$.  The second join results in relation $R[ADFG] = R[AD]
\bowtie R[AFG]$.  This relation has functional dependency $A \to DFG$, and it is in 3NF
(which is also tidy).

Unfortunately, it is not trivial to calculate all possible decomposition trees for a given
dataset.  \textcolor{red}{Further formalization and study is under progress.}

\subsection{Data semantics and interpretation}

In the rest of the book, we focus on a statistical view of the data.  Besides the
functional dependencies, we also consider the statistical dependencies of data.  For
instance, attributes $A$ and $B$ might not be functionally dependent, but they might exits
unknown $P(A, B)$ that we can estimate from the data.  Each observed value of a key can
represent a instance of a random variable, and the other attributes can represent
measured attributes or calculated properties.

For data analysis, it is very important to understand the relationships between the
observations.  For example, we might want to know if the observations are independent, if
they are identically distributed, or if there is a known selection bias.  We might also
want to know if the observations are dependent on time, and if there are hidden variables
that affect the observations.

Following wrong assumptions can lead to wrong conclusions.  For example, if we assume that
the observations are independent, but they are not, we might underestimate the variance of
the estimators.

Although we not focus on time series, we must consider the temporal dependence of the
observations.  For example, we might want to know how the observation $x_t$ is affected by
$x_{t-1}$, $x_{t-2}$, and so on.  We might also want to know if Markov property holds,
and if there is periodicity and seasonality in the data.

For the sake of the scope of this book, we suggest that any prediction on temporal data
should be done in the state space, where it is safer to assume that observations are
independent and identically distributed.  This is a common practice in reinforcement
learning and deep learning. Takens' theorem\footfullcite{Takens1980} allows you to
reconstruct the state space of a dynamical system using time-delay embedding. Given a
single observed time series, you can create a multidimensional representation of the
underlying dynamical system by embedding the time series in a higher-dimensional space.
This embedding can reveal the underlying dynamics and structure of the system.

\section{Unstructured data}

Unstructured data are data that do not have a predefined data model or are not organized
in a predefined manner.  For example, text, images, and videos are unstructured data.

Every unstructured dataset can be converted into a structured dataset.  However, the
conversion process is not always straightforward nor lossless.  For example, we can
convert a text into a structured dataset by counting the number of occurrences of each
word.  However, we lose the order of the words in the text.

The study of unstructured data is, for the moment, out of the scope of this book.

\chapter{Data science project}
\label{chap:project}

% %TODO: do not forget to talk about doucmentation and deployment

\chapterprecishere{\raggedleft\textup{with contributions from} \textsc{Johnny C. Marques}
  \\[5mm] Figured I could throw myself a pity party or go back to school and learn the computers.
  \par\raggedleft--- \textup{Don Carlton}, Monsters University (2013)}

First of all, \emph{a data science project is a software project}.  The difference between a data
science software and a traditional software is that some components of the former is
constructed from data.  This means that part of the solution cannot be designed from the
knowledge of the domain expert, but must be extracted from the data.  Sometimes, a
traditional solution would be too complex to design, and it is more efficient to infer it
from the data.

One good example of project is a spam filter.  The spam filter is a software that
classifies emails into two categories: spam and non-spam.  A traditional approach would be
to design a set of filters that are known to be effective.  However, the effectiveness of
the filters is limited by the knowledge of the designer and is cumbersome to maintain.  A
data science approach is to design a software that learns the filters from the data,
rather than from the knowledge of the designer.  The software is trained using a set of
emails that are already classified as spam or non-spam.

In this chapter, we discuss common methodologies for data science projects.  We also
present the concept of agile methodologies and the SCRUM framework.  We finally propose an
extension to SCRUM adapted for data science projects.

% TODO:
% - the need of methodologies for data science projects
% - CRISP-DM
% - Nina's approach
% - Agile
% - Problems and advantages of SCRUM
% - Our approach

\begin{mainbox}{Chapter remarks}

  \boxsubtitle{Contents}

  \startcontents[chapters]
  \printcontents[chapters]{}{1}{}
  \vspace{1em}

  \boxsubtitle{Context}

  \begin{itemize}
    \itemsep0em
    \item Data science project is a software project.
    \item Data science methodologies focus on the data analysis process.
    \item Industry demands not only data analysis but also software development.
  \end{itemize}

  \boxsubtitle{Objectives}

  \begin{itemize}
    \itemsep0em
    \item Explore the common methodologies for data science projects.
    \item Understand the agile methodologies and the SCRUM framework.
    \item Propose an extension to SCRUM adapted for data science projects.
  \end{itemize}

  \boxsubtitle{Takeways}

  \begin{itemize}
    \itemsep0em
    \item \dots
  \end{itemize}
\end{mainbox}

{}
\clearpage

\section{CRISP-DM}

CRISP-DM\footnote{Official guide available at
\url{https://www.ibm.com/docs/it/SS3RA7_18.3.0/pdf/ModelerCRISPDM.pdf}.} is a methodology
for data mining projects.  It is an acronym for Cross Industry Standard Process for Data
Mining.  It is a methodology that was developed in the 1990s by IBM, and it is still
widely used today.

CRISP-DM is a cyclic process.  The process is composed of six phases:
\begin{enumerate}
  \itemsep0em
  \item Business understanding: this is the phase where the project objectives are
    defined.  The objectives must be defined in a way that is measurable.  The phase also
    includes the definition of the project plan.
  \item Data understanding: this is the phase where the data is collected and explored.
    The data is collected from the data sources, and it is explored to understand its
    characteristics.  The phase also includes the definition of the data quality
    requirements.
  \item Data preparation: this is the phase where the data is prepared for the modeling
    phase.  The data is cleaned, transformed, and aggregated.  The phase also includes the
    definition of the modeling requirements.
  \item Modeling: this is the phase where the model is trained and validated.  The model is
    trained using the prepared data, and it is validated using the validation data.  The
    phase also includes the definition of the evaluation requirements.
  \item Evaluation: this is the phase where the model is evaluated.  The model is evaluated
    using the evaluation data.  The phase also includes the definition of the deployment
    requirements.
  \item Deployment: this is the phase where the model is deployed.  The model is deployed
    using the deployment requirements.  The phase also includes the definition of the
    monitoring requirements.
\end{enumerate}

\Cref{fig:cripdm} shows a diagram of the CRISP-DM process.  Note that the process is
cyclic and completly focused on the data.  The process do not address the software
development aspects of the project.

\begin{figurebox}[label=fig:cripdm]{Diagram of the CRISP-DM process.}
  \centering
  \begin{tikzpicture}
    \node (1) at (0, 0) {};
    \node (2) at (4, -4) {};
    \node (3) at (0, -8) {};
    \node (4) at (-4, -4) {};

    \node [block] (bu) at (-1.5, -2) {Business understanding};
    \node [block] (du) at (1.5, -2) {Data understanding};
    \path [line] (bu) -- (du);
    \path [line] (du) -- (bu);

    \node [block] (dp) at (2.5, -3.5) {Data preparation};
    \node [block] (m) at (2.5, -5) {Modeling};
    \path [line] (dp) -- (m);
    \path [line] (m) -- (dp);

    \path [line] (du) -- (dp);

    \node [block] (e) at (0, -6.5) {Evaluation};
    \node [block] (d) at (-2.5, -4.25) {Deployment};

    \path [line] (m) -- (e);
    \path [line] (e) -- (d);
    \path [line] (e) -- (bu);

    \node [draw, circle, fill=gray, text centered, text=white] at (0, -4.25) {Data};

    \path [bigarrow] (1.east) to[out=0, in=90] (2.60);
    \path [bigarrow] (2.-60) to[out=-90, in=0] (3.east);
    \path [bigarrow] (3.west) to[out=180, in=-90] (4.-120);
    \path [bigarrow] (4.120) to[out=90, in=180] (1.west);
  \end{tikzpicture}
  \tcblower
  Each block represents a phase of the CRISP-DM process.  Data is the central element of
  the process.  Arrows represent the transitions between the phases.
\end{figurebox}

The CRISP-DM methodology is a good starting point for data science projects.  However, it
does not mean that should be followed strictly.  The process is cyclic and flexible, and
adaptations are possible at any stage of the process.

\section{ZN approach}

\textcite{Zumel2019} also propose a methodology for data science projects --- which we
call the ZN approach.  Besides
describing each step in a data science project, they further address the roles of each
individual involved in the project.  They state that data science projects are always
collaborative, as they require domain expertise, data expertise, and software expertise.
The requirements are dynamic, and the project has many exploratory phases.  Usually,
projects based on data are urgent, and they must be completed in a short time --- not
only due to the business requirements, but also because the data changes over time.
The authors state that agile methodologies are suitable (and necessary) for data science
projects.

\subsection{Roles of the ZN approach}

In their approach, five roles are defined.

\paragraph{Project sponsor}  It is the main stakeholder of the project, the one that needs the
results of the project.  He represents the business interests and champions the project.
The project is considered successful if the sponsor is satisfied.  Note that, ideally, the
sponsor can not be the data scientist, but someone that is not involved in the development
of the project.  However, he needs to be able to express \emph{quantitatively} the business
goals and participate actively in the project.

\paragraph{Client}  The client is the domain expert.  He represents the end users'
interests.  In a small project, he is usually the sponsor.  He translates the daily
activities of the business into the technical requirements of the software.

\paragraph{Data scientist}  The data scientist is the one that sets and executes the
analytic strategy.  He is the one that communicates with the sponsor and the client,
effectively connecting all the roles.  In small projects, he can also act as the developer
of the software.  However, in large projects, he is usually the project manager.
Although it is not required to be a domain expert, the data scientist must be able to
understand the domain of the problem.  He must be able to understand the business goals and
the client's requirements.  Most importantly, he must be able to define and to solve the
right tasks.

\paragraph{Data architect}  The data architect is the one that manages data and data storage.
He usually is involved in more than one project, so it is not an active participant.  He
that receives instructions to adapt the data storage and means to collect data.

\paragraph{Operations}  The operations role is the one that manages infrastructure and
deploys final project results.  He is responsible to define requirements such as response
time, programming language, and the infrastructure to run the software.

\subsection{Processes of the ZN approach}

\citeauthor{Zumel2019}'s model is similar to CRISP-DM, but emphasizes that back-and-forth
is possible at any stage of the process.  \Cref{fig:zumel} shows a diagram of the process.
The phases are:
\begin{itemize}
  \itemsep0em
  \item Define the goal: what problem are we trying to solve?
  \item Collect and manage data: what information do we need?
  \item Build the model: find patterns in the data that may solve the problem.
  \item Evaluate the model: is the model good enough to solve the problem?
  \item Present results and document: establish that we can solve the problem and how we
    did it. (This step is a differentiator from CRISP-DM.  In ZN approach, result
    presentation is essential; data scientists must be able to communicate their results
    effectively to the client/sponsor.)
  \item Deploy the model: make the model available to the end users.
\end{itemize}

\begin{figurebox}[label=fig:zumel]{Diagram of the data science process proposed by \textcite{Zumel2019}.}
  \centering
  \begin{tikzpicture}
    \node (1) at (0, 0) {};
    \node (2) at (1, -1) {};
    \node (3) at (0, -2) {};
    \node (4) at (-1, -1) {};
    \path [bigarrow] (1.east) to[out=0, in=90] (2.60);
    \path [bigarrow] (2.-60) to[out=-90, in=0] (3.east);
    \path [bigarrow] (3.west) to[out=180, in=-90] (4.-120);
    \path [bigarrow] (4.120) to[out=90, in=180] (1.west);

    \node [block] (dg) at (0, 1) {Define the goal};
    \node [block] (cm) at (3, -0.25) {Collect and manage data};
    \node [block] (bm) at (3, -1.75) {Build the model};
    \node [block] (em) at (0, -3) {Evaluate the model};
    \node [block] (pd) at (-3, -1.75) {Present results};
    \node [block] (dm) at (-3, -0.25) {Deploy the model};

    \path [dline] (dg) to[out=0, in=90] (cm.north);
    \path [dline] (cm) -- (bm);
    \path [dline] (bm.south) to[out=-90, in=0] (em.east);
    \path [dline] (em.west) to[out=180, in=-90] (pd.south);
    \path [dline] (pd) -- (dm);
    \path [dline] (dm.north) to[in=180, out=90] (dg.west);
  \end{tikzpicture}

  \tcblower
  Each block represents a phase of the data science process.  The emphasis is on the
  cyclic nature of the process.  Arrows represent the transitions between the phases, that
  can be back-and-forth.
\end{figurebox}

\section{Agile methodology}

Agile is a methodology for software development.  It is an alternative to the waterfall
methodology.  The waterfall methodology is a sequential design where each phase
must be completed before the next phase can begin.

The four values of agile manifesto are:
\begin{itemize}
  \itemsep0em
  \item Individuals and interactions over processes and tools;
  \item Working software over comprehensive documentation;
  \item Customer collaboration over contract negotiation;
  \item Responding to change over following a plan.
\end{itemize}

\section{SCRUM framework}

SCRUM is an agile framework for software development.  It is a process framework for
managing complex projects.  It is a lightweight, which means that it
provides just enough guidance to be effective.

Many consider that SCRUM is not adequate for data science projects.  The main reason is
that SCRUM is designed for projects where the requirements are known in advance.  Also,
that data science projects have exploratory phases, which are not well supported by SCRUM.

I argue that this view is wrong.  SCRUM is a framework, and it is designed to be adapted to
the needs of the project.  SCRUM is not a rigid process.  In the following, I propose an
extension to SCRUM that makes it suitable for data science projects.

(In real-world, most developers do not have hacking-level skills.  They are not autonomous
enough to work without a plan.  This is especially true for ``data scientists,'' who are
often not even developers.  SCRUM is a good compromise between the need for autonomy and
the need for a detailed plan.  Project methodology is needed to ensure that the project is
completed in time and within budget.)

\section{Our approach}

The previously mentioned methodologies lack the focus on the software development aspects of
the data science project.  For instance, CRISP-DM defines the stages only of the data
mining process, i.e. it does not explicitly address user interface or data collection.
\citeauthor{Zumel2019}'s approach addresses data collection and presentation of results, but
delegates the software development to the operations role, barely mentioning it.  SCRUM is
a good framework for software development, but it is not designed for data science
projects.  It lacks the exploratory phases of data science projects.

Thus, we propose an extension to SCRUM that makes it suitable for data science projects.
The extension is based on the following observations:
\begin{itemize}
  \itemsep0em
  \item Data science projects have exploratory phases;
  \item Data itself is a component of the solution;
  \item The solution is usually modularized, parts of it are constructed from data while the
    other parts are constructed like traditional software;
  \item The solution is usually deployed as a service, that must be maintained and
    monitored.
\end{itemize}

Moreover, we add two other values besides the agile manifesto values.  They are:
\begin{itemize}
  \itemsep0em
  \item Model confidence/understanding over model performance;
  \item Code version control over interactive environments.
\end{itemize}

The first value is based on the observation that the model performance is not the most
important aspect of the model.  The most important aspect is the being sure that the model
behaves as expected (and sometimes why it behaves as expected).  It is not uncommon to find
models that seems to perform well during evaluation steps\footnote{Of course, when
evaluation is not performed correctly.}, but that are not suitable for production.

The second value is based on the observation that interactive environments are not suitable
for the development of the model search code, for instance.  Interactive environments
auxiliate the exploratory phases, but the final version of the code must be version
controlled.  Often, we hear stories that models cannot be reproduced because the code that
generated them are not runnable anymore.  This is a serious problem, and it is not
acceptable for maintaining a software solution.

These observations and values are the basis of our approach.  The roles and principles of
our approach are described in the following sections.

\subsection{The roles of our approach}

\textcolor{red}{Combine SCRUM roles with the roles defined by \textcite{Zumel2019}.}

\begin{tablebox}[label=tab:roles]{Roles of our approach.}
  \centering
  \rowcolors{2}{black!10!white}{}
  \begin{tabular}{lll}
    \toprule
    \textbf{Our approach} & \textbf{SCRUM} & \textbf{ZM} \\
    \midrule
    Sponsor & Product owner & Project sponsor \\
    Client & Stakeholder & Client \\
    Data scientist & Scrum master & Data scientist \\
    Dev Team & & Data architect/operations \\
    \bottomrule
  \end{tabular}
  \tcblower
  The roles of SCRUM are associated with the roles defined by \textcite{Zumel2019}.
  In our approach, the data scientist leads the development team and interacts with the sponsor
  and the client.  The development team includes people with both database and software
  engineering expertise.
\end{tablebox}

\subsection{The principles of our approach}

\begin{enumerate}
  \item Modularize the solution. Usually, in four main modules: frontend, backend,
    dataset, and model search.  The frontend is the user interface.  The backend is the
    server-side code.  The dataset is the data that is used to train the model.  The model
    search is the code that searches for the best model.
  \item Version control everything.  This includes the code, the data, and the
    documentation. The most used tool for code version control is Git.  For datasets,
    extensions to Git exist, such as DVC\footnote{\url{https://dvc.org/}}.  One important aspect
    is to version control the model search code.  Interactive environments such as Jupyter
    notebooks are not suitable for this purpose.  They can be used to draft the code, but
    the final version must be version controlled.
  \item Continuous integration and continuous deployment.  This means that the code is
    automatically (or at least semi-automatically) tested and deployed.  The backend and
    frontend code is tested using unit tests.  The model search code is tested using
    validation methods such as cross-validation and Bayesian analysis on the discovered
    models.  Usually the model search code is very computationally intensive, and it is
    not feasible to run it on every commit.  Instead, it is run periodically, for example
    once a day.  If the clould infrastructure required to run the model search code is not
    available to automate validation and deploymen, at least make sure that the code is
    easily runnable.  This means that the code must be well documented, and that the
    required infrastructure must be well documented.  Also aggregate commands using a
    Makefile or a similar tool.  Pay attention on the dependences between dataset and the
    model training.  If the dataset changes significantly, not only the deployed model
    must be retrained, but the model search algorithm may need to be rethought.
  \item Reports as deliverables.  During sprints, the deliverables of data exploration are
    reports.  The reports must be version controlled and must be reproducible.  The reports
    must be generated in a way that is understandable by the client and the sponsor.
  \item Setup quantitative goals.  Do not fall on the trap of forever improving the model.
    Instead, setup quantitative goals for the model performance.  For example, the model
    must have a precision of at least 90\%.  Once you reach the goal, prioritize other
    tasks.
  \item Measure \emph{exactly} what you want.  During model validation, use your own
    metrics based on the project goals.  Usually, more than one metric is needed, and they
    might be conflicting.  Use strategies to balance the metrics, such as Pareto
    optimization.  Beware of the metrics that are most used in the literature.  They might not
    be suitable for your project.  Notice that during model training, some methods are
    limited to the loss functions that they can optimize.  If possible, choose a method
    that can optimize the loss function that you want.  Even if you are not explicitly
    optimizing the wanted metric, you might find a model that performs well on that metric.
    That is a reason validation is important.
  \item Report model stability and performance variance.  Understanding the limitations
    and characteristics of the model is more important than the model performance.  For
    example, if the model performance is high, but the model is unstable, it is not
    suitable for production.  Also, in some scenarios, interpretability is more important than
    performance.
  \item In user interface, mask data-science-specific terminology.  Usually, data science
    software gives the user the option to choose the model.  In order to avoid confusion,
    the user interface must mask the data-science-specific terminology.  This helps non
    experts to use the software consciously.
  \item Monitor model performance in production.  If possible setup feedback from the user
    interface.  Avoid automation of model releases because concept drift usually requires
    exploratory analysis.
  \item Use the appropriate backend.  REST API vs websocket.  The choice depends on the
    requirements of the project.  REST API is more suitable for stateless requests, while
    websocket is more suitable for stateful requests.  For example, if the user interface
    must be updated in real-time, websocket is more suitable.  If the user interface is
    used to submit batch requests, REST API is more suitable.
\end{enumerate}

\subsection{Solution search framework}

TODO Move part of the \cref{chap:planning} here, dropping the sampling
strategy but bringing the main defitions.

\chapter{Structured data}
\label{chap:data}

As one expects, when we measure a phenomenon, the resulting data come in many different
formats.  For example, we can measure the temperature of a room using a thermometer.  The
resulting data are numbers.  We can assess English proficiency using an essay test.  The
resulting data are texts.  We can register relationships between proteins and their
functions.  The resulting data are graphs.  Thus, it is essential to understand the nature
of the data we are working with.

The most common data format is the \emph{structured data}.  Structured data refers to
information that is organized in a tabular format.
We restrict the kind of information we store in each cell, i.e., the data type of each
measurement.  The data type restricts the operations we can
perform on the data.  For example, we can perform arithmetic operations on numbers, but
not on text.  All cells in the same column must share the same data type.

In this chapter, I discuss the most common data types and the most common data formats.
More specifically, we are interested in how the semantics of the data are encoded in the
data format.  Database normalization and tidy data are two concepts that are crucial for
the understanding of structured data.

As a result, the reader will be equipped with the mindset to perform data tasks
--- collection, integration, tidying, and exploration --- well.

\begin{mainbox}{Chapter remarks}

  \boxsubtitle{Contents}

  \startcontents[chapters]
  \printcontents[chapters]{}{1}{}
  \vspace{1em}

  \boxsubtitle{Context}

  \begin{itemize}
    \itemsep0em
    \item Data comes in many different formats.
    \item Good data analysis requires understanding the data types and their meanings.
  \end{itemize}

  \boxsubtitle{Objectives}

  \begin{itemize}
    \itemsep0em
    \item Understand the most common data types and formats.
    \item Enable the reader to perform data tasks well by associating data format and semantics.
  \end{itemize}

  \boxsubtitle{Takeaways}

  \begin{itemize}
    \itemsep0em
    \item The choice of the observational unit is not always straightforward.
    \item Format and types must reflect the information the solution will ``see'' in
      production.
  \end{itemize}
\end{mainbox}

{}
\clearpage

\section{Data types}

The most common classification of data types is Stevens' types: nominal, ordinal,
interval, and ratio.  Nominal data are data that can be classified into categories.
Ordinal data are data that can be classified into categories and ordered.  Interval data
are data that can be classified into categories, ordered, and measured in fixed units.
Ratio data are data that can be classified into categories, ordered, measured in fixed
units, and have a true zero.  In practice, they differ on the logical and arithmetic operations
we can perform on them.

\begin{tablebox}[label=tab:stevens]{Stevens' types.}
  \centering
  \rowcolors{2}{black!10!white}{}
  \begin{tabular}{cc}
    \toprule
    \textbf{Data type} & \textbf{Operations} \\
    \midrule
    Nominal & $=$ \\
    Ordinal & $=, <$ \\
    Interval & $=, <, +, - $ \\
    Ratio & $=, <, +, -, \times, \div$ \\
    \bottomrule
  \end{tabular}
  \tcblower
  Stevens' types are a classification of data types based on the operations we can perform
  on them.
\end{tablebox}

\Cref{tab:stevens} summarizes the allowed operations for each of Stevens' types.  All types
enable equality comparison.  Ordinal data can also be tested in terms of their order, but
they do not allow quantitative difference.  Interval data, on the other hand, allow
addition and subtraction.  Finally, the true zero of ratio data enables us to calculate
relative differences (multiplication and division).

For example, colors are nominal data.  We can classify colors into categories, but we cannot
order them.  A categorical variable that classifies sizes into small, medium, and large is
ordinal data.  We can order the sizes, but we cannot say that the difference between small
and medium is the same as the difference between medium and large.  Temperature in Celsius
is interval data.  We can order temperatures, and we can say that the difference between
10 and 20 degrees is the same as the difference between 20 and 30 degrees.  However, we
cannot say that 20 degrees is twice as hot as 10 degrees.  Finally, weight is ratio data.
We can order weights, we can say that the difference between 10 and 20 kilograms is the
same as the difference between 20 and 30 kilograms, and we can say that 20 kilograms is
twice as heavy as 10 kilograms.

Nonetheless, Stevens' types do not exhaust all possibilities for data types.  For example,
probabilities are bounded at both ends, and thus do not tolerate arbitrary scale shifts.
\textcite{Paul1993}\footfullcite{Paul1993} provide interesting insights about data types.  Although I do not
agree with all his points, I think it is a good read.  In particular, I agree with
his criticism of statements that data types are evident from the data independent of the
questions asked.  The same data can be interpreted in different ways depending on the
context and the goals of the analysis.

However, I do not agree with the idea that good data analysis does not assume data types.
I think that data scientists should be aware of the data types they are working with and
how they affect the analysis.  With no bias, there is no learning.  There is no such
thing as a ``bias-free'' analysis; the amount of possible combinations of assumptions
easily grows out of control.  The data scientist must take responsibility for the
consequences of their assumptions.  Good assumptions and hypotheses are a key part of the
data science methodology.

When we work with structured data, two concepts are very important: database normalization
and tidy data.  Database normalization is mainly focused on the data storage.  Tidy data is
mainly focused on the requirements of data for analysis.  Both concepts have their
mathematical and logical foundations and tools for data handling.

\section{Database normalization}
\label{sec:normalization}

Database normalization is the process of organizing the columns and tables of a relational
database to minimize data redundancy and improve data integrity.  The need for database
normalization comes from the fact that the same data can be stored in many different ways.

Normal form is a state of a database that is free of certain types of data redundancy.
Before studying normal forms, we need to understand basic concepts in database theory
and the basic operations in relational algebra.

\subsection{Relational algebra}

Relational algebra is a theory that uses algebraic structures to manipulate relations.
Consider the following concepts.

\paragraph{Relation}  A relation is a table with rows and columns that represent
an entity.  Each row, or tuple, is assumed to appear only once in the relation.  Each
column, or attribute, is assumed to have a unique name.

\paragraph{Projection}  The projection of a relation is the operation that returns a
relation with only the columns specified in the projection.  For example, if we have a
relation $X[A, B, C]$ and we perform the projection $\pi_{A, C}(X)$, we get a
relation with only the columns $A$ and $C$, i.e., $X[A, C]$.  The number of rows
in the resulting relation might be less than the number of rows in the original relation
because of repeated rows.

\paragraph{Join}  The (natural) join of two relations is the operation that returns a
relation with the columns of both relations.  For example, if we have two relations $S[U
\cup V]$ and $T[U \cup W]$, where $U$ is the common set of attributes, the join $S \bowtie T$
of $S$ and $T$ is the relation with tuples $(u, v, w)$ such that $(u, v) \in S$ and $(u,
w) \in T$.  The generalized join is built up out of binary joins:  $\bowtie \left\{ R_1,
R_2, \dots, R_n \right\} = R_1 \bowtie R_2 \bowtie \dots \bowtie R_n$. Since the join
operation is associative and commutative, we can parenthesize however we want.

\paragraph{Functional dependency}  A functional dependency is a constraint between two
sets of attributes in a relation.  It is a statement that if two tuples agree on certain
attributes, then they must agree on another attribute.  Specifically, the \emph{functional
dependency} $U \to V$ holds in $R$ if and only if for every pair of tuples $t_1$ and $t_2$
in $R$ such that $t_1[U] = t_2[U]$, it is also true that $t_1[V] = t_2[V]$.

\paragraph{Multi-valued dependency}  A multi-valued dependency constrains
two sets of attributes in a relation.  The
\emph{multi-valued dependency} $U \twoheadrightarrow V$ holds in $R$ if and only if $R =
R[UV] \bowtie R[UW]$, where $W$ are the remaining attributes.  Note, however, that unlike
functional dependencies, multi-valued dependencies are not simple to interpret, so we
restrict our discussion to its mathematical properties\footnote{In fact, one might
naively think that a multi-valued dependency is a functional dependency between
many attributes.  However, this is not the case.}.

\paragraph{Join dependency}  A join dependency is a constraint between subsets of
attributes (not necessarily disjoint) in a relation.  $R$ obeys the join dependency $*
\left\{ X_1, X_2, \dots, X_n \right\}$ if $R = {\bowtie \left\{ R[X_1], R[X_2], \dots,
R[X_n] \right\}}$.

\subsection{Normal forms}

The normal forms are a series of progressive conditions that a relation must satisfy to
be considered normalized.  The normal forms are cumulative, i.e., a relation that is in
$n$-th normal form is also in $(n-1)$-th normal form.  The normal forms are a way to
reduce redundancy and improve data integrity.

\paragraph{First normal form (1NF)}  A relation is in 1NF if and only if all attributes
are atomic.  An attribute is atomic if it is not a set of attributes.  For example, the
relation $R[A, B, C]$ is in 1NF if and only if $A$, $B$, and $C$ are atomic.

\paragraph{Second normal form (2NF)}  A relation is in 2NF if and only if it is in 1NF
and every non-prime attribute is fully functionally dependent on the primary key.  A
non-prime attribute is an attribute that is not part of the primary key.  A primary key
is a set of attributes that uniquely identifies a tuple.  A non-prime attribute is fully
functionally dependent on the primary key if it is functionally dependent on the primary
key and not on any subset of the primary key.  For example, the relation $R[U \cup V]$ is
in 2NF if and only if $U \to X,~\forall X \in V$ and there is no $W \subset U$ such that
$W \to X,~\forall X \in V$.

\paragraph{Third normal form (3NF)}  A relation is in 3NF if and only if it is in 2NF
and every non-prime attribute is non-transitively dependent on the primary key.  A
non-prime attribute is non-transitively dependent on the primary key if it is not
functionally dependent on another non-prime attribute.  For example, the relation $R[U
\cup V]$ is in 3NF if and only if $U$ is the primary key and there is no $X \in V$ such
that $X \to Y,~\forall Y \in V$.

\paragraph{Boyce-Codd normal form (BCNF)}  A relation $R$ with attributes $X$ is in BCNF
if and only if it is in 2NF and for each nontrivial functional dependency $U \to V$ in
$R$, the functional dependency $U \to X$ is in $R$.  In other words, a relation is in BCNF
if and only if every functional dependency is the result of keys.

\paragraph{Fourth normal form (4NF)}  A relation $R$ with attributes $X$ is in 4NF if
and only if it is in 2NF and for each nontrivial multi-valued dependency $U \twoheadrightarrow
V$ in $R$, the functional dependency $U \to X$ is in $R$.  In other words, a relation is
in 4NF if and only if every multi-valued dependency is the result of keys.

\paragraph{Projection join normal form (PJNF)} A relation $R$ with attributes $X$ is in
PJNF\footnote{Also known as fifth normal form (5NF).  The authors themselves prefer
the term PJNF because it emphasizes the operations to which the normal form applies.}
if and only if it is in 2NF and
the set of key dependencies\footnote{Key dependency is a functional dependency in the form
$K \to X$, where $X$ encompasses all attributes of the relation.} of $R$ implies each join
dependency of $R$.  The PJNF guarantees that the
table cannot be decomposed without losing information (except by decompositions based on
keys).

The idea behind the definition of BCNF and 4NF is slightly different from the
PJNF.  In fact, if we consider that for each key dependency implies a join dependency, the
relation is in the so-called overstrong projection-join normal
form\footfullcite{Fagin1979}.  Such a level of normalization does not improve data storage
or eliminate inconsistencies.  In practice, it means that if a relation is in PJNF,
careless joins --- i.e., those that violate a join dependency --- produce
inconsistent results.

\paragraph{Simple example}  Consider the 2NF relation $R[A, B, C, D]$ with functional
dependencies $A \to B,~B \to C,~C \to D$.  The relation is not in 3NF because $C$ is
transitively dependent on $A$.  To normalize it, we can decompose it into the
relations $R_1[A, B, C]$ and $R_2[C, D]$.  Now, $R_2$ is in 3NF and $R_1$ is in 2NF, but
not in 3NF.  We can decompose $R_1$ into the relations $R_3[A, B]$ and $R_4[B, C]$.
The original relation can be reconstructed by $\bowtie \left\{ R_2, R_3, R_4 \right\}$.

\paragraph{Illustrative example of data integrity}  Consider a relation of students and
their grades.  The relation contains the attributes ``student'', ``course'', ``course
credits'', and ``grade''.  The primary key is the composite of ``student'' and ``course''.
The functional dependencies are ``student'' and ``course'' determine ``grade'', and
``course'' determines ``course credits''.  The relation is in 2NF but not 3NF.

\begin{tablebox}[label=tab:student-grade-illustration]{Student grade relation.}
  \centering
  \rowcolors{2}{black!10!white}{}
  \begin{tabular}{cccc}
    \toprule
    \textbf{Student} & \textbf{Course} & \textbf{Course credits} & \textbf{Grade} \\
    \midrule
    Alice & Math & 4 & A \\
    Alice & Physics & 3 & B \\
    Bob & Math & 4 & B \\
    Bob & Physics & 3 & A \\
    \bottomrule
  \end{tabular}
  \tcblower
  An example of a relation of students and their grades in 2NF.
\end{tablebox}

\Cref{tab:student-grade-illustration} shows an example of possible values of the relation.
If we decide to change the course credits of the course ``Math'' to 5, we must update the
two rows; otherwise, the relation will be inconsistent.  A 3NF relation (see
\cref{tab:student-grade-illustration2}) would have the attributes ``course'' and ``course
credits'' in a separate relation, avoiding the possibility of data inconsistency.  If
needed, the relation would be reconstructed by a join operation.

\begin{tablebox}[label=tab:student-grade-illustration2]{Student grade relation in 3NF.}
  \centering
  \rowcolors{2}{black!10!white}{}
  \begin{tabular}{cc}
    \toprule
    \textbf{Course} & \textbf{Course credits} \\
    \midrule
    Math & 4 \\
    Physics & 3 \\
    \bottomrule
  \end{tabular}
  \quad
  \rowcolors{2}{black!10!white}{}
  \begin{tabular}{ccc}
    \toprule
    \textbf{Student} & \textbf{Course} & \textbf{Grade} \\
    \midrule
    Alice & Math & A \\
    Alice & Physics & B \\
    Bob & Math & B \\
    Bob & Physics & A \\
    \bottomrule
  \end{tabular}
  \tcblower
  An example of a relation of students and their grades in 3NF.
\end{tablebox}

\paragraph{Invalid join example} Consider the 2NF relation $R[ABC]$\footnote{Here we abbreviate ${A,
B, C}$ as $ABC$.} such that the primary key is the composite of $A$, $B$, and $C$.  The
relation is thus in the 4NF, as no column is a determinant of another column.  Suppose,
however, the following constraint: if $(a, b, c')$, $(a, b', c)$, and $(a', b, c)$ are in
$R$, then $(a, b, c)$ is also in $R$.  This can be illustrated if we consider $A$ as a
agent, $B$ as a product, and $C$ as a company.  If an agent $a$ represents companies $c$ and
$c'$, and product $b$ is in his portfolio, then assuming both companies make $b$, $a$
must offer $b$ from both companies.

The relation is not in PJNF, as the join dependency $* \left\{ AB, AC, BC \right\}$ is not
implied by the primary key.  (In fact, the only functional dependency is the trivial $ABC
\to ABC$.)  In this case, to avoid redundancies and inconsistencies, we must split the
relation into the relations $R_1[AB]$, $R_2[AC]$, and $R_3[BC]$.

It is interesting to notice that in this case, the relation $R_1 \bowtie R_2$ might
contain tuples that do not make sense in the context of the original relation.  For
example, if $R_1$ contains $(a, b)$ and $R_2$ contains $(a, c')$, the join contains
$(a, b, c')$, which might not be a valid tuple in the original relation if $(b, c')$ is
not in $R_3$.

\begin{hlbox}{Important note on PJNF}
  \em
  This is very important to notice, as it is a common mistake to assume
  that the join of the decomposed relations always contains valid tuples.
\end{hlbox}

\paragraph{Valid joins example}  Consider the 2NF relation $R[A, B, C, D, E]$ with the functional
dependencies $A \to D$, $AB \to C$, and $B \to E$.  To make it PJNF, we can decompose it
into the relations $R_1[A, D]$, $R_2[A, B, C]$, and $R_3[B, E]$.  The original relation can
be reconstructed by $\bowtie \left\{ R_1, R_2, R_3 \right\}$.  However, unlike the
previous example, the join of the decomposed relations always contains valid tuples
--- excluding degenerate joins, where there are no common attributes.
The reason is that all join dependencies implied by the key dependencies are trivial when
reduced\footnote{I am investigating a formal proof based on \fullcite{Vincent1997}.}.

\section{Tidy data}

It is estimated that 80\% of the time spent on data analysis is spent on data preparation.
Usually, the same process is repeated many times in different datasets. The idea is that
organized data carries the meaning of the data, reducing the time spent on handling
the data to get it into the right format for analysis.

Tidy data, proposed by \textcite{Wickham2014}\footfullcite{Wickham2014},
is a data format that provides a standardized way to organize data values within
a dataset.  The main advantage of tidy data is that it provides clear semantics with a focus
on only one view of the data.

Many data formats might be ideal for particular tasks, such as raw data, dense tensors, or
normalized databases.  However, most statistical and machine learning methods
require a particular data format.  Tidy data is a data format that is suitable for those
tasks.

% \begin{hlbox}{Wickham's thoughts on tidy data}
%   \em
%   Like families, tidy datasets are all alike but every messy dataset is messy in its own
%   way.
% \end{hlbox}

In an unrestricted table, the meaning of rows and columns is not fixed.  In a tidy table,
the meaning of rows and columns is fixed.  The semantics are more restrictive than usually
required for general tabular data.

\begin{tablebox}[label=tab:simple-messy]{Example of same data in different formats.}
  \centering
  \rowcolors{2}{black!10!white}{}
  \begin{tabular}{ccc}
    \toprule
     & \textbf{Cases (2019)} & \textbf{Cases (2020)} \\
    \midrule
    \textbf{Brazil} & 100 & 200 \\
    \textbf{USA} & & 400 \\
    \bottomrule
  \end{tabular}
  \\[1em]
  \begin{tabular}{ccc}
    \toprule
    & \textbf{Brazil} & \textbf{USA} \\
    \midrule
    \textbf{Cases (2019)} & 100 & \\
    \textbf{Cases (2020)} & 200 & 400 \\
    \bottomrule
  \end{tabular}
  \tcblower
  The same data in different formats.  Both are considered messy by Wickham.
\end{tablebox}

\Cref{tab:simple-messy} shows an example of the same data in different formats.  Although
they emphasize different aspects (especially for visualization) of the data, both contain
the same amount of information.  They are considered messy by Wickham because the meaning
of the rows and columns is not fixed.

It is based on the idea that a dataset is a collection of values, where:
\begin{itemize}
  \itemsep0em
  \item Each \emph{value} belongs to a variable and an observation.
  \item Each \emph{variable}, represented by a column, contains all values that measure
    the same attribute across (observational) units.
  \item Each \emph{observation}, represented by a row, contains all values measured on the
    same unit across attributes.
  \item \emph{Attributes} are the characteristics of the units, e.g., height, temperature,
    duration.
  \item \emph{Observational units} are the individual entities being measured, for
    instance, a person, a day, an experiment.
\end{itemize}
Table \ref{tab:tidy} summarizes the main concepts.

\begin{tablebox}[label=tab:tidy]{Tidy data concepts.}
  \centering
  \rowcolors{2}{black!10!white}{}
  \begin{tabular}{cccc}
    \toprule
    \textbf{Concept} & \textbf{Structure} & \textbf{Contains} & \textbf{Across} \\
    \midrule
    Variable & Column & Same attribute & Units \\
    Observation & Row & Same unit & Attributes \\
    \bottomrule
  \end{tabular}
\end{tablebox}

\begin{tablebox}[label=tab:simple-tidy]{Example of tidy data.}
  \centering
  \rowcolors{2}{black!10!white}{}
  \begin{tabular}{ccc}
    \toprule
    \textbf{Country} & \textbf{Year} & \textbf{Cases} \\
    \midrule
    Brazil & 2019 & 100 \\
    Brazil & 2020 & 200 \\
    USA & 2019 & \\
    USA & 2020 & 400 \\
    \bottomrule
  \end{tabular}
  \tcblower
  An example of tidy data from the data in \cref{tab:simple-messy}.
\end{tablebox}

If we follow this structure, the meaning of the data is implicit in the table.
\Cref{tab:simple-tidy} shows the same data in a tidy format.  The table is now longer, but
the variables and observations are clear from the table itself.

However, it is not always trivial to organize data in a tidy format.  Usually, we have
more than one level of observational units, each one represented by a table.  Moreover,
there might exist more than one way to define what the observational units in a dataset
are\footnote{Although Wickham himself implies that there is only one possible way to
define the observational units of the dataset.}.

To organize data in a tidy format, one can consider that:
\begin{itemize}
  \itemsep0em
  \item Attributes are functionally related among themselves --- e.g., Z is a linear
    combination of X and Y, or X and Y are correlated, or $P(X, Y)$ follows some joint distribution.
  \item Units can be grouped or compared --- e.g., person A is taller than person B, or
    the temperature in day 1 is higher than in day 2.
\end{itemize}

A particular point that tidy data do not address is that values in a column might not be
in the same scale or unit of measurement\footnote{Observational unit is not the same
concept as unit of measurement.}.  For example, a column might contain the
temperature in an experiment, and another column might contain the unit of measurement
that was used to measure the temperature.  This is a common problem in databases, and it
must be addressed for machine learning and statistical methods to work properly.

Note that the order of the rows and columns is not important.  However, it might be
convenient to sort data in a particular way to facilitate understanding.  For
instance, one usually expects that the first columns are \emph{fixed
variables}\footnote{Closely related (and potentially the same as) key in database
theory.} --- i.e., variables that are not the result of a measurement but that describe the
experimental design ---, and the last columns
are \emph{measured variables}.  Also, arranging rows by some variable might highlight some
pattern in the data.

Usually, columns are named --- the collection of all column names is called the
header, while rows are usually numerically indexed.

\subsection{Common messy datasets}
\label{sub:messy}

\textcite{Wickham2014}\footfullcite{Wickham2014} lists some common problems with messy
datasets and how to tidy them.  In this subsection, we focus on the problems and the
tidy solutions.  The data handling operations that enable us to tidy the data are
presented in \cref{chap:handling}.  Readers interested in a step-by-step guide for data
tidying are encouraged to read \textcite{Wickham2023}\footfullcite{Wickham2023}.

The problems are summarized in the following.

\clearpage
\subsubsection{Headers are values, not variable names}  For example, consider
\cref{tab:messy1}.  This table is not tidy because the column headers are values, not
variable names.  This format is frequently used in presentations since it is more compact.
It is also useful to perform matrix operations. However, it is not appropriate for general
analysis.

\begin{tablebox}[label=tab:messy1]{Messy table, from Pew Forum dataset, where headers are values, not variable names.}
  \centering
  \rowcolors{2}{black!10!white}{}
  \begin{tabular}{l r r r c}
    \toprule
    \textbf{Religion} & \textbf{<\$10k} & \textbf{\$10-20k} & \textbf{\$20-30k} & \textbf{\dots} \\
    \midrule
    Agnostic & 27 & 34 & 60 & \dots \\
    Atheist & 12 & 27 & 37 & \dots \\
    Buddhist & 27 & 21 & 30 & \dots \\
    \dots & \dots & \dots & \dots & \dots \\
    \bottomrule
  \end{tabular}
\end{tablebox}

To make it tidy, we can transform it into the \cref{tab:tidy1} by explicitly introducing
variables \emph{Income} and \emph{Frequency}.
Note that the table is now longer, but it is also narrower.  This is a common pattern when
fixing this kind of issue.  The table is now tidy because the column headers are variable
names, not values.

\begin{tablebox}[label=tab:tidy1]{Tidy version of \cref{tab:messy1} where values are correctly moved.}
  \centering
  \rowcolors{2}{black!10!white}{}
  \begin{tabular}{l l r}
    \toprule
    \textbf{Religion} & \textbf{Income} & \textbf{Frequency} \\
    \midrule
    Agnostic & <\$10k & 27 \\
    Agnostic & \$10-20k & 34 \\
    Agnostic & \$20-30k & 60 \\
    \dots & \dots & \dots \\
    Atheist & <\$10k & 12 \\
    Atheist & \$10-20k & 27 \\
    Atheist & \$20-30k & 37 \\
    \dots & \dots & \dots \\
    \bottomrule
  \end{tabular}
\end{tablebox}

\clearpage
\subsubsection{Multiple variables are stored in one column}  For example, consider the
\cref{tab:messy2}.  This table is not tidy because the column --- interestingly called
\emph{column} --- contains multiple variables.  This format is frequent, and sometimes the
column name contains the names of the variables.  Sometimes it is very hard to separate
the variables.

\begin{tablebox}[label=tab:messy2]{Messy table, from TB dataset, where multiple variables are stored in one column.}
  \centering
  \rowcolors{2}{black!10!white}{}
  \begin{tabular}{l l l r c}
    \toprule
    \textbf{country} & \textbf{year} & \textbf{column} & \textbf{cases} & \textbf{\dots} \\
    \midrule
    AD & 2000 & m014 & 0 & \dots \\
    AD & 2000 & m1524 & 0 & \dots \\
    AD & 2000 & m2534 & 1 & \dots \\
    AD & 2000 & m3544 & 0 & \dots \\
    \dots & \dots & \dots & \dots \\
    \bottomrule
  \end{tabular}
\end{tablebox}

To make it tidy, we can transform it into the \cref{tab:tidy2}.  Two columns are created
to contain the variables \emph{Sex} and \emph{Age}, and the old column is removed.  The
table keeps the same number of rows, but it is now wider.  This is a common pattern when
fixing this kind of issue.  The new version usually fixes the issue of correctly
calculating ratios and frequency.

\begin{tablebox}[label=tab:tidy2]{Tidy version of \cref{tab:messy2} where values are correctly moved.}
  \centering
  \rowcolors{2}{black!10!white}{}
  \begin{tabular}{l l l l r c}
    \toprule
    \textbf{country} & \textbf{year} & \textbf{sex} & \textbf{age} & \textbf{cases} & \textbf{\dots} \\
    \midrule
    AD & 2000 & m & 0--14 & 0 & \dots \\
    AD & 2000 & m & 15--24 & 0 & \dots \\
    AD & 2000 & m & 25--34 & 1 & \dots \\
    AD & 2000 & m & 35--44 & 0 & \dots \\
    \dots & \dots & \dots & \dots & \dots \\
    \bottomrule
  \end{tabular}
\end{tablebox}

\clearpage
\subsubsection{Variables are stored in both rows and columns}  For example, consider the
\cref{tab:messy3}.  This is the most complicated case of messy data.  Usually, one of the
columns contains the names of the variables, in this case the column \emph{element}.

\begin{tablebox}[label=tab:messy3]{Messy table, adapted from molten weather dataset, where variables are stored in both rows and columns.}
  \centering
  \rowcolors{2}{black!10!white}{}
  \begin{tabular}{llllcccc}
    \toprule
    \textbf{id} & \textbf{year} & \textbf{mo.} & \textbf{element} & \textbf{d1} & \textbf{d2} & \textbf{\dots} & \textbf{d31} \\
    \midrule
    MX17004 & 2010 & 1 & tmax &    & 24 & \dots & 27 \\
    MX17004 & 2010 & 1 & tmin & 14 &    & \dots &    \\
    MX17004 & 2010 & 2 & tmax & 27 & 24 & \dots & 27 \\
    MX17004 & 2010 & 2 & tmin & 14 &    & \dots & 13 \\
    \dots & \dots & \dots & \dots & \dots & \dots & \dots & \dots \\
    \bottomrule
  \end{tabular}
\end{tablebox}

To fix this issue, we must first decide which column contains the names of the variables.
Then, we must lengthen the table in function of the variables (and potentially their
names), as seen in \cref{tab:tidy3a}.

\begin{tablebox}[label=tab:tidy3a]{Partial solution to tidy \cref{tab:messy3}. Note that
  the table is now longer.}
  \centering
  \rowcolors{2}{black!10!white}{}
  \begin{tabular}{lllc}
    \toprule
    \textbf{id} & \textbf{date} & \textbf{element} & \textbf{value} \\
    \midrule
    MX17004 & 2010-01-01 & tmax &    \\
    MX17004 & 2010-01-01 & tmin & 14 \\
    MX17004 & 2010-01-02 & tmax & 24 \\
    MX17004 & 2010-01-02 & tmin &    \\
    \dots & \dots & \dots & \dots \\
    \bottomrule
  \end{tabular}
\end{tablebox}

\clearpage
Afterwards, we widen the table in function of their names.  Finally, we remove
implicit information, as seen in \cref{tab:tidy3b}.

\begin{tablebox}[label=tab:tidy3b]{Tidy version of \cref{tab:messy3} where values are correctly moved.}
  \centering
  \rowcolors{2}{black!10!white}{}
  \begin{tabular}{llcc}
    \toprule
    \textbf{id} & \textbf{date} & \textbf{tmin} & \textbf{tmax} \\
    \midrule
    MX17004 & 2010-01-01 & 14 &    \\
    MX17004 & 2010-01-02 &    & 24 \\
    \dots & \dots & \dots & \dots \\
    \bottomrule
  \end{tabular}
\end{tablebox}

\subsubsection{Multiple types of observational units are stored in the same table}  For
example, consider the \cref{tab:messy4}.  It is very common during data collection that
many observational units are registered in the same table.

\begin{tablebox}[label=tab:messy4]{Messy table, adapted from billboard dataset, where multiple types of observational units are stored in the same table.}
  \centering
  \rowcolors{2}{black!10!white}{}
  \begin{tabular}{lllll}
    \toprule
    \textbf{year} & \textbf{artist} & \textbf{track} & \textbf{date} & \textbf{rank} \\
    \midrule
    2000 & 2 Pac & Baby Don't Cry & 2000-02-26 & 87 \\
    2000 & 2 Pac & Baby Don't Cry & 2000-03-04 & 82 \\
    2000 & 2 Pac & Baby Don't Cry & 2000-03-11 & 72 \\
    2000 & 2 Pac & Baby Don't Cry & 2000-03-18 & 77 \\
    \dots & \dots & \dots & \dots & \dots \\
    2000 & 2Ge+her & The Hardest\dots & 2000-09-02 & 91 \\
    2000 & 2Ge+her & The Hardest\dots & 2000-09-09 & 87 \\
    2000 & 2Ge+her & The Hardest\dots & 2000-09-16 & 92 \\
    \dots & \dots & \dots & \dots & \dots \\
    \bottomrule
  \end{tabular}
\end{tablebox}

To fix this issue, we must ensure that each observation unit is moved to a different table.
Sometimes, it is useful to create unique identifiers for each observation.
The separation avoids several types of potential inconsistencies.  However, take into
account that during data analysis, it is possible that we have to denormalize them.  The
two resulting tables are shown in \cref{tab:tidy4a} and \cref{tab:tidy4b}.

\begin{tablebox}[label=tab:tidy4a]{Tidy version of \cref{tab:messy4} containing the observational unit \emph{track}.}
  \centering
  \rowcolors{2}{black!10!white}{}
  \begin{tabular}{lll}
    \toprule
    \textbf{track id} & \textbf{artist} & \textbf{track} \\
    \midrule
    1 & 2 Pac & Baby Don't Cry \\
    2 & 2Ge+her & The Hardest Part Of Breaking Up \\
    \dots & \dots & \dots \\
    \bottomrule
  \end{tabular}
\end{tablebox}

\begin{tablebox}[label=tab:tidy4b]{Tidy version of \cref{tab:messy4} containing the observational unit \emph{rank of the track in a certain week}.}
  \centering
  \rowcolors{2}{black!10!white}{}
  \begin{tabular}{lll}
    \toprule
    \textbf{track id} & \textbf{date} & \textbf{rank} \\
    \midrule
    1 & 2000-02-26 & 87 \\
    1 & 2000-03-04 & 82 \\
    1 & 2000-03-11 & 72 \\
    1 & 2000-03-18 & 77 \\
    \dots & \dots & \dots \\
    2 & 2000-09-02 & 91 \\
    2 & 2000-09-09 & 87 \\
    2 & 2000-09-16 & 92 \\
    \dots & \dots & \dots \\
    \bottomrule
  \end{tabular}
\end{tablebox}

\subsubsection{A single observational unit is stored in multiple tables}  For example, consider
\cref{tab:messy5a,tab:messy5b}.  It is very common during data
collection that a single observational unit is stored in multiple tables.  Usually, the
table (or file) itself represents the value of a variable.  When columns are compatible,
it is straightforward to combine the tables.

\begin{tablebox}[label=tab:messy5a]{Messy tables, adapted from nycflights13 dataset, where
  a single observational unit is stored in multiple tables.  Assume that the origin
  filename is called \texttt{2013.csv}.}
  \centering
  \rowcolors{2}{black!10!white}{}
  \begin{tabular}{llll}
    \toprule
    \textbf{month} & \textbf{day} & \textbf{time} & \textbf{\dots} \\
    \midrule
    1 & 1 & 517 & \dots \\
    1 & 1 & 533 & \dots \\
    1 & 1 & 542 & \dots \\
    1 & 1 & 544 & \dots \\
    \dots & \dots & \dots & \dots \\
    \bottomrule
  \end{tabular}
\end{tablebox}

\begin{tablebox}[label=tab:messy5b]{Messy tables, adapted from nycflights13 dataset, where
  a single observational unit is stored in multiple tables.  Assume that the origin
  filename is called \texttt{2014.csv}.}
  \centering
  \rowcolors{2}{black!10!white}{}
  \begin{tabular}{llll}
    \toprule
    \textbf{month} & \textbf{day} & \textbf{time} & \textbf{\dots} \\
    \midrule
    1 & 1 & 830 & \dots \\
    1 & 1 & 850 & \dots \\
    1 & 1 & 923 & \dots \\
    1 & 1 & 1004 & \dots \\
    \dots & \dots & \dots & \dots \\
    \bottomrule
  \end{tabular}
\end{tablebox}

To fix this issue, we must first make the columns compatible.  Then, we can combine the
tables adding a new column that identifies the origin of the data.  The resulting table is
shown in \cref{tab:tidy5}.

\begin{tablebox}[label=tab:tidy5]{Tidy data where \cref{tab:messy5a,tab:messy5b} are combined.}
  \centering
  \rowcolors{2}{black!10!white}{}
  \begin{tabular}{lllll}
    \toprule
    \textbf{year} & \textbf{month} & \textbf{day} & \textbf{time} & \textbf{\dots} \\
    \midrule
    2013 & 1 & 1 & 517 & \dots \\
    2013 & 1 & 1 & 533 & \dots \\
    2013 & 1 & 1 & 542 & \dots \\
    2013 & 1 & 1 & 544 & \dots \\
    \dots & \dots & \dots & \dots & \dots \\
    2014 & 1 & 1 & 830 & \dots \\
    2014 & 1 & 1 & 850 & \dots \\
    2014 & 1 & 1 & 923 & \dots \\
    2014 & 1 & 1 & 1004 & \dots \\
    \dots & \dots & \dots & \dots & \dots \\
    \bottomrule
  \end{tabular}
\end{tablebox}

\clearpage
\section{Bridging normalization, tidiness, and data theory}
\label{sub:bridge}

First and foremost, both concepts, normalization and tidy data, are not in conflict.

In data normalization, given a set of functional, multivalued, and join dependencies, there
exists a normal form that is free of redundancy.  In tidy data,
\textcite{Wickham2023}\footfullcite{Wickham2023} also state that there is only one way to organize the given data.

\textcite{Wickham2014}\footfullcite{Wickham2014} state that tidy data is 3NF.  However, he does not provide a
formal proof.  Since tidy data focuses on data analysis and not on data storage, I argue
that there is more than one way to organize the data in a tidy format.  It actually
depends on what you define as the observational unit.

Moreover, both of them are related to the philosophical concept of substance (ousia) ---
see \cref{sub:phenomena}.
Entities and observational units are substances while attributes are predicates.
Each tuple or observation is a primary substance, i.e., a substance that contrasts with
everything else, particular, individual.

We can also understand primary keys and fixed variables as the same concept.  They both
describe the sample uniquely.  They connect the entities/observational
units to the remaining attributes.  They also should never be fed into a learning
machine (more details in \cref{chap:slt}), since they are individual and thus not
appropriate to generalize.

\begin{tablebox}[label=fig:bridge]{Terms in different contexts.}
  \centering
  \rowcolors{2}{black!10!white}{}
  \begin{tabular}{ccc}
    \toprule
    \textbf{Relations} & \textbf{Tidy data} & \textbf{Philosophy} \\
    \midrule
    Entities & Observational units & Substance \\
    Tuple & Observation & Primary substance \\
    %Attribute & Attribute & Predicate \\
    Primary key & Fixed variables & Univocal name \\
    Non-prime attr. & Measured variable & Predicate \\
    \bottomrule
  \end{tabular}
  \tcblower
  Equivalence (or similarity) of data-related terms in different contexts.
  The ontological understanding of the data influences the way it is organized.
\end{tablebox}

\Cref{fig:bridge} summarizes the equivalence (or similarity) of terms in different
contexts.

\subsection{Tidy or not tidy?}
\label{sub:tidy-not-tidy}

Consider the following example.  We want to study the \emph{phenomenon} of temperature in a
certain city.  We fix three sensors in different locations to measure the temperature.  We
collect data three times a day.  If we consider as the observational unit the
event of measuring the temperature, we can organize the data in a tidy format as shown in
\cref{tab:temp1}.

\begin{tablebox}[label=tab:temp1]{Tidy data where the observational unit is the event of measuring the temperature.}
  \centering
  \rowcolors{2}{black!10!white}{}
  \begin{tabular}{lllc}
    \toprule
    \textbf{date} & \textbf{time} & \textbf{sensor} & \textbf{temperature} \\
    \midrule
    2023-01-01 & 00:00 & 1 & 20 \\
    2023-01-01 & 00:00 & 2 & 21 \\
    2023-01-01 & 00:00 & 3 & 22 \\
    2023-01-01 & 08:00 & 1 & 21 \\
    2023-01-01 & 08:00 & 2 & 22 \\
    2023-01-01 & 08:00 & 3 & 23 \\
    \dots & \dots & \dots & \dots \\
    \bottomrule
  \end{tabular}
\end{tablebox}

However, since the sensors are fixed, we can consider the observational unit as the
\emph{temperature at some time}.  In this case, we can organize the data in a tidy format
as shown in \cref{tab:temp2}.

\begin{tablebox}[label=tab:temp2]{Tidy data where the observational unit is the temperature at some time.}
  \centering
  \rowcolors{2}{black!10!white}{}
  \begin{tabular}{llccc}
    \toprule
    \textbf{date} & \textbf{time} & \textbf{temp. 1} & \textbf{temp. 2} & \textbf{temp. 3} \\
    \midrule
    2023-01-01 & 00:00 & 20 & 21 & 22 \\
    2023-01-01 & 08:00 & 21 & 22 & 23 \\
    \dots & \dots & \dots & \dots & \dots \\
    \bottomrule
  \end{tabular}
\end{tablebox}

In both cases, one can argue that the data is also normalized.  In the first case, the
primary key is the composite of the columns \emph{date}, \emph{time}, and \emph{sensor}.
In the second case, the primary key is the composite of the columns \emph{date} and
\emph{time}.

One can state that the first form is more appropriate, since it is flexible enough to add more
sensors or sensor-specific attributes (using an extra table).  However, the second form is very natural for machine learning and statistical
methods.  Given the definition of tidy data, I believe both forms are correct.  It is
just a matter of what ontological view you have of the data.

\begin{tablebox}[label=tab:body1]{Tidy data for measurements of a person's body.}
  \centering
  \rowcolors{2}{black!10!white}{}
  \begin{tabular}{lccc}
    \toprule
    \textbf{name} & \textbf{chest} & \textbf{waist} & \textbf{hip} \\
    \midrule
    Alice & 90 & 70 & 100 \\
    Bob & 100 & 110 & 110 \\
    \dots & \dots & \dots & \dots \\
    \bottomrule
  \end{tabular}
\end{tablebox}

Still, one can argue that the sensors share the same nature and thus only the first
form is correct (or can even insist that the more flexible form is the correct one).
Consider however the data in \cref{tab:body1}.  The observational unit is the person, and
the attributes are the body measurements.

\begin{tablebox}[label=tab:body2]{Another tidy data for measurements of a person's body.}
  \centering
  \rowcolors{2}{black!10!white}{}
  \begin{tabular}{llc}
    \toprule
    \textbf{name} & \textbf{body part} & \textbf{measurement} \\
    \midrule
    Alice & chest & 90 \\
    Alice & waist & 70 \\
    Alice & hip & 100 \\
    Bob & chest & 100 \\
    Bob & waist & 110 \\
    Bob & hip & 110 \\
    \dots & \dots & \dots \\
    \bottomrule
  \end{tabular}
\end{tablebox}

If we apply the same logic of \cref{tab:temp1}, data in \cref{tab:body1} becomes
\cref{tab:body2}.  Now, the observational unit is the measurement of a body part of a
given person.  Now, we can easily include more body parts.  Let us say that we want to add
the head circumference.  We just need to include rows such as ``Alice, head, 50'' and
``Bob, head, 55''.  Moreover, what if we want to add the height of the person? Should we
create another table (with ``name'' and ``height'') or should we consider ``height''
as another body part (even though it seems weird to consider the full body a part of the
body)?

In the first version of the data (\cref{tab:temp1}), it would be trivial to include head
circumference and height.  In the second version, the choice becomes inconvenient.  This
table seems ``overly tidy''.  If the first fits well for the analysis, it should be
preferred.

In summary, tidiness is a matter of perspective.

\subsection{Change of observational unit}
\label{sub:change-unit}

Another very interesting conjecture is whether we can formalize the eventual \emph{change
of observational unit} in terms of the order that joins and grouping operations are
performed.

Consider the following example: the relation $R[A, B, C, D, E]$ and the functional
dependencies $A \to D$, $B \to E$, and $AB \to C$.  The relation can be normalized up to
3NF by following one of the decomposition trees shown in \cref{fig:decomp}.
Every decomposition tree must take into account that the join of the projections is
lossless and dependency preserving.

\begin{figurebox}[label=fig:decomp]{Decomposition trees for the relation $R[ABCDE]$ and
  the functional dependencies $A \to D$, $B \to E$, and $AB \to C$ to reach 3NF.}
  \centering
  \begin{tikzpicture}
    \node (root1) at (0, 0) {ABCDE}
      child {node {AD}}
      child {node {ABCE}
        child {node {BE}}
        child {node {ABC}}};
    \node (root2) at (3, 0) {ABCDE}
      child {node {BE}}
      child {node {ABCD}
        child {node {AD}}
        child {node {ABC}}};
  \end{tikzpicture}
\end{figurebox}

Note that the decomposition that splits first $R[ABC]$ is not valid, since the resulting
relation $R[AB]$ is not a consequence of a functional dependency; see
\cref{fig:wrongdecomp}.

\begin{figurebox}[label=fig:wrongdecomp]{Invalid decomposition trees for the relation $R[ABCDE]$.}
  \centering
  \begin{tikzpicture}
    \node (root1) at (0, 0) {ABCDE}
      child {node {ABC}}
      child {node {ABDE}
        child {node {AD}}
        child {node {ABE}
          child {node {BE}}
          child {node[gray] {AB}}}};
    \node (root2) at (3, 0) {ABCDE}
      child {node {ABC}}
      child {node {ABDE}
        child {node {BE}}
        child {node {ABD}
          child {node {AD}}
          child {node[gray] {AB}}}};
  \end{tikzpicture}
  \tcblower
  We consider the functional dependencies $A \to D$, $B \to E$, and $AB \to C$.
  Note that $R[AB]$ is not a consequence of a functional dependency.
\end{figurebox}

In this kind of relation schema, we have a set of key attributes, here $\mathcal{K} = AB$,
and a set of non-prime attributes, here $\mathcal{N} = CDE$.  Note that the case
$\mathcal{K} \cap \mathcal{N} = \emptyset$ is the simplest one we can have.

Observe, however, that transitive dependencies\footnote{Actually, when an attribute is
both key and non-prime, some joins may generate invalid tables.} and complex join
dependencies restrict even further the joins we are allowed to perform.
% \textcolor{red}{Further formalization and study is under progress.}

Now, consider a very common case: in our dataset, keys are unknown.  Let $A$ be a student
id, $B$ be the course id, $D$ be the student age, $E$ be the course load, and $C$ be the
student grade at the course.  If only $CDE$ is known, the table $R[CDE]$ is already tidy
--- and the observational unit is the enrollment --- once there is no key to perform any
kind of normalization.  This happens in many cases where privacy is a concern.

\begin{tablebox}[label=tab:student]{Example of a dataset where the observational unit is the student.}
  \centering
  \rowcolors{2}{black!10!white}{}
  \begin{tabular}{llll}
    \toprule
    \textbf{A (student)} & \textbf{B (course)} & \textbf{C (grade)} & \textbf{E (load)} \\
    \midrule
    1 & 1 & 7 & 60 \\
    1 & 2 & 8 & 30 \\
    2 & 1 & 7 & 60 \\
    2 & 3 & 9 & 40 \\
    \dots & \dots & \dots & \dots \\
    \bottomrule
  \end{tabular}
  \\[1em]
  \begin{tabular}{lll}
    \toprule
    \textbf{A (student)} & \textbf{F (average grade)} & \textbf{G (total load)} \\
    \midrule
    1 & 7.5 & 90 \\
    2 & 8 & 100 \\
    \dots & \dots & \dots \\
    \bottomrule
  \end{tabular}
  \tcblower
  Relation $R[ABCE]$ becomes $R[AFG]$ after the summarization operation.  Now each row
  represents a student (values in $A$ are unique).
\end{tablebox}

But we can also consider that the observational unit is the student.  In this case, we
must perform joins traversing the leftmost decomposition tree in \cref{fig:decomp} from
bottom to top.  After each join, a summarization operation is performed on the relation
considering the student as the observational unit, i.e., over attribute $A$.  The first
join results in relation $R[ABCE]$ and the summarization operation results in a new
relation $R[AFG]$ where $F$ is the average grade and $G$ is the total course load taken by
the student (see \cref{tab:student}).  They are all calculated based on the rows that are grouped in function of $A$.
It is important to notice that, after the summarization operation, all observations must
contain a different value of $A$.  The second join results in relation $R[ADFG] = R[AD]
\bowtie R[AFG]$.  This relation has functional dependency $A \to DFG$, and it is in 3NF
(which is also tidy).

Unfortunately, it is not trivial to calculate all possible decomposition trees for a given
dataset.  It is up to the data scientist to decide which directions to follow.  However,
it is important to notice that the order of the joins and summarization operations are
crucial to the final result.
%\textcolor{red}{Further formalization and study is under progress.}

\section{Data semantics and interpretation}

In the rest of the book, we focus on a statistical view of the data.  Besides the
functional dependencies, we also consider the statistical dependencies of the data.  For
instance, attributes $A$ and $B$ might not be functionally dependent, but they might exist
in an unknown $P(A, B)$ that we can estimate from the data.  Each observed value of a key can
represent an instance of a random variable, and the other attributes can represent
measured attributes or calculated properties.

For data analysis, it is very important to understand the relationships between the
observations.  For example, we might want to know if the observations are independent, if
they are identically distributed, or if there is a known selection bias.  We might also
want to know if the observations are dependent on time, and if there are hidden variables
that affect the observations.

Following wrong assumptions can lead to wrong conclusions.  For example, if we assume that
the observations are independent, but they are not, we might underestimate the variance of
the estimators.

Although we do not focus on time series, we must consider the temporal dependence of the
observations.  For example, we might want to know how the observation $x_t$ is affected by
$x_{t-1}$, $x_{t-2}$, and so on.  We might also want to know if the Markov property holds,
and if there is periodicity and seasonality in the data.

For the sake of the scope of this book, we suggest that any prediction on temporal data
should be done in the state space, where it is safer to assume that observations are
independent and identically distributed.  This is a common practice in reinforcement
learning and deep learning. Takens' theorem\footfullcite{Takens1980} allows you to
reconstruct the state space of a dynamical system using time-delay embedding. Given a
single observed time series, you can create a multidimensional representation of the
underlying dynamical system by embedding the time series in a higher-dimensional space.
This embedding can reveal the underlying dynamics and structure of the system.

\section{Unstructured data}

Unstructured data are data that do not have a predefined data model or are not organized
in a predefined manner.  For example, text, images, and videos are unstructured data.

Every unstructured dataset can be converted into a structured one.  However, the
conversion process is not always straightforward nor lossless.  For example, we can
convert a text into a structured dataset by counting the number of occurrences of each
word\footnote{This is called a bag-of-words approach.}.
However, we lose the order of the words in the text.

The study of unstructured data is, for the moment, out of the scope of this book.

% vim: set spell spelllang=en:

\chapter{Data handling}
\label{chap:handling}

\chapterprecishere{%
  {\fontspec[Scale=2]{Symbola}\color{black!80}\symbol{"1F5E1}}
  It's dangerous to go alone! Take this.
  \par\raggedleft--- \textup{Unnamed Old Man}, The Legend of Zelda}

% Important: avoid the term "data manipulation" as it has a negative connotation
% TODO: review this introduction after finishing the remaining of the chapter

In the previous chapter, I discussed the relationship between data format and data
semantics.  We also saw in \cref{chap:project} that data tasks --- specifically
integration and tidying --- must adjust the available data to reflect the kind of
input we expect in production.

For those tasks, we must be careful with the operations we perform on the data. At the
stage of data preparation, for example, we should never parametrize our data handling
pipeline in terms of information retrieved\footnote{For instance, imputation by the mean
of a column.} by sampling the data.  This is because such operations lead to \gls{leakage}
during evaluation and other biases in our conclusions.

In this chapter, we consider that tables are rectangular data structures in which values
of the same column share the same properties (i.e. the same type, same restrictions, etc.)
and each column has a name.  Moreover, we assume that any value is possibly
\emph{missing}.

From a mathematical definition of such tables, we can define a set of operations that can
be applied to them.  These operations are the building blocks of data handling pipelines:
combinations of operations that transform a dataset into another dataset.
I also highlight some important properties of these operations.

I show how these operations can be combined to create complex data handling pipelines by
using them to solve the issues presented in \cref{sub:messy}.

\begin{mainbox}{Chapter remarks}

  \boxsubtitle{Contents}

  \startcontents[chapters]
  \printcontents[chapters]{}{1}{}
  \vspace{1em}

  \boxsubtitle{Context}

  \begin{itemize}
    \item \dots
  \end{itemize}

  \boxsubtitle{Objectives}

  \begin{itemize}
    \item \dots
  \end{itemize}

  \boxsubtitle{Takeways}

  \begin{itemize}
    \item \dots
  \end{itemize}
\end{mainbox}

{}
\clearpage

\section{Formal structured data}

\newcommand{\domainof}[1]{\mathcal{D}\!\left(#1\right)}
\newcommand{\missing}{\text{?}}
\newcommand{\rowcard}[1][k_1, \dots, k_k]{\operatorname{card}\!\left(#1\right)}

In this section, I present a formal definition of structured data.  This definition is
compatible with the relational model and tidy data presented in \cref{chap:data}.
My definition takes into account the index\footnote{Also called grouping variables.} of
the table, which is a key concept in data handling.  We also consider that values can be
missing.  Repeated values are represented by allowing cells to contain sets of values.

\begin{defbox}{Indexed table}{itable}
An indexed table $T$ is a tuple $(K, H, c)$, where $K = \left\{K_i : i = 1, \dots,
k\right\}$ is the set of index columns, $H$ is the set of (non-index) columns, and $c :
\domainof{K_1} \times \dots \times \domainof{K_k} \times H \to \mathcal{V}$ is the cell function.
Here, $\mathcal{V}$ represents the space of all possible tuples of values, which
may include missing values $\missing$.  Values have arbitrary types, such as integers,
real numbers, strings, etc.
Each index column $K_i$ has a domain $\domainof{K_i}$, which is an enumerable set of
values.
\end{defbox}

A possible row $r$ of the table is thus indexed by a tuple $r = (k_1, \dots,
k_k)$, where $k_i \in \domainof{K_i}$.  Each row has a cardinality $\rowcard[r]$, which
represents how many times the entity represented by the row is present in the table.
A row $r$ with $\rowcard[r] = 0$ is considered to be missing.

A cell is then represented by a row $r$ and a column $h \in H$.  The value of the cell,
$\vec{v} = c(r, h)$ is a tuple of values such that $|\vec{v}| = \rowcard[r]$.

\begin{defbox}{Value matrix}{vmatrix}
The value matrix $V$ of the row $r$ is \[
  \Big[ c(r, h) : h \in H \Big]\text{,}
\] with dimensions $\rowcard[r] \times |H|$, assuming an arbitrary fixed order of the
columns.
\end{defbox}

We assume that value matrices --- and consequently row cardinalities --- are minimal. This
means that there are no \emph{nested row} $v_{i, 1}, \dots, v_{i, |H|}$ in the value matrices
such that $v_{i, j} = \missing$ for all $j$.

From these concepts, we can define the basic operations and properties that can be applied
to tables.

\subsection{Splitting and binding}

Split and bind are very basic operations that can be applied to tables.  They are
inverses of each other and are used to divide and combine tables, respectively.
They are important in the data science process because they play a key role in
data semantics and validation of solutions.

\begin{defbox}{Split operation}{split}
Given an indicator function $s : \domainof{K_1} \times \dots \times \domainof{K_k} \to
\left\{0, 1\right\}$, the split operation creates two tables, $T_0$ and $T_1$, that
contains only the rows for which
$s(r) = 0$ and $s(r) = 1$, respectively.

Mathematically, the split operation is defined as \[
  \operatorname{split}(T, s) = \left(T_0, T_1\right)\text{,}
\] where $T = (K, H, c)$, $T_i = (K, H, c_i)$, and \[
  c_i(r, h) = \begin{dcases}
    c(r, h) & \text{if } s(r) = i \\
    () & \text{otherwise.}
  \end{dcases}
\]
\end{defbox}

\emph{Note that, by definition, the split operation never ``breaks'' a row.  So, the
indices define the indivisible entities of the table.}  The resulting tables are
disjoint:

\begin{defbox}{Disjoint tables}{disjoint-tables}
  Two tables $T_0 = (K, H, c_0)$ and $T_1 = (K, H, c_1)$ are said to be disjoint if
  $\rowcard[r; c_0] = 0$ if $\rowcard[r; c_1] > 0$ for any row $r$, and vice-versa.
\end{defbox}

The binding operation is the inverse of the split operation.  Given two disjoint tables
$T_0 = (K, H, c_0)$ and $T_1 = (K, H, c_1)$, the binding operation creates a new table $T$
that contains all the rows of $T_0$ and $T_1$.

\begin{defbox}{Bind operation}{bind}
  Mathematically, the binding operation is defined as \[
    \operatorname{bind}(T_0, T_1) = (K, H, c)\text{,}
  \] where $T_i = (K, H, c_i)$ and \[ c(r, h) = c_0(r, h) + c_1(r, h)\text{.} \]
  The operator $+$ stands for the tuple concatenation operator%
  \footnote{The order of the concatenation here is not an issue since we guarantee
  that at least one of the operands is empty.}.
\end{defbox}

\emph{Thus, a requirement for the binding operation is that the tables are disjoint in
terms of the row entities they have.}

\paragraph{Premises in real-world applications}

One important aspect of these functions is that we assume that the entities represented by
the rows are indivisible, and that any binding operation will never occur for tables that
share the same entities.

In real-world applications, this is not always true.  Many times, we do not know the
process someone else has used to collect the data.  In these cases, we must be careful
about the guarantees we explain in this chapter.  On the other hand, one can consider the
premises we use as a guideline to design good data collection processes.

We can see data collection as the result of a splitting operation in the universe set of
all possible entities.  This is a good way to think about data collection, as we can try
to ensure that we collect all possible information about the entities we are interested
in.

This, of course, depends on what we define as the index columns of the table.  Consider
the example of collecting information about grades of students.  If we define as the index
columns the student's name and year, we must ensure that we collect all the grades of all
subjects a student has taken in a year.  We do not need, though, to collect information
from all students or all years.  On the other hand, if we define as the index column only
the student's name, we must collect all the grades of all subjects a student has taken in
all years.

In summary, the less variables we define as index columns, the more information we must
collect about each entity.  However, in the next sections, we show that assuming many index columns
lead to restrictions in the operations we can perform on the table.

This conceptual trade-off is important to understand when structuring the problem we are
trying to solve.  Neglecting these issues can lead to strong statistical biases and
incorrect conclusions.

\subsection{Split-invariance}

One property we can study about data handling operations is whether they are distributive
over the bind operation.  This property is called \emph{split-invariance}.

From now on, we will denote \[
  T_0 + T_1 = \operatorname{bind}(T_0, T_1)\text{,}
\] for any tables $T_0$ and $T_1$ to simplify the notation.

\begin{defbox}{Split-invariance}{split-invariance}
An arbitrary data handling operation $f(T)$ is said to be split-invariant
if, for any table $T$ and split function $s$, the following equation holds \[
  f\!\left(T_0 + T_1\right) =
    f\!\left(T_0\right) + f\!\left(T_1\right)\text{,}
\] where $T_0, T_1 = \operatorname{split}\!\left(T; s\right)$.
\end{defbox}

Split-invariance is a desirable property for data handling operations during the data
tasks described in \cref{chap:project}: integration and tidying.  Even while exploring
data, we should take effort to use split-invariant operations.

The reason is that split-invariance ensures that the operation does not depend on the
split performed (usually unknown to us) to create the table we have in hand.  This
property is important to avoid \gls{leakage} or to bias the results of the analysis.

\subsection{Illustrative example}

\begin{tablebox}[label=tab:grades1]{Data table of student grades.}
  \centering
  \rowcolors{2}{black!10!white}{}
  \begin{tabular}{cccc}
    \toprule
    \textbf{student} & \textbf{subject} & \textbf{year} & \textbf{grade} \\
    \midrule
    Alice & Chemistry & 2020 & 6 \\
    Alice & Math & 2019 & 8 \\
    Alice & Physics & 2019 & 7 \\
    Bob & Chemistry & 2018 & ? \\
    Bob & Chemistry & 2019 & 7 \\
    Bob & Math & 2019 & 9 \\
    Bob & Physics & 2019 & 4 \\
    Bob & Physics & 2020 & 8 \\
    Carol & Biology & 2020 & 8 \\
    Carol & Chemistry & 2020 & 3 \\
    Carol & Math & 2020 & 10 \\
    \bottomrule
  \end{tabular}
  \tcblower
  Data collected about student grades.  All information that is available is presented.
\end{tablebox}

Consider the example of data collected about student grades.  \Cref{tab:grades1}
exemplifies all information we can possibly have about the grades of students.  A missing
value in a cell of that table indicates that, for some reason, the information is not
retrievable.

The domain of the variables are:
\begin{itemize}
  \itemsep0em
  \item $\domainof{\text{student}} = \left\{\text{Alice}, \text{Bob}, \text{Carol}\right\}$;
  \item $\domainof{\text{subject}} = \left\{\text{Biology}, \text{Chemistry}, \text{Math},
    \text{Physics}\right\}$;
  \item $\domainof{\text{year}} = \mathbb{Z}$; and
  \item $\domainof{\text{grade}} = \left[0, 10\right] \cup \left\{\missing\right\}$.
\end{itemize}

Of course, in practice, we have no guarantee that the data we have is complete nor the
clear specification of the domain of the variables.  Instead, we must choose good
premises about the data we are working with.

Knowing that the data is complete, we can safely assume that:
\begin{enumerate}
  \itemsep0em
  \item Alice has never taken Biology;
  \item Bob passed Physics, although at the second attempt;
  \item Carol has only taken classes in 2020.
\end{enumerate}

\begin{tablebox}[label=tab:grades2]{Data table of student grades assuming student and subject as indices.}
  \centering
  \rowcolors{2}{black!10!white}{}
  \begin{tabular}{ccccc}
    \toprule
    \textbf{s} & \textbf{student} & \textbf{subject} & \textbf{year} & \textbf{grade} \\
    \midrule
    0 & Alice & Chemistry & (2020) & (6) \\
    1 & Alice & Math & (2019) & (8) \\
    1 & Alice & Physics & (2019) & (7) \\
    0 & Bob & Chemistry & (2018, 2019) & (?, 7) \\
    0 & Bob & Math & (2019) & (9) \\
    1 & Bob & Physics & (2019, 2020) & (4, 8) \\
    0 & Carol & Biology & (2020) & (8) \\
    0 & Carol & Chemistry & (2020) & (3) \\
    1 & Carol & Math & (2020) & (10) \\
    \bottomrule
  \end{tabular}
  \tcblower
  Indexed table with data from \cref{tab:grades1} assuming student and
  subject as indices.  The column $s$ is the split indicator.
\end{tablebox}

Now consider an arbitrary collection mechanism that consider student and subject as the
indices of the table.  \Cref{tab:grades2} shows the table we have in hand.  The column $s$
is the split indicator.  Only rows with $s = 1$ are available to us.

Now, about the statements we made before:
\begin{enumerate}
  \itemsep0em
  \item There is no way we can know if Alice has taken Biology or not.  It could be that
    the data collection mechanism failed to collect this information or that the
    information simply does not exist.
  \item We can safely assume that Bob has passed Physics in the second try, once all
    information about (Bob, Physics) is assumed to be available.
  \item There is no guarantee that Carol has only taken classes in 2020.  It could be that
    some row (Carol, subject) with year different than 2020 is missing in the table.
\end{enumerate}

\begin{tablebox}[label=tab:grades3]{Data table of student grades assuming student as the index.}
  \centering
  \rowcolors{2}{black!10!white}{}
  \begin{tabular}{ccp{2.6cm}p{1.8cm}>{\raggedright\arraybackslash}p{1.2cm}}
    \toprule
    \textbf{s} & \textbf{student} & \textbf{subject} & \textbf{year} & \textbf{grade} \\
    \midrule
    1 & Alice & (Chemistry, Math, Physics) & (2020, 2019, 2019) & (6, 8, 7) \\
    0 & Bob & (Chemistry, Chemistry, Math, Physics, Physics) & (2018, 2019, 2019, 2019, 2020) & (?, 7, 9, 4, 8) \\
    1 & Carol & (Biology, Chemistry, Math) & (2020, 2020, 2020) & (8, 3, 10) \\
    \bottomrule
  \end{tabular}
  \tcblower
  Indexed table with data from \cref{tab:grades1} assuming only student
  as the index.  The column $s$ is the split indicator.
\end{tablebox}

We can be even more restrictive and consider only the student as the index of the table.
Imposing this restriction would difficult the data collection process, but it would
guarantee that we have all information about each student.  \Cref{tab:grades3} shows the
table we have in hand.  As before, the column $s$ is the split indicator and only rows with
$s = 1$ are available to us.

Our conclusions may change again:
\begin{enumerate}
  \itemsep0em
  \item We can safely assume that Alice has never taken Biology, as $\text{Chemistry}
    \not\in c(\text{Alice}, \text{subject})$.
  \item We can assume nothing about Bob's grades, as all information about him is missing.
  \item We can safely assume that Carol has only taken classes in 2020, as $c(\text{Carol},
    \text{year})$ contains only values with 2020.
\end{enumerate}

It is straightforward to see that the less index columns we have, the more information we
have about the present entities.  Also, we can see how important it is the assumptions on
the index columns to the conclusions we can draw from the data.  Consequently,
split-invariant operations can preserve valid conclusions about the data even when
information is missing\footnote{Absence can be due to incomplete data collection or
artificial splitting for validation, consult \cref{chap:planning}.}.

\section{Data handling pipelines}

In the literature and in software documentation, you will find a variety of terms used to
describe data handling operations\footnote{%
  The terminology ``data handling'' itself is not universal.  Some authors and libraries
  call it ``data manipulation'', ``data wrangling'', ``data shaping'', or ``data
  engineering''.  I use the term ``data handling'' because it seems more generic.
  Also, it avoids confusion with the term ``data
  manipulation'' which has a negative connotation in some contexts.}. %
They often refer to the same or similar operations, but the terminology can be confusing.
In this section, I present a summary of these operations mostly based on
\textcite{Wickham2023} definitions\footnote{Which they call \emph{verbs}.}.


During the preparation of data for our project, we will need to perform a set of operations
on possibly multiple datasets.  These operations are organized in a pipeline, where the
outputs of one operation are the inputs of the next one.
Operations are extensively parametrized, for instance, most of them can use predicates to
define the groups, arrangements, or conditions under which they should be applied.

\begin{figurebox}[label=fig:pipeline]{Example of data handling pipeline.}
  \centering
  \begin{tikzpicture}[every node/.style={font=\small, inner sep=4pt}]
    \node (s1) [darkcircle] at (0, 0) {Source 1};
    \node (s2) [darkcircle] at (0, -2) {Source 2};
    \node (f1) [mediumblock] at (2, 0) {$f_1$};
    \node (f2) [mediumblock] at (4, 0) {$f_2$};
    \node (f3) [mediumblock] at (2, -2) {$f_3$};
    \node (f4) [mediumblock] at (4, -2) {$f_4$};
    \node (f5) [mediumblock] at (6, -1) {$f_5$};
    \node (data) [darkcircle, minimum width=15mm] (data) at (8, -1) {Data};

    \path [line] (s1) -- (f1);
    \path [line] (f1) -- (f2);
    \path [line] (f1.east) -- (f4);
    \path [line] (s2) -- (f3);
    \path [line] (f3) -- (f4);
    \path [line] (f2) -- (f5);
    \path [line] (f4) -- (f5);
    \path [line] (f5) -- (data);
  \end{tikzpicture}
  \tcblower
  A data handling pipeline is a set of operations that transform a dataset into
  another dataset.  We can have more than one source dataset and the output is a single
  dataset where each row represents a sample in the observational unit we are interested
  in.
\end{figurebox}

In \cref{fig:pipeline}, we show an example of a data handling pipeline.  The pipeline
starts with two source datasets, Source 1 and Source 2.  The datasets are processed by a
set of operations, $f_1, f_2, f_3, f_4, f_5$, and the output is a single dataset,
Data.  Our goal at the data tasks --- see \cref{sub:workflow} --- is to create a dataset
that is representative of the observational unit we are interested in.  Representative
here means that the dataset is tidy\footnote{Remember that our definition of tidiness
depend on the observational unit.  That means, in practice, that if the original data
sources are in a observational unit different from the one we are interested in, after
joining them, the connecting variables might be dropped to eliminate transitive
dependencies.  Consult \cref{sub:tidy-not-tidy,sub:change-unit}.} and that the priors,
i.e. the distribution of the data is faithful to the real distribution of the phenomenon.

A pipeline is more flexible than a chain of operations because it can handle more complex
structures, where different branches (forks) of processing occur simultaneously, and then
come together (merges) later in the workflow.  For instance, the output of $f_1$ is the
input of $f_2$ and $f_4$ (fork), and $f_5$ has as input the outputs of $f_2$ and $f_4$
(merge).

Pipelines are great conceptual tools to organize the data handling process.  They allow
for the separation of concerns, where each operation is responsible for a single task.
Also, declaring the whole pipeline at once allows for the optimization of the operations
and the use of parallel processing.  This is important when dealing with large datasets.
The declarative approach, opposed to the imperative one, makes it easier to reason about
and maintain the code\footnote{Tidyverse and Polars are examples of
libraries that use a declarative approach to data handling.}.

\section{Split-invariant operations}

We use the following terminology to refer to the data handling parameters:
\begin{itemize}
  \item \textbf{Predicate}: a function that returns a logical value, used to filter
    rows/columns or to define the groups of rows/columns to be processed;
  \item \textbf{Aggregation function}: a function that returns a single value given a vector
    of values (in which, the order of the values may be important);
  \item \textbf{Window function}: a function that returns a vector of values given a vector
    of values in which, the order of the values is important;
  \item \textbf{Expression}: a function that returns a vector of values element-wise, used to create new
    columns or to modify existing ones.
\end{itemize}

% \begin{slidebox}{Data handling pipelines}{}
%   \begin{itemize}
%     \item Data handling operations can be combined to create complex pipelines;
%     \item Operators may be reversible;
%     \item Operators are vectorized;
%     \item They can be parametrized with predicates, aggregation functions, and expressions;
%     \item They operate on datasets and return new datasets as output.
%     \item They are declarative.
%   \end{itemize}
% \end{slidebox}


\subsection{Filtering rows}

Filtering is the process of selecting a subset of rows from a dataset based on a
predicate.  If more than a single predicate is used, they are combined using a logical
operator, such as logical disjunction (or) or logical conjunction (and).

After filtering, the dataset will contain only the rows that satisfy the predicate.
Columns remain unchanged.  This operation is potentially irreversible, as the removed
rows are lost.

In the basic form, each row is treated independently.  For instance, the predicate
\code{age > 18} will select all rows where the value in the \code{age} column is
greater than 18.

However, if the predicate depends on an aggregation or window function, one must specify
the groups and/or the order of the rows.  For instance, the predicate \code{age >
mean(age) group by country} will select the rows where the value in the \code{age}
column is greater than the mean of the \code{age} for each \code{country}. Another
example is the predicate \code{cumsum(price) < 100 sort by date}, which selects the rows
that satisfy the condition that the cumulative sum of the \code{price} column is less
than 100 given the order of the rows defined by the \code{date} column.

The trivial group is the entire dataset, so it is usually not necessary to specify it
explicitly.  However, it is usually not sensible to not specify the order of the rows.

When dealing with real values, be aware of floating-point precision issues.  In other
words, do not use the equality operator to compare real numbers.  Most of libraries
provide operators to compare real numbers within a given tolerance.

\begin{hlbox}{Practical tips}
  \begin{itemize}
    \item Use filtering to remove rows that are not relevant to your analysis;
    \item Use predicates to define the conditions under which rows should be removed;
    \item When aggregation functions are needed to define the predicate, specify the groups and
      the order of the rows;
    \item Be aware of floating-point precision issues when comparing real numbers.
  \end{itemize}
\end{hlbox}

\subsection{Selecting columns}

Selecting is the process of choosing a subset of columns from a dataset.  The remaining
columns are discarded.  This operation is not reversible, as the discarded columns are
lost.  Rows remain unchanged.

There are two main ways to select columns: by name or by predicate.  The former is the
most common and is used to select a fixed set of columns.  The latter is used to select
columns that satisfy a given condition, i.e., the values in the columns are used to
determine which columns should be selected.

When selecting columns by name, one can use a list of column names or a regular
expression\footnote{Regular expressions are very general and powerful, but they are also
complex and error-prone.  An alternative is to use some form of hierarchical naming,
such as \code{type.column} to express groups of columns.}.
The latter is useful when the column names follow a pattern that reflects the semantics of
the columns.  For instance,
one can use the regular expression \code{col[0-9]+} to select all columns whose names
start with \code{col} followed by one or more digits.

When selecting columns by predicate, one can use a function that returns a logical value
to define the condition under which a column should be selected.  For instance, one can
use the predicate \code{isnumeric} to select all columns that contain numeric values.
Notice, however, that the predicate is applied to each column independently and returns a
single logical value for each column.

Like filtering, selecting predicates might contain aggregation functions.  Although it is
theorically possible to consider the order of the values in the columns, it is not common
to do so.  (Especially because one would need to assume that the rows are previously
sorted by some criterion.) Groups, however, never make sense in this context, once the
predicate is applied to each column independently.

Depending on the context, it may be useful to ``drop'' columns instead of selecting them.
This is the same as selecting all columns except the ones specified.  This is useful when
the number of columns to be dropped is small compared to the total number of columns.
Strictly speaking, we just need to negate the predicate or the regular expression used to
select the columns.

Finally, it is very common to find libraries and framework in which the order of the
columns is important.  As a result, columns can be selected by position as well.
I find this practice error-prone and I recommend avoiding it whenever possible.

\begin{hlbox}{Practical tips}
  \begin{itemize}
    \item Use selecting to remove columns that are not relevant to your analysis;
    \item Use column names or regular expressions (or hierarchical names) to select columns;
    \item Use predicates (many to one, with no aggregation functions) to define the conditions
      under which columns should be selected;
    \item Avoid depending on the order of the columns.
  \end{itemize}
\end{hlbox}

\subsection{Mutating columns}

Mutating is the process of creating new columns.  The operation is reversible, as the
original columns are kept.  The new columns are added to the dataset.

The values in the new column are determined by an expression.  The expression is a
function that returns a vector of values given the values in the other columns.  The
expression can be a simple function, such as \code{y = x + 1}, or a more complex
function, such as
\begin{center}
  \code{y = ifelse(x > 0, 1, 0)}.
\end{center}
Here, \code{x} and \code{y} are
the names of an existing and the new column, respectively.

One may also use an aggregation and window function in the expression. This is particularly
useful when performing mutation considering a group.  In this case, the returned value is
repeated (aggregation function) for each row of the same group.  Like in filtering, the
more explicit you can be about order and groups, the better.

For example, the expression
\begin{center}
  \code{y = cumsum(x) group by category sort by date}
\end{center}
will create a new column \code{y} with the cumulative sum of the \code{x} column for each
\code{category} given the order of the rows defined by the \code{date} column.

Sometimes, the same expression can be used to create multiple columns.  This is useful
when the new columns are related.  To do so, one first specifies the columns in the same way as
when selecting columns.  Then, one needs to specify a rule to name the new columns.
For instance, \code{x\_new = x + 1 across x matches \textasciicircum{}col[0-9]+\$}.

Practically speaking, mutation can overwrite existing columns.  This is useful when the
new column is a replacement for the old one.  Formally, overwriting is just a sequence of
mutation and selection operations.

\begin{hlbox}{Practical tips}
  \begin{itemize}
    \item Use mutating to create new columns that are relevant to your analysis;
    \item Use expressions to define the values of the new columns;
    \item Use aggregation and window functions in the expression to create new columns based on
      groups and order;
    \item Use the same expression to create multiple columns when the new columns are related.
  \end{itemize}
\end{hlbox}

\subsection{Aggregating rows}

We can aggregate the rows of a dataset to create a new dataset with fewer rows.    The
operation is not reversible, as the discarded rows are lost.  The columns are also lost,
only the new aggregate columns remain.

The values in the new columns are determined by an aggregation function.  Like filtering
and mutation, the aggregation function can be parametrized by specifying a group and/or an
order.

The resulting dataset will contain one row for each group.  The values in the new columns
are determined by the aggregation function applied to the values in the other columns.
All columns that define the groups are usually kept in the resulting dataset.  In this
case, as expected, values of such columns are equal for all rows in the same group.

For instance, the aggregation function \code{mean(x) group by category} will create a
new dataset with one row for each different value of \code{category} and a new column
with the mean of the \code{x} column for each group.

\begin{hlbox}{Practical tips}
  \begin{itemize}
    \item Use aggregation to summarize the data in a dataset;
    \item Use aggregation functions to define the values of the new columns;
    \item Other columns are lost;
    \item Use the group and order parameters to define the groups and the behavior of the
      aggregation function.
  \end{itemize}
\end{hlbox}

\subsection{Binding datasets}

One trivial, yet important, operation is to bind datasets.  This is the process of
combining two or more datasets into a single dataset.  The operation is reversible, as the
original datasets are kept.  The new dataset contains all the rows and columns of the
original datasets.

There are two ways to bind datasets: by rows or by columns.  The former is used to
combine datasets that have exactly the same columns but represent different parts of the
same dataset.  The latter is used to combine datasets that comprise the same observations
(rows) but captures different aspects of the same dataset.

When binding datasets by rows, the datasets must have the same columns\footnote{In
practice, it is usually required that they share the same order of the columns as well.
This is not a theoretical requirement, but a common limitation of most libraries.}.
The resulting dataset will contain all the rows of the original datasets.  The columns
remain unchanged.  It is a good practice to create a new column that represents the source
of each row.  For instance, if each table represents data collected in a different year,
one can create a new column \code{year} that contains the year of the data.

When binding datasets by columns, the datasets must have the same number of rows.  Each
matching row represent the same observation\footnote{Practically speaking, either the
order of the rows or a key column is used to match the rows of the datasets.  In both
situations, this is equivalent to a join operation by the row number or the key column;
assuming that both datasets contains the same observations.}. The resulting dataset will
contain all the columns of the original datasets.  The rows remain unchanged.

\begin{hlbox}{Practical tips}
  \begin{itemize}
    \item Use binding to combine datasets that represent different parts of the same dataset;
    \item Use binding by rows to combine datasets that have the same columns --- in this
      case, create a new column that represents the source of each row;
    \item Use binding by columns to combine datasets that have the same number of rows.
  \end{itemize}
\end{hlbox}

{\color{red} Talk about splitting as the reverse function, and the reason why missing
columns may be a problem. Example of the unit of measurement.}

\subsection{Joining datasets}

Joining is the process of combining two datasets into a single dataset based on common
columns.  The operation may not be reversible, consult \cref{sec:normalization} for more
details.

The join of two tables is the operation that returns a new table with the columns of both
tables.  Let \code{U} be the common set of columns.  For each occurring value of
\code{U} in the first table, the operation will look for the same value in the second
table.  If it finds it, it will create a new row with the columns of both tables.  If it
does not find it, no row will be created.  This operation assumes that values in \code{U}
are unique in each table.

The variation described above is usually called natural or inner join.  Three other
variations are possible.
\begin{itemize}
  \item Left join: for each occurring value of \code{U} in the first table, the operation
    will look for the same value in the second table.  If it finds it, it will create a new
    row with the columns of both tables.  If it does not find it, it will create a new row
    with the columns of the first table and missing values for the columns of the second
    table.
  \item Right join: the same as the left join, but the roles of the tables are reversed.
  \item Outer join: for each different value of \code{U} in both tables, the operation
    will create a new row with the columns of both tables.  If a value is missing in one
    table, it will be filled with a missing value.
\end{itemize}

\begin{hlbox}{Practical tips}
  \begin{itemize}
    \item Use joining to integrate datasets;
    \item Be aware of the risks of joining datasets (\cref{sec:normalization}), for
      example, that some joins may create invalid rows;
    \item Use the appropriate variation of the join operation in applications.
  \end{itemize}
\end{hlbox}

\subsection{Pivoting and unpivoting}

Another important operation is to pivot and unpivot datasets.  These are the processes of
transforming a dataset from a long format to a wide format and vice versa.  The operations
are reversible and they are the inverse of each other.

Pivoting requires to specify a name column --- whose discrete and finite possible values
will become the names of the new columns --- and a value column --- whose values will be
spread across the rows.  All remaining columns are considered to be keys, uniquely
identifying each row of new the dataset.

Unpivoting\footnote{Which \citeauthor{Wickham2023} call pivot longer.} is the reverse
operation.  One must specify all the columns whose names are the values of the before
called name column.  The values of these columns will be gathered into a new column.
As before, all remaining columns are considered to be keys.

In practical applications, where not all remaining columns are keys, one must aggregate
rows beforehand.

\begin{tablebox}[label=tab:pivot]{Pivoting example.}
  \begin{minipage}{0.45\textwidth}
    \centering
    \rowcolors{2}{black!10!white}{}
    \begin{tabular}{ccc}
      \toprule
      \textbf{name} & \textbf{year} & \textbf{value} \\
      \midrule
      A & 2019 & 1 \\
      A & 2020 & 2 \\
      A & 2021 & 3 \\
      B & 2019 & 4 \\
      B & 2020 & 5 \\
      B & 2021 & 6 \\
      \bottomrule
    \end{tabular}
  \end{minipage}
  \hfill
  \begin{minipage}{0.45\textwidth}
    \centering
    \rowcolors{2}{black!10!white}{}
    \begin{tabular}{cccc}
      \toprule
      \code{name} & \code{2019} & \code{2020} & \code{2021} \\
      \midrule
      A & 1 & 2 & 3 \\
      B & 4 & 5 & 6 \\
      \bottomrule
    \end{tabular}
  \end{minipage}
  \tcblower
  The left table is in the long format and the right table is in the wide format.  The
  name column is \code{year} and the value column is \code{value}.
\end{tablebox}

\Cref{tab:pivot} shows an example of pivoting.  The left table is in the long format and
the right table is in the wide format.  The name column is \code{year}, the value column
is \code{value}, and the remaining column is \code{name} which is an unique identifier
of the rows in the wide format.

\begin{hlbox}{Practical tips}
  \begin{itemize}
    \item Use pivoting to transform datasets from a long format to a wide format;
    \item Use unpivoting to transform datasets from a wide format to a long format;
    \item Be aware of the need to aggregate rows before unpivoting.
  \end{itemize}
\end{hlbox}

\section{An algebra for statistical transformations}

In recent years, some researchers made an effort to create a formal algebra for
statistical transformations.  The idea is to create a set of operations that can be
combined to create complex statistical transformations.  This is similar to the idea of
relational algebra, which is a set of operations that can be combined to create complex
queries.

The difference between relational algebra and a formal algebra for statistical
transformations is that the latter is more complex.  This is because statistical
transformations are more complex than queries.  For instance, the concept of missing data
is not present in relational algebra, but it is in statistical transformations.

\textcite{Song2021}, for example, propose a formal paradigm for statistical data
transformation.  They present a data model, an algebra, and a formal language.  Their goal
is to create a standard for statistical data transformation that can be used by different
statistical software.

However, in my opinion, the major deficiency of their work is that they mostly try to
``reverse engineer'' the operations that are commonly used in statistical software.  This
is useful for the translation of code between different software, but it is not productive
to advance in the theoretical understanding of statistical transformations.

If one ought to tackle the challenge of formally expressing statistical transformations, I
think one should start from the basic operations.  Basic operations mean that they are
irreducible, i.e., they cannot be expressed as a sequence of other operations.

Some thoughts about it:
\begin{itemize}
  \item Binding columns can be expressed as a join operation, thus it is not a basic
    operation.
  \item Some software provide features that can be better expressed in other (often simpler) ways.  Row
    naming is an example.  It is useful to keep track of the origin of each row, but names
    can be just another column.  I argue for excluding row naming in a formal algebra.
  \item Some operations are very useful and recurring, even if they are not basic.  Such
    operations must be omitted from the formal algebra for the sake of simplicity.
    However, any software that implements a language for the formal algebra can provide
    syntax sugar for these operations.
  \item Not defining your algebra in terms of a specific programming language is a good
    practice.  This is because the algebra is a theoretical concept and should be
    independent of any implementation.  It also gives opportunities to rethink the
    things that commonly done in a specific way.  This can lead to new insights and
    correct error-prone practices.
  \item Pivoting seems to be ``different'' enough to the other operations to be considered
    in the set of basic operations.  However, it is not hard to see that they can be
    rewritten as combinations with the meta tables containing the possible values of the
    attributes (or some sort of aggregation function).
\end{itemize}

% vim: spell spelllang=en

\chapter{Learning from data}
\label{chap:slt}
\glsresetall

\chapterprecishere{%
  To  understand  God's  thoughts  we  must study statistics, for these are the measure of His purpose.
  \par\raggedleft--- \textup{Florence Nightingale}, her diary}

As we discussed before, in this book, I focus on the problem of inferring a solution for a
predictive task from data.  In this chapter, we introduce the basic concepts of the
\gls{slt}, a general framework for predictive learning tasks.

More specifically, we discuss the \emph{inductive learning} approach, which involves
deriving general rules from specific observations.

We also formally establish the learning problem, and we define the two most common
predictive tasks: binary data classification and regression estimation.  We discuss the
optimal solutions for these tasks in an ideal (although unrealistic) scenario where the
distributions of the data are known.

Moreover, we discuss principles that guide the learning process when the distribution of
the data is unknown.  From those principles, we discuss the properties and limitations of
the learning process.

Finally, we realize those concepts for simple linear problems, explaining two basic
algorithms for the learning process: the perceptron and the maximal margin classifier.
% TODO: if regression is shown, update here with adaline and svr

\begin{mainbox}{Chapter remarks}

  \boxsubtitle{Contents}

  \startcontents[chapters]
  \printcontents[chapters]{}{1}{}
  \vspace{1em}

  \boxsubtitle{Context}

  \begin{itemize}
    \itemsep0em
    \item Inductive reasoning is the process of deriving general rules from specific
      observations.
  \end{itemize}

  \boxsubtitle{Objectives}

  \begin{itemize}
    \itemsep0em
    \item Define the learning problem and the common predictive tasks.
    \item Understand the main principles that guide the learning process.
  \end{itemize}

  \boxsubtitle{Takeways}

  \begin{itemize}
    \itemsep0em
    \item Optimal solutions establish how good a solution can possibly be.
    \item Reducing erro is not enough to guarantee a good solution.
    \item Controlling the complexity of the model is crucial to generalize.
  \end{itemize}
\end{mainbox}

{}
\clearpage

\section{Introduction} % TODO: I don't like this title

Several problems can be addressed by techniques that use data somehow.  Once we focus on
one particular problem --- inductive learning ---, we need to define the scope of the
tasks we are interested in.  Let us start from the broader fields to the more specific
ones.

\Gls{ai} is a very broad field, including not only the study of algorithms
that exhibit intelligent behavior, but also the study of the behavior of intelligent
systems.  For instance, it encompasses the study of optimization methods, bioinspired algorithms,
robotics, philosophy of mind, and many other topics.  We are interested in the subfield of
artificial intelligence that studies algorithms that exhibit some form of intelligent
behavior.

A more specific subfield of \gls{ai} is \gls{ml}, which studies algorithms that
enable computers to automatically learn and improve their performance on a task from
experience, without being explicitly programmed by a human being.

Programming a computer to play chess is a good example of the difference between
traditional \gls{ai} and \gls{ml}.  In traditional \gls{ai}, a human programmer
would write a program that contains the rules of chess and the strategies to play the game.
The algorithm might even ``search'' among the possible moves to find the best one.  In
\gls{ml}, the programmer would write a program that learns to play chess by playing
against itself, against other programs, or even from watching games played by humans.
The system would learn the rules of chess and the strategies to play the game by itself.

This field is particularly useful when the task is too complex to be solved by
traditional programming methods or when we do not know how to solve the task.
Among the many tasks that can be addressed by \gls{ml}, we can specialize even more.

Predictive learning is the \gls{ml} paradigm that focuses on making predictions about
outcomes (sometimes about the future) based on historical data.  Predictive tasks
involve predicting the value of a target variable based on the values of one or more
input variables\footnote{Descriptive learning, which is out of the scope of this book,
focuses on describing the relationships between variables in the data without the
need of a target variable.}.

Depending on the reasoning behind the learning algorithms, we can divide the learning
field into two main approaches: \emph{inductive learning} and \emph{transductive
learning}\footnote{Trasduction is the process obtaining specific knowledge from specific
observation, and it is not the focus of this book.}.

Inductive learning involves deriving general rules from specific observations.  The
general rules can make predictions about \emph{any} new instances.  Such an approach
is exactly what we want to apply the project methodology we described in
\cref{sec:our-approach}:  the solution is the general rule inferred from the data.

\begin{figurebox}[label=fig:learning]{Organizational chart of the learning field.}
  \centering
  \begin{tikzpicture}
    \draw[outline] (0,0) circle (30mm) node {};
    \node[below] at (0, 2.6) {artificial intelligence};
    \draw[outline] (0,-0.5) circle (25mm) node {};
    \node[below] at (0, 1.6) {machine learning};
    \draw[outline] (0,-1) circle (20mm) node {};
    \node[below] at (0, 0.5) {predictive learning};
    \draw[outline] (0,-1.5) circle (15mm) node {};
    \node[below] at (0, -1.0) {inductive learning};
  \end{tikzpicture}
  \tcblower
  Artificial intelligence studies algorithms that exhibit intelligent behavior and the
  behavior of intelligent systems.  Machine learning is a subfield of artificial
  intelligence that studies algorithms that enable computers to automatically learn from
  data.  Predictive learning which focuses on making predictions about outcomes given
  known input data.  Inductive learning is a yet more specific type of learning that
  involves deriving general rules from specific observations.
\end{figurebox}

\Cref{fig:learning} give us a hierarchical view of the learning field.  Alternatives ---
such as descriptive learning in opposition to predictive learning, or transductive
learning in opposition to inductive learning --- are out of the scope of this book.

Maybe the most general (and useful) framework for predictive learning is \gls{slt}.
In this chapter, we will introduce the basic concepts of this theory and discuss the
properties of the main \gls{ml} methods.

\section{The learning problem}

Consider the set
\begin{equation}
  \label{eq:training-set}
  \big\{(\vec{x}_i, y_i) : i = 1, \dots, n \big\}
\end{equation}
where each sample $i$ is associated with a feature vector $\vec{x}_i \in \mathcal{X}$ and a target variable
$y_i \in \mathcal{Y}$.  We assume that samples are random independent identically
distributed (i.i.d.) observations drawn according to $$\Prob(x, y) = \Prob(y \mid x) \Prob(x)\text{.}$$
Both distributions $\Prob(x)$ and $\Prob(y \mid x)$ are fixed but unknown.

This is equivalent to the original \gls{slt} setup stated by \textcite{Vapnik1999b}, where
a generator produce random vectors $\vec{x}$ according to a fixed but unknown
probability distribution $\Prob(x)$ and a supervisor returns an output value $y$ for every
input vector $x$ according to a conditional distribution function $\Prob(y \mid x)$, also fixed but
unknown.

Moreover, note that this setup is compatible with the idea of tidy data and 3NF (see
\cref{sub:bridge}). Of course, we assume $X, Y$ are only the measured variables (or
non-prime attributes).  In practice, it means that we left aside the keys in the learning
process.

In terms of the tables defined in \cref{sec:formal-structured-data}, any row $r$ in the
table $T = (K, H, c)$, in the desired observational unit, such that $\rowcard[r] > 0$, and
$h \in H$ the chosen target variable, we have a corresponding target $y = c(r, h)$ and a
feature vector $\vec{x}$ corresponds to the tuple $$\big(c(r, h') : h' \in H \setminus
\left\{ h \right\}\big)\text{.}$$  Similarly, the variables $K$ that describe each unit
are left aside, as it does not make sense to infer general rules from them.

From the statistical point-of-view, learning problems consist of answering questions about
the distribution of the data.

\subsection{Learning tasks}

In terms of predictive learning, given de before-mentioned scenario, we can refine our
goals by tackling specific tasks\footnote{I consider tasks as well-defined subproblems of
a higher-level problem.}.

Consider a \emph{learning machine} capable of generating a set of functions, or
\emph{models}, $f(x; \theta) \equiv f_\theta(x)$, for a set of parametrizations $\theta
\in \Theta$ and such that $f_\theta : \mathcal{X} \rightarrow \mathcal{Y}$.  In a learning
task, we must choose, among all possible $f_\theta$, the one that predicts the target
variable the best possible way.

In order to learn, we must first define the \emph{loss} (or discrepancy) $\mathcal{L}$
between the response $y$ to a given input $x$, drawn from $\Prob(x, y)$, and the
response provided by the learned function.

Then, given the \emph{risk function}
\begin{equation}
  \label{eq:risk}
  R(\theta) = \int \mathcal{L}(y, f_\theta(x))\, d\!\Prob(x, y)\text{,}
\end{equation}
the goal is to find the function $f_\theta$ that minimizes $R(\theta)$
where the only available information is the \emph{training set} given by \eqref{eq:training-set}.

This formulation encompasses many specific tasks. I focus on the two of them which I
believe are the most fundamental ones: \emph{binary data classification}\footnote{In
\gls{slt}, Vapnik calls it \emph{pattern recognition}.} and \emph{regression
estimation}\footnote{We are not talking about \emph{regression analysis}; regression
estimation is closer to the \emph{scoring} task definition by \fullcite{Zumel2019}.}.  (I
left aside the density estimation problem, once it is not addressed in the remaining of
the book.)

\subsubsection{Binary data classification task}

In this task, the output $y$ takes on
only two possible values, zero or one\footnote{Alternatively, negative class is
represented by $-1$ and positive class by $1$.} --- called the negative and the positive
class, respectively ---, and the functions $f_\theta$ are indicator
functions. Choosing the loss
\begin{equation*}
  \mathcal{L}(y, f_\theta(x)) = \begin{cases}
    0 & \text{if } y = f_\theta(x) \\
    1 & \text{if } y \neq f_\theta(x)\text{,}
  \end{cases}
\end{equation*}
the risk $\eqref{eq:risk}$ becomes the probability of
classification error.  The function $f_\theta$, in this case, is called \emph{classifier}
and $y$ is called the \emph{label}.

\subsubsection{Regression estimation task}

In this task, the output $y$ is a real value and the functions $f_\theta$ are real-valued
functions.  The loss function is the squared error
\[
  \mathcal{L}(y, f_\theta(x)) = \big(y - f_\theta(x)\big)^2\text{.}
\]
In \cref{sec:optimal-solution}, we show that the function that minimizes the risk with such
loss function is the so-called \emph{regression}.
The estimator $f_\theta$ of the regression, in this case, is called a \emph{regressor}.

\subsection{A few remarks}

These two tasks are pretty general and can be applied to a wide range of problems.  The
modeling of the task at hand and choice of the loss function is crucial to the success of
the learning process.

About these learning tasks, we can make a few remarks.

\paragraph{Supervised and semisupervised learning}
In both cases, classification and regression estimation, the learning task is to find the function
that maps the input data to the output data in the best possible way.  Although the
learning machine described generate models in a \emph{supervised} manner --- i.e. target
is known for all samples in the training set ---, there are
alternative ways to solve the inductive learning problem, such as the \emph{semisupervised}
approach, where the model can be trained with a small subset of labeled data and a large
subset of unlabeled data --- that is, data whose outputs $y$ are unknown.

\paragraph{Generative and discriminative models}
Any learning machine generates a model that describes the relationship between the input
and output data.  This model can be generative or discriminative.  Generative models
describe the joint probability distribution $\Prob(x, y)$ and can also be used to generate new
data.  Discriminative models, on the other hand, describe the conditional probability
distribution $\Prob(y \mid x)$ directly and can only be used to make predictions. Generative models are
usually much more complex than discriminative models\footnote{Since modeling $\Prob(x, y)$
indirectly models $\Prob(y \mid x)$ and $\Prob(x)$.}, but they hold more information about
the data.  If you only
need to solve the predictive problem, prefer a discriminative model.

\paragraph{Multiclass classification}
In the binary classification task, the output $y$ is
a binary variable.  However, it is possible to have a multiclass classification task,
where  $y$ can take on more than two possible values.  Although some learning methods can
address directly the multiclass classification task, it is possible to transform the
problem into a binary classification task.  The most common method is the
\emph{one-versus-all} method where we train $l$ binary classifiers, one for each class,
and the class with the highest score is the predicted class.  Another method is the
\emph{one-versus-one} method, where we train $l(l-1)/2$ binary classifiers, one for each
pair of classes, and the class with the most votes is the predicted class.
As one should expected, dealing with more than two classes is more complex than dealing
with only two classes.  If possible, prefer to deal with binary classification tasks first.

\paragraph{Number of inputs and outputs}
Note that the definition of the learning problem does not restrict the number of inputs
and outputs.  The input data can be a scalar, a vector, a matrix, or a tensor, and the
output as well.  The learning machine must be able to handle the input and output data
according to the problem.

% TODO: when discussing bias
% \paragraph{Parametric vs nonparametric models}
% The learning machine generates a set of functions $f_\theta$ where $|\theta|$ can be fixed
% or not.  If $|\theta|$ is always fixed, the model is called \emph{parametric}.  If
% $|\theta|$ is not fixed beforehand, the model is called \emph{nonparametric}.  Parametric
% models are usually simpler and faster, but they are less flexible.  In other words, it is
% up to the researcher to choose the best model ``size'' for the problem.  If the model is
% too small, it will not be able to capture the complexity of the data.  If the model is too
% large, it will be too complex, too slow to train and might overfit to the data.
% Nonparametric models are more flexible, but they usually require more data to be trained.

\section{Optimal solutions}
\label{sec:optimal-solution}

In this section, I show that the optimal solutions for the tasks of binary data
classification and regression estimation depend only on $\Prob(y \mid x)$ (i.e.
discriminative models).  This is useful
to understand how good a solution can possibly be and to derive practical solutions in the
next sections.

\subsection{Bayes classifier}

The optimal solution for the binary data classification task is the \emph{Bayes
classifier}, which minimizes the probability of classification error.  The Bayes
classifier is defined as
\begin{equation*}
  f_\text{Bayes}(x) = \argmax_{y \in \mathcal{Y}} \Prob(y \mid x)\text{.}
\end{equation*}

We can easily see that the Bayes classifier is the optimal solution for the binary data
classification task.  The probability of classification error for an arbitrary classifier
$f$ is
\begin{equation*}
  R(f) = \int \mathbb{1}_{f(x) \neq y}\, d\!\Prob(x, y) =
    \iint \mathbb{1}_{f(x) \neq y}\, d\!\Prob(y | x)\, d\!\Prob(x)\text{,}
\end{equation*}
where $\mathbb{1}_{\cdot}$ is the indicator function that returns one if the condition is
true and zero otherwise.  Let $b(x) = \Prob(y = 1 \mid x)$, we have that
\[
  \int \mathbb{1}_{f(x) \neq y}\, d\!\Prob(y | x) =
    b(x) \mathbb{1}_{f(x) = 0} + (1 - b(x)) \mathbb{1}_{f(x) = 1}\text{,}
\]
which only of the terms is nonzero for each $x$.  Thus, the risk is minimized by choosing
a classifier that $f(x) = 1$ if $b(x) > 1 - b(x)$ and $f(x) = 0$ otherwise.  This is the
Bayes classifier.

Consequently, the \emph{Bayes error rate}, or irreducible error, is the lowest possible loss for any
classifier in a given problem.  The Bayes error rate sums the errors of the
Bayes classifier for each class:
\begin{equation*}
  R_\text{Bayes} = \int \left[
    b(x) \mathbb{1}_{f_\text{Bayes}(x) = 0} +
    \left(1 - b(x)\right) \mathbb{1}_{f_\text{Bayes}(x) = 1}
  \right] d\!\Prob(x)\text{.}
\end{equation*}

We know that $f_\text{Bayes}(x) = 1$ if $b(x) > 0.5$ and $f_\text{Bayes}(x) = 0$ otherwise.
Thus, the Bayes error rate can be rewritten as
\begin{equation*}
  R_\text{Bayes} = \int \min\left\{ b(x), 1 - b(x) \right\}\, d\!\Prob(x)\text{.}
\end{equation*}

\begin{figurebox}[label=fig:bayes-classifier]{Bayes classifier illustration.}
  \centering
  \begin{tikzpicture}
    \begin{axis}[
        ticks=none,
        axis x line=bottom,
        axis y line=left,
        xlabel={$x$},
        ymax=0.4,
        xmin=-2.2, xmax=1.4,
      ]
      % P(x|y=0)
      \addplot+[fill=gray, draw=black, opacity=0.2, smooth, mark=none] coordinates {
        (-2, 0.1) (-1.5, 0.2) (-1, 0.35) (-0.5, 0.2) (0, 0.1)
      };
      \node at (axis cs:-1, 0.2) {$\Prob(x \mid y = 0)$};
      % P(x|y=1)
      \addplot+[fill=gray, draw=black, opacity=0.4, smooth, mark=none] coordinates {
        (-0.8, 0.1) (-0.3, 0.2) (0.2, 0.35) (0.7, 0.2) (1.2, 0.1)
      };
      \node at (axis cs:0.2, 0.2) {$\Prob(x \mid y = 1)$};
      % Bayes
      \draw[dashed, gray] (axis cs:-0.3, 0) -- (axis cs:-0.3, 0.4);
    \end{axis}
  \end{tikzpicture}
  \tcblower
  The Bayes classifier is the line that separates the two classes.  The Bayes error is a
  result of the darker area in which the distributions of the classes intersect.
\end{figurebox}

\Cref{fig:bayes-classifier} illustrates the Bayes classifier and its error rate.
The vertical line represents the Bayes classifier that separates the classes the best way
possible in the space of the feature vectors $x$.  Since the distributions $\Prob(x \mid y
= 0)$ and $\Prob(x \mid y = 1)$ may intersect, there is a region where the Bayes
classifier cannot always predict the class correctly.

\subsection{Regression function}
\label{sec:regression-function}

In the regression estimation task, the goal is to approximate the optimal solution, called
\emph{regression function},
\begin{equation}
  \label{eq:regression-function}
  r(x) = \int y\, d\!\Prob(y \mid x)\text{,}
\end{equation}
that is the expected value of the target variable $y$ given the input $x$.

It is easy to show that the regression function minimizes the risk \eqref{eq:risk} with
loss
\[
  \mathcal{L}(y, r(x)) = \big(y - r(x)\big)^2\text{.}
\]
The risk functional for an arbitrary function $f$ is
\begin{multline*}
  R(f) =
    \int \big(y - f(x)\big)^2\, d\!\Prob(x, y) =\\
    \int y^2\, d\!\Prob(y) -
    2 \int f(x) \left[ \int y\, d\!\Prob(y \mid x) \right] d\!\Prob(x) +
    \int f(x)^2\, d\!\Prob(x)\text{,}
\end{multline*}
however we can substitute $r(x)$ for the inner integral and obtain
\begin{multline*}
  R(f) =
    \int y^2\, d\!\Prob(y) - 2 \int f(x) r(x)\, d\!\Prob(x) + \int f(x)^2\, d\!\Prob(x) = \\
    \int y^2\, d\!\Prob(y) + \int \left[ f(x)^2 - 2 f(x) r(x) \right]\, d\!\Prob(x)\text{.}
\end{multline*}
Once the first term is a constant, the risk is minimized by minimizing
\[
  f(x)^2 - 2 f(x) r(x)\text{.}
\]

Deriving the last expression with respect to $f(x)$ and setting it to zero, we obtain
\begin{equation*}
  \frac{d}{d f(x)} \Big[ f(x)^2 - 2 f(x) r(x) \Big] = 2 f(x) - 2 r(x) = 0 \Rightarrow
  f(x) = r(x)\text{.}
\end{equation*}

Like for the Bayes classifier, the stochastic nature of the data leads to an irreducible
error in the regression estimation task.  We have that
\begin{equation*}
  R(r) = \int \big(y - r(x)\big)^2\, d\!\Prob(x, y) =
    \int y^2\, d\!\Prob(y) - \int r(x)^2\, d\!\Prob(x)\text{,}
\end{equation*}
where the first term is
\[
  \E\!\left[y^2\right] = \Var(y) + \E\!\left[y\right]^2
\]
and the second term is
\[
  \E\left[\E\!\left[y \mid x\right]^2\right] =
    \Var\!\left(\E\!\left[y \mid x\right]\right) + \E\!\left[\E\!\left[y \mid x\right]\right]^2 =
    \Var\!\left(\E\!\left[y \mid x\right]\right) + \E\!\left[y\right]^2\text{.}
\]
Thus, the irreducible error is
\[
  R(r) = \Var(y) - \Var\!\left(\E\!\left[y \mid x\right]\right) \text{.}
\]

The interpretation of the irreducible error comes from the law of the total variance:
\begin{equation*}
  \Var(y) = \E\!\left[ \Var(y \mid x) \right] + \Var\!\left(\E\!\left[y \mid x\right]\right)\text{,}
\end{equation*}
where the first term is known as the unexplained variance and the second term, as the
explained variance.  The equality $R(r) = \E\!\left[ \Var(y \mid x) \right]$
captures the idea that this variance is the intrinsic uncertainty that cannot be further
reduced.

\begin{figurebox}[label=fig:explained-unexplained-variance]{Unexplained variance is the
  error of the regression.}
  \centering
  \begin{tikzpicture}
    \begin{axis}[
        axis lines=middle,
        xlabel={$x$},
        ylabel={$y$},
        xmin=-0.2, xmax=1.5,
        ymin=-0.5, ymax=1.8,
        xtick={0, 0.5, 1},
        ytick={0, 0.5, 1},
        domain=0:1]

      \addplot[thick] {x} node[right] {$r(x) = x$};

      \addplot [draw=none, fill=gray, opacity=0.2] {x + 1} \closedcycle;
      \addplot [draw=none, fill=gray, opacity=0.2] {x - 1} \closedcycle;
      \draw[Stealth-Stealth, gray] (axis cs:0.5,0.5) -- (axis cs:0.5, 1.5) node[midway, right] {$\sigma = 1$};
      \node at (axis cs:0.5,1.3) [anchor=west] {Unexplained variance};

     \addplot[samples=2, style={dashed}] {0.5} node[midway, below, anchor=north west] {Explained variance};
    \end{axis}
  \end{tikzpicture}
  \tcblower
  Expected error of the regression function for data generated by
  $\Prob(y \mid x) \Prob(x)$ such that $\Prob(y \mid x) = \mathcal{N}(x, 1)$ and
  $\Prob(x) = \mathcal{U}(0, 1)$.  The regression function is $r(x) = x$.
\end{figurebox}

\Cref{fig:explained-unexplained-variance} illustrates the irreducible error of the
regression function for arbitrary distributions $\Prob(y \mid x)$ and $\Prob(x)$.
In this case $\E\!\left[ \Var(y \mid x) \right] = 1$, which means that points drawn
by $\Prob(x, y)$ are distributed around the regression function $r(x)$ with a standard
deviation of one.  The explained variance is the spread of the regression across the
x-domain.

\section{ERM inductive principle}

It is very interesting to study the optimal solution for the learning tasks, but in the
real-world, we do not have access to the distributions $\Prob(x)$ and $\Prob(y \mid x)$.
We must rely on the training data $(x_1, y_1), \dots, (x_n, y_n)$ to infer a solution.

In the following sections, for the sake of simplicity, let $z$ describe the pair $(x, y)$
and $L(z, \theta)$, a generic loss function for the model $f_\theta$.  Note that the
training dataset is thus a set of $n$ i.i.d. samples $z_1, \dots, z_n$.

Since the distribution $\Prob(z)$ is unknown, the risk functional $R(\theta)$ is replaced by
the \emph{empirical risk functional}
\begin{equation}
  \label{eq:empirical-risk}
  R_n(\theta) = \frac{1}{n} \sum_{i=1}^n L(z_i, \theta)\text{.}
\end{equation}

Approximating $R(\theta)$ by the empirical risk functional $R_n(\theta)$ is the so called
\gls{erm} inductive principle.  The ERM principle is the basis of the \gls{slt}.

Traditional methods, such as least-squares, maximum likelihood, and maximum a posteriori are
all realizations of the ERM principle for specific loss functions and hypothesis spaces.

\subsection{Consistency of the learning process}

One important question about the ERM principle is the consistency of the learning process.
Consistency means that, given a sufficient number of samples, the empirical risk
functional $R_n(\theta)$ converges to the true risk functional $R(\theta)$ over the
hypothesis space $\Theta$.

The consistency of the ERM principle is guaranteed by the uniform (two-sided)
convergence\footnote{ Actually, only a weaker one-sided uniform is needed; consult a
detailed explanation in chapter 2 of \fullcite{Vapnik1999b}.  The equivalence is a
consequence of the key theorem of learning proved by Vapnik and Chervonenkis in 1989 and
later translated to English in \fullcite{Vapnik1991}.} of the empirical risk functional
$R_n(\theta)$ to the true risk functional $R(\theta)$ over the hypothesis space $\Theta$.
The uniform convergence is defined as
\[
  \lim_{n \to \infty} \Prob\!\left(
    \sup_{\theta \in \Theta} \Big| R_n(\theta) - R(\theta_{n_0}) \Big| > \epsilon
  \right) = 0\text{.}
\]

\subsection{Rate of convergence}

Beyond consistency, it is also useful to understand the rate at which $R_n(\theta)$
converges to $R(\theta)$ as the sample size $n$ increases.  It is possible for a learning
machine be consistent but have a slow convergence rate, which means that a large number of
samples is needed to achieve a good solution.

The asymptotic rate of convergence of the empirical risk functional $R_n(\theta)$ is
fast if, for any $n > n_0$, the exponential bound
\[
  \Prob\!\big(R(\theta_n) - R(\theta) > \epsilon\big) < \exp\!\left( - c n \epsilon^2 \right)
\]
holds true, where $c$ is a positive constant.

\subsection{VC entropy}

Let $L(z, \theta)$, $\theta \in \Theta$, be a set of bounded loss functions, i.e.
\[
  \left| L(z, \theta) \right| < M\text{,}
\]
for some constant $M$ and all $z$ and $\theta$.  One can construct $n$-dimensional vectors
\[
  l(z_1, \dots, z_n; \theta) = \big[ L(z_1, \theta), \dots, L(z_n, \theta) \big]\text{.}
\]
Once the loss functions are bounded, this set of vectors belongs to a $n$-dimensional
cube and has finite minimal $\epsilon$-net\footnote{An $\epsilon$-net is a set of points
that are $\epsilon$-close to any point in the set.}.

Consider the quantity $N(z_1, \dots, z_n; \Theta, \epsilon)$ that counts the number of
elements of the minimal $\epsilon$-net of that set of vectors.
Once the quantity $N$ is a random variable, we can define the VC entropy as
\[
  H(n; \Theta, \epsilon) = \E\!\left[ \ln N(z_1, \dots, z_n; \Theta, \epsilon) \right]\text{,}
\]
where $z_i$ are i.i.d. samples drawn from some $\Prob(z)$.

If $L(z, \theta)$, $\theta \in \Theta$ is a set of indicator functions (binary
classification task),  we measure the diversity of this set using the quantity $N(z_1,
\dots, z_n; \Theta)$ that counts the number of different separations of the given sample
can be made by the functions.  In this case, the minimal $\epsilon$-net for $\epsilon$ < 1
does not depend on $\epsilon$ and is a subset of the vertices of the unit cube.

A necessary and sufficient condition for the uniform convergence is
\begin{equation}
  \label{eq:uniform-convergence}
  \lim_{n \to \infty} \frac{H(n; \Theta, \epsilon)}{n} = 0\text{,}
\end{equation}
for all $\epsilon > 0$.

The VC entropy measures the complexity of the hypothesis space $\Theta$.  The intuition
behind the need of a decreasing VC entropy with increasing numbers of observations is
related to the nonfalsifiability of the learning machine.  For instance, the set of
functions that can always separate the training data perfectly (contains all the vertices
of the cube) is nonfalsifiable because
it implies that the minimum of the empirical risk is zero independently of the value of
the true risk.

\subsection{Growing function and VC dimension}

It turns out that we can guarantee both the uniform convergence and the fast rate of
convergence independently of $\Prob(z)$.  Actually,
\[
  \lim_{n \to \infty} \frac{G(n; \Theta)}{n} = 0
\]
is the necessary and sufficient condition, where
\[
  G(n; \Theta) = \ln \sup_{z_1, \dots, z_n} N(z_1, \dots, z_n; \Theta)\text{,}
\]
is the \emph{growth function} of the hypothesis space $\Theta$.

\citeauthor{Vapnik1968}\footfullcite{Vapnik1968} showed that the growth function either
satisfies
\[
  G(n; \Theta) = n \ln 2
\]
or is bounded by
\[
  G(n; \Theta) \leq h \left( \ln \frac{l}{n} + 1 \right)\text{,}
\]
where $h$ is an integer.  Thus, the growth function is either linear or logarithmic in
$n$.  In the first case, we say that the \emph{VC dimension} of the hypothesis space is
infinite, and in the second case, the VC dimension is $h$.

A finite VC dimension is enough to imply both consistency and a fast rate of convergence.

\subsubsection{Intuitions about the VC dimension}
\label{sub:vc-dimension-intuition}

% A figure with lines separating separating points in a plane
\begin{figurebox}[label=fig:vc-dimension]{VC dimension of a set of lines in the plane.}
  \centering
  \begin{tikzpicture}
    \begin{axis}[
        axis lines=middle,
        xmin=-1, xmax=1,
        ymin=-1, ymax=1,
        xtick={0},
        ytick={0},
        domain=-1:1]

      \addplot[only marks, mark=*, mark size=2pt] coordinates {
        (-0.5, 0.3) (-0.3, -0.5) (0.7, -0.5)
      };

      % diagonal lines separating points
      \addplot[dashed] {x} node[right] {$\theta_1$};
      \addplot[dashed] {-x} node[right] {$\theta_2$};
      \addplot[dashed] {-0.3} node[right] {$\theta_3$};

    \end{axis}
  \end{tikzpicture}
  \tcblower
  The VC dimension of the lines in the plane is equal to 3, since a line can shatter 3
  points in all 8 possible ways, but not four points.
\end{figurebox}

For a set of indicator functions, the VC dimension is the maximum number of vectors that
can be \emph{shattered} by the functions.  If, for any $n$, there is a set of $n$
vectors that can be shattered by the functions, the VC dimension is infinite.
We say that $h$ vectors can be shattered if they can be separated into two classes in all
$2^h$ possible ways.  \Cref{fig:vc-dimension} illustrates the VC dimension of a set of
lines in the plane.

\begin{figurebox}[label=fig:vc-sin]{High-frequency sine wave functions have infinite VC dimension.}
  \centering
  \begin{tikzpicture}
    \begin{axis}[
        axis lines=middle,
        xlabel={$x$},
        ylabel={$y$},
        xmin=-1.2, xmax=1.2,
        ymin=-1.2, ymax=1.2,
        xtick={0},
        ytick={0},
        domain=-1:1,
        samples=100]

      \addplot[smooth] {sin(deg(24 * x))};
      \addplot[dashed, smooth] {sin(deg(4 * x))};

      \addplot[only marks, mark=*, mark size=2pt] coordinates {
        (-0.942, 0.586) (-0.562, -0.780) (-0.337, -0.976)
        (0.313, 0.950) (0.562, 0.780) (0.942, -0.586)
      };
    \end{axis}
  \end{tikzpicture}
  \tcblower
  High-frequency sine wave approximate well any set of points even though they may come
  from a low-frequency sine wave or any other function.
\end{figurebox}

One misconception about the VC dimension is that it is related to the number of parameters
of the model.  The VC dimension is related to the complexity of the hypothesis space, not
to the number of parameters.  For instance, the VC dimension of functions
\[
  f(z; \theta) = \mathbb{1}_{\sin \theta x > 0}
\]
is infinite, even though the parameter $\theta$ is a scalar.  See \cref{fig:vc-sin}.
By increasing the frequency $\theta$ of the sine wave, the function can approximate any set
of points.

This opens remarkable opportunities to find good solutions containing a huge number of
parameters\footnote{Sometimes, like in a linear model, the number of parameters is
proportional to the number of dimensions of the feature vector.} but with a finite VC
dimension.  % TODO: discuss non-parametric here?

\section{SRM inductive principle}

The \gls{erm} principle is a powerful tool to study the generalization ability of the
learning process.  By generalization ability, we mean the ability of the learning machine
to predict the output of new data that was not seen during the training process.  However,
it relies on the hypothesis that the number of samples tends to infinity.

In fact, \citeauthor{Vapnik1999b}\footfullcite{Vapnik1999b} summarizes the bounds for the
generalization ability of learning machines in the following way\footnote{ For the sake of
the arguments, we consider only the expression for bounded losses and an hypothesis space
with infinite number of functions.  Rigorously, the loss function may not be bounded;
consult the original work for the complete expressions.}:
\begin{equation}
  \label{eq:generalization-bound}
  R(\theta_n) \leq R_n(\theta_n) + \frac{B \mathcal{E}}{2} \left(
    1 + \sqrt{1 + \frac{4 R_n(\theta_n)}{B \mathcal{E}}}
  \right)\text{,}
\end{equation}
with
\[
  \mathcal{E} = 4 \frac{
    h \left( \ln \frac{2 n}{h} + 1 \right) - \ln \frac{\eta}{4}
  }{n}\text{,}
\]
where $B$ is the upper bound of the loss function, $h$ is the VC dimension of the
hypothesis space, $n$ is the number of samples.  The term $\eta$ is the confidence level,
i.e. the inequality holds with probability $1 - \eta$.

It is easy to see that as the number of samples $n$ increases, the empirical risk
$R_n(\theta_n)$ approaches the true risk $R(\theta_n)$.  Also, the greater the VC
dimension $h$, the greater the term $\mathcal{E}$, decreasing the generalization ability
of the learning machine.

In other words, if $\nicefrac{n}{h}$ is small, a small empirical risk does not guarantee a small value
for the actual risk.  A consequence is that we need to minimize both terms of the
right-hand side of the inequality \cref{eq:generalization-bound} to achieve a good
generalization ability.

% A table comparing overfit and underfit and the empirical risk and confidence interval
\begin{tablebox}[label=tab:overfit-underfit]{Overfitting and underfitting.}
  \centering
  \begin{tabular}{lll}
    \toprule
    \textbf{Problem} & \textbf{Empirical risk} & \textbf{Confidence interval} \\
    \midrule
    Underfitting & High & Low \\
    Overfitting & Low & High \\
    \bottomrule
  \end{tabular}
  \tcblower
  Two problems that can arise in the learning process are underfitting and overfitting.
  Underfitting occurs when the model is too simple (low VC dimension) and cannot capture
  the complexity of the training data (high empirical risk).  Overfitting occurs when the
  model is too complex (high VC dimension increases the confidence interval) and fits the
  training data almost perfectly (low empirical risk).
\end{tablebox}

Failure to balance the optimization of these terms leads to two problems: underfitting and
overfitting.  \Cref{tab:overfit-underfit} summarizes the problems.

The \gls{srm} principle consists of minimizing both the empirical risk (optimizing the
parameters of the model) and the confidence interval (controlling VC dimension).

Let $\Theta_k \subset \Theta$ and
\[
  S_k = \left\{ L(z, \theta) : \theta \in \Theta_k \right\}
\]
such that
\[
  S_1 \subset S_2 \subset \dots \subset S_n \subset \dots\text{,}
\]
satisfying
\[
  h_1 \leq h_2 \leq \dots \leq h_n \leq \dots\text{,}
\]
where $h_k$ is the finite VC dimension\footnote{Note that the VC dimension considering the
whole set $\Theta$ might be infinite.  Moreover, in the original formulation, the sets
$S_k$ also need to satisfy some bounds; read more in chapter 4 of \fullcite{Vapnik1999b}.}
of each set $S_k$.  This is called an \emph{admissible structure}.

\begin{figurebox}[label=fig:srm-tradeoff]{SRM trade-off.}
  \centering
  \begin{tikzpicture}
    \begin{axis}[
        axis lines=middle,
        xlabel={$k$},
        ylabel={Risk},
        ytick={0},
        yticklabels={},
        xtick={0.1, 0.5, 1},
        xticklabels={$h_1$, $h^*$, $h_n$},
        grid=both,
        xmin=0, xmax=1.5,
        ymin=0, ymax=1.5,
        domain=0.1:1]

      \addplot[smooth] {(x - 1)^2 + 0.1} node[right, text width=1cm] {Empirical risk};
      \addplot[smooth] {x^2 + 0.1} node[right, text width=2cm] {Confidence interval};
      \addplot[smooth, thick] {x^2 + (x - 1)^2 + 0.2} node[left, text width=2cm] {Risk upper bound};

    \end{axis}
  \end{tikzpicture}
  \tcblower
  The upper bound of the risk is the sum of the empirical risk and the confidence
  interval.  The smallest bound is found for some $k^*$ in the admissible structure.
\end{figurebox}

Given the observations $z_1, \dots, z_n$, the \gls{srm} principle chooses the function
$L(z, \theta_n^k)$ that minimizes the empirical risk $R_n(\theta_n^k)$ in the subset
$S_k$ for which the guaranteed risk --- upper bound considering the confidence interval
--- is minimal.  This is a trade-off between the quality of the approximation and the
complexity of the approximating function --- see \cref{fig:srm-tradeoff}.

\subsection{Bias invariance trade-off}

The trade-off that the \gls{srm} principle deals with is the general case of the so called
\emph{bias-variance trade-off}.  The bias-variance trade-off is a well-known concept in
machine learning that describes the relationship between different kinds of errors a model
can have.

The \emph{bias error} comes from failure to capture relevant relationship between features
and target outputs.  The \emph{variance error} comes from erroneously modeling the random
noise in the training data.

The terms bias and variance (and the irreducible error) are clearly illustrated by
studying the particular regression estimation task.

Consider a learning machine that produces a function $\hat{f}(x; D)$ based on the
training set $$D = \big\{(x_1, y_1), \dots, (x_n, y_n)\big\}$$ such that
\[
  y_i = f(x_i) + \epsilon\text{,}
\]
for a fixed function $f$ and a random noise $\epsilon$ with zero mean and variance
$\sigma^2$, where $x_i$ are i.i.d. samples drawn from some distribution $\Prob(x)$.

Also, consider that $\bar{f}(x)$ is the expected value of the function $\hat{f}(x; D)$
over all possible training sets $D$, i.e.
\[
  \bar{f}(x) = \int \hat{f}(x; D)\, d\!\Prob(D)\text{.}
\]
(Note that the models themselves are the random variable we are studying here.)

For any model $\hat{f}$, the expected (squared) error for a particular sample $(x, y)$,
$\E_D\!\left[ \big( y - \hat{f}(x; D) \big)^2 \right]$, is
\begin{align}
  \notag \int \big( y - \hat{f}(x) \big)^2\, d\!\Prob(D, \epsilon)
    &= \int \big( y - f(x) + f(x) - \hat{f}(x) \big)^2\, d\!\Prob(D, \epsilon) \\
    &= \label{eq:term1}\int \big( y - f(x) \big)^2\, d\!\Prob(D) \\
    &+ \label{eq:term2}\int \big( f(x) - \hat{f}(x) \big)^2\, d\!\Prob(D) \\
    &+ \label{eq:term3}2 \int \big( y - f(x) \big)\,\big( f(x) - \hat{f}(x) \big)\, d\!\Prob(D, \epsilon)\text{.}
\end{align}

The term \eqref{eq:term1} is the irreducible error:
\begin{equation}
  \label{eq:irreducible-error}
  \int \big( y - f(x) \big)^2\, d\!\Prob(D) =
  \int \big( f(x) + \epsilon - f(x) \big)^2\, d\!\Prob(D) =
  \int \epsilon^2\, d\!\Prob(D) = \sigma^2\text{.}
\end{equation}
As the best solution is $f$ itself, the error that comes from the noise is unavoidable.

The term \eqref{eq:term3} is null:
\begin{multline*}
  \int \big( y - f(x) \big)\,\big( f(x) - \hat{f}(x) \big)\, d\!\Prob(D, \epsilon) =
  \int \epsilon\,\big( f(x) - \hat{f}(x) \big)\, d\!\Prob(D, \epsilon) = \\
  \cancelto{0}{\int \epsilon\, d\!\Prob(\epsilon)} \int \big( f(x) - \hat{f}(x) \big)\, d\!\Prob(D) = 0\text{,}
\end{multline*}
since $\Prob(D)$ and $\Prob(\epsilon)$ are independent and $\E[\epsilon] = 0$ by
definition.

We can apply a similar strategy to analyse the term \eqref{eq:term2}:
\begin{align}
  \notag \int \big( f(x) - \hat{f}(x) \big)^2\, d\!\Prob(D)
    &= \int \big( f(x) - \bar{f}(x) + \bar{f}(x) - \hat{f}(x) \big)^2\, d\!\Prob(D) \\
    &= \label{eq:term4}\int \big( f(x) - \bar{f}(x) \big)^2\, d\!\Prob(D) \\
    &+ \label{eq:term5}\int \big( \bar{f}(x) - \hat{f}(x) \big)^2\, d\!\Prob(D) \\
    &+ \label{eq:term6}2 \int \big( f(x) - \bar{f}(x) \big)\,\big( \bar{f}(x) - \hat{f}(x) \big)\, d\!\Prob(D)\text{.}
\end{align}

Now, the term \eqref{eq:term6} is also null:
\begin{multline*}
  \int \big( f(x) - \bar{f}(x) \big)\,\big( \bar{f}(x) - \hat{f}(x; D) \big)\, d\!\Prob(D) = \\
  \big( f(x) - \bar{f}(x) \big) \int \big( \bar{f}(x) - \hat{f}(x; D) \big)\, d\!\Prob(D) = \\
  \big( f(x) - \bar{f}(x) \big) \cancelto{0}{\left( \bar{f}(x) - \int \hat{f}(x; D)\, d\!\Prob(D) \right)} = 0\text{,}
\end{multline*}
since $\bar{f}(x)$ is the expected value of $\hat{f}(x; D)$.

The term \eqref{eq:term4} does not depend on the training set, so
\begin{equation}
  \label{eq:bias}
  \int \big( f(x) - \bar{f}(x) \big)^2\, d\!\Prob(D) =
  \big( f(x) - \bar{f}(x) \big)^2\text{.}
\end{equation}
This term is the square of the bias of the models.

The term \eqref{eq:term5} is the variance of the function $\hat{f}(x; D)$:
\begin{multline}
  \label{eq:model-variance}
  \int \big( \bar{f}(x) - \hat{f}(x; D) \big)^2\, d\!\Prob(D) =\\
  \E_D\!\left[ \big( \bar{f}(x) - \hat{f}(x; D) \big)^2 \right] =
  \Var_D\!\left( \hat{f}(x; D) \right)\text{.}
\end{multline}

Finally, putting all together --- i.e.  \cref{eq:irreducible-error,eq:bias,eq:model-variance}
---, we have that the expected error for a particular sample $(x, y)$ is
\[
  \E_D\!\left[ \big( y - \hat{f}(x; D) \big)^2 \right] =
    \sigma^2 +
    \big( f(x) - \E\!\left[ \hat{f}(x; D) \right) \big]^2 +
    \Var_D\!\left( \hat{f}(x; D) \right)\text{.}
\]

The irreducible error is the regression error that cannot be reduced by any model --- see
\cref{sec:regression-function}.  The bias error is the error that one expects from the
model acquired by the learning machine and that we observe in the training data --- i.e.
the empirical risk.  The variance error, which does not depend on the real function $f$
but on the models the learning machine can generate, is the error that comes from how
different the models can be from each other --- i.e. the confidence interval the come from
the VC dimension.

\subsection{Regularization}

Also related to the \gls{srm} principle is the concept of \emph{regularization}.
Regularization encourages models to learn robust patterns within the data rather than
memorizing it.

Regularization techniques usually modify the loss function by adding a penalty term that
depends on the complexity of the model.  So, instead of minimizing the empirical risk
$R_n(\theta)$, the learning machine minimizes the regularized empirical risk
\[
  R_n(\theta) + \lambda \Omega(\theta)\text{,}
\]
where $\Omega(\theta)$ is the complexity of the model and $\lambda$ is a hyperparameter
that controls the trade-off between the empirical risk and the complexity.
Note that the regularization term acts a proxy for the confidence interval in the
\gls{srm} principle.  However, regularization is often justified by common sense or
intuition, rather than by strong theoretical arguments.

Other approaches that indirectly control the complexity of the model --- such as early
stopping, dropout, ensembles, and pruning --- are often called implicit regularization.

\section{Linear problems}

To realize the concepts of the \gls{srm} principle in practice, we consider linear
classification tasks.

For the examples in the following subsections, we use the datasets for the AND
and the XOR problem --- see \cref{tab:and-xor}.
The AND problem is linearly separable, while the XOR problem is not.

\begin{tablebox}[label=tab:and-xor]{AND and XOR datasets.}
  \centering
  \begin{minipage}{0.45\textwidth}
    \centering
    \rowcolors{2}{black!10!white}{}
    \begin{tabular}{ccc}
      \toprule
      $x_1$ & $x_2$ & $y = x_1 \land x_2$ \\
      \midrule
      0 & 0 & 0 \\
      0 & 1 & 0 \\
      1 & 0 & 0 \\
      1 & 1 & 1 \\
      \bottomrule
    \end{tabular}
  \end{minipage}
  \begin{minipage}{0.45\textwidth}
  \centering
  \rowcolors{2}{black!10!white}{}
  \begin{tabular}{ccc}
    \toprule
    $x_1$ & $x_2$ & $y = x_1 \oplus x_2$ \\
    \midrule
    0 & 0 & 0 \\
    0 & 1 & 1 \\
    1 & 0 & 1 \\
    1 & 1 & 0 \\
    \bottomrule
  \end{tabular}
  \end{minipage}
  \tcblower
  The AND and XOR datasets are binary classification datasets where the output $y$ is the
  ``logical AND'' and the ``exclusive OR'' of the inputs $x_1$ and $x_2$, i.e.
  $y = x_1 \land x_2$ and $y = x_1 \oplus x_2$.
\end{tablebox}

We show two learning machines that implement the \gls{srm} principle in different ways:
\begin{itemize}
  \itemsep0em
  \item The perceptron, which fixes the complexity of the model and tries to minimize the
    empirical risk; and
  \item The maximal margin classifier, which fixes the empirical risk --- in this case,
    zero -- and tries to minimize the confidence interval.
\end{itemize}

\subsection{Perceptron}
\label{sub:perceptron}

The perceptron is a linear classifier that generates a hyperplane that separates the
classes in the feature space.  It is a parametric model and the learning process minimizes the empirical risk
by adjusting its fixed set of parameters.

\begin{defbox}{Parametric model}{parametric}
  If the learning machine generates a set of functions $f_\theta$ where the number of
  parameters $|\theta|$ is always fixed, the models are called \emph{parametric}.
\end{defbox}

Parametric models are usually simpler and faster to fit, but they are less flexible.  In
other words, it is up to the researcher to choose the best model ``size'' for the problem.
If the model is too small, it will not be able to capture the complexity of the data.  If
the model is too large, it tends to be too complex, too slow to train and might overfit to
the data.  Note, however, that the VC dimension and number of parameters are not the same
thing --- consult \cref{sub:vc-dimension-intuition}.

The perceptron model (with two inputs) is
\begin{equation*}
  f(x_1, x_2; \vec{w} = \left[w_0, w_1, w_2\right]) = u(w_0 + w_1 x_1 + w_2 x_2)\text{,}
\end{equation*}
where $u$ is the heaviside step function
\begin{equation*}
  u(x) = \begin{cases}
    1 & \text{if } x > 0\text{,} \\
    0 & \text{otherwise.}
  \end{cases}
\end{equation*}
The parameters $\theta = \vec{w}$ are called the weights of the perceptron.
The equation $\vec{w} \cdot \vec{x} = 0$, where $\vec{x} = [1, x_1,
x_2]$, is the equation of a hyperplane.

\begin{figurebox}[label=fig:perceptron-and]{Perceptron decision boundaries in the AND dataset.}
  \centering
  \begin{tikzpicture}
    \begin{axis}[
        axis x line=bottom,
        axis y line=left,
        xlabel={$x_1$},
        ylabel={$x_2$},
        width=0.6\textwidth,
        height=0.6\textwidth,
        xtick={0, 1},
        ytick={0, 1},
        xmin=-0.5, xmax=1.5,
        ymin=-0.5, ymax=1.5,
      ]
      \addplot+[only marks, mark=-, color=black, mark size=3pt] coordinates {
        (0, 0) (0, 1) (1, 0)
      };
      \addplot+[only marks, mark=+, color=black, mark size=3pt] coordinates {
        (1, 1)
      };
      \addplot+[domain=0:1.5, mark=none, black, thick] {1.1 - 0.6 * x};
    \end{axis}
  \end{tikzpicture}
  \tcblower
  The perceptron assumes that the classes are linearly separable.
  The hyperplane that separates the classes comes from the weights of the model.
  In this case, $w_0 = -1.1$, $w_1 = 0.6$, and $w_2 = 1$.
\end{figurebox}

In \cref{fig:perceptron-and}, we show the hyperplane (in this case, a line) that the model
with weights $\vec{w} = [-1.1, 0.6, 1]$ generates in this feature space. Such a choice
of parameters correctly classifies the AND dataset, see \cref{tab:and-perceptron}.

\begin{tablebox}[label=tab:and-perceptron]{Truth table for the predictions of the perceptron in the AND dataset.}
  \centering
  \rowcolors{2}{black!10!white}{}
  \begin{tabular}{ccc|cc}
    \toprule
    $x_1$ & $x_2$ & $y$ & $-1.1 + x_1 + x_2$ & $\hat{y}$ \\
    \midrule
    0 & 0 & 0 & -1.1 & 0 \\
    0 & 1 & 0 & -0.1 & 0 \\
    1 & 0 & 0 & -0.5 & 0 \\
    1 & 1 & 1 & 0.5 & 1 \\
    \bottomrule
  \end{tabular}
  \tcblower
  The perceptron model with parameters $w_0 = 1.1$, $w_1 = -1$, and $w_2 = -1$
  correctly classifies the AND dataset.
\end{tablebox}

\begin{figurebox}[label=fig:perceptron-xor]{Perceptron decision boundaries in the XOR
  dataset.}
  \centering
  \begin{tikzpicture}
    \begin{axis}[
        axis x line=bottom,
        axis y line=left,
        xlabel={$x_1$},
        ylabel={$x_2$},
        width=0.6\textwidth,
        height=0.6\textwidth,
        xtick={0, 1},
        ytick={0, 1},
        xmin=-0.5, xmax=1.5,
        ymin=-0.5, ymax=1.5,
      ]
      \addplot+[only marks, mark=-, color=black, mark size=3pt] coordinates {
        (0, 0) (0, 1) (1, 0)
      };
      \addplot+[only marks, mark=+, color=black, mark size=3pt] coordinates {
        (1, 1)
      };
      \addplot+[domain=0:1.5, mark=none, black, thick] {-0.5 + x};
    \end{axis}
  \end{tikzpicture}
  \tcblower
  The XOR dataset is not linearly separable.
  The hyperplane that separates the classes comes from the weights of the model.
  In this case, $w_0 = -0.5$, $w_1 = 1$, and $w_2 = -1$.
  There is no way to classify the XOR dataset correctly with a perceptron.
\end{figurebox}

In \cref{fig:perceptron-xor}, we show the hyperplane that the model $\vec{w} = [-0.5, 1, -1]$
generates for the XOR dataset.  As one can see, the classes are not linearly separable,
and the perceptron model fails to classify the dataset correctly, see \cref{tab:xor-perceptron}.

\begin{tablebox}[label=tab:xor-perceptron]{Truth table for the predictions of the perceptron.}
  \centering
  \rowcolors{2}{black!10!white}{}
  \begin{tabular}{ccc|cc}
    \toprule
    $x_1$ & $x_2$ & $y$ & $-0.5 + x_1 - x_2$ & $\hat{y}$ \\
    \midrule
    0 & 0 & 0 & -0.5 & 0 \\
    0 & 1 & 1 & -1.5 & 0 \\
    1 & 0 & 1 & 0.5 & 1 \\
    1 & 1 & 0 & -0.5 & 0 \\
    \bottomrule
  \end{tabular}
  \tcblower
  The perceptron model with parameters $w_0 = -0.5$, $w_1 = 1$, and $w_2 = -1$
  fails to classify the XOR dataset correctly --- as any other perceptron would do.
\end{tablebox}

It is easy to see that there are an infinite number of hyperplanes that can separate the
classes in the AND dataset.  The training procedure of the perceptron is a simple
algorithm that adjusts the weights of the model to find one of these hyperplanes ---
effectively minimizing the empirical risk.  The algorithm updates the weights iteractively
for each sample that is misclassified, repeating the samples as many times as necessary.
It stops when all samples are correctly classified.

For a binary classification problem and a perceptron with weights $\vec{w}$, there are 4
situations for a given sample $\vec{x}$ and $y$:
\begin{enumerate}
  \itemsep0em
  \item $y = 0$ and $u(\vec{w} \cdot \vec{x})= 0$;
  \item $y = 0$ and $u(\vec{w} \cdot \vec{x})= 1$;
  \item $y = 1$ and $u(\vec{w} \cdot \vec{x})= 0$;
  \item $y = 1$ and $u(\vec{w} \cdot \vec{x})= 1$.
\end{enumerate}

By definition, the algorithm must update the weights when the situation 2 or 3 occurs.
Let $e = y - u(\vec{w} \cdot \vec{x})$ be the error of the model for a given sample.

In situation 2, we have that $\vec{w} \cdot \vec{x} > 0$ which means that the angle
$\alpha$ between the vectors $\vec{w}$ and $\vec{x}$ is less than $90^\circ$, once
$\|\vec{w}\|\|\vec{x}\|\alpha\theta > 0 \implies \cos\alpha > 0 \implies \alpha <
90^\circ$.  To increase the angle between the vectors, we can subtract $\eta\vec{x}$ from
$\vec{w}$, for some small $\eta > 0$ --- see \cref{fig:perceptron-w2}.  The error
here is $e = -1$.

\begin{figurebox}[label=fig:perceptron-w2]{Angle between $\vec{w}$ and $\vec{x}$ in a posite output.}
  \centering
  \begin{tikzpicture}
    \draw[-Stealth] (0, 0) -- (2, 0) node[right] {$\vec{x}$};
    \draw[-Stealth] (0, 0) -- (1, 1.8) node[right] {$\vec{w}$};
    \draw[-Stealth, dashed] (1, 1.8) -- (-0.4, 1.8) node[above] {$-\eta\vec{x}$};
    \draw[-Stealth, thick, gray] (0, 0) -- (-0.4, 1.8) node[left] {$\vec{w}'$};
    \draw (0.4, 0) arc (0:59:0.4) node[right] {$\alpha$};
  \end{tikzpicture}
  \tcblower
  A positive output for the perceptron with weights $\vec{w}$ and input $\vec{x}$ means
  that the angle between the vectors is less than $90^\circ$.  To increase the angle
  between the vectors, we can subtract $\eta\vec{x}$ from $\vec{w}$, for some small $\eta
  > 0$.
\end{figurebox}

Similarly, in situation 3, we have that $\vec{w} \cdot \vec{x} < 0$ which means that the
angle $\alpha$ between the vectors $\vec{w}$ and $\vec{x}$ is greater than $90^\circ$.
To decrease the angle between the vectors, we can add $\eta\vec{x}$ from $\vec{w}$, for
some small $\eta > 0$ --- see \cref{fig:perceptron-w3}.  Now, the error is $e = 1$.

\begin{figurebox}[label=fig:perceptron-w3]{Angle between $\vec{w}$ and $\vec{x}$ in a negative output.}
  \centering
  \begin{tikzpicture}
    \draw[-Stealth] (0, 0) -- (2, 0) node[right] {$\vec{x}$};
    \draw[-Stealth, thick, gray] (0, 0) -- (1, 1.8) node[right] {$\vec{w}'$};
    \draw[-Stealth, dashed] (-0.4, 1.8) -- (1, 1.8) node[above] {$\eta\vec{x}$};
    \draw[-Stealth] (0, 0) -- (-0.4, 1.8) node[left] {$\vec{w}$};
    \draw (0.4, 0) arc (0:101:0.4) node[above] {\phantom{a }$\alpha$};
  \end{tikzpicture}
  \tcblower
  A negative output for the perceptron with weights $\vec{w}$ and input $\vec{x}$ means
  that the angle between the vectors is greater than $90^\circ$.  To decrease the angle
  between the vectors, we can add $\eta\vec{x}$ from $\vec{w}$, for some small $\eta
  > 0$.
\end{figurebox}

From those observations, we can derive a general update rule
\[
  \vec{w}' = \vec{w} + \eta e \vec{x}\text{,}
\]
where $\eta$ is a small positive number that controls the step size of the algorithm.
Note that this rule works even for cases 1 and 4, where the error is zero.

The algorithm converges given $\eta$ sufficiently small and the dataset is linearly
separable.  Note that the algorithm do not make any effort to reduce the confidence
interval.

The perceptron is (possibly) the simplest artificial neural network.  More complex
networks can be built by stacking perceptrons in layers and adding non-linear activation
functions.  The training strategies for those network are usually based on reducing the
empirical risk using the gradient descent algorithm while controlling the complexity of
the model with regularization techniques\footnote{To counterbalance the potential
``excess'' of neurons, techniques like $l_2$ regularization ``disable'' some neurons by
pressuring their weights to zero.}.  Consult \cref{sec:mlp}.

% TODO: adaline for regression

\subsection{Maximal margin classifier}

\section{Closing remarks}

The \gls{srm} principle is a powerful tool to understand the generalization ability of
learning machines.  The principle not only explains many of the empirical results in
\gls{ml} but also provides a theoretical framework to guide the development of new
learning machines.

Many powerful methods have been proposed in the literature --- e.g. support vector
machines, boosting, and deep learning --- that can deal with complex nonlinear problems.
I encourage the reader to dive into the literature to learn more about these methods and
the theoretical principles behind them.  Some comments about a few methods are given in
\cref{chap:learning-machines}.

% vim: spell spelllang=en

\chapter{Data preprocessing}
\label{chap:preprocess}
\glsresetall

\chapterprecishere{I find your lack of faith disturbing.
  \par\raggedleft--- \textup{Darth Vader}, Star Wars: Episode IV -- A New Hope (1977)}

\begin{mainbox}{Chapter remarks}

  \boxsubtitle{Contents}

  \startcontents[chapters]
  \printcontents[chapters]{}{1}{}
  \vspace{1em}

  \boxsubtitle{Context}

  \begin{itemize}
    \itemsep0em
    \item \dots
  \end{itemize}

  \boxsubtitle{Objectives}

  \begin{itemize}
    \itemsep0em
    \item \dots
  \end{itemize}

  \boxsubtitle{Takeaways}

  \begin{itemize}
    \itemsep0em
    \item \dots
  \end{itemize}
\end{mainbox}

{}
\clearpage

\section{Introduction}

In \cref{chap:data,chap:handling}, we discussed data semantics and the tools to
handle data.  They provide the grounds for preparation of the data as we described in the
data sprint tasks in \cref{sub:workflow}.  However, the focus is to guarantee that the
data is tidy and in the observational unit of interest, not to prepare it for modeling.

As a result, although data might be appropriate for the learning tasks we described in
\cref{chap:slt} --- in the sense that we know what the feature vectors and the target
variable are ---, they might not be suitable for the machine learning methods we will use.

One simple example is the perceptron (\cref{sub:perceptron}) that assumes that all
input variables are real numbers.  If the data contains categorical variables, we must
convert them to numerical variables before applying the perceptron.

For this reason, the solution sprint tasks in \cref{sub:workflow} include not only the
learning tasks but also the \emph{data preprocessing} tasks, which are dependent on the
chosen machine learning methods.

\begin{defbox}{Data preprocessing}{preprocessing}
  The process of adjusting the data to make it suitable for a particular learning machine
  or, at the least, to ease the learning process.
\end{defbox}

This is done by applying a series of operations to the data, like in data handling.  The
difference here is that some of the parameters of the operations are not fixed rather they
are fit from a data sampling.  Once fitted, the operations can be applied to
new data, sample by sample.

As a result, a data processing technique acts in three steps:
\begin{enumerate}
  \itemsep0em
  \item \textbf{Fitting}: The parameters of the operation are adjusted to the training
    data (which has already been integrated and tidied, represents well the phenomenon of
    interest, and each sample is in the correct observational unit);
  \item \textbf{Adjustment}: The training data is adjusted according to the fitted
    parameters, eventually, changing the sampling size and distribution;
  \item \textbf{Applying}: The operation is applied to new data, sample by sample.
\end{enumerate}

Understanding these steps and correctly defining the behavior of each of them is crucial
to avoid \gls{leakage} and to guarantee that the model will behave as expected in
production.

\subsection{Formal definition}

Let $T = (K, H, c)$ be a table that represents the data in the desired observational unit
--- as defined in \cref{sec:formal-structured-data}.  In this chapter, without loss of
generality --- as the keys are not used in the modeling process ---, we can consider $K =
\{1, 2, \dots\}$ such that $\rowcard[i] = 0$ if, and only if, $i > n$.  That means that
every row $r \in \{1, \dots, n\}$ is present in the table.

A data preprocessing strategy $F$ is a function that takes a table $T = (K, H, c)$ and
returns a adjusted table $T' = (K', H', c')$ and a fitted \emph{preprocessor} $f(z; \phi)
= f_\phi(z)$ such that $$z \in \bigtimes_{h\, \in\, H} \domainof{h} \cup \{?\}$$ and $\phi$ are
the fitted parameters of the operation.  Similarly, $z' = f_\phi(z)$, called the
preprocessed tuple, satisfies $$z' \in \bigtimes_{h'\, \in\, H'} \domainof{h'} \cup
\{?\}\text{.}$$ Note that we make no restrictions on the number of rows in the adjusted
table, i.e., preprocessing techniques can change the number of rows in the training table.

In practice, strategy $F$ is a chain of dependent preprocessing operations $F_1$, \dots,
$F_m$ such that, given $T = T^{(0)}$, each operation $F_i$ is applied to the table
$T^{(i-1)}$ to obtain $T^{(i)}$ and the fitted preprocessor $f_{\phi_i}$.  Thus, $T' =
T^{(m)}$ and $$f(z; \phi = \{\phi_1, \dots, \phi_m\}) = \left(f_{\phi_1} \circ \dots \circ
f_{\phi_m}\right)(z)\text{,}$$ where $\circ$ is the composition operator.  I say that
they are dependent since none of the operations can be applied to the table without the
previous ones.

\subsection{Degeneration}

The objective of the fitted preprocessor is to adjust the data to make it suitable for the
model.  However, sometimes it can not achieve this goal for a particular input $z$.  This
can happen for many reasons, such as unexpected values, information ``too incomplete'' to
make a prediction, etc.

Formally, we say that the preprocessor $f_\phi$ degenerates over tuple $z$ if it outputs
$z' = f_\phi(z)$ such that $z' = (?, \dots, ?)$.  In practice, that means that the
preprocessor decided that it has no strategy to adjust the data to make it suitable for
the model.  For the sake of simplicity, if any step $f_{\phi_i}$ degenerates over
tuple $z^{(i)}$, the whole preprocessing chain degenerates\footnote{Usually, this is
implemented as an exception or similar programming mechanism.} over $z = z^{(0)}$.

Consequently, in the implementation of the solution, the developer must chose a default
behavior for the model when the preprocessing chain degenerates over a tuple.  It can
be as simple as returning a default value or as complex as redirecting the tuple to a
different pair of preprocessor and model.  Sometimes, the developer can choose to
integrate this as an error or warning in the user application.

\subsection{Data preprocessing tasks}

The most common data preprocessing tasks can be divided into three categories:
\begin{itemize}
  \itemsep0em
  \item Data cleaning;
  \item Data sampling; and
  \item Data transformation. % colocar enhancement aqui
\end{itemize}

In the next sections, I address some of the most common data preprocessing tasks
in each of these categories.  I present them at the order they are usually applied in the
preprocessing, but note that the order is not fixed and can be changed according to the
needs of the problem.

\section{Data cleaning}

Data cleaning is the process of removing errors and inconsistencies from the data.  This is
usually done to make the data more reliable for training and to avoid bias in the learning
process.  Usually, such errors and inconsistencies make the learning machines ``confused''
and can lead to poor performance models.

Also, it includes the process of dealing with missing information, which most machine
learning methods do not cope with.  Solutions range from the simple removal of the
observations with missing data to the creation of information to encode the missing data.

\subsection{Invalid and inconsistent data}

% TODO: move this somewhere when we talk about data handling and/or tidying
% Sometimes, during data collection, information is recorded using special codes.  For
% instance, the value 9999 might be used to indicate that the data is missing.  Such codes
% must be replaced with more appropriate values before modeling.  If a single variable
% encodes more than one concept, new variables must be created to represent each concept.

There are a few, but important, tasks to be done during data preprocessing in terms of
invalid and inconsistent data --- note that we assume that most of the issues in terms of
the semantics of the data have been solved in the data handling phase.  Especially in
production, the developer must be aware of the behavior of the model when it faces
information that is not supposed to be present in the data.

One of the tasks is to assert that physical quantities are dealt with standard units.  One must
check whether all columns that store physical quantities have the same unit of
measurement.  If not, one must convert the values to the same unit.  A summary of this
preprocessing task is presented in \cref{tab:unit-conversion}.

\begin{tablebox}[label=tab:unit-conversion]{Unit conversion preprocessing task.}
  \centering
  \rowcolors{2}{black!10!white}{}
  \begin{tabular}{lp{6cm}}
    \toprule
    \multicolumn{2}{c}{\textbf{Unit conversion}} \\
    \midrule
    % \textbf{Requirements} &
    %   A variable with the physical quantity and a variable with the unit of measurement. \\
    \textbf{Goal} &
      Convert physical quantities into the same unit of measurement. \\
    \textbf{Fitting} &
      None. User must declare the units to be used and, if appropriate, the conversion
      factors. \\
    \textbf{Adjustment} &
      Training set is adjusted sample by sample, independently. \\
    \textbf{Applying} &
      Preprocessor converts the numerical values and drop the unit of measurement column.  \\
    \bottomrule
  \end{tabular}
\end{tablebox}

Moreover, if one knows that a variable must follow a specific range of values, we can check
whether the values are within this range.  If not, one must replace the values with
missing data or with the closest valid value.  Alternatively, one can discard the
observation based on that criterion.  Consult \cref{tab:range-check} for a summary of this
operation.

\begin{tablebox}[label=tab:range-check]{Range check preprocessing task.}
  \centering
  \rowcolors{2}{black!10!white}{}
  \begin{tabular}{lp{6cm}}
    \toprule
    \multicolumn{2}{c}{\textbf{Range check}} \\
    \midrule
    % \textbf{Requirements} &
    %   A numerical variable. \\
    \textbf{Goal} &
      Check whether the values are within the expected range. \\
    \textbf{Fitting} &
      None. User must declare the valid range of values. \\
    \textbf{Adjustment} &
      Training set is adjusted sample by sample, independently.  If appropriate,
      degenerated samples are removed. \\
    \textbf{Applying} &
      Preprocessor checks whether the value $x$ of a variable are within the range $[a,
      b]$.  If not, it replaces $x$ with: (a) missing value $?$, (b) the closest valid
      value $\max(a, \min(b, x))$, or (c) degenerates (discards the observation). \\
    \bottomrule
  \end{tabular}
\end{tablebox}

Another common problem in inconsistent information is that the same category might be
represented by different strings.  This is usually done by creating a dictionary that maps
the different names to a single one, using standardizing lower or upper case, removing
special characters, or more advanced fuzzy matching techniques --- see
\cref{tab:text-standardization}.

\begin{tablebox}[label=tab:text-standardization]{Category standardization preprocessing task.}
  \centering
  \rowcolors{2}{black!10!white}{}
  \begin{tabular}{lp{6cm}}
    \toprule
    \multicolumn{2}{c}{\textbf{Category standardize}} \\
    \midrule
    % \textbf{Requirements} &
    %   A categorical variable. \\
    \textbf{Goal} &
      Create a dictionary and/or function to map different names to a single one. \\
    \textbf{Fitting} &
      None. User must declare the mapping. \\
    \textbf{Adjustment} &
      Training set is adjusted sample by sample, independently. \\
    \textbf{Applying} &
      Preprocessor replaces the categorical variables $x$ of a variable with the mapped
      value $f(x)$ that implements case standardization, special character removal, and/or
      dictionary fuzzy matching. \\
    \bottomrule
  \end{tabular}
\end{tablebox}

Note that these techniques parameters are not fitted from the data, but rather are fixed
from the problem definition.  As a result, they could be done in the data handling phase.
The reason we put them here is that the new data in production usually come with the
same issues.  Having the fixes programmed into the preprocessor makes it easier to
guarantee that the model will behave as expected in production.

\subsection{Missing data}

Since most models cannot handle missing data, it is crucial to deal with it in the data
preprocessing.

There are four main strategies to deal with missing data:
\begin{itemize}
  \itemsep0em
  \item Remove the observations (rows) with missing data;
  \item Remove the variables (columns) with missing data;
  \item Just impute the missing data;
  \item Use an indicator variable to mark the missing data and impute it.
\end{itemize}

Removing rows and columns are commonly used when the number of missing data is small
compared to the total number of rows or columns.  However, be aware that removing rows
``on demand'' can
artificially change data distribution, especially when the missing data is not missing at
random.  Row removal suffers from the same problem as any filtering operations
(degeneration) in the preprocessing step; the developer must specify a default behavior
for the model when a row is discarded in production.  In the case of column removal, the
preprocessor just learns to drop the columns that have missing data during fitting.

Imputing the missing data is usually done by replacing the missing values with some
statistic of the available values in the column, such as the mean, the median, or the
mode\footnote{More sophisticated methods can be used, such as the k-nearest neighbors
algorithm, for example, consult \fullcite{Troyanskaya2001}.}.  This is a simple and
effective strategy, but it can introduce bias in the data, especially when the number of
samples with missing data is large.
Also, it is not suitable when one is not sure whether the missing data is missing because
of a systematic error or phenomenon.

For this case, creating an indicator variable is a good strategy.  This is done by creating
a new column that contains a logical value indicating whether the data is missing or
not\footnote{Some kind of imputation is still needed, but we expect the model to deal
better with it since it can decide using both the indicator and the original variable.}.
By doing so, the model can learn the importance of the missing data.

% \footnote{\color{red}Sometimes the indicator variable is already present: pregnancy and sex
% example.}.

Many times the indicator variable is already present in the data.  For instance, in a
dataset that contains information about pregnancy, let us say the number of days since
the last pregnancy.  This information certainly be missing if sex is male
or number of children is zero.  In this case, no new indicator variable is needed.

\subsection{Outliers}

Outliers are observations that are significantly different from the other observations.
They can be caused by errors in the data collection process or by the presence of a
different phenomenon.  In both cases, it is important to deal with outliers before
modeling.

There are many outliers detection methods, such as the Z-score, the IQR, and the DBSCAN.
% TODO

Like filtering operations in the pipeline, the developer must specify a default behavior
for the model when an outlier is detected in production.

% \begin{slidebox}{Data cleaning}{}
%   \begin{itemize}
%     \item Data cleaning is the process of removing errors and inconsistencies from the data;
%     \item Use the following strategies to deal with missing data:
%       \begin{itemize}
%         \item Remove the rows with missing data;
%         \item Remove the columns with missing data;
%         \item Impute the missing data;
%         \item Use an indicator variable to mark the missing data.
%       \end{itemize}
%     \item Replace special codes with more appropriate values;
%     \item Create a dictionary to map different names to a single one;
%     \item Check whether all columns that store physical quantities have the same unit of
%       measurement;
%     \item Check whether the values are within the expected range;
%     \item Use outlier detection methods to deal with outliers.
%   \end{itemize}
% \end{slidebox}

\section{Data transformation}

One important task in data handling is data transformation.  This is the process of adjusting
the format and the types of the data to make it suitable for analysis.

Before data transformation we must make sure that data is tidy, i.e., to have
each variable in a column and each observation in a row.  Remember that, depending on the
problem definition, we target a particular observational unit.  Having a clear picture of
the observational unit is important to define the columns and the rows of the dataset.

Then, when the data format is acceptable, we can perform a series of operations to make the
column's types and values suitable for modeling.  The reason for this is that most
machine learning methods require the input variables to follow some restrictions.  For
instance, some methods require the input variables to be real numbers, others require the
input variables are in a specific range, etc.

% \subsection{Reshaping}
%
% \textcolor{red}{TODO: pipeline exceptions: like pivoting and aggregating are kept outside
% the pipeline.}
%
% Reshaping is the process of changing the format of the data.  The most common reshaping
% operations are pivoting and unpivoting, which we have already discussed.  However, there
% are other reshaping operations that are useful in practice.
%
% For instance, one can reshape a dataset by splitting a column into multiple columns.  This
% is useful when a column contains multiple values that should be separated.  This can be
% done with mutation with appropriate expressions.  Some frameworks might provide special functions
% to do this, usually called splitting functions.
%
% We can also consider reshaping the operations of filtering, selecting, and aggregating.
% Filtering is usually done to reduce the scope of the data, given some conditions on the
% variables.  Selecting is usually done to remove irrelevant variables or highly correlated
% ones.  Aggregating in a reshaping task is usually applied together with pivoting to change the
% observational unit of the dataset.

% \begin{slidebox}{Reshaping}{}
%   \begin{itemize}
%     \item Reshaping is the process of changing the format of the data;
%     \item The most common reshaping operations are pivoting and unpivoting;
%     \item Other common operation include:
%       \begin{itemize}
%         \item Splitting a column into multiple columns;
%         \item Filtering to reduce the scope of the data;
%         \item Selecting to remove irrelevant variables or highly correlated ones;
%         \item Aggregating to change the observational unit of the dataset.
%       \end{itemize}
%   \end{itemize}
% \end{slidebox}

\subsection{Type conversion}

Type conversion is the process of changing the type of the values in the columns.  This
is usually done to make the data suitable for modeling.  For instance, some machine
learning methods require the input variables to be real numbers.

The most common type conversions are from categorical to numerical and from numerical to
categorical.  The former is usually done by creating dummy variables, i.e., a new column
for each possible value of the categorical variable.  This transformation is also known as
one-hot encoding.  The latter is usually done by binning (discretization and quantization
other concepts) the numerical variable, either by
frequency or by range.

% \begin{slidebox}{Type conversion}{}
%   \begin{itemize}
%     \item Type conversion is the process of changing the type of the values in the columns;
%     \item Use one-hot encoding to convert categorical variables to numerical;
%     \item Use binning to convert numerical variables to categorical.
%   \end{itemize}
% \end{slidebox}

\subsection{Normalization}

Normalization is the process of scaling the values in the columns.  This is usually done to
keep data in a specific range or to make the data comparable.  For instance, some machine
learning methods require the input variables to be in the range $[0, 1]$.

The most common normalization methods are standardization and rescaling.  The former is done
by subtracting the mean and dividing by the standard deviation of the values in the column.
The latter is performed so the values are in a specific range, usually $[0, 1]$ or $[-1, 1]$.

\begin{hlbox}{Clamping after rescaling}
  In production, it is common to clamp the values after rescaling.  This is done to avoid
  the model to make predictions that are out of the range of the training data.
\end{hlbox}

Related to normalization is the log transformation.  This is usually done to make the data
more symmetric or to reduce the effect of outliers.  The log transformation is the process
of taking the logarithm of the values in the column.

% \begin{slidebox}{Normalization}{}
%   \begin{itemize}
%     \item Normalization is the process of scaling the values in the columns;
%     \item Use standardization to make the values have mean 0 and standard deviation 1;
%     \item Use rescaling to make the values be in a specific range;
%     \item Use the log transformation to make the data more symmetric or to reduce the effect
%       of outliers.
%   \end{itemize}
% \end{slidebox}

\subsection{Sampling}

Sampling is the process of selecting a random subset of the data.  This is usually done to
reduce the size of the data or to create a balanced dataset.  For instance, some machine
learning methods are heavily affected by the number of observations in each class.
Also, some methods are computationally expensive and a smaller dataset might be enough to
solve the problem.

The most common sampling methods are random sampling and resampling\footnote{Resampling is
the process of sampling with replacement, sometimes called bootstrapping.}.  The former is
done by selecting a random subset of the data.  The latter is done by selecting a random
subset of the data with replacement.

While random sampling is useful to reduce the size of the data, resampling can be used to
increase the size of the data.  (Although this has some caveats.)  Moreover, resampling
can also create variations of the original dataset with the same distribution of the
values.

More advanced sampling methods are usually used to create balanced datasets.  For
instance, one can use the SMOTE algorithm\footfullcite{chawla2002smote} to create
synthetic observations of the minority class.

\textcolor{red}{In production, just ignore.}

\subsection{Dimensionality reduction}

Dimensionality reduction is the process of reducing the number of variables in the data.
This is usually done to reduce the complexity of the model or to identify irrelevant
variables.  The so-called \emph{curse of dimensionality} is a common problem in machine
learning, where the number of variables is much larger than the number of observations.

There are two main types of dimensionality reduction algorithms: feature selection and
feature extraction.  The former is done by selecting a subset of the variables that leads
to the best models.  The latter is done by creating new variables that are combinations
of the original ones.

Feature selection can be performed before modeling (filter), together with the model
search (wrapper), or as a part of the model itself (embedded).

Feature extraction is usually done by linear methods, such as principal component analysis
(PCA), or by non-linear methods, such as convolution layers and autoencoders.  These methods are able to
compress the information in the data into a smaller number of variables.

% \begin{slidebox}{Dimensionality reduction}{}
%   \begin{itemize}
%     \item Dimensionality reduction is the process of reducing the number of variables in the data;
%     \item Use feature selection to select a subset of the variables that leads to the best models;
%     \item Use feature extraction to create new variables that are combinations of the original ones.
%   \end{itemize}
% \end{slidebox}

% \begin{hlbox}{Practice!}
%   Can you identify which data transformation operations are used to make datasets
%   presented in \cref{chap:data} tidy?
% \end{hlbox}

\section{Data enhancement}

Data integration is the process of combining data from different sources into a single
dataset.  This is usually done to create a more complete dataset or to create a dataset
with a different observational unit.

To perform integration, consider the discussions in \cref{sec:normalization,sub:bridge}.

Additionally, one must consider the following points:
\begin{itemize}
  \item Sometimes the same column may have different names in different datasets.  Redundant
    columns must be removed.
  \item Separate datasets that share the same variables usually happen because there is a
    hidden variable that is not present in the datasets.  During integration, the new
    variable must be created.
\end{itemize}

% \begin{slidebox}{Data integration}{}
%   \begin{itemize}
%     \item Data integration is the process of combining data from different sources into a single dataset;
%     \item Not every join is possible, consider the discussions in \cref{sec:normalization,sub:bridge};
%     \item Remove redundant columns;
%     \item Create new variables to represent the hidden variables.
%   \end{itemize}
% \end{slidebox}

In the data handling pipeline, data integration is useful for data enhancement.  This is
the process of adding new columns to the dataset or single instances.  For example,
imagine that in the tidy data we have a column with the zip code of the customers.  We can
use this information to join (in this case a left join) a dataset with social and economic
information about the region of the zip code.

\section{Comments on unstructured data}

% vim: spell spelllang=en

\chapter{Solution validation}
\label{chap:planning}
\glsresetall

\chapterprecishere{%
  All models are wrong, but some are useful.
  \par\raggedleft--- \textup{George E. P. Box}, Robustness in Statistics}

Once we have defined what an inductive problem is and the means to solve it, we need to
think about how to validate the solution.

In this chapter, we present the experimental planning one can use in the data-driven
parts of a data science project.  \emph{Experimental planning}  in the context of data
science involves designing and organizing experiments to gather performance data
systematically in order to reach specific goals or test hypotheses.

The reason we need to plan experiments is that data science is experimental, i.e. we
usually lack a theoretical model that can predict the outcome of a given algorithm on a
given dataset.  This is why we need to run experiments to gather performance data and make
inferences from it.  The stochastic nature of data and of the learning process makes it
more difficult to predict the outcome of a given algorithm on a given dataset.  Robust
experimental planning is essential to ensure that the results of the experiments are
reliable and can be used to make decisions.

Moreover, we need to understand the main metrics that are used to evaluate the performance
of a solution --- i.e. the pair preprocessor and model.  Each learning task has different
metrics and the goals of the project will determine which metrics are more important.

There is not a single way to plan experiments, but there are some common steps that can
be followed to design a good experimental plan.  In this chapter, we present a
framework for experimental planning that can be used in most data science projects
for inductive problems.

\begin{mainbox}{Chapter remarks}

  \boxsubtitle{Contents}

  \startcontents[chapters]
  \printcontents[chapters]{}{1}{}
  \vspace{1em}

  \boxsubtitle{Context}

  \begin{itemize}
    \item \dots
  \end{itemize}

  \boxsubtitle{Objectives}

  \begin{itemize}
    \item \dots
  \end{itemize}

  \boxsubtitle{Takeaways}

  \begin{itemize}
    \item \dots
  \end{itemize}
\end{mainbox}

{}
\clearpage

\section{Evaluation}
\label{sec:evaluation}

One fundamental step in the validation of a data driven solution for a task is the
\emph{evaluation} of the pair preprocessor and model. This chapter presents strategies to
measure performance of
classifiers and regressors, and how to interpret the results.

We consider the following setup.  Let $$D = \{(\vec{x}_i, y_i)\}_{i=1,\dots,n}$$ be a
dataset\footnote{We assume that the dataset is tidy, in the appropriate observational
unit, and representative of the phenomenon of interest.}, where $\vec{x}_i$ is a feature
vector and $y_i$ is the target value.  We assume
that the dataset is split into a training set, given by indices
$\mathcal{I}_\text{training} \in \{1, \dots, n\}$, and a test set, given by indices
$\mathcal{I}_\text{test} \in \{1, \dots, n\}$, such that $$\mathcal{I}_\text{training}
\cap \mathcal{I}_\text{test} = \emptyset$$ and $$\mathcal{I}_\text{training} \cup
\mathcal{I}_\text{test} = \{1,\dots,n\}\text{.}$$

For evaluation, we consider a data preprocessing technique $F$ and a learning machine
$M$.  The following steps are taken:
\begin{enumerate}
  \item Preprocessing technique $F$ is applied to the training set $D_\text{training} =
    \{(\vec{x}_i, y_i) : i \in \mathcal{I}_\text{training}\}$ to obtain a preprocessed
    training set $D'_\text{training}$ and a fitted preprocessor $f(\vec{x}, y; \phi)
    \equiv f_\phi(\vec{x}, y)$ --- consult \cref{chap:preprocess}.
  \item The learning machine $M$ is trained on the preprocessed training set
    $D'_\text{training}$ to obtain a model $f(\vec{x}'; \theta) \equiv
    f_\theta(\vec{x}')$ --- consult \cref{chap:slt}.
  \item The preprocessor $f_\phi$ is applied to the test set $D_\text{test} =
    \{(\vec{x}_i, y_i) : i \in \mathcal{I}_\text{test}\}$ to obtain a preprocessed test
    set $D'_\text{test} = \{(\vec{x}'_i, y'_i) : i \in \mathcal{I}_\text{test}\}$ such
    that $(\vec{x}'_i, y'_i) = f_\phi(\vec{x}_i, y_i)$.
  \item The model $f_\theta$ is used to make predictions on the preprocessed test set
    $D'_\text{test}$ to obtain predicted values $\hat{y}_i = f_\theta(\vec{x}_i)$ for all
    $i \in \mathcal{I}_\text{test}$.
\end{enumerate}

Note that, by comparing $\hat{y}_i$ with $y_i$ for all $i \in \mathcal{I}_\text{test}$, we
evaluate how good the choice of $\phi$ (parameters of the preprocessor) and $\theta$
(parameters of the model) was.

\subsection{Binary classification evaluation}

In order to assess the quality of a solution for a binary classification task, we need to know which
samples in the test set were classified into which classes.  This information is
summarized in the \emph{confusion matrix}, which is the basis for performance metrics in
classification tasks.

\subsubsection{Confusion matrix}

The confusion matrix is a table where the rows represent the true classes and the columns
represent the predicted classes.  The diagonal of the matrix represents the correct
classifications, while the off-diagonal elements represent errors.  For binary
classification, the confusion matrix is given by
\begin{equation*}
  \begin{blockarray}{cccc}
    & & \multicolumn{2}{c}{\text{Predicted}} \\
    & & 1 & 0 \\
    \begin{block}{l c (c c)}
      \text{Expected} & 1 & \text{TP} & \text{FN} \\
      & 0 & \text{FP} & \text{TN} \\
    \end{block}
  \end{blockarray}
\end{equation*}
where TP is the number of true positives
$$|\{ i \in \mathcal{I}_\text{test} : y_i = 1 \land \hat{y}_i = 1 \}|\text{,}$$
TN is the number of true negatives
$$|\{ i \in \mathcal{I}_\text{test} : y_i = 0 \land \hat{y}_i = 0 \}|\text{,}$$
FN is the number of false negatives
$$|\{ i \in \mathcal{I}_\text{test} : y_i = 1 \land \hat{y}_i = 0 \}|\text{,}$$
and FP is the number of false positives
$$|\{ i \in \mathcal{I}_\text{test} : y_i = 0 \land \hat{y}_i = 1 \}|\text{.}$$

\subsubsection{Performance metrics}

From the confusion matrix, we can derive several performance metrics.  Each of them focus
on different aspects of the classification task, and the choice of the metric depends on
the problem at hand.  Each metric prioritizes different types of errors and yields
a value between 0 and 1, where 1 is the best possible value.

\paragraph{Accuracy} is the proportion of correct predictions over the total number of
samples in the test set, given by
\begin{equation*}
  \text{Accuracy} = \frac{\text{TP} + \text{TN}}{\text{TP} + \text{TN} + \text{FP} + \text{FN}}\text{.}
\end{equation*}
This metric is simple and easy to interpret: a classifier with an accuracy of 1 is
perfect, while a classifier with an accuracy of 0.5 misses half of the predictions.
Accuracy assigns the same weight to any kind of error --- i.e. false positives and false
negatives.  As a result, if the proportion of positive and negative samples is imbalanced,
the value of accuracy may become misleading.  Let $r$ be the ratio of positive samples in
the test set --- consequently, $1-r$ is the ratio of negative samples ---, then a
classifier that correctly predicts all positive samples and none of the negative samples
will have an accuracy of $r$.  If $r$ is close to 1, the classifier will have a high value
of accuracy even if it is not good at predicting the negative class.

This issue is not impeditive for the usage of accuracy in imbalanced datasets, but one
needs to be aware that accuracy values lower than $\max(r, 1-r)$ are not better than
guessing.

\paragraph{Balanced accuracy} aims to solve this interpretation issue of the accuracy.  It
is the average of the true positive rate (TPR) and the true negative rate (TNR), given by
\begin{equation*}
  \text{Balanced Accuracy} = \frac{\text{TPR} + \text{TNR}}{2}\text{,}
\end{equation*}
where
\[
  \text{TPR} = \frac{\text{TP}}{\text{TP} + \text{FN}}\text{,}
\]
and
\[
  \text{TNR} = \frac{\text{TN}}{\text{TN} + \text{FP}}\text{.}
\]
Each term penalizes a different type of error independently: TPR penalizes false
negatives, while TNR penalizes false positives.  Balanced accuracy is useful when the cost
of errors on the minority class is higher than the cost of errors on the majority class.
This way, any value greater than 0.5 is better than random guessing.

A limitation of the balanced accuracy is that it ``automatically'' assigns the weight of
errors based on the class proportion, which may not be the best choice for the problem.
Other metrics focus only one of the classes and are more flexible to adjust the weight of
errors.

\paragraph{Precision} is an asymmetric metric that focuses on the positive class.  It is
the proportion of true positive predictions over the total number of samples predicted as
positive, given by
\begin{equation*}
  \text{Precision} = \frac{\text{TP}}{\text{TP} + \text{FP}}\text{.}
\end{equation*}
This metric is useful when the cost of false alarm is high, as it quantifies the
ability of the classifier to avoid false positives.  For example, in a medical diagnosis
task, precision is important to avoid unnecessary treatments (false positive diagnosis).
Semantically, precision measures how confident we can be that a positive prediction is
actually positive.  Note that it measures nothing about the ability of the classifier in
terms of the negative predictions.

\paragraph{Recall} is another asymmetric metric that also focuses on the positive class.
It is the proportion of true positive predictions over the total number of
samples that are actually positive, given by
\begin{equation*}
  \text{Recall} = \text{TPR} = \frac{\text{TP}}{\text{TP} + \text{FN}}\text{.}
\end{equation*}
This metric is useful when the cost of missing a positive sample is high, as it quantifies the
ability of the classifier to avoid false negatives.  It can also be interpreted as the
``completeness'' of the classifier: how many positive samples were correctly retrieved.
For example, in a medical diagnosis task, recall is important to avoid missing a
diagnosis.

\paragraph{F-score} is way of balancing both kinds of errors, false positives and false
negatives, while maintaining the focus on the positive class. It is the weighted harmonic
mean of precision and recall given by
\begin{equation*}
  \text{F-score}(\beta) = \text{F}_\beta\text{-score} =
    \frac%
      {(1 + \beta^2) \cdot \text{Precision} \cdot \text{Recall}}
      {\beta^2 \cdot \text{Precision} + \text{Recall}}\text{,}
\end{equation*}
where $\beta > 0$ is a parameter that controls the weight of precision in the metric.
The most common value for $\beta$ is 1, which gives the F$_1$-score.  Higher values of
$\beta$ give more weight to precision ($\beta > 1$), while lower values give more weight
to recall ($\beta < 0$).

\paragraph{Specificity} goes in the opposite direction of recall, focusing on the negative
class.  It is the proportion of true negative predictions over the total
number of samples that are actually negative, given by
\begin{equation*}
  \text{Specificity} = \text{TNR} = \frac{\text{TN}}{\text{TN} + \text{FP}}\text{.}
\end{equation*}
This metric is very common in the medical literature, but less common in other contexts.
The probable reason is that it is easier to interpret the metrics that focus on the
positive class, as the negative class is usually the majority class --- and, thus, less
interesting.

\subsubsection{Interpretation of metrics}

\begin{tablebox}[label=tab:classification-metrics]{Summary of the properties of
  data classification performance metrics.}
  \centering
  \rowcolors{2}{black!10!white}{}
  \begin{tabular}{l c c}
    \toprule
    \textbf{Metric} & \textbf{Focus} & \textbf{Interpretation} \\
    \midrule
    Accuracy           & Symmetric & Penalizes all \\
    Balanced Accuracy  & Symmetric & Penalizes all \\
    Recall (TPR)       & Positive & Penalizes FN \\
    Precision          & Positive & Penalizes FP \\
    F-score            & Positive & Penalizes all \\
    Specificity (TNR)  & Negative & Penalizes FP \\
    % Fall-out (FPR)     & Negative & Not affected & Penalizes TN \\
    % FPR = 1 - TNR
    \bottomrule
  \end{tabular}
\end{tablebox}

A common misconception about the asymmetric metrics (especially precision and recall) is
that they are robust to class imbalance.  This is not true.  Moreover, they benefit
classifiers that prefer the positive class.  As a result, they should be preferred when
the positive class is the minority class.

\begin{tablebox}[label=tab:classification-metrics-ex]{Behavior of classification
  performance metrics for different classifiers.}
  \centering
  \rowcolors{2}{black!10!white}{}
  \begin{tabular}{l c c c}
    \toprule
    \textbf{Metric} & \textbf{Imbalance} & \textbf{Guess 1} & \textbf{Guess 0} \\
    \midrule
    Accuracy & Affected & $r$ & $1 - r$ \\
    Balanced Accuracy & Not affected & $0.5$ & $0.5$ \\
    Recall (TPR) & Not affected & $1$ & $0$ \\
    Precision & Affected & $r$ & $0/0 = 0$ \\
    F$_1$-score & Affected & $\frac{2r}{1 + r}$ & 0 \\
    Specificity (TNR) & Not affected & $0$ & $1$ \\
    \bottomrule
  \end{tabular}
  \tcblower
  Performance of different classifiers in the example of a dataset with ratio $r$ of
  positive and $1-r$ of negative samples.
\end{tablebox}

% TODO: things get bad if majority class is positive

\subsection{Regression estimation evaluation}

Performance measures for regression tasks are usally calculated based on the error or residual
$\epsilon_i = \hat{y}_i - y_i$ for all $i \in \mathcal{I}_\text{test}$.  The most common measures
are the mean absolute error, mean squared error.

\paragraph{Mean absolute error} is the average of the absolute values of the residuals,
given by
\begin{equation*}
  \text{MAE} = \frac{1}{n} \sum_{i=1}^n | \epsilon_i |\text{.}
\end{equation*}
This measure is easy to interpret and gives an idea of the average error of the model.

\paragraph{Mean squared error} is the average of the squared residuals, given by
\begin{equation*}
  \text{MSE} = \frac{1}{n} \sum_{i=1}^n \epsilon_i^2\text{.}
\end{equation*}
This measure penalizes large errors more than the mean absolute error, as the squared
residuals are summed.

\paragraph{Root mean squared error} is the square root of the mean squared error, given by
\begin{equation*}
  \text{RMSE} = \sqrt{\text{MSE}}\text{.}
\end{equation*}
This measure is in the same unit as the target variable, which makes it easier to
interpret.

% Deixar de fora.
% \paragraph{Coefficient of determination} is a measure of how well the model explains the
% variance of the target variable, given by
% \begin{equation*}
%   R^2 = 1 - \frac{1}{(n-1)s^2} \sum_{i=1}^n \epsilon_i^2\text{,}
% \end{equation*}
% where $s^2$ is the sample variance of the target variable $y_i : i \in
% \mathcal{I}_\text{test}$.  The coefficient of determination ranges from $-\infty$ to 1,
% where 1 indicates a perfect fit and 0 indicates that the model is as good as the mean of
% the target variable.  Negative values indicate that the model is worse than the mean.
% TODO: falar que não é diretamente comparável,
% TODO: assimétricas precisam manter as prioris

% Sometimes, \dots
% Falar do RMSLE, plotar a curva em função da diferença. RMSLE incurs a larger penalty for
% the underestimation of the Actual variable than the Overestimation.

\subsection{Probabilistic classification evaluation}

A particular case of the regression estimation is when we want to estimate the
probability\footnote{Although the term probability is used, the output of the regressor
does not need to be a probability in the strict sense.  It is a confidence level in the
interval $[0, 1]$ that can be interpreted as a probability.} of a sample belonging to the
positive class --- i.e. $y = 1$.  In this case, the output of the model should be a
probability in the interval $[0, 1]$.  We can use a threshold $\tau$ to convert the
probabilities into binary predictions.  The default threshold is usually $\tau = 0.5$ ---
a sample is positive if the probability is greater than or equal to 0.5 and negative
otherwise ---, but it can be adjusted to change the trade-off between recall and
specificity. A low threshold, $\tau \approx 0$, will increase recall at the expense of
specificity, while a high threshold, $\tau \approx 1$, will increase specificity at the
expense of recall.

Thus, any regressor $f_R : \mathcal{X} \rightarrow [0, 1]$ can be converted into a binary
classifier $f_C : \mathcal{X} \rightarrow \{0, 1\}$ by comparing the output with the
threshold $\tau$:
\begin{equation*}
  f_C(\vec{x}; \tau) = \begin{cases}
    1 & \text{if } f_R(\vec{x}) \geq \tau\text{,} \\
    0 & \text{otherwise}\text{.}
  \end{cases}
\end{equation*}

Before we discuss the performance measures for probabilistic classifiers, let us define
some rates that are used in the evaluation.  The true positive rate (TPR) is the proportion
of true positive predictions over the total number of samples that are actually positive,
\begin{equation*}
  \text{TPR} = \frac{\text{TP}}{\text{TP} + \text{FN}}\text{,}
\end{equation*}
and the false positive rate (FPR) is the proportion of false positive predictions over the
total number of samples that are actually negative,
\begin{equation*}
  \text{FPR} = \frac{\text{FP}}{\text{FP} + \text{TN}}\text{.}
\end{equation*}

The performance of the possible variations of the classifiers can be evaluated using
appropriate measures.  Consider the example below of a given test set and the predictions
of a regressor.  We first sort the samples by the predicted probabilities and then
% TODO: define TPR, FPR, FNR somewhere else
calculate the true positive rate and false positive rate for each threshold.
We need to consider only thresholds equal to the unique predicted values to understand the
variations.

\begin{tablebox}[label=tab:prob-reg-example]{Illustrative example of probability .}
  \centering
  \rowcolors{2}{black!10!white}{}
  \begin{tabular}{rrrr}
    \toprule
    \textbf{Predicted/Threshold} & \textbf{Expected} & \textbf{TPR} & \textbf{FPR}  \\
    \midrule
    - / $\infty$ & - & $0/5$ & $0/5$ \\
    $0.98$       & 1 & $1/5$ & $0/5$ \\
    $0.97$       & 1 & $2/5$ & $0/5$ \\
    $0.80$       & 0 & $2/5$ & $1/5$ \\
    $0.72$       & 1 & $3/5$ & $1/5$ \\
    $0.70$       & 1 & $4/5$ & $1/5$ \\
    $0.66$       & 0 & $4/5$ & $2/5$ \\
    $0.52$       & 0 & $4/5$ & $3/5$ \\
    $0.40$       & 1 & $5/5$ & $3/5$ \\
    $0.25$       & 0 & $5/5$ & $4/5$ \\
    $0.10$       & 0 & $5/5$ & $5/5$ \\
    \bottomrule
  \end{tabular}
\end{tablebox}

From this table, we can calculate the performance measures that are useful for
probabilistic classifiers.

\subsubsection{Receiver operating characteristic}

The receiver operating characteristic (ROC) curve is a graphical representation of the
trade-off between the true positive rate and the false positive rate as the threshold
$\tau$ is varied.  The ROC curve is obtained by plotting the TPR against the FPR for all
possible thresholds.  \Cref{fig:roc-example} is the ROC curve for the example in
\cref{tab:prob-reg-example}.

\begin{figurebox}[label=fig:roc-example]{Illustrative example of ROC curve.}
  \centering
  \begin{tikzpicture}
    \datavisualization [
      scientific axes=clean,
      visualize as line/.list={curve, diagonal},
      x axis={label={FPR}, include value=0.0, include value=1.0, length=5cm},
      y axis={label={TPR}, include value=0.0, include value=1.0, length=5cm},
      all axes={grid},
      style sheet=vary dashing,
    ] data [set=curve] {
      % based on the table above
      x, y
      0.0, 0.0
      0.0, 0.2
      0.0, 0.4
      0.2, 0.4
      0.2, 0.6
      0.2, 0.8
      0.4, 0.8
      0.6, 0.8
      0.6, 1.0
      0.8, 1.0
      1.0, 1.0
    } data [set=diagonal] {
      x, y
      0.0, 0.0
      1.0, 1.0
    };
  \end{tikzpicture}
  \tcblower
  ROC curve for the example in \cref{tab:prob-reg-example}.  The diagonal line represents
  a random classifier, and points above the diagonal are better than random.
\end{figurebox}

The ROC curve is useful to explore the trade-off between recall and specificity.  The
diagonal line represents a random classifier, and points above the diagonal are better
than random.  The area under the ROC curve (AUC) is a possible measure of the performance
of the classifier.  The AUC is scale invariant, which means that it measures how well
predictions are ranked, rather than their absolute values.  In our example, the AUC is
$0.8$.

\subsubsection{Detection error trade-off}

The detection error trade-off (DET) curve is a graphical representation of the trade-off
between the false positive rate and the false negative rate (FNR),
\begin{equation*}
  \text{FNR} = \frac{\text{FN}}{\text{TP} + \text{FN}} = 1 - \text{TPR}\text{.}
\end{equation*}
The DET curve is similar to the ROC curve, but by plotting only the FPR and FNR, it gives
a better view of the ``cost'' (errors) of different thresholds.  The DET curve is
especially useful when the cost of false positives and false negatives is different.
The DET curve of our example is shown in \cref{fig:det-example}.

\begin{figurebox}[label=fig:det-example]{Illustrative example of DET curve.}
  \centering
  \begin{tikzpicture}
    \datavisualization [
      scientific axes=clean,
      visualize as line,
      x axis={label={FPR}, include value=0.0, include value=1.0, length=5cm},
      y axis={label={FNR}, include value=0.0, include value=1.0, length=5cm},
      all axes={grid},
    ] data {
      % based on the table above
      x, y
      0.0, 1.0
      0.0, 0.8,
      0.0, 0.6,
      0.2, 0.6,
      0.2, 0.4,
      0.2, 0.2,
      0.4, 0.2,
      0.6, 0.2,
      0.6, 0.0,
      0.8, 0.0,
      1.0, 0.0,
    };
  \end{tikzpicture}
  \tcblower
  DET curve for the example in \cref{tab:prob-reg-example}.  The diagonal line represents
  a random classifier, and points below the diagonal are better than random.
\end{figurebox}

Usually, the DET curve is plotted in a normal deviate scale~\parencite{Martin1997}.  In
this scale, the axes are transformed to show the error rates in a more linear way.

\begin{figurebox}[label=fig:det-example-normal]{Illustrative example of DET curve (normal deviate scale).}
  \centering
  \begin{tikzpicture}
    \datavisualization [
      scientific axes=clean,
      visualize as line,
      x axis={%
        label={FPR},
        include value=0.001, include value=0.999,
        scaling=-3 at 0cm and 3 at 5cm,
        ticks={%
        %   major={at={0.001, 0.005, 0.02, 0.05, 0.1, 0.2, 0.5, 0.8, 0.9, 0.95, 0.98, 0.995, 0.999}},
          tick typesetter/.code={%
            \pgfmathprintnumber{##1}$\sigma$
          },%
        },%
      }%
    ] data {
      x,y
      -3.090232306167813,3.090232306167813
      -2.5758293035489,2.5758293035489
      -2.0537489106318225,2.053748910631822
      -1.6448536269514726,1.6448536269514715
      -1.2815515655446008,1.2815515655446008
      -0.8416212335729142,0.8416212335729144
      0,0
      0.8416212335729144,-0.8416212335729142
      1.2815515655446008,-1.2815515655446008
      1.6448536269514715,-1.6448536269514726
      2.053748910631822,-2.0537489106318225
      2.5758293035489,-2.5758293035489
      3.090232306167813,-3.090232306167813
    };
  \end{tikzpicture}
\end{figurebox}

\subsection{Other variations}

Some other points:
\begin{itemize}
  \item measures for classification are asymmetric (benefit positive, positive should be minority);
  \item prefer measures that work well with averaging and good to compare;
  \item Be careful with priors.
  \item multiclass how to evaluate?
  \item customize to address the real problem.
\end{itemize}


\section{An experimental plan for data science}

Like any other experimental science, data science requires a well-thought-out experimental
plan to ensure that evaluation results are reliable and can be used to make decisions.
Failure to use well the resources we have at hand --- i.e. the limited amount of data ---
can lead to incorrect conclusions about the performance of a solution.

There are important elements that should be considered when designing an experimental
plan.  These elements are:
\begin{itemize}
  \item \textbf{Hypothesis}: The main question that the experiment aims to validate.
    In this chapter, we address common questions in data science projects and how to
    validate them.
  \item \textbf{Data}: The dataset that will be used in the experiment.  In
    \cref{chap:fundamental,chap:data}, we address topics about collecting and organizing data.
    In \cref{chap:handling}, we address topics about preparing the data for the
    experiments.
  \item \textbf{Solution search algorithm} Techniques that find a solution for the task.
    We use the term ``search'' because, the chosen algorithm aim at optimizing both the
    parameters of the preprocessing chain and the ones of the model.  The theoretical
    basis for these techniques is in \cref{chap:preprocess,chap:slt}.
  \item \textbf{Performance measuring}: The metric that will be used to evaluate the
    performance of the model.  Refer to \cref{sec:evaluation} for the main metrics used in
    binary classification and regression estimation tasks.
\end{itemize}

A general example of a description of an experimental plan is ``What is the probability of
the technique $A$ to find a model that reaches a performance $X$ in terms of metric $Y$ in
the real-world given dataset $Z$ as training set (assuming $Z$ is a representative
dataset)?''

Another example is ``Is technique $A$ better than technique $B$ for finding a model that
predicts the output with $D$ as a training set in terms of metric $E$?''

In the next sections, we consider these two cases: \emph{estimating expected performance}
and \emph{comparing algorithms}.  Before that, we discuss a strategy to make the best use
of the finite amount of data we have available.

\subsection{Sampling strategy}

When dealing with a data-driven solution, the available data is a representation of the
real world.  So, we have to make the best use of the data we have to estimate how good our
solution is expected to be in production.

As we have seen, the more data we use to search for a solution, the better the solution is
expected to be.  Thus, we use the whole dataset for deploying a solution.  But, what
method for preprocessing and learning should we use?  How well that technique is
expected to perform in the real world?

Let us say we fix a certain technique, let us call it $A$.  Let $M$ be the solution found
by $A$ using the whole dataset $D$.  If we assess $M$ using the whole dataset $D$, the
performance $p$ we get is optimistic.  This is because $M$ has been trained and tested
on the same data.

One could argue that we could use a hold-out set to estimate the performance of $M$ ---
i.e. splitting the dataset into a training set and a test set once.  However, this does
not solve the problem.  The performance $p$ we in the test set might be an overestimation
or an underestimation of the performance of $M$ in production.  This is because the
randomly chosen test set might be an ``outlier'' in the representation of the real world,
containing cases that are too easy or too hard to predict.

The correct way to estimate the performance of $M$ is to address performance as a
random variable, since both the data and the learning process are stochastic.
By doing so, we can study the distribution of the performance, not particular values.

As any statistical evaluation, we need to generate samples
of the performance of the possible solutions that $A$ is able to obtain. To do so, we use
a sampling strategy to generate datasets $D_1, D_2, \ldots$ from $D$.  Each
dataset is further divided into a training set and a test set, which must be disjoint.
Each training set is thus used to find solution --- $M_1, M_2, \ldots$ for each
training set --- and the test set is used to evaluate the performance --- $p_1, p_2,
\ldots$ for each test set --- of the solution.  The test set emulates the real-world
scenario, where the model is used to make predictions on new data.

The most common sampling strategy is the \emph{cross-validation}.  It assumes that data is
independent and identically distributed (i.i.d.).  The cross-validation technique divides
the dataset into $r$ folds randomly, with the same size.  Each part (fold) is used as a
test set once and as a training set $r-1$ times.  So, first we use as training set folds
$2, 3, \ldots, r$ and as test set fold $1$.  Then, we use as training set folds $1, 3,
\ldots, r$ and as test set fold $2$. And so on.  See \cref{fig:cross-validation}.

\begin{figurebox}[label=fig:cross-validation]{Cross-validation}
  \centering
  \begin{tikzpicture}
    \foreach \i in {1, 2, 3, 4} {
      \node at (2 * \i, 0) {Fold \i};
      \foreach \j in {1, 2, 3, 4} {
        % if \i == \j, then it is the test set
        \ifnum\i=\j
          \node [smallblock, minimum width=16mm, minimum height=6mm] (fold\i\j) at (2 * \i, -\j) {Test};
        \else
          \node [smalldarkblock, minimum width=16mm, minimum height=6mm] (fold\i\j) at (2 * \i, -\j) {Training};
        \fi
      }
    }
    \foreach \j in {1, 2, 3, 4} {
      \node [draw, dashed, fit={(fold1\j) (fold4\j)}] {};
    }
  \end{tikzpicture}
  \tcblower
  Cross-validation is a technique to sample training and test sets.  It divides the
  dataset into $r$ folds, using $r-1$ folds as a training set and the remaining fold as a
  test set.
\end{figurebox}

If possible, one should use repeated cross-validation, where this process is repeated many
times, each having a different fold partitioning chosen at random.  Also, when dealing with
classification problems, we should use stratified cross-validation, where the distribution
of the classes is preserved in each fold.

% TODO Performance Visualization
% Also, the application of a single cross-validation sampling enables us to create a
% predicted vector for the whole dataset.  This is done by concatenating the predictions for
% each fold.  (Note however that the predictions are not totally independent, as they share
% some training data.  This dependency should be taken into account when analyzing the
% results.) This vector can be used to perform hypothesis tests --- like McNemar's test, see
% \cref{sub:comparison} --- or to plot ROC (Receiver Operating Characteristic) curves or DET
% (Detection Error Tradeoff) curves --- see \cref{sec:evaluation}.

\subsection{Collecting evidence}

Once we understand the sampling strategy, we can design the experimental plan to collect
evidence about the performance of the solution.  The plan involves the following steps.

The solution search algorithm $A$ involves both a
given data preprocessing chain and a machine learning method.  Both of them generate a
different result for each dataset $D_k$ used as an input.  In other words, the parameters
$\phi$ of the data preprocessing step are adjusted --- see \cref{chap:preprocess} --- and the
parameters $\theta$ of the machine learning model are adjusted --- see \cref{chap:slt}.
These parameters, $\left[\phi_k, \theta_k\right]$ are the solution $M_k$, and must be
calculated exclusively using the training set $D_{k,\text{train}}$.

Once the parameters $\phi_k$ and $\theta_k$ are fixed, we apply them
in the test set $D_{k,\text{test}}$.  For each sample $(x_i, y_i) \in D_{k,\text{test}}$,
we calculate the prediction $\hat{y}_i = f_{\phi,\theta}(x_i)$.  The target value $y$ is
called the ground-truth or expected outcome.

Given a performance metric $R$, for each dataset $D_k$, we calculate
$$p_k = R\!\left(\left[y_i : i\right], \left[\hat{y}_i : i\right]\right)\text{.}$$
Note that, by definition, $p_k$ is free of \gls{leakage}, as $\left[\phi_k,
\theta_k\right]$ are found without the use of the data in $D_{k,\text{test}}$ and to
calculate $\hat{y}_i$ we use only $x_i$ (with no target $y_i$).

For a detailed explanation of this process for each sampling, consult
\cref{sec:evaluation}.
A summary of the experimental plan for estimating expected performance is shown in
\cref{fig:plan-single}.

\begin{figurebox}[label=fig:plan-single]{Experimental plan for estimating expected performance of a solution.}
  \centering
  \begin{tikzpicture}
    \node [darkcircle] (data) at (0, 0) {Data};
    \node [block] (sampling) at (0, -2) {Sampling strategy};
    \path [line] (data) -- (sampling);

    \foreach \i in {1, 2, 4, 5} {
      \draw [dashed] (-7 + 2 * \i, -4.5) rectangle (-5.1 + 2 * \i, -3.5);
      \path [line] (sampling) -- (-6.1 + 2 * \i, -3.5);

      \node [smalldarkblock] (train\i) at (-6.4 + 2 * \i, -4) {Training};
      \node [smallblock] (test\i) at (-5.6 + 2 * \i, -4) {Test};

      \path [line] (-6.1 + 2 * \i, -4.5) -- (-6.1 + 2 * \i, -5.5);
    }
    \node [anchor=center] at (0, -4) {\dots};

    \draw [dashed] (-5, -5.5) rectangle (4.9, -10.5);

    \node [smalldarkblock, font=\small, inner sep=4pt] (train) at (-4, -7) {Training};
    \node [smallblock, inner sep=4pt] (test) at (-4, -9) {Test (no target)};

    \draw [dashed] (-3, -6) rectangle (3, -8);
    \node [anchor=south] at (0, -6.1) {Solution search algorithm};

    \node [block] (handling) at (-1.5, -7) {Data handling pipeline};
    \node [block] (learning) at (1.5, -7) {Machine learning};
    \node (model) at (4, -7) {%
      % bracket array with \theta and \phi
      $\left[
      \begin{array}{c}
        \phi \\
        \theta \\
      \end{array}
      \right]$
    };

    \path [line] (train) -- (handling);
    \path [line] (handling) -- (learning);
    \path [line, dashed] (3, -7) -- (model);

    \node [block] (preprocess) at (-1.5, -9) {Preprocessor};
    \node [block] (prediction) at (1.5, -9) {Model};

    \path [line, dashed] (handling) -- (preprocess);
    \path [line, dashed] (learning) -- (prediction);

    \path [line] (test) -- (preprocess);
    \path [line] (preprocess) -- (prediction);

    \node [smallblock, inner sep=4pt] (predicted) at (4, -9) {predictions};
    \node (performance) at (4, -10) {$p$};
    \path [line] (prediction) -- (predicted);

    \node [smallblock, inner sep=4pt] (labels) at (-4, -10) {Test (target)};
    \path [line] (labels) -- (performance);
    \path [line] (predicted) -- (performance);

    \node (perfs) at (-4.2, -12) {%
      $\left[
        \begin{array}{c}
          p_1 \\
          p_2 \\
          \vdots \\
        \end{array}
      \right]$
    };

    \node [block] (hypothesis) at (-1, -12) {Hypothesis test};

    \path [line, dashed] (-4.2, -10.5) -- (perfs);
    \path [line] (perfs) -- (hypothesis);
  \end{tikzpicture}
  \tcblower
  % A brief description of the experimental plan.
  The experimental plan for estimating the expected performance of a solution involves
  sampling the data, training and testing the solution, evaluating the performance, and
  validating the results.
\end{figurebox}

Finally, we can study the sampled performance values $p_1, p_2, \ldots$ like any other
statistical data to proof (or disproof) the hypothesis.  This process is called
validation.

\begin{defbox}{Validation}{validation}
  While we call evaluation the process of assessing the performance of a solution using a
  test set; validation, on the other hand, is the process of interpreting or confirming
  the meaning of the evaluation results.  Validation is the process of determining the
  degree to which the evaluation results support the intended use of the solution (unseen
  data).
\end{defbox}

The results are not the ``real'' performance of the solution
$M$ in the real world, as that would require new data to be collected.  However, we can
safely interpret the performance samples as being sampled from the same distribution as
the real-world performance of the solution $M$.

% TODO: Visualization of results
% Talk about summary statistics, visualization (boxplot, roc and det curves), and Bayesian
% analysis.

\subsection{Estimating expected performance}

We have seem that we need a process of interpreting or confirming the meaning of the
evaluation results.
Sometimes, it is as simple as calculating the mean and standard deviation of the
performance samples.  Other times, we need to use more sophisticated techniques, like
hypothesis tests or Bayesian analysis.

Let us say our goal is to reach a certain performance threshold $p_0$.  After an
experiments done with $10$ repeated $10$-fold cross-validation, we have the average
performance $\bar{p}$ and the standard deviation $\sigma$.  If $\bar{p} - \sigma \gg
p_0$, it is very likely that the solution will reach the threshold in production.
Although this is not a formal validation, it is a good and likely enough indication.

Also, it is common to use visualization techniques to analyze the results.  Box plots are
a good way to see the distribution of the performance samples.

A more sophisticated technique is to use Bayesian analysis.  In this case, we use the
performance samples to estimate the probability distribution of the performance of the
algorithm.  This distribution can be used to calculate the probability of the performance
being better than a certain threshold.

\textcite{Benavoli2017}\footfullcite{Benavoli2017} propose an interesting Bayesian test that accounts for the
overlapping training sets in the cross-validation\footnote{%
This is actually a particular case of the proposal in the paper, where the authors
consider the comparison between two performance vector --- which is the case described in
\cref{sub:comparison}.}.
Let $z_k = p_k - p^{*}$ be the
difference between the performance of the $k$-th fold and the performance goal $p^{*}$,
a generative model for the data is
\begin{equation*}
  \vec{z} = \vec{1}\mu + \vec{v}\text{,}
\end{equation*}
where $\vec{z} = (z_1, z_2, \ldots, z_n)$ is the vector performance gains, $\vec{1}$ is a
vector of ones, $\mu$ is the parameter of interest (the mean performance gain), and
$\vec{v} \sim \operatorname{MVN}(0, \Sigma)$ is a multivariate normal noise with zero mean
and covariance matrix $\Sigma$.  The covariance matrix $\Sigma$ is characterized as
\begin{equation*}
  \Sigma_{ii} = \sigma^2\text{,}\quad
  \Sigma_{ij} = \sigma^2\rho\text{,}
\end{equation*}
for all $i \neq j \in \{1, 2, \ldots, n\}$, where $\rho$ is the correlation (between folds)
and $\sigma^2$ is the variance.  The likelihood model of the data is
\begin{equation*}
  \Prob(\vec{z} \mid \mu, \Sigma) =
    \exp\left(-\frac{1}{2}(\vec{z} - \vec{1}\mu)^T \Sigma^{-1} (\vec{z} - \vec{1}\mu)\right)
    \frac{1}{(2\pi)^{n/2} \sqrt{\lvert \Sigma \rvert}}\text{.}
\end{equation*}
According to them, such likelihood does not allow to estimate the correlation from data,
as the maximum likelihood estimate of $\rho$ is zero regardless of the observations.
Since $\rho$ is not identifiable, the authors suggest using the heuristic where $\rho$ is
the ratio between the number of folds and the total number of performance samples.

To estimate the probability of the performance of the solution being greater than the
threshold, we first estimate the parameters $\mu$ and $\nu = \sigma^{-2}$ of the
generative model.  \textcite{Benavoli2017} consider the prior
\begin{equation*}
  \Prob(\mu, \nu \mid \mu_0, \kappa_0, a, b) = \operatorname{NG}(\mu, \nu; \mu_0, \kappa_0, a, b)\text{,}
\end{equation*}
that is a Normal-Gamma distribution with parameters $(\mu_0, \kappa_0, a, b)$.  This is a
conjugate prior to the likelihood model.  Choosing the prior parameters $\mu_0 = 0$,
$\kappa_0 \to \infty$, $a = -1/2$, and $b = 0$, the posterior distribution of $\mu$ is a
location-scale Student distribution.  Mathematically, we have
\begin{equation*}
  \Prob(\mu \mid \vec{z}, \mu_0, \kappa_0, a, b) =
    \operatorname{St}(\mu; n - 1, \bar{z}, \left(
      \frac{1}{n} + \frac{\rho}{1 - \rho}
    \right)s^2)\text{,}
\end{equation*}
where
\begin{equation*}
  \bar{z} = \frac{1}{n} \sum_{i=1}^n z_i\text{,}
\end{equation*}
and
\begin{equation*}
  s^2 = \frac{1}{n - 1} \sum_{i=1}^{n-1} (z_i - \bar{z})^2\text{.}
\end{equation*}

Thus, validating that the solution obtained by the algorithm in production will surpass
the threshold $p^{*}$ consists of calculating the probability
\begin{equation*}
  \Prob(\mu > 0 \mid \vec{z}) > \gamma\text{,}
\end{equation*}
where $\gamma$ is the confidence level.

Note that the Bayesian analysis is a more sophisticated technique than the null hypothesis
significance testing, as it allows us to estimate the probability of the performance of
the solution being better than a certain threshold.

% TODO: fix all textcite
Also, be aware that the choice of the model and the prior distribution can affect the
results.  \textcite{Benavoli2017} suggest using 10 repetitions of 10-fold cross-validation
to estimate the parameters of the generative model.  They also show experimental evidence
that their choices are robust to the choice of the prior distribution.  However, one
should be aware of the limitations of the model.

\subsection{Comparing strategies}
\label{sub:comparison}

\textcolor{red}{Talk about paired dataset samples.}

When we have two or more strategies to solve a problem, we need to compare them to see
which one is better.  This is a common situation in data science projects, as we usually
have many techniques to solve a problem.

One way to look at this problem is to consider that the algorithm\footnote{That includes
both data handling and machine learning.} $A$ has \emph{hyperparameters} $\lambda \in
\Lambda$.  A hyperparameter here is a parameter that is not learned by the algorithm, but
is set by the user.  For example, the number of neighbors in a k-NN algorithm is a
hyperparameter.  For the sake of generality, we can consider that the hyperparameters may
also include different learning algorithms or data handling pipelines.

Let us say we have a baseline algorithm $A(\lambda_0)$ --- for instance, something that is
in production, the result of the last sprint or a well-known algorithm --- and a new candidate algorithm $A(\lambda)$.
Suppose $\vec{p}(\lambda_0)$ and $\vec{p}(\lambda)$ are the performance vectors of the
baseline and the candidate algorithms, respectively, that are calculated using the same
strategy described in \cref{sec:expected-performance}.  We can validate whether the
candidate is better than the baseline by
\begin{equation*}
  \Prob(\mu > 0 \mid \vec{z}) > \gamma\text{,}
\end{equation*}
where $\vec{z}$ is now $\vec{p}(\lambda) - \vec{p}(\lambda_0)$ and $\gamma$ is the confidence
level.  Also, the expected performance gain of the candidate algorithm is $\mu$ --- if
negative, the performance loss.

This strategy can be applied iteratively to compare many algorithms.  For example, we can
compare $A(\lambda_1)$ with $A(\lambda_0)$, $A(\lambda_2)$ with $A(\lambda_1)$, and so on,
keeping the best algorithm found so far as the baseline. In the cases where the confidence
level is not reached, but the expected performance gain is positive, we can consider
additional characteristics of the algorithms, like the interpretability of the model, the
computational cost, or the ease of implementation, to decide which one is better. However,
one should pay attention whether the probability
\begin{equation*}
  \Prob(\mu < 0 \mid \vec{z})
\end{equation*}
is too high or not.  Always ask yourself if the risk of the performance loss is worth in
the real-world scenario.

\subsection{About nesting experiments}

Mathematically speaking, there is no difference between assessing the choice of
$\left[\phi, \theta\right]$ and the choice of $\lambda$.  Thus, some techniques --- like
grid search --- can be used to find the best hyperparameters using a nested experimental
plan.

The idea is the same, we assess how good is the expected choice of the
hyperparameter-optimization technique $B$ to find the appropriate hyperparameters.  Similarly,
the choice of the hyperparameters and the parameters that goes to production is the
application of $B$ to the whole dataset.  However, never use the choices of the
hyperparameters in the experimental plan to make decisions about what goes to production.
(The same is true for the parameters $\left[\phi, \theta\right]$ in the traditional case.)

\textcolor{red}{We can unnest the search by interpreting the options as different
algorithms two by two.}

% vim: spell spelllang=en


\renewcommand{\theHchapter}{A\arabic{chapter}} % workaround for hyperref
\appendix
\chapter{Preliminaries}
\label{chap:preliminaries}

\chapterprecishere{%
  Maar ik maak steeds wat ik nog niet kan om het te leeren kunnen.\par\raggedleft---
  \textup{Vincent van Gogh}, The Complete Letters of Vincent Van Gogh, Volume Three}

Foundamental concepts in data science come from a variety of fields, including
mathematics, statistics, computer science, optimization theory, and information theory.
This chapter provides a brief overview of the main computational, mathematical and statistical
concepts in data science.

The goal is not to provide a comprehensive treatment of these topics, but to consolidate
notations and definitions that are used throughout the book.  The reader is encouraged to
consult the references provided at the end of each topic for a more in-depth treatment.
Statisticians with strong programming background and computer scientists with strong
statistics background will probably not find much new here.

I first introduce the main concepts in algorithms and data structures, which are the
building blocks of computational thinking.  Then, I present the basic concepts in set
theory and linear algebra, which are important mathematical foundations for data science.
Finally, I introduce the main concepts in probability theory, the cornerstone of
statistical learning and inference.

If your are familiar with these topics, you can safely skip this chapter.  Otherwise, I
encourage you to read it carefully, as it will help you understand the rest of the book.

\begin{mainbox}{Chapter remarks}

  \boxsubtitle{Contents}

  \startcontents[chapters]
  \printcontents[chapters]{}{1}{}
  \vspace{1em}

  \boxsubtitle{Context}

  \begin{itemize}
    \item Data science relies on a variety of mathematical and computational concepts.
    \item The main concepts are algorithms, data structures, set theory, linear algebra,
      and probability theory.
  \end{itemize}

  \boxsubtitle{Objectives}

  \begin{itemize}
    \item Introduce a brief overview of the main computational, mathematical and
      statistical concepts in data science.
    \item Remind the reader the main definitions and properties of these concepts.
    \item Consolidate notations and definitions that are used throughout the book.
  \end{itemize}
\end{mainbox}

{}
\clearpage

\section{Algorithms and data structures}

Algorithms are step-by-step procedures for solving a problem.  They are used to
manipulate data structures, which are ways of organizing data to solve problems.
They are realized in programming languages, which are formal languages that can be used
to express algorithms.

My suggestion of a comprehensive book about algorithms and data structures is
\textcite{Cormen2022}\footfullcite{Cormen2022}.  An alternative for beginners is
\textcite{Guttag2021}\footfullcite{Guttag2021}.

\subsection{Computational complexity}

The computational complexity of an algorithm is the amount of resources it uses to
run as a function of the size of the input.  The most common resources are time and
space.

Usually, we are interested in the asymptotic complexity of an algorithm, i.e. how
the complexity grows as the size of the input grows.  The most common notation for
asymptotic complexity is the Big-O notation.

\paragraph{Big-O notation}  Let $f$ and $g$ be functions from the set of natural numbers
to the set of real numbers, i.e. $f, g : \mathbb{N} \rightarrow \mathbb{R}$.  We say that $f$ is
$O(g)$ if there exists a constant $c > 0$ such that $f(n) \leq c g(n)$ for all $n \geq
n_0$, where $n_0$ is a natural number.
We can order functions by their asymptotic complexity.  For example, $O(1) < O(\log n) <
O(n) < O(n \log n) < O(n^2) < O(2^n) < O(n!)$.  Throughout this book, we consider
$\log n = \log_2 n$, i.e. whenever the base of the logarithm is not specified, it is
assumed to be $2$.

The asymptotic analysis of algorithms is usually done in the worst-case scenario, i.e.
the maximum amount of resources the algorithm uses for any input of size $n$.  Thus,
it gives us an upper bound on the complexity of the algorithm.  In other words, an
algorithm with complexity $O(g(n))$ is guaranteed to run in at most $c g(n)$ time for some
constant $c$.

It does not mean, for instance, that an algorithm with time complexity $O(n)$ will always run
faster than an algorithm with time complexity $O(n^2)$, but that the former will run faster
for a large enough input size.

An important property of the Big-O notation is that
\[
  O(f) + O(g) = O(\max(f, g))\text{,}
\]
i.e. if an algorithm has two sequential steps with time complexity $O(f)$ and $O(g)$, the
highest complexity is the one that determines the overall complexity.

\subsection{Algoritmic paradigms}

Some programming techniques are used to solve a wide variety of problems.  They are called
algorithmic paradigms.  The most common ones are listed below.

\paragraph{Divide and conquer}  The problem is divided into smaller subproblems that are
solved recursively.  The solutions to the subproblems are then combined to give a solution
to the original problem.  Some example algorithms are merge sort, quick sort, and binary
search.

\begin{algobox}[label=alg:bsearch]{Binary search algorithm.}
  \KwData{A sorted array $\vec{a} = \left[a_1, a_2, \dots, a_n\right]$ and a key $x$}
  \KwResult{True if $x$ is in $\vec{a}$, false otherwise}
  $l \gets 1$\;
  $r \gets n$\;
  \While{$l \leq r$}{
    $m \gets \left\lfloor \frac{l + r}{2} \right\rfloor$\;
    \If{$x = a_m$}{
      \Return{\emph{true}}
    }
    \eIf{$x < a_m$}{
      $r \gets m - 1$\;
    }{
      $l \gets m + 1$\;
    }
  }
  \Return{\emph{false}}
  \tcblower
  An iterative algorithm that searches for a key in a sorted array.
\end{algobox}

Consider as an example the \cref{alg:bsearch} that solves the binary search problem.  Given a
$n$-elements sorted array $\vec{a} = \left[a_1, a_2, \dots, a_n\right]$, $a_1 \leq a_2 \leq
\dots \leq a_n$, and a key $x$, the algorithm returns true if $x$ is in $A$ and false
otherwise.  The algorithm works by dividing the array in half at each step and comparing
the key with the middle element.  Each time the key is not found, the search space is
reduced by half.

\begin{algobox}[label=alg:bsearch2]{Recursive binary search algorithm.}
  \SetKwProg{Fn}{function}{ is}{end}
  \Fn{$\operatorname{bsearch}(\left[a_1, a_2, \dots, a_n\right], x)$}{
    \If{$n = 0$}{
      \Return{\emph{false}}
    }
    $m \gets \left\lfloor \frac{n}{2} \right\rfloor$\;
    \If{$x = a_m$}{
      \Return{\emph{true}}
    }
    \eIf{$x < a_m$}{
      \Return{$\operatorname{bsearch}(\left[a_1, \dots, a_{m-1}\right], x)$}
    }{
      \Return{$\operatorname{bsearch}(\left[a_{m+1}, \dots, a_{n}\right], x)$}
    }
  }
  \tcblower
  A recursive algorithm that searches for a key in a sorted array.  Note that trivial
  conditions --- $n = 0$ and key found --- are handled first, so the recursion stops
  when the problem is small enough.
\end{algobox}

Divide and conquer algorithms can be implemented using recursion.  \emph{Recursion}
is also a algorithmic paradigm where a function calls itself to solve smaller instances of the
same problem.  The recursion stops when the problem is small enough to be solved directly.

\Cref{alg:bsearch2} shows a recursive implementation of the binary search algorithm.
The smaller instances, or so called \emph{base cases}, are when the array is empty or the key
is found in the middle.  Other conditions --- key is smaller or greater than the middle element ---
are handled by calling the function recursively with the left or right half of the array.

% \paragraph{Dynamic programming}  The problem is divided into overlapping subproblems, and
% the solutions to the subproblems are only solved once. The subproblems are then optimized
% to find the overall solution.  Some example algorithms are the Bellman-Ford algorithm,
% Floyd-Warshall algorithm, and the Knapsack problem.

This solution --- both algorithms --- has a worst-case time complexity of $O(\log n)$. The
search space is halved at each step, thus, in the $i$-th iteration, the remaining number
of elements in the array is $n / 2^{i-1}$.  In the worst-case, the algorithm stops when the
search space has size $1$ or smaller, i.e.
\[
  \frac{n}{2^{i-1}} = 1 \implies
  i = 1 + \log n\text{.}
\]

Note this strategy leads to such a low time complexity that we can solve large instances
of the problem in a reasonable amount of time.  Consider the case of an array with $2^{64}
= 18{,}446{,}744{,}073{,}709{,}551{,}616$ elements, the algorithm will find the key in at most $65$
steps.

\paragraph{Greedy algorithms}  The problem is solved with incremental steps, each of which
is locally optimal.  The overall solution is not guaranteed to (but might) be optimal.  Some example
algorithms are Dijkstra's algorithm and Prim's algorithm.  Greedy algorithm are usually
very efficient in terms of time complexity --- see more in the following.

One example of suboptimal greedy algorithm is a heuristic solution for the knapsack problem.
The \emph{knapsack problem} is a combinatorial optimization problem where the goal is to
maximize the value of items in a knapsack without exceeding its capacity.  The problem is
mathematically defined as
\[
  \text{maximize }~\sum_{i = 1}^n v_i x_i\text{,}
\]
\[
  \text{subject to }~\sum_{i = 1}^n w_i x_i \leq W\text{,}
\]
where $v_i$ is the value of item $i$, $w_i$ is the weight of item $i$, $x_i$ is a binary
variable that indicates if item $i$ is in the knapsack, and $W$ is the capacity of the
knapsack.

\begin{algobox}[label=alg:knapsack]{Heuristic solution for the knapsack problem.}
  \KwData{A list of $n$ items, each with a value $v_i$ and a weight $w_i$, and a capacity $W$}
  \KwResult{The binary variable $x_i$ for each item $i$ that maximizes the total value}
  Sort the items in decreasing order of value\;
  $V \gets 0$\;
  $x_i \gets 0,~\forall i$\;
  \For{$i \gets 1$ \KwTo $n$}{
    \If{$w_i \leq W$}{
      $x_i \gets 1$\;
      $V \gets V + v_i$\;
      $W \gets W - w_i$\;
    }
  }
  \Return{$x_i,~\forall i$}
  \tcblower
  A greedy algorithm that solves suboptimally the knapsack problem.  The algorithm iterates over
  the items in decreasing order of value and puts the item in the knapsack if it fits.
\end{algobox}

An algorithm that finds a suboptimal solution for the knapsack problem is shown in
\cref{alg:knapsack}.
It iterates over the items in decreasing order of value and puts the item in the
knapsack if it fits.  The algorithm is suboptimal because there might exist small-value
items that, when combined, would fit in the knapsack and yield a higher total value.

The most costly operation in the algorithm is the sorting of the items in
decreasing order of value, which has a time complexity\footnote{%
Considering the worst-case time complexity of the sorting algorithm, consult
\fullcite{Cormen2022} for more details.} of $O(n \log n)$.

\paragraph{Brute force}  The problem is solved by trying all possible solutions.  Most of
the time, brute force algorithms have exponential time complexity, leading to impractical
solutions for large instances of the problem.  On the other hand, brute force algorithms
are usually easy to implement and understand, as well as guaranteed to find the optimal
solution.

In the previous example, a brute force algorithm for the knapsack problem would
try all possible combinations of items and select the one that maximizes the total value
without exceeding the capacity.  One can easily see that the time complexity of such an
algorithm is $O(2^n)$, where $n$ is the number of items, as there are $2^n$ possible
combinations of items.  Such an \emph{exhaustive search} is impractical for large $n$,
but it is guaranteed to find the optimal solution.

One should avoid brute force algorithms whenever possible, as they are usually too costly
to be practical.  However, they are useful for small instances of the problem, for
verification of the results of other algorithms, and for educational purposes.

\paragraph{Backtracking}  The problem is solved incrementally, one piece at a time.  If a
piece does not fit, it is removed and replaced by another piece.  Some example algorithms
are the naïve solutions for N-queens problem and for the Sudoku problem.  Backtracking, as
a special case of brute force, often leads to exponential (or worse) time complexity.

Many times, backtracking algorithms are combined with other techniques to reduce the
search space and make the algorithm more efficient.  For example, the backtracking
algorithm for the Sudoku problem is combined with constraint propagation to reduce the
number of possible solutions.

A Sudoku puzzle consists of an $n \times n$ grid, divided into $n$ subgrids of size
$\sqrt{n} \times \sqrt{n}$.  The goal is to fill the grid with numbers from $1$ to $n$
such that each row, each column, and each subgrid contains all numbers from $1$ to $n$ but
no repetitions.  The most common grid size is $9 \times 9$.

\begin{figurebox}[label=fig:sudoku]{Backtracking to solve a Sudoku puzzle.}
  \centering
  \begin{tikzpicture}
    % a tree like structure with the Sudoku puzzle at the root, and tries to fill each cell
    % with a number, backtracking when a number does not fit

    % an almost complete Sudoku puzzle
    \node (root) at (1, 0) {$\begin{array} {|c|c|c|c|}
      \hline
      ? &   &   & 1 \\
      \hline
      1 & 2 & 4 &   \\
      \hline
        & 4 & 1 & 2 \\
      \hline
      2 &   &   &   \\
      \hline
      \end{array}$};
    % node with first cell filled with 1
    \node (n1) at (-3.6, -3) {$\begin{array} {|c|c|c|c|}
      \hline
      \color{gray} 1 &   &   & \color{gray} 1 \\
      \hline
      \color{gray} 1 & 2 & 4 &   \\
      \hline
        & 4 & 1 & 2 \\
      \hline
      2 &   &   &   \\
      \hline
      \end{array}$};
    % node with first cell filled with 2
    \node (n2) at (-1.2, -3) {$\begin{array} {|c|c|c|c|}
      \hline
      \color{gray} 2  &   &   & 1 \\
      \hline
      1 & \color{gray} 2 & 4 &   \\
      \hline
        & 4 & 1 & 2 \\
      \hline
      \color{gray} 2 &   &   &   \\
      \hline
      \end{array}$};
    % node with first cell filled with 3
    \node (n3) at (1.2, -5) {$\begin{array} {|c|c|c|c|}
      \hline
      3  & ? &   & 1 \\
      \hline
      1 & 2 & 4 &   \\
      \hline
        & 4 & 1 & 2 \\
      \hline
      2 &   &   &   \\
      \hline
      \end{array}$};
    % node with first cell filled with 4
    \node (n4) at (3.6, -3) {$\begin{array} {|c|c|c|c|}
      \hline
      4  & ?  &   & 1 \\
      \hline
      1 & 2 & 4 &   \\
      \hline
        & 4 & 1 & 2 \\
      \hline
      2 &   &   &   \\
      \hline
      \end{array}$};
    % node with the second cell filled with 1
    \node (n5) at (-3.6, -8) {$\begin{array} {|c|c|c|c|}
      \hline
      3 & \color{gray} 1 &   & \color{gray} 1 \\
      \hline
      \color{gray} 1 & 2 & 4 &   \\
      \hline
        & 4 & 1 & 2 \\
      \hline
      2 &   &   &   \\
      \hline
      \end{array}$};
    % node with the second cell filled with 2
    \node (n6) at (-1.2, -8) {$\begin{array} {|c|c|c|c|}
      \hline
      3 & \color{gray} 2 &   & 1 \\
      \hline
      1 & \color{gray} 2 & 4 &   \\
      \hline
        & 4 & 1 & 2 \\
      \hline
      2 &   &   &   \\
      \hline
      \end{array}$};
    % node with the second cell filled with 3
    \node (n7) at (1.2, -8) {$\begin{array} {|c|c|c|c|}
      \hline
      \color{gray} 3 & \color{gray} 3 &   & 1 \\
      \hline
      1 & 2 & 4 &   \\
      \hline
        & 4 & 1 & 2 \\
      \hline
      2 &   &   &   \\
      \hline
      \end{array}$};
    % node with the second cell filled with 4
    \node (n8) at (3.6, -8) {$\begin{array} {|c|c|c|c|}
      \hline
      3 & \color{gray} 4 &   & 1 \\
      \hline
      1 & 2 & 4 &   \\
      \hline
        & \color{gray} 4 & 1 & 2 \\
      \hline
      2 &   &   &   \\
      \hline
      \end{array}$};
    % etc
    \node (dots) at (3.6, -5) {\dots};
    % arrows
    \draw[-Stealth] (root.west) -- (n1.north);
    \draw[-Stealth] (n1.north) -- (root.south west);
    \draw[-Stealth] (root.south west) -- (n2.north);
    \draw[-Stealth] (n2.north) -- (root.south);
    \draw[-Stealth] (root.south) -- (n3.north west);

    \draw[-Stealth] (n3.north west) -- (n5.north);
    \draw[-Stealth] (n5.north) -- (n3.south west);
    \draw[-Stealth] (n3.south west) -- (n6.north);
    \draw[-Stealth] (n6.north) -- (n3.south);
    \draw[-Stealth] (n3.south) -- (n7.north);
    \draw[-Stealth] (n7.north) -- (n3.south east);
    \draw[-Stealth] (n3.south east) -- (n8.north);
    \draw[-Stealth] (n8.north) -- (n3.north east);

    \draw[-Stealth] (n3.north east) -- (root.south east);
    \draw[-Stealth] (root.south east) -- (n4.north);

    \draw[-Stealth] (n4.south) -- (dots);
  \end{tikzpicture}
  \tcblower
  A Sudoku puzzle --- in this case, $4 \times 4$ --- is solved by trying all possible
  numbers in each cell and backtracking when a number does not fit.  The solution is found
  when all cells are filled and the constraints are satisfied.  Arrows indicate the
  backtracking steps.  The question mark indicates an empty cell that needs to be filled
  at that step.  Constraints violation are shown in gray.
\end{figurebox}

An illustration of backtracking to solve a $4 \times 4$ Sudoku puzzle\footnote{%
Smaller puzzles are more didactic, but the same principles apply to larger puzzles.}
is shown in \cref{fig:sudoku}.
The puzzle is solved by trying all possible numbers in each cell and backtracking when a
number does not fit.  The solution is found when all cells are filled and the constraints
are satisfied.  Arrows indicate the the steps of the backtracking algorithm.  Every time a
constraint is violated --- indicated in gray ---, the algorithm backtracks to the previous
cell and tries a different number.

One can easily see that a puzzle with $m$ missing cells has $n^m$ possible solutions.  For
small values of $m$ and $n$, the algorithm is practical, but for large values, it becomes
too costly.

\subsection{Data structures}

Data structures are ways of organizing data to solve problems.  The most common ones
are listed below.  A comprehensive material about the properties and implementations of
data structures can be found in \textcite{Cormen2022}\footfullcite{Cormen2022}.

\paragraph{Arrays}  An array is a homogeneous collection of elements that are accessed by
an integer index.  The elements are usually stored in contiguous memory locations.
In the scope of this book, it is equivalent to a mathematical vector whose element's type
are not necessarily numerical.  Thus, a $n$-elements array $\vec{a}$ is denoted by
$\left[a_1, a_2, \dots, a_n\right]$, where the $i$ in $a_i$ is the index of the element.

% \paragraph{Linked lists}  A linked list is a collection of elements called nodes.  Each
% node contains a value and a pointer to the next node in the list.  The first node is
% called the head, and the last node is called the tail.  The tail points to a null
% reference.

\paragraph{Stacks}  A stack is a collection of elements that are accessed in a
\gls{lifo} order.  Elements are added to the top of the stack and
removed from the top of the stack.  In other words, only two operations are
allowed: push (add an element to the top of the stack) and pop (remove the top element).
Only the top element is accessible.

\paragraph{Queues}  A queue is a collection of elements that are accessed in a
\gls{fifo} order.  Elements are added to the back of the queue and
removed from the front of the queue.  The two operations allowed are enqueue (add an
element to the back of the queue) and dequeue (remove the front element).  Only the
front and back elements are accessible.

\paragraph{Trees}  A tree is a collection of nodes.  Each node contains a value and a list
of references to its children.  The first node is called the root.  A node with no
children is called a leaf.  No cycles are allowed in a tree, i.e. a child cannot be an
ancestor of its parent. The most common type of tree is the binary tree, where each node
has at most two children.

Mathematically, a binary tree is a recursive data structure.  A binary tree is either
empty or consists of a root node and two binary trees, called the left and right
children.   Thus, a binary tree $T$ is
\[
  T = \begin{cases}
    \emptyset & \text{if it is empty, or} \\
    \left(v, T_l, T_r\right) & \text{if it has a value $v$ and two children $T_l$ and $T_r$.}
  \end{cases}
\]
Note that the left and right children are themselves binary trees.  If $T$ is a leaf,
then $T_l = T_r = \emptyset$.

This properties makes it easy to represent a binary tree using parentheses notation.  For
example, $(1, (2, \emptyset, \emptyset), (3, \emptyset, \emptyset))$ is a binary tree
with root $1$, left child $2$, and right child $3$.

\paragraph{Graphs}  A graph is also a collection of nodes.  Each node contains a value and a
list of references to its neighbors; the references are called edges.  A graph can be
directed or undirected.  A graph is directed if the edges have a direction.

Mathematically, a graph is a pair $G = (V, E)$, where $V$ is a set of vertices and $E \in
V \times V$ is a set of edges.  An edge is a pair of vertices, i.e. $e = (v_i, v_j)$,
where $v_i, v_j \in V$. If the graph is directed, the edge is an ordered pair, i.e. $e =
(v_i, v_j) \neq (v_j, v_i)$.

Not only each node can hold a value, but also each edge can have a weight.  A weighted
graph is a graph where there exists a function $w : E \rightarrow \mathbb{R}$ that assigns
a real number to each edge.

\begin{figurebox}[label=fig:graph]{A graph with four vertices and five edges.}
  \centering
  \begin{tikzpicture}
    % a graph with four vertices and five edges
    \node (v1) at (0, 0) {1};
    \node (v2) at (2, 0) {2};
    \node (v3) at (2, 2) {3};
    \node (v4) at (0, 2) {4};
    \draw[->] (v1) -- (v2);
    \draw[->] (v2) -- (v3);
    \draw[->] (v3) -- (v4);
    \draw[->] (v4) -- (v1);
    \draw[->] (v1) -- (v3);
  \end{tikzpicture}
  \tcblower
  A graph with four vertices and five edges.  Vertices are numbered from $1$ to $4$, and
  edges are represented by arrows.  The graph is directed, as the edges have a direction.
\end{figurebox}

A graphical representation of a directed graph with four vertices and five edges is shown
in \cref{fig:graph}.  The vertices are numbered from $1$ to $4$, and the edges are
represented by arrows.

Another common representation of a graph is the adjacency matrix.  An adjacency matrix is
a square matrix $A$ of size $n \times n$, where $n$ is the number of vertices.  The
$i, j$-th entry of the matrix is $1$ if there is an edge from vertex $i$ to vertex $j$,
and $0$ otherwise.  The adjacency matrix of the graph in \cref{fig:graph} is
\[
  A = \begin{pmatrix}
    0 & 1 & 1 & 0 \\
    0 & 0 & 1 & 0 \\
    0 & 0 & 0 & 1 \\
    1 & 0 & 0 & 0
  \end{pmatrix}\text{.}
\]

% \paragraph{Map} A map is a collection of key-value pairs.  The keys are unique, and
% each key is associated with a value.  The keys are used to access the values.

\section{Set theory}

A set is a collection of elements.  The elements of a set can be anything, including
other sets.  The elements of a set are unordered, and each element is unique.  The
most common notation for sets is the curly braces notation, e.g. $\{1, 2, 3\}$.

Some special sets are listed below.

\paragraph{Universe set}  The universe set is the set of all elements in a given context.
It is denoted by $\Omega$.

\paragraph{Empty set}  The empty set is the set with no elements.  It is denoted by
the symbol $\emptyset$.  Depending on the context, it can also be denoted by $\{\}$.

\subsection{Set operations}

The basic operations on sets are union, intersection, difference, and complement.

\paragraph{Union}  The union of two sets $A$ and $B$ is the set of elements that are in
$A$ or $B$.  It is denoted by $A \cup B$.  For example, the union of $\{1, 2, 3\}$ and
$\{3, 4, 5\}$ is $\{1, 2, 3, 4, 5\}$.

\paragraph{Intersection}  The intersection of two sets $A$ and $B$ is the set of elements
that are in both $A$ and $B$.  It is denoted by $A \cap B$.  For example, the intersection
of $\{1, 2, 3\}$ and $\{3, 4, 5\}$ is $\{3\}$.

\paragraph{Difference}  The difference of two sets $A$ and $B$ is the set of elements
that are in $A$ but not in $B$.  It is denoted by $A \setminus B$.  For example, the
difference of $\{1, 2, 3\}$ and $\{3, 4, 5\}$ is $\{1, 2\}$.

\paragraph{Complement}  The complement of a set $A$ is the set of elements that are not
in $A$.  It is denoted by $A^c = \Omega \setminus A$.

\paragraph{Inclusion}  Inclusion is a relation between sets.  A set $A$ is included in a
set $B$ if all elements of $A$ are also elements of $B$.  It is denoted by $A \subseteq B$.

\subsection{Set operations properties}

Union and intersection are commutative, associative and distributive.  Thus, given sets
$A$, $B$, and $C$, the following statements hold:
\begin{itemize}
  \item \emph{Commutativity:} $A \cup B = B \cup A$ and $A \cap B = B \cap A$;
  \item \emph{Associativity:} $(A \cup B) \cup C = A \cup (B \cup C)$ and $(A \cap B) \cap C = A \cap (B \cap C)$;
  \item \emph{Distributivity:} $A \cup (B \cap C) = (A \cup B) \cap (A \cup C)$ and $A \cap (B \cup C) = (A \cap B) \cup (A \cap C)$.
\end{itemize}

The difference operation can be expressed in terms of union and intersection as
\[
  A \setminus B = A \cap B^c\text{.}
\]

The complement of the union of two sets is the intersection of their complements, i.e.
\[
  (A \cup B)^c = A^c \cap B^c\text{.}
\]
Similarly, the complement of the intersection of two sets is the union of their complements, i.e.
\[
  (A \cap B)^c = A^c \cup B^c\text{.}
\]
This property is known as De Morgan's laws.

In terms of inclusion, given sets $A$, $B$, and $C$, the following statements hold:
\begin{itemize}
  \item \emph{Reflexity:} $A \subseteq A$;
  \item \emph{Antisymmetry:} $A \subseteq B$ and $B \subseteq A$ if and only if $A = B$;
  \item \emph{Transitivity:} $A \subseteq B$ and $B \subseteq C$ implies $A \subseteq C$.
\end{itemize}

\subsection{Relation to Boolean algebra}

Set operations are closely related to Boolean algebra.  In Boolean algebra, the elements
of a set are either true or false, many times represented by $1$ and $0$, respectively.
The union operation is equivalent to the logical OR operation, expressed by the symbol
$\lor$; and the intersection operation is equivalent to the logical AND operation,
expressed by the symbol $\land$.  The compliment operation is equivalent to the logical
NOT operation, expressed by the symbol $\lnot$.

The distributive property of set operations is equivalent to the distributive property of
Boolean algebra.  Important properties like De Morgan's laws also hold in Boolean algebra,
i.e. $\lnot (A \lor B) = \lnot A \land \lnot B$ and $\lnot (A \land B) = \lnot A \lor
\lnot B$.

Boolean algebra is the foundation of digital electronics and computer science.  The
logical operations are implemented in hardware using logic gates, and the logical
operations are used in programming languages to control the flow of a program.

Reader interested in more details about Boolean algebra and Discrete Mathematics
should consult \textcite{Rosen2018}\footfullcite{Rosen2018}.

\section{Linear algebra}

Linear algebra is the branch of mathematics that studies vector spaces and linear
transformations.  It is a fundamental tool in many areas of science and engineering.
The basic objects of linear algebra are vectors and matrices.  A common textbook that
covers the subject in depth is \textcite{Strang2023}\footfullcite{Strang2023}.

\paragraph{Vector}  A vector is an ordered collection of numbers.  It is denoted by a bold
lowercase letter, e.g. $\vec{v} = [v_i]_{i= 1,\dots, n}$ is a vector of length $n$.

\paragraph{Matrix}  A matrix is a rectangular collection of numbers.  It is denoted by an
uppercase letter, e.g. $A = (a_{ij})_{i = 1, \dots, n;~j = 1, \dots, m}$ is the matrix
with $n$ rows and $m$ columns.

\paragraph{Tensor}  Tensors are generalizations of vectors and matrices.  A tensor of rank
$k$ is a multidimensional array with $k$ indices.  Scalars are tensors of rank $0$,
vectors are tensors of rank $1$, and matrices are tensors of rank $2$.  Tensors are
commonly used in machine learning and physics.

\subsection{Operations}

The main operations in linear algebra are presented below.

\paragraph{Addition}  The sum of two vectors $\vec{v}$ and $\vec{w}$ is the vector
$\vec{v} + \vec{w}$ whose $i$-th entry is $v_i + w_i$.  The sum of two matrices $A$ and
$B$ is the matrix $A + B$ whose $i, j$-th entry is $a_{ij} + b_{ij}$.  (The same rules apply
to subtraction.)

\paragraph{Scalar multiplication}  The product of a scalar $\alpha$ and a vector $\vec{v}$
is the vector $\alpha \vec{v}$ whose $i$-th entry is $\alpha v_i$.  Similarly, the product of a
scalar $\alpha$ and a matrix $A$ is the matrix $\alpha A$ whose $i, j$-th entry is
$\alpha a_{ij}$.

\paragraph{Dot product}  The dot product of two vectors $\vec{v}$ and $\vec{w}$ is the
scalar $$\vec{v} \cdot \vec{w} = \sum_{i = 1}^n v_i w_i\text{.}$$  The dot product is also called
the inner product.

\paragraph{Matrix multiplication}  The product of two matrices $A$ and $B$ is the matrix
$C = A B$ whose $i, j$-th entry is $$c_{ij} = \sum_{k = 1}^n a_{ik} b_{kj}\text{.}$$
The number of columns of $A$ must be equal to the number of rows of $B$, and the
resulting matrix $C$ has the same number of rows as $A$ and the same number of columns as $B$.
Unless otherwise stated, we consider the vector $\vec{v}$ with length $n$ as a column
matrix, i.e. a matrix with one column and $n$ rows.

\paragraph{Transpose}  The transpose of a matrix $A$ is the matrix $A^T$ whose $i, j$-th
entry is the $j, i$-th entry of $A$.  If $A$ is a square matrix, then $A^T$ is the
matrix obtained by reflecting $A$ along its main diagonal.

\paragraph{Determinant}  The determinant of a square matrix $A$ is a scalar that is a
measure of the (signed) volume of the parallelepiped spanned by the columns of $A$.  It is
denoted by $\det(A)$ or $|A|$.

The determinant is nonzero if and only if the matrix is invertible and the linear map
represented by the matrix is an isomorphism -- i.e. it preserves the dimension of the
vector space.  The determinant of a product of matrices is the product of their
determinants.

Particularly, the determinant of a $2 \times 2$ matrix $\begin{pmatrix} a & b \\ c & d
\end{pmatrix}$ is $$\begin{vmatrix} a & b \\ c & d \end{vmatrix} = ad - bc\text{.}$$

\paragraph{Inverse matrix}  An $n \times n$ matrix $A$ has an inverse $n \times n$ matrix
$A^{-1}$ if
\[
  A A^{-1} = A^{-1} A = I_n\text{,}
\]
where $I_n$ is the $n \times n$ identity matrix, i.e. a matrix whose diagonal entries are
$1$ and all other entries are $0$. If such a matrix exists, $A$ is said
\emph{invertible}.  A square matrix that is not invertible is called singular. A square
matrix with entries in a field is singular if and only if its determinant is zero.

To calculate the inverse of a matrix, we can use the formula
\[
  A^{-1} = \frac{1}{\det(A)} \operatorname{adj}(A)\text{,}
\]
where $\operatorname{adj}(A)$ is the adjugate (or adjoint) of $A$, i.e. the transpose of the cofactor matrix
of $A$.

The cofactor of the $i, j$-th entry of a matrix $A$ is the determinant of the matrix
obtained by removing the $i$-th row and the $j$-th column of $A$, multiplied by $(-1)^{i
+ j}$.

In the case of a $2 \times 2$ matrix, the inverse is
\[
  \begin{pmatrix}
    a & b \\
    c & d
  \end{pmatrix}^{-1} = \frac{1}{ad - bc}
  \begin{pmatrix}
    d & -b \\
    -c & a
  \end{pmatrix}\text{.}
\]

\subsection{Systems of linear equations}

A system of linear equations is a collection of linear equations that share their
unknowns.  It is usually written in matrix form as $A \vec{x} = \vec{b}$, where $A$ is a
matrix of constants, $\vec{x}$ is a vector of unknowns, and $\vec{b}$ is a vector of
constants.

The system has a unique solution if and only if the matrix $A$ is invertible.  The
solution is $\vec{x} = A^{-1} \vec{b}$.

% \subsection{Matrix decompositions}
%
% Matrix decompositions are factorizations of matrices into matrices with special
% properties.  They are used to solve linear systems, compute inverses, and compute
% eigenvalues and eigenvectors.
%
% \textcolor{red}{Verify!}
%
% \paragraph{Singular value decomposition} The singular value decomposition (SVD) of a
% matrix $A$ is a factorization of the form
% \begin{equation}
%   \label{eq:svd}
%   A = U \Sigma V^T\text{,}
% \end{equation}
% where $U$ and $V$ are orthogonal matrices and $\Sigma$ is a diagonal
% matrix with non-negative real numbers on the diagonal.  The singular values are the
% diagonal entries of $\Sigma$.
%
% \paragraph{Eigenvalue decomposition}  The eigenvalue decomposition of a matrix $A$
% is a factorization of the form
% \begin{equation}
%   \label{eq:eigdec}
%   A = Q \Lambda Q^{-1}\text{,}
% \end{equation}
% where $Q$ is a square matrix whose columns are the eigenvectors of $A$, and
% $\Lambda$ is a diagonal matrix whose diagonal entries are the eigenvalues of
% $A$.
%
% \paragraph{Cholesky decomposition}  The Cholesky decomposition of a positive-definite
% matrix $A$ is a factorization of the form
% \begin{equation}
%   \label{eq:chol}
%   A = L L^T\text{,}
% \end{equation}
% where $L$ is a lower triangular matrix with real and positive diagonal entries.
%
% \paragraph{QR decomposition}  The QR decomposition of a matrix $A$ is a
% factorization of the form
% \begin{equation}
%   \label{eq:qr}
%   A = Q R\text{,}
% \end{equation}
% where $Q$ is an orthogonal matrix and $R$ is an upper triangular matrix.
%
% \paragraph{LU decomposition}  The LU decomposition of a square matrix $A$ is a
% factorization of the form
% \begin{equation}
%   \label{eq:lu}
%   A = L U\text{,}
% \end{equation}
% where $L$ is a lower triangular matrix with unit diagonal entries and $U$ is
% an upper triangular matrix.

\subsection{Eigenvalues and eigenvectors}

An eigenvalue of an $n \times n$ square matrix $A$ is a scalar $\lambda$ such that there exists a
non-zero vector $\vec{v}$ satisfying
\begin{equation}
  \label{eq:eig}
  A \vec{v} = \lambda \vec{v}\text{.}
\end{equation}
The vector $\vec{v}$ is called an eigenvector of $A$ corresponding to $\lambda$.

The eigenvalues of a matrix are the roots of its characteristic polynomial, i.e. the
roots of the polynomial $\det(A - \lambda I_n) = 0$, where $I_n$ is the $n \times n$ identity matrix.

\section{Probability}

Probability is the branch of mathematics that studies the likelihood of events.  It is
used to model uncertainty and randomness.  The basic objects of probability are events
and random variables.

For a comprehensive material about probability theory, the reader is referred to
\textcite{Ross2019}\footfullcite{Ross2019} and \textcite{Ross2023}\footfullcite{Ross2023}.

\subsection{Axioms of probability and main concepts}

The Kolmogorov axioms of probability are the foundation of probability theory.
They are
\begin{enumerate}
  \item The probability of an event $A$ is a non-negative real number, i.e. $\Prob(A) \geq 0$;
  \item The probability of the sample space $\Omega$\footnote{The set of all possible
    events.} is one, i.e. $\Prob(\Omega) = 1$; and
  \item The probability of the union of disjoint events, $A \cap B = \emptyset$, is
    the sum of the probabilities of the events, i.e. $\Prob(A \cup B) = \Prob(A) + \Prob(B)$.
\end{enumerate}

\paragraph{Sum rule}
A particular consequence of the third axiom is the addition law of probability.
If $A$ and $B$ are not disjoint, then
\begin{equation*}
  \Prob(A \cup B) = \Prob(A) + \Prob(B) - \Prob(A \cap B)\text{.}
\end{equation*}

\paragraph{Joint probability}

The joint probability of two events $A$ and $B$ is the probability that both events
occur.  It is denoted by $\Prob(A, B) = \Prob(A \cap B)$.

\paragraph{Law of total probability}

The law of total probability states that if $B_1, \dots, B_n$ are disjoint events
such that $\cup_{i = 1}^n B_i = \Omega$, then for any event $A$, we have that
$$\Prob(A) = \sum_{i = 1}^n \Prob(A, B_i)\text{.}$$

\paragraph{Conditional probability}

The conditional probability of an event $A$ given an event $B$ is the probability
that $A$ occurs given that $B$ occurs.  It is denoted by $\Prob(A \mid B)$.

\paragraph{Independence}

Two events $A$ and $B$ are independent if the probability of $A$ given $B$ is the
same as the probability of $A$, i.e. $\Prob(A \mid B) = \Prob(A)$.  It is equivalent to
$\Prob(A, B) = \Prob(A) \cdot \Prob(B)$.

\paragraph{Bayes' rule}

Bayes' rule is a formula that relates the conditional probability of an event $A$
given an event $B$ to the conditional probability of $B$ given $A$.  It is
\begin{equation}
  \label{eq:bayes}
  \Prob(A \mid B) = \frac{\Prob(B \mid A) \cdot \Prob(A)}{\Prob(B)}\text{.}
\end{equation}
Bayes' rule is one of the most important formulas in probability theory and is used
in many areas of science and engineering.  Particularly, for data science, it is
used in Bayesian statistics and machine learning.

\subsection{Random variables}

A random variable is a function that maps the sample space $\Omega$ to the real
numbers.  It is denoted by a capital letter, e.g. $X$.

Formally, let $X : \Omega \rightarrow E$ be a random variable.  The
probability that $X$ takes on a value in a set $A \subseteq E$ is
\begin{equation}
  \label{eq:rv}
  \Prob(X \in A) = \Prob(\{\omega \in \Omega : X(\omega) \in A\})\text{.}
\end{equation}

If $E = \mathbb{R}$, then $X$ is a continuous random variable.  If $E = \mathbb{Z}$,
then $X$ is a discrete random variable.  The random variable $X$ is said to follow
a certain probability distribution $P$ --- denoted by $X \sim P$ --- given by its
probability mass function or probability density function --- see below.

\paragraph{Probability mass function}

The \gls{pmf} of a discrete random variable $X$ is the
function $p_X : \mathbb{Z} \rightarrow [0, 1]$ defined by
\begin{equation}
  \label{eq:pmf}
  p_X(x) = \Prob(X = x)\text{.}
\end{equation}

\paragraph{Probability density function}

The \gls{pdf} of a continuous random variable $X$ is the
function $f_X : \mathbb{R} \rightarrow [0, \infty)$ defined by
\begin{equation}
  \label{eq:pdf}
  \Prob(a \leq X \leq b) = \int_a^b f_X(x) dx\text{.}
\end{equation}

\paragraph{Cumulative distribution function}

The \gls{cdf} of a random variable $X$ is the function
$F_X : \mathbb{R} \rightarrow [0, 1]$ defined by
\begin{equation}
  \label{eq:cdf}
  F_X(x) = \Prob(X \leq x)\text{.}
\end{equation}

\subsection{Expectation and moments}

Expectation is a measure of the average value of a random variable.  Moments are
measures of the shape of a probability distribution.

\paragraph{Expectation}  The expectation of a random variable $X$ is the average
value of $X$.  It is denoted by $\E[X]$.  By definition, it is
\begin{equation*}
  \E[X] = \sum_{x} x \cdot p_X(x)\text{,}
\end{equation*}
if $X$ is discrete, or
\begin{equation*}
  \E[X] = \int_{-\infty}^{\infty} x \cdot f_X(x) dx\text{,}
\end{equation*}
if $X$ is continuous.

The main properties of expectation are listed below.

The expectation operator is linear.  Given two random variables $X$ and $Y$ and a real
number $c$, we have
\begin{equation*}
  \E[c X] = c \E[X]\text{,}
\end{equation*}
\begin{equation*}
  \E[X + c] = \E[X] + c\text{,}
\end{equation*}
and
\begin{equation*}
  \E[X + Y] = \E[X] + \E[Y]\text{.}
\end{equation*}

Under a more general setting, given a function $g : \mathbb{R} \rightarrow \mathbb{R}$,
the expectation of $g(X)$ is
\begin{equation*}
  \E[g(X)] = \sum_{x} g(x) \cdot p_X(x)\text{,}
\end{equation*}
if $X$ is discrete, or
\begin{equation*}
  \E[g(X)] = \int_{-\infty}^{\infty} g(x) \cdot f_X(x) dx\text{,}
\end{equation*}
if $X$ is continuous.

\paragraph{Variance}  The variance of a random variable $X$ is a measure of how
spread out the values of $X$ are.  It is denoted by $\Var(X)$.  By definition, it is
\begin{equation}
  \label{eq:variance}
  \Var(X) = \E\!\left[\left(X - \E[X]\right)^2\right]\text{.}
\end{equation}

Note that, as a consequence, the expectation of $X^2$ --- called second moment --- is
\[
  \E[X^2] = \Var(X) + \E[X]^2\text{,}
\]
since
\begin{align*}
  \Var(X)
    &= \E\!\left[\left(X - \E[X]\right)^2\right] \\
    &= \E\!\left[X^2 - 2 X \E[X] + \E[X]^2\right] \\
    &= \E[X^2] - 2 \E[X] \E[X] + \E[X]^2 \\
    &= \E[X^2] - \E[X]^2\text{.}
\end{align*}

Higher moments are defined similarly.  The $k$-th moment of $X$ is
\begin{equation*}
  \E[X^k] = \sum_{x} x^k \cdot p_X(x)\text{,}
\end{equation*}
if $X$ is discrete, or
\begin{equation*}
  \E[X^k] = \int_{-\infty}^{\infty} x^k \cdot f_X(x) dx\text{,}
\end{equation*}
if $X$ is continuous.

\paragraph{Sample mean}  The sample mean is the average of a sample of random variables.
Given a sample $X_1, \dots, X_n$ such that $X_i \sim X$ for all $i$, the sample mean is
\begin{equation*}
  \bar{X} = \frac{1}{n} \sum_{i = 1}^n X_i\text{.}
\end{equation*}

\paragraph{Law of large numbers}  The law of large numbers states that the average of
a large number of independent and identically distributed (i.i.d.) random variables converges
to the expectation of the random variable.  Mathematically,
\begin{equation*}
  \lim_{n \rightarrow \infty} \frac{1}{n} \sum_{i = 1}^n X_i = \E[X]\text{,}
\end{equation*}
given $X_i \sim X$ for all $i$.

\paragraph{Sample variance}  The sample variance is a measure of how spread out the
values of a sample are.  Given a sample $X_1, \dots, X_n$ such that $X_i \sim X$ for all
$i$, the sample variance is
\begin{equation*}
  S^2 = \frac{1}{n - 1} \sum_{i = 1}^n (X_i - \bar{X})^2\text{.}
\end{equation*}
Note that the denominator is $n - 1$ instead of $n$ to correct the bias of the sample
variance.

\paragraph{Sample standard deviation}  The sample standard deviation is the square root
of the sample variance, i.e. $S = \sqrt{S^2}$.

\paragraph{Sample skewness}  The skewness is a measure of the asymmetry of a probability
distribution.  The sample skewness is based on the third moment of the sample.  Given a
sample $X_1, \dots, X_n$ such that $X_i \sim X$ for all $i$, the sample skewness is
\begin{equation*}
  \text{Skewness} = \frac{\frac{1}{n} \sum_{i = 1}^n (X_i - \bar{X})^3}{S^3}\text{.}
\end{equation*}
Skewness is zero for a symmetric distribution, positive for a right-skewed distribution,
and negative for a left-skewed distribution.

\paragraph{Sample kurtosis}  The kurtosis is a measure of the tailedness of a probability
distribution.  The sample kurtosis is based on the fourth moment of the sample.  Given a
sample $X_1, \dots, X_n$ such that $X_i \sim X$ for all $i$, the sample kurtosis is
\begin{equation*}
  \text{Kurtosis} = \frac{\frac{1}{n} \sum_{i = 1}^n (X_i - \bar{X})^4}{S^4} - 3\text{.}
\end{equation*}
Kurtosis is positive if the tails are heavier than a normal distribution, and negative if
the tails are lighter.

\subsection{Common probability distributions}

Several phenomena in nature and society can be modeled as random variables.  Some
distributions are frequently used to model these phenomena.  The main ones are
listed below.

\paragraph{Bernoulli distribution}  The Bernoulli distribution is a discrete
distribution with two possible outcomes, usually called success and failure.  It is
parametrized by a single parameter $p \in [0, 1]$, which is the probability of
success.  It is denoted by $\text{Bern}(p)$.

The expected value of $X \sim \text{Bern}(p)$ is $\E[X] = p$, and the variance is
$\Var(X) = p(1 - p)$.

\paragraph{Poisson distribution}  The Poisson distribution is a discrete distribution
that models the number of events occurring in a fixed interval of time or space.  It is
parametrized by a single parameter $\lambda > 0$, which is the average number of events
in the interval.  It is denoted by $\text{Poisson}(\lambda)$.

The probability mass function of $X \sim \text{Poisson}(\lambda)$ is
\begin{equation}
  \label{eq:poisson}
  p_X(x) = \frac{e^{-\lambda} \lambda^x}{x!}\text{.}
\end{equation}

The expected value of $X \sim \text{Poisson}(\lambda)$ is $\E[X] = \lambda$, and the
variance is $\Var(X) = \lambda$.

\paragraph{Normal distribution} The normal distribution is a continuous distribution
with a bell-shaped density.  It is parametrized by two parameters, the mean $\mu \in
\mathbb{R}$ and the standard deviation $\sigma > 0$.  It is denoted by
$\mathcal{N}(\mu, \sigma^2)$.

The special case where $\mu = 0$ and $\sigma = 1$ is called the standard normal
distribution.  It is denoted by $\mathcal{N}(0, 1)$.

The probability density function of $X \sim \mathcal{N}(\mu, \sigma^2)$ is
\begin{equation}
  \label{eq:normal}
  f_X(x) = \frac{1}{\sqrt{2 \pi \sigma^2}} \exp\left(-\frac{(x - \mu)^2}{2 \sigma^2}\right)\text{.}
\end{equation}

The expected value of $X \sim \mathcal{N}(\mu, \sigma^2)$ is $\E[X] = \mu$, and the
variance is $\Var(X) = \sigma^2$.

\paragraph{Central limit theorem}  The central limit theorem states that the normalized
version of the sample mean converges to a standard normal distribution\footnote{This
statement of the central limit theorem is known as the Lindeberg-Levy CLT.  There are
other versions of the central limit theorem, some more general and some more
restrictive.}. Given $X_1, \dots, X_n$ i.i.d. random variables with mean $\mu$ and finite
variance $\sigma^2 < \infty$,
\begin{equation*}
  \sqrt{n} (\bar{X} - \mu) \sim \mathcal{N}(0, \sigma^2)\text{,}
\end{equation*}
as $n \rightarrow \infty$.  In other words, for a large enough $n$, the distribution of
the sample mean gets closer\footnote{Formally, this is called convergence in distribution,
refer to \fullcite{Billingsley1995} for more details.} to a normal distribution with mean
$\mu$ and variance $\sigma^2/n$.

The central limit theorem is one of the most important results in probability theory and
statistics.  Its implications are fundamental in many areas of science and engineering.

\paragraph{T distribution} The T distribution is a continuous distribution with a
bell-shaped density.  It is parametrized by a single parameter $\nu > 0$, called the
degrees of freedom.  It is denoted by $\mathcal{T}(\nu)$.

% The probability density function of $X \sim \mathcal{T}(\nu)$ is
% \begin{equation}
%   \label{eq:t}
%   f_X(x) = \frac{\Gamma\left(\frac{\nu + 1}{2}\right)}{\sqrt{\nu \pi} \Gamma\left(\frac{\nu}{2}\right)}
%   \left(1 + \frac{x^2}{\nu}\right)^{-\frac{\nu + 1}{2}}\text{.}
% \end{equation}

The T distribution generalizes to the three parameter location-scale t distribution
$\mathcal{T}(\mu, \sigma^2, \nu)$, where $\mu$ is the location parameter and $\sigma$ is
the scale parameter.  Thus, given $X \sim \mathcal{T}(\nu)$, we have that
$\mu + \sigma X \sim \mathcal{T}(\mu, \sigma^2, \nu)$.

Note that $$\lim_{\nu \rightarrow \infty} \mathcal{T}(\nu) = \mathcal{N}(0, 1)\text{.}$$
Thus, the T distribution converges to the standard normal distribution as the degrees of
freedom go to infinity.

\paragraph{Gamma distribution} The Gamma distribution is a continuous distribution with a
right-skewed density.  It is parametrized by two parameters, the shape parameter $\alpha
> 0$ and the rate parameter $\beta > 0$.  It is denoted by $\text{Gamma}(\alpha, \beta)$.

The probability density function of $X \sim \text{Gamma}(\alpha, \beta)$ is
\begin{equation}
  \label{eq:gamma}
  f_X(x) = \frac{\beta^\alpha x^{\alpha - 1} e^{-\beta x}}{\Gamma(\alpha)}\text{,}
\end{equation}
where $\Gamma(\alpha)$ is the gamma function, defined by
\begin{equation}
  \label{eq:gammaf}
  \Gamma(\alpha) = \int_0^\infty t^{\alpha - 1} e^{-t} dt\text{.}
\end{equation}

In Bayesian analysis, the gamma distribution is commonly used as a conjugate prior.
A conjugate prior is a prior distribution that, when combined with the likelihood,
results in a posterior distribution that is of the same family as the prior.

\subsection{Permutations and combinations}

For the sake of the reference, we present some definitions and formulas from
combinatorics. Combinatorics is the branch of mathematics that studies the counting of
objects.

\paragraph{Factorial}  The factorial of a non-negative integer $n$ is the product of all
positive integers up to $n$.  It is denoted by $n!$.  By definition, $0! = 1$.

\paragraph{Permutation}  A permutation is an arrangement of a set of elements.  The
number of permutations of $n$ elements is $n!$.  Permutations are used in combinatorics
to count the number of ways to arrange a set of elements.

\paragraph{Combination}  A combination is a selection of a subset of elements from a set.
The number of combinations of $k$ elements from a set of $n$ elements is $$\binom{n}{k} =
\frac{n!}{k!(n - k)!}\text{.}$$  Combinations are used in combinatorics to count the
number of ways to select a subset of elements from a set.  The binomial coefficient
$\binom{n}{k}$ is also called a choose function.

% \section{Optimization}
%
% Optimization is the process of finding the best solution to a problem.  The best
% solution is called the (global) optimum.  The optimum can be a maximum or a minimum.
% Sometimes, we are interested in finding a local optimum, which is the best solution
% in a neighborhood of the current solution.  Also, we might be interested in finding
% a solution that is good enough, i.e. a solution that is close to the optimum.
% In this case, we use heuristics to search in the solution space.
%
% \subsection{Minimization of convex functions}
%
% Maybe the simplest case of optimization is the minimization of a convex function.
% A function $f : \mathbb{R}^n \rightarrow \mathbb{R}$ is convex if for any two points
% $\vec{v}$ and $\vec{w}$ in the domain of $f$, the line segment connecting them lies
% above the graph of $f$.  Mathematically, it means that
% $$f(t\vec{v} + (1 - t) \vec{w}) \leq t f(\vec{v}) + (1 - t) f(\vec{w})$$ for all $t \in [0, 1]$.
%
% \subsection{Gradient descent}
%
% Gradient descent is an iterative algorithm for finding the minimum of a function.
% It is based on the observation that the gradient of a function points in the
% direction of the steepest descent.  The algorithm is
% \begin{equation}
%   \label{eq:gd}
%   \vec{w}(t + 1) = \vec{w}(t) - \alpha \nabla f(\vec{w}(t))\text{,}
% \end{equation}
%
% \subsection{Constraint optimization}
%
% Techniques like Lagrange multipliers, penalty methods, and barrier methods are used to
% handle constrained optimization problems in data science.
%
% \subsection{Convex optimization}
%
% Convex optimization problems, where the objective function and the constraints are convex,
% have efficient algorithms that guarantee global optimality.

% \subsection{Gradient descent algorithm}
%
% Let $f(\vec{w})$, $f : \mathbb{R}^n \rightarrow \mathbb{R}$, be an objective function that
% we are trying to minimize.  We know that
% $f$ is convex, of class $\mathcal{C}^2$, and its gradient $\nabla f$ is Lipschitz continuous with Lipschitz
% constant $L > 0$.
%
% We want to show that $\lim_{t\rightarrow\infty} f(\vec{w}(t)) = f^{*}$ where $f^{*}$
% is the global minimum of $f$ and $$\vec{w}(t+1) = \vec{w}(t) - \alpha \nabla f(\vec{w}(t))\mbox{,}$$
% for any initial condition $\vec{w}(0)$ and $0 < \alpha \leq \frac{1}{L}$.
%
% Convexity implies that for any two points $\vec{v}$ and $\vec{w}$ in the domain of
% $f$, the line segment connecting them lies above the graph of $f$.  Mathematically, it
% means that $$f(t\vec{v} + (1 - t) \vec{w}) \leq t f(\vec{v}) + (1 - t)
% f(\vec{w})$$ for all $t \in [0, 1]$.
%
% The Lipschitz continuity condition means that the gradient of $f(\vec{w})$ does not change too rapidly.
% Formally, $$\left\|\nabla f(\vec{v}) - \nabla f(\vec{w})\right\| \leq L \|\vec{v} - \vec{w}\|\mbox{,}$$
% for all $\vec{v}$ and $\vec{w}$ in the domain of $f$.  This is a rather weak
% assumption, and it means that the gradient can not change arbitrarily fast.
%
% Since $f$ is convex and twice differentiable, its Hessian is a positive semidefinite
% matrix, and thus its norm is its largest eigenvalue.
%
% A consequence of the Lipschitz continuity for a $\mathcal{C}^2$ function $f$ is that for
% any $\vec{v}$ and $\vec{w}$, we have that
% \begin{equation}
%   \label{eq:lcg1}
%   \vec{v}^T \nabla^2 f(w) \vec{v} \leq L \|v\|^2\text{.}
% \end{equation}
% It means that the eigenvalues of the Hessian are bounded above by $L$.
%
% \paragraph{Descent lemma.}  For $f$, a the multivariate Taylor expansion is that
% $$f(w) = f(v)$$

\chapter{Topics on learning machines}
\label{chap:learning-machines}
\glsresetall


\chapterprecishere{Oh, the depth of the riches and wisdom and knowledge of God! How
  unsearchable are his judgments and how inscrutable his ways!
  \par\raggedleft--- \textup{Romans 11:33} (ESV)}

This appendix is under construction.  Topics like the kernel trick, back-propagation, and
other machine learning algorithms will be discussed here.

{}
\clearpage

\section{Multi-layer perceptron}

The \gls{mlp} is a non-linear classifier that generates a set of hyperplanes
that separates the classes.  In order to simplify the understanding, consider the
that the activation function of the hidden layer is the discrete step function
\begin{equation*}
  \sigma(x) = \begin{cases}
    1 & \text{if } x > 0 \\
    0 & \text{otherwise.}
  \end{cases}
\end{equation*}
A model with two neurons in the hidden layer (effectively the combination of three
perceptrons) is
\begin{multline*}
  f(x_1, x_2; \theta = \left\{ \vec{w}^{(1)}, \vec{w}^{(2)}, \vec{w}^{(3)} \right\}) = \\
  \sigma\left(
    \vec{w}^{(3)} \cdot \left[1, \sigma(\vec{w}^{(1)} \cdot \vec{x}), \sigma(\vec{w}^{(2)} \cdot \vec{x})\right]
  \right)\text{.}
\end{multline*}

The parameters $\vec{w}^{(1)}$ and $\vec{w}^{(2)}$ represent the hyperplanes that separate
the classes in the hidden layer, and $\vec{w}^{(3)}$ represents how the hyperplanes are
combined to generate the output.  If we set $\vec{w}^{(1)} = [-0.5, 1, -1]$ (like the
perceptron in the previous example) and $\vec{w}^{(2)} = [-0.5, -1, 1]$, we use the third neuron
to combine the results of the first two neurons.  This way, a possible solution for the
XOR problem is setting $\vec{w}^{(3)} = [0, 1, 1]$.

\begin{figurebox}[label=fig:mlp]{MLP class boundaries for the XOR problem.}
  \centering
  \begin{tikzpicture}
    \begin{axis}[
        axis x line=bottom,
        axis y line=left,
        xlabel={$x_1$},
        ylabel={$x_2$},
        width=0.6\textwidth,
        height=0.6\textwidth,
        xtick={0, 1},
        ytick={0, 1},
        grid=both,
        xmin=-0.5, xmax=1.5,
        ymin=-0.5, ymax=1.5,
      ]
      \addplot+[only marks, mark=+, mark size=3pt] coordinates {
        (0, 1) (1, 0)
      };
      \addplot+[only marks, mark=*, mark size=3pt] coordinates {
        (0, 0) (1, 1)
      };
      \addplot+[domain=0:1.5, mark=none, black] {-0.5 + x};
      \addplot+[domain=-0.5:1.5, mark=none, black] {0.5 + x};
    \end{axis}
  \end{tikzpicture}
  \tcblower
  \Gls{mlp} with two neurons in the hidden layer generates two linear hyperplanes that
  separate the classes, effectively solving the XOR problem.
\end{figurebox}

\begin{tablebox}[label=tab:xor-mlp]{Truth table for the predictions of the MLP.}
  \centering
  \rowcolors{2}{black!10!white}{}
  \begin{tabular}{ccc|cccc}
    \toprule
    $x_1$ & $x_2$ & $y$ & \nth{1} neuron & \nth{2} neuron & $\hat{y}$ \\
    \midrule
    0 & 0 & 0 & 0 & 0 & 0 \\
    0 & 1 & 1 & 0 & 1 & 1 \\
    1 & 0 & 1 & 1 & 0 & 1 \\
    1 & 1 & 0 & 0 & 0 & 0 \\
    \bottomrule
  \end{tabular}
  \tcblower
  Preditions of the \gls{mlp} for the XOR problem.  The output of the \nth{1} and \nth{2}
  neurons are hyperplanes that separate the classes in the hidden layer, which are
  combined by the \nth{3} neuron to generate the correct output.
\end{tablebox}

\Cref{fig:mlp,tab:xor-mlp} show the class boundaries and the predictions of the MLP for
the XOR problem.

Note that there are many possible solutions for the XOR problem using the MLP.
Learning strategies like back-propagation are used to find the optimal parameters for
the model and regularization techniques, like $l_1$ and $l_2$ regularization, are used to
prevent overfitting.

% TODO: backpropagation and regularization
% TODO: talk about deep learning

Deep learning is the study of neural networks with many layers.  The idea is to use many
layers to learn not only the boundaries that separate the classes (or the function that
maps inputs and outputs) but also the features that are relevant to the problem.
A complete discussion of deep learning can be found in
\textcite{Goodfellow2016}\footfullcite{Goodfellow2016}.

\clearpage
\subsection{Decision trees}

The decision tree is a non-linear classifier that generates a set of hyperplanes that
are orthogonal to the axes.  Consider the decision tree in \cref{fig:tree-and}.

\begin{figurebox}[label=fig:tree-and]{Decision tree representation.}
  \centering
  \begin{tikzpicture}
    \node[decision] (x1) at (0, 0) {$x_1$};
    \node[block] (n1) at (-2, -1.5) {$\hat{y} = 0$};
    \node[decision] (x2) at (2, -1.5) {$x_2$};
    \node[block] (n2) at (0, -3) {$\hat{y} = 0$};
    \node[block] (p) at (4, -3) {$\hat{y} = 1$};

    \draw (x1) -| (n1) node [midway, above] {$\leq 0.5$};
    \draw (x1) -| (x2) node [midway, above] {$>0.5$};
    \draw (x2) -| (n2) node [midway, above] {$\leq 0.5$};
    \draw (x2) -| (p) node [midway, above] {$>0.5$};
  \end{tikzpicture}
  \tcblower
  The decision tree that solves the AND problem.
\end{figurebox}

\begin{figurebox}[label=fig:tree-bias]{Decision tree spatial representation.}
  \centering
  \begin{tikzpicture}
    \begin{axis}[
        axis x line=bottom,
        axis y line=left,
        xlabel={$x_1$},
        ylabel={$x_2$},
        width=0.6\textwidth,
        height=0.6\textwidth,
        xtick={0, 1},
        ytick={0, 1},
        grid=both,
        xmin=-0.5, xmax=1.5,
        ymin=-0.5, ymax=1.5,
      ]
      \addplot+[only marks, mark=+, mark size=3pt] coordinates {
        (1, 1)
      };
      \addplot+[only marks, mark=*, mark size=3pt] coordinates {
        (0, 0) (0, 1) (1, 0)
      };
      \addplot+[domain=0.5:1.5, mark=none, black, thick] {0.5};
      \addplot+[mark=none, black, thick] coordinates {(0.5, -0.5) (0.5, 1.5)};
    \end{axis}
  \end{tikzpicture}
  \tcblower
  The decision tree learning bias is the assumption that the classes can be separated with
  hyperplanes orthogonal to the axes.
\end{figurebox}

The spatial representation of the decision tree is shown in \cref{fig:tree-bias}.  The
decision tree generates hyperplanes that are orthogonal to the axes, which is the learning
bias of the model.

Decision tree are nonparametric models, one can easily increase the depth of the tree to
fit the data, generating as many hyperplanes as necessary to separate the classes.
Training a decision tree with a large depth can lead to overfitting, so it is important to
use techniques like depth limit and pruning to prevent this from happening.

% \subsection{$k$-nearest neighbors learning bias}
%
% The $k$-nearest neighbors ($k$-NN) is a non-linear nonparametric classifier that
% generate arbitrarily complex decision boundaries by ``memoring'' the training data.
% The behavior of the boundaries depends on the value of $k$ and the distance metric one
% uses to find the nearest neighbors of a point.
%
% \begin{figurebox}[label=fig:1nn-bias]{1-NN learning bias.}
%   \centering
%   \begin{tikzpicture}
%     \begin{axis}[
%         axis x line=bottom,
%         axis y line=left,
%         xlabel={$x_1$},
%         ylabel={$x_2$},
%         width=0.6\textwidth,
%         height=0.6\textwidth,
%         xtick={0, 1},
%         ytick={0, 1},
%         grid=both,
%         xmin=-0.5, xmax=1.5,
%         ymin=-0.5, ymax=1.5,
%       ]
%       \addplot+[only marks, mark=+, mark size=3pt] coordinates {
%         (1, 1)
%       };
%       \addplot+[only marks, mark=*, mark size=3pt] coordinates {
%         (0, 0) (0, 1) (1, 0)
%       };
%       \addplot+[domain=0.5:1.5, mark=none, black, thick] {0.5};
%       \addplot+[mark=none, black, thick] coordinates {(0.5, 0.5) (0.5, 1.5)};
%     \end{axis}
%   \end{tikzpicture}
%   \tcblower
%   In this particular case, the 1-NN boundaries match the decision tree boundaries.
% \end{figurebox}
%
% As $k$ increases the boundaries become smoother:
% \href{https://images.squarespace-cdn.com/content/v1/5d782753c70af105c29a9b14/1580261947016-XODPUVKWPGGMJJMAXSNF/Screen+Shot+2020-01-28+at+8.38.55+PM.png}{example}.
%
% See illustration: \href{https://scikit-learn.org/stable/auto_examples/classification/plot_classifier_comparison.html}{here}.

% vim: spell spelllang=en


\backmatter
\emergencystretch=1em
\printbibliography
\printglossary

\end{document}

% vim: spell spelllang=en
