\chapter{Topics on learning machines}
\label{chap:learning-machines}

This appendix is under construction.  Topics like the kernel trick, back-propagation, and
other machine learning algorithms will be discussed here.

\section{Multi-layer perceptron}
\label{sec:mlp}

The \gls{mlp} is a non-linear classifier that generates a set of hyperplanes
that separates the classes.  In order to simplify understanding, consider
that the activation function of the hidden layer is the discrete step function
\begin{equation*}
  \sigma(x) = \begin{cases}
    1 & \text{if } x > 0 \\
    0 & \text{otherwise.}
  \end{cases}
\end{equation*}
A model with two neurons in the hidden layer (effectively the combination of three
perceptrons) is
\begin{multline*}
  f(x_1, x_2; \theta = \left\{ \vec{w}^{(1)}, \vec{w}^{(2)}, \vec{w}^{(3)} \right\}) = \\
  \sigma\left(
    \vec{w}^{(3)} \cdot \left[1, \sigma(\vec{w}^{(1)} \cdot \vec{x}), \sigma(\vec{w}^{(2)} \cdot \vec{x})\right]
  \right)\text{.}
\end{multline*}

The parameters $\vec{w}^{(1)}$ and $\vec{w}^{(2)}$ represent the hyperplanes that separate
the classes in the hidden layer, and $\vec{w}^{(3)}$ represents how the hyperplanes are
combined to generate the output.  If we set weights $\vec{w}^{(1)} = [-0.5, 1, -1]$ (like the
perceptron in the previous example) and $\vec{w}^{(2)} = [-0.5, -1, 1]$, we use the third neuron
to combine the results of the first two neurons.  This way, a possible solution for the
XOR problem is setting $\vec{w}^{(3)} = [0, 1, 1]$.

\begin{figurebox}[label=fig:mlp]{MLP class boundaries for the XOR problem.}
  \centering
  \begin{tikzpicture}
    \begin{axis}[
        axis x line=bottom,
        axis y line=left,
        xlabel={$x_1$},
        ylabel={$x_2$},
        width=0.6\textwidth,
        height=0.6\textwidth,
        xtick={0, 1},
        ytick={0, 1},
        xmin=-0.5, xmax=1.5,
        ymin=-0.5, ymax=1.5,
      ]
      \addplot+[only marks, mark=+, color=black, mark size=3pt] coordinates {
        (0, 1) (1, 0)
      };
      \addplot+[only marks, mark=-, color=black, mark size=3pt] coordinates {
        (0, 0) (1, 1)
      };
      \addplot+[domain=0:1.5, mark=none, black, thick] {-0.5 + x};
      \addplot+[domain=-0.5:1.5, mark=none, black, thick] {0.5 + x};
    \end{axis}
  \end{tikzpicture}
  \tcblower
  \Gls{mlp} with two neurons in the hidden layer generates two linear hyperplanes that
  separate the classes, effectively solving the XOR problem.
\end{figurebox}

\begin{tablebox}[label=tab:xor-mlp]{Truth table for the predictions of the MLP.}
  \centering
  \rowcolors{2}{black!10!white}{}
  \begin{tabular}{ccc|cccc}
    \toprule
    $x_1$ & $x_2$ & $y$ & \nth{1} neuron & \nth{2} neuron & $\hat{y}$ \\
    \midrule
    0 & 0 & 0 & 0 & 0 & 0 \\
    0 & 1 & 1 & 0 & 1 & 1 \\
    1 & 0 & 1 & 1 & 0 & 1 \\
    1 & 1 & 0 & 0 & 0 & 0 \\
    \bottomrule
  \end{tabular}
  \tcblower
  Predictions of the \gls{mlp} for the XOR problem.  The output of the \nth{1} and \nth{2}
  neurons are hyperplanes that separate the classes in the hidden layer, which are
  combined by the \nth{3} neuron to generate the correct output.
\end{tablebox}

\Cref{fig:mlp,tab:xor-mlp} show the class boundaries and the predictions of the MLP for
the XOR problem.

Note that there are many possible solutions for the XOR problem using the MLP.
Learning strategies like back-propagation are used to find the optimal parameters for
the model and regularization techniques, like $l_1$ and $l_2$ regularization, are used to
prevent overfitting.

% TODO: backpropagation and regularization
% TODO: talk about deep learning

Deep learning is the study of neural networks with many layers.  The idea is to use many
layers to learn not only the boundaries that separate the classes (or the function that
maps inputs and outputs) but also the features that are relevant to the problem.
A complete discussion of deep learning can be found in
\textcite{Goodfellow2016}\footfullcite{Goodfellow2016}.

\clearpage
\section{Decision trees}

The decision tree is a non-linear classifier that generates a set of hyperplanes that
are orthogonal to the axes.  Consider the decision tree in \cref{fig:tree-and}.

\begin{figurebox}[label=fig:tree-and]{Decision tree representation.}
  \centering
  \begin{tikzpicture}
    \node[decision] (x1) at (0, 0) {$x_1$};
    \node[block] (n1) at (-2, -1.5) {$\hat{y} = 0$};
    \node[decision] (x2) at (2, -1.5) {$x_2$};
    \node[block] (n2) at (0, -3) {$\hat{y} = 0$};
    \node[block] (p) at (4, -3) {$\hat{y} = 1$};

    \draw (x1) -| (n1) node [midway, above] {$\leq 0.5$};
    \draw (x1) -| (x2) node [midway, above] {$>0.5$};
    \draw (x2) -| (n2) node [midway, above] {$\leq 0.5$};
    \draw (x2) -| (p) node [midway, above] {$>0.5$};
  \end{tikzpicture}
  \tcblower
  The decision tree that solves the AND problem.
\end{figurebox}

\begin{figurebox}[label=fig:tree-bias]{Decision tree spatial representation.}
  \centering
  \begin{tikzpicture}
    \begin{axis}[
        axis x line=bottom,
        axis y line=left,
        xlabel={$x_1$},
        ylabel={$x_2$},
        width=0.6\textwidth,
        height=0.6\textwidth,
        xtick={0, 1},
        ytick={0, 1},
        xmin=-0.5, xmax=1.5,
        ymin=-0.5, ymax=1.5,
      ]
      \addplot+[only marks, mark=+, black, mark size=3pt] coordinates {
        (1, 1)
      };
      \addplot+[only marks, mark=-, black, mark size=3pt] coordinates {
        (0, 0) (0, 1) (1, 0)
      };
      \addplot+[domain=0.5:1.5, mark=none, black, thick] {0.5};
      \addplot+[mark=none, black, thick] coordinates {(0.5, -0.5) (0.5, 1.5)};
    \end{axis}
  \end{tikzpicture}
  \tcblower
  Decision trees assume that the classes can be separated with
  hyperplanes orthogonal to the axes.
\end{figurebox}

The spatial representation of the decision tree is shown in \cref{fig:tree-bias}.
Decision trees are a type of classifier that generates a set of hyperplanes orthogonal to the axes.

Decision trees are nonparametric models, one can easily increase the depth of the tree to
fit the data, generating as many hyperplanes as necessary to separate the classes.
Training a decision tree with a large depth can lead to overfitting, so it is important to
use techniques like depth limit and pruning to prevent this from happening.

% \subsection{$k$-nearest neighbors learning bias}
%
% The $k$-nearest neighbors ($k$-NN) is a non-linear nonparametric classifier that
% generate arbitrarily complex decision boundaries by ``memoring'' the training data.
% The behavior of the boundaries depends on the value of $k$ and the distance metric one
% uses to find the nearest neighbors of a point.
%
% \begin{figurebox}[label=fig:1nn-bias]{1-NN learning bias.}
%   \centering
%   \begin{tikzpicture}
%     \begin{axis}[
%         axis x line=bottom,
%         axis y line=left,
%         xlabel={$x_1$},
%         ylabel={$x_2$},
%         width=0.6\textwidth,
%         height=0.6\textwidth,
%         xtick={0, 1},
%         ytick={0, 1},
%         grid=both,
%         xmin=-0.5, xmax=1.5,
%         ymin=-0.5, ymax=1.5,
%       ]
%       \addplot+[only marks, mark=+, mark size=3pt] coordinates {
%         (1, 1)
%       };
%       \addplot+[only marks, mark=*, mark size=3pt] coordinates {
%         (0, 0) (0, 1) (1, 0)
%       };
%       \addplot+[domain=0.5:1.5, mark=none, black, thick] {0.5};
%       \addplot+[mark=none, black, thick] coordinates {(0.5, 0.5) (0.5, 1.5)};
%     \end{axis}
%   \end{tikzpicture}
%   \tcblower
%   In this particular case, the 1-NN boundaries match the decision tree boundaries.
% \end{figurebox}
%
% As $k$ increases the boundaries become smoother:
% \href{https://images.squarespace-cdn.com/content/v1/5d782753c70af105c29a9b14/1580261947016-XODPUVKWPGGMJJMAXSNF/Screen+Shot+2020-01-28+at+8.38.55+PM.png}{example}.
%
% See illustration: \href{https://scikit-learn.org/stable/auto_examples/classification/plot_classifier_comparison.html}{here}.

% vim: spell spelllang=en
