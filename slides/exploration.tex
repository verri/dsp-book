% slides/exploration.tex - Data Exploration

% slides/preamble.tex - Shared preamble for all slide decks
% Metropolis theme with grayscale style matching the book

\documentclass[aspectratio=169]{beamer}

\usetheme{metropolis}

% ---------- Colors (grayscale) ----------
\definecolor{bookdark}{gray}{0.2}
\definecolor{bookgray}{gray}{0.5}
\definecolor{booklight}{gray}{0.92}

\setbeamercolor{normal text}{fg=bookdark, bg=white}
\setbeamercolor{alerted text}{fg=bookdark}
\setbeamercolor{frametitle}{fg=white, bg=bookdark}
\setbeamercolor{title separator}{fg=bookgray}
\setbeamercolor{progress bar}{fg=bookgray, bg=booklight}
\setbeamercolor{block title}{fg=white, bg=bookgray}
\setbeamercolor{block body}{fg=bookdark, bg=booklight}
\setbeamercolor{block title alerted}{fg=white, bg=bookdark}
\setbeamercolor{block body alerted}{fg=bookdark, bg=booklight}
\setbeamercolor{block title example}{fg=white, bg=bookgray}
\setbeamercolor{block body example}{fg=bookdark, bg=booklight}

\setbeamertemplate{frame numbering}[fraction]

% ---------- Fonts (matching the book) ----------
\usepackage[T1]{fontenc}
\usepackage{fontspec}
\usepackage[warnings-off={mathtools-colon,mathtools-overbracket}]{unicode-math}

\setmathfont{STIXTwoMath}[
  Extension={.otf},
  Path={./fonts/},
  Scale=1]

\setsansfont{STIXTwoText}[
  Extension={.otf},
  Path={./fonts/},
  UprightFont={*-Regular},
  BoldFont={*-Bold},
  ItalicFont={*-Italic},
  BoldItalicFont={*-BoldItalic}]

\setmainfont{STIXTwoText}[
  Extension={.otf},
  Path={./fonts/},
  UprightFont={*-Regular},
  BoldFont={*-Bold},
  ItalicFont={*-Italic},
  BoldItalicFont={*-BoldItalic}]

\setmonofont{CourierPrime}[
  Extension={.ttf},
  Path={./fonts/},
  UprightFont={*-Regular},
  BoldFont={*-Bold},
  ItalicFont={*-Italic},
  BoldItalicFont={*-BoldItalic},
  Scale=0.9]

% ---------- Packages ----------
\usepackage{amsmath}
\usepackage{mathtools}
\usepackage{graphicx}
\usepackage{booktabs}

% ---------- TikZ ----------
\usepackage{tikz}
\usetikzlibrary{shapes, arrows.meta, positioning, shapes.geometric, fit}

\tikzset{%
  decision/.style={draw, diamond, text centered, minimum height=0.5cm, minimum width=1cm},
  block/.style={rectangle, draw, text width=6em, text centered, rounded corners, minimum height=3em},
  mediumblock/.style={rectangle, draw, text width=3em, text centered, rounded corners, minimum height=2em},
  darkblock/.style={block, fill=gray, text=white},
  smallblock/.style={rectangle, rounded corners, draw, font=\tiny, minimum height=1em, inner sep=2pt},
  smalldarkblock/.style={smallblock, fill=gray, text=white},
  darkcircle/.style={draw, circle, fill=gray, text centered, text=white},
  smallcircle/.style={draw, circle, text centered, font=\tiny},
  smalldarkcircle/.style={smallcircle, fill=gray, text=white},
  line/.style={draw, -latex},
  dline/.style={draw, latex-latex},
  bigarrow/.style={draw, -latex, line width=3pt, gray},
}

% ---------- Math operators ----------
\DeclareMathOperator*{\argmax}{arg\,max}
\DeclareMathOperator*{\argmin}{arg\,min}
\DeclareMathOperator{\Prob}{P}
\DeclareMathOperator{\E}{E}
\DeclareMathOperator{\Var}{Var}
\DeclareMathOperator{\sign}{sign}
\DeclareMathOperator{\clamp}{clamp}

% ---------- Hyperref ----------
\usepackage{hyperref}
\hypersetup{colorlinks, urlcolor=bookgray, linkcolor=bookdark}

% ---------- Commands ----------
\renewcommand{\vec}[1]{\mathbf{#1}}
\newcommand{\code}[1]{\colorbox{black!10!white}{\texttt{#1}}}

% ---------- Book info, license, and disclaimer ----------
\newcommand{\bookframe}{%
  \begin{frame}{About these slides}
    These slides are companion material for the book

    \vspace{0.3cm}
    \begin{center}
      \textbf{Data Science Project: An Inductive Learning Approach}\\[2pt]
      Prof.~Dr.~Filipe A. N. Verri\\[4pt]
      \url{https://leanpub.com/dsp}
    \end{center}

    \vspace{0.3cm}
    All intellectual content comes from the book and is not AI-generated.
    Slides were produced with the assistance of
    \href{https://claude.ai/code}{Claude Code}.

    \vspace{0.3cm}
    {\small
    Licensed under
    \href{https://creativecommons.org/licenses/by-nc/4.0/}{CC BY-NC 4.0}.
    You are free to modify and redistribute this work as long as you give
    proper credit and do not use it for commercial purposes.}
  \end{frame}%
}

% ---------- TikZ color presets (used by book figures) ----------
\colorlet{circle edge}{black!50}
\colorlet{circle area}{black!20}
\tikzset{
  filled/.style={fill=circle area, draw=circle edge, thick},
  outline/.style={draw=circle edge, thick},
}


\title{Data Exploration}
\subtitle{Data Science Project: An Inductive Learning Approach}
\author{Prof.~Dr.~Filipe A. N. Verri}
\date{}

\begin{document}

\maketitle
\bookframe

% ---- Epigraph ----

\begin{frame}{}
  \vfill
  \begin{quote}
    The greatest value of a picture is when it forces us to notice what we
    never expected to see.
    \begin{flushright}
      --- John W. Tukey, \textit{Exploratory Data Analysis}
    \end{flushright}
  \end{quote}
  \vfill
\end{frame}

% ---- Overview ----

\begin{frame}{Overview}
  \begin{columns}[T]
    \begin{column}{0.48\textwidth}
      \textbf{Contents}
      \begin{itemize}
        \item Introduction and goals
        \item Descriptive statistics
        \item Data visualization
        \item Graphical integrity and data-ink ratio
        \item Turning charts into stories
      \end{itemize}
    \end{column}
    \begin{column}{0.48\textwidth}
      \textbf{Objectives}
      \begin{itemize}
        \item Understand what to explore and what not to
        \item Use descriptive statistics effectively
        \item Create honest, clear visualizations
        \item Communicate findings as stories
      \end{itemize}
    \end{column}
  \end{columns}
\end{frame}

% ===========================================================================
\section{Introduction}
% ===========================================================================

\begin{frame}{When does exploration happen?}
  Two main moments:
  \begin{enumerate}
    \item \textbf{Before data organization} --- understand the original
      phenomena and data sources
    \item \textbf{After data organization} --- validate that transformations
      achieved expected results
  \end{enumerate}

  \vspace{0.5cm}
  In practice, also during intermediate steps as a form of \textbf{debugging}.
\end{frame}

\begin{frame}{Exploratory Data Analysis (EDA)}
  \begin{itemize}
    \item Term coined by Tukey (1977)\footfullcite{Tukey1977}
    \item Informal study of data: graphics and elaborate statistical summaries
    \item Counterpoint to dominance of confirmatory statistical methods
    \item Original context: statistical modeling, not machine learning
    \item Two classical goals: data description and model formulation
    \item In our approach: model formulation is delegated to the ML algorithm
  \end{itemize}
\end{frame}

\begin{frame}{The fine line}
  \begin{alertblock}{Key tension}
    Deepen understanding enough to know the data and the problem, but not so
    much as to bias the final solution.
  \end{alertblock}

  \vspace{0.3cm}
  \begin{itemize}
    \item ML principle: minimize human interference in the learning process
    \item Excessive domain-knowledge influence can limit discovery
    \item A variable that experts deem unimportant may contain valuable
      information
  \end{itemize}
\end{frame}

% ---------------------------------------------------------------------------
\subsection{Goals of data exploration}
% ---------------------------------------------------------------------------

\begin{frame}{Goal: understand units and variables}
  \textbf{Fixed (index) variables:}
  \begin{itemize}
    \item Identify each one's meaning, individually and combined
    \item Possible observational units (e.g.\ enrollment, class)
    \item Domain: categorical, ideally finite and known \emph{a priori}
    \item Inspect unique values --- but beware: sample $\neq$ full domain
  \end{itemize}

  \vspace{0.3cm}
  \textbf{Measured (non-index) variables:}
  \begin{itemize}
    \item Used directly in the ML problem
    \item Analyze type, domain, and distribution
    \item Descriptive statistics and visualizations
  \end{itemize}
\end{frame}

\begin{frame}{Goal: investigate inconsistencies}
  \textbf{Fixed variables:}
  \begin{itemize}
    \item No duplicate rows (unique identification)
    \item No missing keys
    \item Missing observations vs.\ missing key parts
  \end{itemize}

  \vspace{0.3cm}
  \textbf{Measured variables:}
  \begin{itemize}
    \item Extreme values --- measurement errors or irrelevant phenomena?
    \item Missing values --- which preprocessing techniques are needed?
    \item Compare observed characteristics with prior knowledge
  \end{itemize}
\end{frame}

\begin{frame}{Goal: validate data manipulations}
  \begin{itemize}
    \item Data handling operations can produce counter-intuitive results
    \item Explore both original and transformed data
    \item Example: after pivoting, check unique values and distributions
    \item Interactive environments (e.g.\ Jupyter) as debugging tools
    \item Step-by-step inspection of intermediate transformations
  \end{itemize}
\end{frame}

% ---------------------------------------------------------------------------
\subsection{Non-goals of data exploration}
% ---------------------------------------------------------------------------

\begin{frame}{Non-goals of data exploration}
  \begin{exampleblock}{Do not model or infer relationships}
    \begin{itemize}
      \item Correlation analysis to decide variable importance
      \item Fitting regression models to understand relationships
      \item These tasks belong to the solution search algorithm
    \end{itemize}
  \end{exampleblock}

  \begin{exampleblock}{Do not test statistical hypotheses}
    \begin{itemize}
      \item Testing if a variable follows a specific distribution adds little
      \item ML algorithms are robust to deviations
      \item Prefer experimental validation of the solution
    \end{itemize}
  \end{exampleblock}

  \begin{exampleblock}{Do not manipulate data}
    \begin{itemize}
      \item Exploration and manipulation are distinct tasks
      \item Exploratory transformations must not persist
    \end{itemize}
  \end{exampleblock}
\end{frame}

% ---------------------------------------------------------------------------
\subsection{Communication in the exploration report}
% ---------------------------------------------------------------------------

\begin{frame}{Communication in the exploration report}
  Three-step approach:
  \begin{enumerate}
    \item Formulate a \textbf{question} about the data
    \item Choose \textbf{tools and techniques} to answer it
    \item \textbf{Communicate} the answer via statistics and visualizations
  \end{enumerate}

  \vspace{0.5cm}
  Key principles:
  \begin{itemize}
    \item \textbf{Simplicity} --- avoid excess information and complex charts
    \item \textbf{Relevance} --- the right answer to the wrong question is
      useless
    \item \textbf{Clarity} --- no ambiguity, no unnecessary jargon
    \item \textbf{Objectivity} --- each piece answers one question only
  \end{itemize}
\end{frame}

% ===========================================================================
\section{Descriptive statistics}
% ===========================================================================

\begin{frame}{Descriptive statistics}
  Numeric summaries of specific properties of the data distribution:
  \begin{itemize}
    \item \textbf{Location}: mean, median, mode, min, max
    \item \textbf{Frequency}: counts per category
    \item \textbf{Dispersion}: standard deviation, entropy, kurtosis
  \end{itemize}

  \vspace{0.3cm}
  Main uses in exploration:
  \begin{itemize}
    \item Validate the domain of variables against expectations
    \item Detect class imbalance or underrepresented categories
    \item Assess information content of each variable
  \end{itemize}
\end{frame}

\begin{frame}{Dispersion and information content}
  \textbf{Low dispersion} (std.\ dev.\ $\approx 0$, low entropy):
  \begin{itemize}
    \item Variable carries little discriminative information
    \item Most observations share similar values
    \item May need to collect more diverse data
  \end{itemize}

  \vspace{0.3cm}
  \textbf{High dispersion}:
  \begin{itemize}
    \item Too many unique or spread-out values
    \item Model may struggle to generalize
    \item Excess information can be as harmful as lack of it
    \item Address during preprocessing
  \end{itemize}

  \vspace{0.3cm}
  Always present statistics within a \textbf{storytelling} context --- never
  as isolated tables without interpretation.
\end{frame}

% ===========================================================================
\section{Data visualization}
% ===========================================================================

\begin{frame}{Why visualize?}
  \begin{itemize}
    \item The visual cortex absorbs large amounts of information quickly
    \item Charts and diagrams make data more intuitive than raw numbers
    \item Powerful for identifying patterns, anomalies, and relationships
  \end{itemize}

  \vspace{0.3cm}
  \begin{alertblock}{But beware}
    Visualizations can be ambiguous, poorly chosen, or intentionally
    misleading (``lying with statistics'').
  \end{alertblock}
\end{frame}

% ---------------------------------------------------------------------------
\subsection{Graphical integrity}
% ---------------------------------------------------------------------------

\begin{frame}{Graphical integrity --- misleading scales}
  \centering
  \begin{tikzpicture}
    \begin{axis}[
      axis x line=bottom,
      axis y line=left,
      ybar,
      ymin=8,
      width=6cm,
      height=5cm,
      bar width=0.4cm,
      enlarge x limits=0.2,
      ylabel={Measure},
      xlabel={Year},
      xtick={1,2,3},
      ]
      \addplot[color=black, fill=gray] coordinates {(1, 9) (2, 10) (3, 12)};
    \end{axis}
  \end{tikzpicture}
  \hfill
  \begin{tikzpicture}
    \begin{axis}[
      axis x line=bottom,
      axis y line=left,
      ybar,
      ymin=0,
      width=6cm,
      height=5cm,
      bar width=0.4cm,
      enlarge x limits=0.2,
      ylabel={Measure},
      xlabel={Year},
      xtick={1,2,3},
      ]
      \addplot[color=black, fill=gray] coordinates {(1, 9) (2, 10) (3, 12)};
    \end{axis}
  \end{tikzpicture}

  \vspace{0.3cm}
  \small
  Left: axis not starting at zero exaggerates differences.\\
  Right: axis starting at zero gives an honest representation.
\end{frame}

\begin{frame}{Tufte's six principles of graphical integrity}
  \begin{enumerate}
    \item Physical representation must be proportional to values
    \item Labels, explanations, and events must be clear and visible
    \item Show variation in data, never in design or scale
    \item For monetary values, adjust for inflation/deflation
    \item Dimensions in chart $\leq$ dimensions in data
    \item Charts must not omit context relevant for interpretation
  \end{enumerate}
  \vspace{0.3cm}
  \hfill\small Tufte (2001)\footfullcite{Tufte2001}
\end{frame}

% ---------------------------------------------------------------------------
\subsection{Data-ink ratio}
% ---------------------------------------------------------------------------

\begin{frame}{Data-ink ratio}
  Core principle: \textbf{above all, show the data.}

  \begin{block}{Definition (Tufte, 2001)}
    \[
      \mathrm{DIR} = \frac{\mathrm{DI}}{\mathrm{DI} + \mathrm{RI} +
      \mathrm{NDI}}
    \]
    \begin{itemize}
      \item $\mathrm{DI}$: non-erasable data-ink
      \item $\mathrm{RI}$: redundant data-ink
      \item $\mathrm{NDI}$: non-data ink
    \end{itemize}
  \end{block}

  \vspace{0.3cm}
  $1 - \mathrm{DIR}$ = proportion of the chart that can be removed without
  losing information.
\end{frame}

\begin{frame}{Maximizing data-ink ratio}
  \textbf{Erase non-data ink} (within reason):
  \begin{itemize}
    \item Remove unnecessary axis lines, grids, legends, labels
    \item $\mathrm{DIR}$ increases as $\mathrm{NDI} \to 0$
    \item But: grids can help reading without a ruler
  \end{itemize}

  \vspace{0.5cm}
  \textbf{Erase redundant data-ink} (within reason):
  \begin{itemize}
    \item Labeled axes $\to$ no need to label each bar
    \item Exception: periodic data may benefit from repetition
    \item Highlighting specific data points can be useful
  \end{itemize}
\end{frame}

% ---------------------------------------------------------------------------
\subsection{Chart editing}
% ---------------------------------------------------------------------------

\begin{frame}{Chart editing --- iterative process}
  \begin{enumerate}
    \item Idealize the visualization and map variables to dimensions
    \item Produce an initial chart
    \item Iteratively remove ink and improve integrity
    \item Adjust until the chart clearly communicates its intention
  \end{enumerate}
\end{frame}

\begin{frame}{Example: supercomputer performance (before editing)}
  \centering
  \begin{tikzpicture}
    \pgfplotsset{
      major grid style={color=black}, minor grid style={color=black},
      tick style={color=black},
      cycle list={{black, mark=x}, {black, mark=*}}}
    \begin{axis}[
      ymajorgrids,
      yminorgrids,
      minor tick num=1,
      ylabel={$R_\text{max}$ (TFLOPS)},
      xlabel={Year},
      ymode=log,
      width=0.85\textwidth,
      height=0.65\textheight,
      ]
      \addplot table [
        only marks, x=Year, y=Rmax, col sep=comma, color=black,
      ] {data/supercomputer2a.csv};
      \addplot table [
        only marks, x=Year, y=Rmax, col sep=comma, color=black,
      ] {data/supercomputer2b.csv};
    \end{axis}
  \end{tikzpicture}
\end{frame}

\begin{frame}{Editing steps}
  \begin{itemize}
    \item Display performance as powers of two ($2^{12}, 2^{14}, \ldots$)
      to emphasize doubling
    \item Remove vertical grid --- log base-2 scale is self-explanatory
    \item Use uniform markers (short horizontal lines) --- focus on trend,
      not individual bests
    \item Fix axis labels (years without thousands separator)
    \item Replace bounding box with axis arrows for cleaner design
    \item Add Moore's Law trend line and annotations
  \end{itemize}
\end{frame}

\begin{frame}{Example: supercomputer performance (after editing)}
  \centering
  \begin{tikzpicture}[
      style={every pin/.style={font=\footnotesize, text width=2cm, align=center}},
    ]
    \pgfplotsset{cycle list={{black, mark=-}, {black, mark=o}}}
    \begin{axis}[
      axis lines=middle,
      ylabel={$R_\text{max}$ (TFLOPS)},
      xlabel={Year},
      xtick={2015, 2017, 2019, 2021, 2023},
      xmin=2014,
      xmax=2025,
      ymin=11,
      ymax=22,
      ytick={12, 14, 16, 18, 20},
      yticklabels={$2^{12}$, $2^{14}$, $2^{16}$, $2^{18}$, $2^{20}$},
      xticklabels={2015, 2017, 2019, 2021, 2023},
      width=0.85\textwidth,
      height=0.7\textheight,
      ]
      \addplot table [
        only marks, x=Year, y=Rmax, col sep=comma, color=black,
      ] {data/supercomputer.csv};
      \addplot[domain=2015:2023, black!20] {(x-2015)/2+11.70959};
      \node [draw,coordinate,pin=above:{Frontier}] at (axis cs:2021,20.187) {};
      \node [draw,coordinate,pin=above:{Sunway TaihuLight}] at (axis cs:2016,16.5051) {};
    \end{axis}
  \end{tikzpicture}
\end{frame}

\begin{frame}{Alternative version with discrete grid}
  \centering
  \begin{tikzpicture}[
      style={every pin/.style={font=\footnotesize, text width=2cm, align=center}},
    ]
    \pgfplotsset{
      major grid style={color=black!10}, minor grid style={color=black!10},
      cycle list={{black, mark=-}, {black, mark=o}}}
    \begin{axis}[
      axis x line=bottom,
      axis y line=left,
      axis line style={draw=none},
      grid=both,
      ylabel={$R_\text{max}$ (TFLOPS)},
      xlabel={Year},
      xtick={2015, 2017, 2019, 2021, 2023},
      xmin=2014,
      xmax=2025,
      ymin=11,
      ymax=22,
      ytick={12, 14, 16, 18, 20},
      yticklabels={$2^{12}$, $2^{14}$, $2^{16}$, $2^{18}$, $2^{20}$},
      xticklabels={2015, 2017, 2019, 2021, 2023},
      width=0.85\textwidth,
      height=0.7\textheight,
      ]
      \addplot table [
        only marks, x=Year, y=Rmax, col sep=comma, color=black,
      ] {data/supercomputer.csv};
      \addplot[domain=2015:2023, black!60] {(x-2015)/2+11.70959};
      \node [draw,coordinate,pin=above:{Frontier}] at (axis cs:2021,20.187) {};
      \node [draw,coordinate,pin=above:{Fugaku}] at (axis cs:2020,18.7537) {};
    \end{axis}
  \end{tikzpicture}
\end{frame}

\begin{frame}{Intentionality of graphical elements}
  \begin{itemize}
    \item Keep redundant/non-data elements when they help communicate
      \textbf{intent}
    \item Grids $\to$ ``compare values''
    \item Annotations $\to$ ``highlight this point''
    \item Trend lines $\to$ ``observe this relationship''
    \item A chart becomes cluttered when it tries to communicate
      \textbf{too many things at once}
    \item Prefer simpler charts highlighting a single phenomenon
  \end{itemize}
\end{frame}

% ---------------------------------------------------------------------------
\subsection{Colors and contrast}
% ---------------------------------------------------------------------------

\begin{frame}{Colors and contrast}
  \begin{itemize}
    \item Color variation must be \textbf{perceptually
      uniform}\footfullcite{Crameri2020}
    \item Neighboring colors should have equal perceptual differences
    \item Must remain legible for colorblind readers and in grayscale
    \item Use palettes with monotonic lightness
      gradient\footfullcite{Crameri2023}
    \item Scientific color maps available for most programming
      languages
  \end{itemize}
\end{frame}

% ---------------------------------------------------------------------------
\subsection{Practical aspects and digital reports}
% ---------------------------------------------------------------------------

\begin{frame}{Practical aspects (Wilke, 2019)}
  Wilke's guide for digital data
  visualization\footfullcite{Wilke2019}:
  \begin{itemize}
    \item Compares versions of the same chart
    \item Labels problematic figures as:
      \begin{itemize}
        \item \textbf{Ugly} --- aesthetic problems, but clear and informative
        \item \textbf{Bad} --- perception problems (confusing, misleading)
        \item \textbf{Wrong} --- mathematical errors or misinterpretation
      \end{itemize}
    \item Some classifications are subjective
    \item Focus: honest, clear, professional design
  \end{itemize}
\end{frame}

% ---------------------------------------------------------------------------
\subsection{Turning charts into stories}
% ---------------------------------------------------------------------------

\begin{frame}{Turning charts into stories}
  A story creates an emotional reaction through ordered observations.

  \vspace{0.3cm}
  \textbf{Story arc}: tension of a question $\to$ resolution.

  \vspace{0.3cm}
  \textbf{OCAR format} (Opening--Challenge--Action--Resolution):
  \begin{enumerate}
    \item \textbf{Opening} --- introduce the topic
    \item \textbf{Challenge} --- raise a question
    \item \textbf{Action} --- present the chart
    \item \textbf{Resolution} --- state the conclusion
  \end{enumerate}

  \vspace{0.3cm}
  Other formats: LDR (Lead--Development--Resolution),\\
  ABDCE (Action--Background--Development--Climax--Ending).
\end{frame}

\begin{frame}{OCAR example}
  \begin{enumerate}
    \item[\textbf{O}] Moore's Law states that the number of transistors in a
      microprocessor doubles every two years, typically reflecting exponential
      performance growth.
    \item[\textbf{C}] Does supercomputer performance follow this trend?
    \item[\textbf{A}] The chart shows $R_\text{max}$ of the top 10
      supercomputers per year (2015--2023).
    \item[\textbf{R}] Despite outliers (Sunway TaihuLight, Frontier), the
      trend line confirms performance doubles roughly every two years.
  \end{enumerate}
\end{frame}

% ===========================================================================
\section{Final comments}
% ===========================================================================

\begin{frame}{Advanced methods}
  \begin{itemize}
    \item PCA, t-SNE, clustering, association rules
    \item Valuable for EDA, but use with caution
    \item When the goal is predictive, exploration can become an end in itself
    \item May find \emph{interesting} but not \emph{important} features
    \item Prefer simple, widely known methods
  \end{itemize}
\end{frame}

\begin{frame}{Takeaways}
  \begin{itemize}
    \item Explore to \textbf{understand} and \textbf{validate}, not to model
    \item Descriptive statistics and visualizations within a storytelling
      context
    \item Graphical integrity: proportional, labeled, contextual
    \item Maximize data-ink ratio --- remove what does not inform
    \item Choose colors that are perceptually uniform and accessible
    \item Structure reports as stories (OCAR)
    \item The exploration report must be concise and objective --- it
      communicates progress to stakeholders
  \end{itemize}
\end{frame}

% ---- End ----

\begin{frame}[standout]
  Questions?
\end{frame}

\end{document}
