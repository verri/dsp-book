% slides/validation.tex - Chapter 8: Solution Validation

% slides/preamble.tex - Shared preamble for all slide decks
% Metropolis theme with grayscale style matching the book

\documentclass[aspectratio=169]{beamer}

\usetheme{metropolis}

% ---------- Colors (grayscale) ----------
\definecolor{bookdark}{gray}{0.2}
\definecolor{bookgray}{gray}{0.5}
\definecolor{booklight}{gray}{0.92}

\setbeamercolor{normal text}{fg=bookdark, bg=white}
\setbeamercolor{alerted text}{fg=bookdark}
\setbeamercolor{frametitle}{fg=white, bg=bookdark}
\setbeamercolor{title separator}{fg=bookgray}
\setbeamercolor{progress bar}{fg=bookgray, bg=booklight}
\setbeamercolor{block title}{fg=white, bg=bookgray}
\setbeamercolor{block body}{fg=bookdark, bg=booklight}
\setbeamercolor{block title alerted}{fg=white, bg=bookdark}
\setbeamercolor{block body alerted}{fg=bookdark, bg=booklight}
\setbeamercolor{block title example}{fg=white, bg=bookgray}
\setbeamercolor{block body example}{fg=bookdark, bg=booklight}

\setbeamertemplate{frame numbering}[fraction]

% ---------- Fonts (matching the book) ----------
\usepackage[T1]{fontenc}
\usepackage{fontspec}
\usepackage[warnings-off={mathtools-colon,mathtools-overbracket}]{unicode-math}

\setmathfont{STIXTwoMath}[
  Extension={.otf},
  Path={./fonts/},
  Scale=1]

\setsansfont{STIXTwoText}[
  Extension={.otf},
  Path={./fonts/},
  UprightFont={*-Regular},
  BoldFont={*-Bold},
  ItalicFont={*-Italic},
  BoldItalicFont={*-BoldItalic}]

\setmainfont{STIXTwoText}[
  Extension={.otf},
  Path={./fonts/},
  UprightFont={*-Regular},
  BoldFont={*-Bold},
  ItalicFont={*-Italic},
  BoldItalicFont={*-BoldItalic}]

\setmonofont{CourierPrime}[
  Extension={.ttf},
  Path={./fonts/},
  UprightFont={*-Regular},
  BoldFont={*-Bold},
  ItalicFont={*-Italic},
  BoldItalicFont={*-BoldItalic},
  Scale=0.9]

% ---------- Packages ----------
\usepackage{amsmath}
\usepackage{mathtools}
\usepackage{graphicx}
\usepackage{booktabs}

% ---------- TikZ ----------
\usepackage{tikz}
\usetikzlibrary{shapes, arrows.meta, positioning, shapes.geometric, fit}

\tikzset{%
  decision/.style={draw, diamond, text centered, minimum height=0.5cm, minimum width=1cm},
  block/.style={rectangle, draw, text width=6em, text centered, rounded corners, minimum height=3em},
  mediumblock/.style={rectangle, draw, text width=3em, text centered, rounded corners, minimum height=2em},
  darkblock/.style={block, fill=gray, text=white},
  smallblock/.style={rectangle, rounded corners, draw, font=\tiny, minimum height=1em, inner sep=2pt},
  smalldarkblock/.style={smallblock, fill=gray, text=white},
  darkcircle/.style={draw, circle, fill=gray, text centered, text=white},
  smallcircle/.style={draw, circle, text centered, font=\tiny},
  smalldarkcircle/.style={smallcircle, fill=gray, text=white},
  line/.style={draw, -latex},
  dline/.style={draw, latex-latex},
  bigarrow/.style={draw, -latex, line width=3pt, gray},
}

% ---------- Math operators ----------
\DeclareMathOperator*{\argmax}{arg\,max}
\DeclareMathOperator*{\argmin}{arg\,min}
\DeclareMathOperator{\Prob}{P}
\DeclareMathOperator{\E}{E}
\DeclareMathOperator{\Var}{Var}
\DeclareMathOperator{\sign}{sign}
\DeclareMathOperator{\clamp}{clamp}

% ---------- Hyperref ----------
\usepackage{hyperref}
\hypersetup{colorlinks, urlcolor=bookgray, linkcolor=bookdark}

% ---------- Commands ----------
\renewcommand{\vec}[1]{\mathbf{#1}}
\newcommand{\code}[1]{\colorbox{black!10!white}{\texttt{#1}}}

% ---------- Book info, license, and disclaimer ----------
\newcommand{\bookframe}{%
  \begin{frame}{About these slides}
    These slides are companion material for the book

    \vspace{0.3cm}
    \begin{center}
      \textbf{Data Science Project: An Inductive Learning Approach}\\[2pt]
      Prof.~Dr.~Filipe A. N. Verri\\[4pt]
      \url{https://leanpub.com/dsp}
    \end{center}

    \vspace{0.3cm}
    All intellectual content comes from the book and is not AI-generated.
    Slides were produced with the assistance of
    \href{https://claude.ai/code}{Claude Code}.

    \vspace{0.3cm}
    {\small
    Licensed under
    \href{https://creativecommons.org/licenses/by-nc/4.0/}{CC BY-NC 4.0}.
    You are free to modify and redistribute this work as long as you give
    proper credit and do not use it for commercial purposes.}
  \end{frame}%
}

% ---------- TikZ color presets (used by book figures) ----------
\colorlet{circle edge}{black!50}
\colorlet{circle area}{black!20}
\tikzset{
  filled/.style={fill=circle area, draw=circle edge, thick},
  outline/.style={draw=circle edge, thick},
}


\title{Solution Validation}
\subtitle{Data Science Project: An Inductive Learning Approach}
\author{Prof.~Dr.~Filipe A. N. Verri}
\date{}

\begin{document}

\maketitle
\bookframe

% ---- Epigraph ----

\begin{frame}{}
  \vfill
  \begin{quote}
    All models are wrong, but some are useful.
    \begin{flushright}
      --- George E. P. Box, \textit{Robustness in Statistics}
    \end{flushright}
  \end{quote}
  \vfill
\end{frame}

% ---- Overview ----

\begin{frame}{Overview}
  \begin{columns}[T]
    \begin{column}{0.48\textwidth}
      \textbf{Contents}
      \begin{itemize}
        \item Evaluation
        \item An experimental plan for data science
      \end{itemize}
    \end{column}
    \begin{column}{0.48\textwidth}
      \textbf{Objectives}
      \begin{itemize}
        \item Understand the importance of experimental planning
        \item Learn the main evaluation metrics
        \item Learn how to design an experimental plan
      \end{itemize}
    \end{column}
  \end{columns}
\end{frame}

% ===========================================================================
\section{Evaluation}
% ===========================================================================

\begin{frame}{Evaluation setup}
  \begin{enumerate}
    \item \textbf{Preprocessing}: apply $F$ to training set $\to$ fitted
      preprocessor $f_\phi$
    \item \textbf{Learning}: train machine $M$ on adjusted training data $\to$
      model $f_\theta$
    \item \textbf{Transformation}: apply $f_\phi$ to test set (no access to
      target $y$)
    \item \textbf{Prediction}: $\hat{y}_i = f_\theta(f_\phi(\vec{x}_i))$
    \item \textbf{Evaluation}: compare $\hat{y}_i$ with $y_i$ on test set
  \end{enumerate}

  \vspace{0.3cm}
  Training and test sets must be \textbf{disjoint}:
  $\mathcal{I}_\text{train} \cap \mathcal{I}_\text{test} = \emptyset$
\end{frame}

% ---------------------------------------------------------------------------
\subsection{Binary classification evaluation}
% ---------------------------------------------------------------------------

\begin{frame}{Confusion matrix}
  \centering
  \rowcolors{2}{black!10!white}{}
  \begin{tabular}{ll|cc}
    \toprule
    & & \multicolumn{2}{c}{\textbf{Predicted}} \\
    & & 1 & 0 \\
    \midrule
    \multirow{2}{*}{\textbf{Expected}} & 1 & TP & FN \\
    & 0 & FP & TN \\
    \bottomrule
  \end{tabular}

  \vspace{0.5cm}
  \raggedright
  \small
  \begin{columns}[T]
    \begin{column}{0.45\textwidth}
      \begin{itemize}
        \item \textbf{TP}: true positives
        \item \textbf{TN}: true negatives
      \end{itemize}
    \end{column}
    \begin{column}{0.45\textwidth}
      \begin{itemize}
        \item \textbf{FP}: false positives
        \item \textbf{FN}: false negatives
      \end{itemize}
    \end{column}
  \end{columns}
\end{frame}

\begin{frame}{Accuracy and balanced accuracy}
  \textbf{Accuracy} --- proportion of correct predictions:
  \[
    \text{Accuracy} = \frac{\text{TP} + \text{TN}}
      {\text{TP} + \text{TN} + \text{FP} + \text{FN}}
  \]

  \vspace{0.2cm}
  \textbf{Balanced accuracy} --- average of per-class rates:
  \[
    \text{Balanced Accuracy} = \frac{\text{TPR} + \text{TNR}}{2}
  \]
  where $\text{TPR} = \frac{\text{TP}}{\text{TP} + \text{FN}}$ and
  $\text{TNR} = \frac{\text{TN}}{\text{TN} + \text{FP}}$.

  \vspace{0.2cm}
  \small
  Accuracy can be misleading with imbalanced classes; balanced accuracy
  gives equal weight to each class.
\end{frame}

\begin{frame}{Precision and recall}
  Both focus on the \textbf{positive class}:

  \vspace{0.3cm}
  \textbf{Precision} --- confidence in positive predictions:
  \[
    \text{Precision} = \frac{\text{TP}}{\text{TP} + \text{FP}}
  \]

  \vspace{0.2cm}
  \textbf{Recall} (TPR) --- completeness of positive retrieval:
  \[
    \text{Recall} = \frac{\text{TP}}{\text{TP} + \text{FN}}
  \]

  \vspace{0.2cm}
  \small
  Precision: avoid false alarms. Recall: avoid missing positives.
\end{frame}

\begin{frame}{F-score and specificity}
  \textbf{F-score} --- weighted harmonic mean of precision and recall:
  \[
    \text{F}_\beta\text{-score} =
      \frac{(1 + \beta^2) \cdot \text{Precision} \cdot \text{Recall}}
        {\beta^2 \cdot \text{Precision} + \text{Recall}}
  \]
  $\beta = 1$: equal weight. $\beta > 1$: favor precision.
  $0 < \beta < 1$: favor recall.

  \vspace{0.3cm}
  \textbf{Specificity} (TNR) --- focuses on the negative class:
  \[
    \text{Specificity} = \frac{\text{TN}}{\text{TN} + \text{FP}}
  \]
\end{frame}

\begin{frame}{Summary of classification metrics}
  \centering\small
  \rowcolors{2}{black!10!white}{}
  \begin{tabular}{l c c}
    \toprule
    \textbf{Metric} & \textbf{Focus} & \textbf{Interpretation} \\
    \midrule
    Accuracy           & Symmetrical & Penalizes all \\
    Balanced Accuracy  & Symmetrical & Penalizes all (weighted) \\
    Recall (TPR)       & Positive & Penalizes FN \\
    Precision          & Positive & Penalizes FP \\
    F-score            & Positive & Penalizes all (weighted) \\
    Specificity (TNR)  & Negative & Penalizes FP \\
    \bottomrule
  \end{tabular}
\end{frame}

\begin{frame}{Metrics for trivial classifiers}
  \centering\small
  \rowcolors{2}{black!10!white}{}
  \begin{tabular}{l c c c}
    \toprule
    \textbf{Metric} & \textbf{Guess 1} & \textbf{Guess 0} & \textbf{Random} \\
    \midrule
    Accuracy$^\dagger$  & $\pi$ & $1 - \pi$ & $0.5$ \\
    Balanced Accuracy   & $0.5$ & $0.5$ & $0.5$ \\
    Recall (TPR)        & $1$ & $0$ & $0.5$ \\
    Precision$^\dagger$ & $\pi$ & --- & $\pi$ \\
    F$_1$-score$^\dagger$ & $\frac{2\pi}{1+\pi}$ & $0$ & $\frac{2\pi}{1+2\pi}$ \\
    Specificity (TNR)   & $0$ & $1$ & $0.5$ \\
    \bottomrule
  \end{tabular}

  \vspace{0.3cm}
  \raggedright
  \small
  $\pi$ = ratio of positive samples. $^\dagger$ = affected by class imbalance.\\
  Prefer asymmetric metrics when the positive class is the \textbf{minority}.
\end{frame}

% ---------------------------------------------------------------------------
\subsection{Regression estimation evaluation}
% ---------------------------------------------------------------------------

\begin{frame}{Regression metrics --- absolute errors}
  Residual: $\epsilon_i = \hat{y}_i - y_i$

  \vspace{0.3cm}
  \textbf{Mean Absolute Error}:
  $\displaystyle\text{MAE} = \frac{1}{n} \sum_{i=1}^n |\epsilon_i|$

  \vspace{0.3cm}
  \textbf{Mean Squared Error}:
  $\displaystyle\text{MSE} = \frac{1}{n} \sum_{i=1}^n \epsilon_i^2$

  \vspace{0.3cm}
  \textbf{Root Mean Squared Error}:
  $\text{RMSE} = \sqrt{\text{MSE}}$

  \vspace{0.3cm}
  \small MAE and RMSE are in the same unit as the target variable.
  MSE/RMSE penalize large errors more.
\end{frame}

\begin{frame}{Regression metrics --- relative errors}
  For strictly positive targets ($y_i > 0$, $\hat{y}_i > 0$):

  \vspace{0.3cm}
  \textbf{Mean Absolute Percentage Error}:
  $\displaystyle\text{MAPE} = \frac{1}{n} \sum_{i=1}^n \frac{|\epsilon_i|}{y_i}$

  \vspace{0.3cm}
  \textbf{Mean Absolute Logarithmic Error}:
  $\displaystyle\text{MALE} = \frac{1}{n} \sum_{i=1}^n
    |\ln\hat{y}_i - \ln y_i|$

  \vspace{0.3cm}
  \small
  Useful when target has large range. MAPE punishes overestimation more.
  MALE is symmetric in multiplicative terms:
  $|\ln\frac{\hat{y}}{y}| = \ln\max\!\left(\frac{\hat{y}}{y},
  \frac{y}{\hat{y}}\right)$.
\end{frame}

\begin{frame}{MAPE vs.\ MALE}
  \centering
  \rowcolors{2}{black!10!white}{}
  \begin{tabular}{rrrrl}
    \toprule
    $\hat{y}$ & $y$ & $|\epsilon|$ & MAPE & $\exp(\text{MALE})$ \\
    \midrule
    100 & 10 & 90 & 9.0 & 10 \\
      1 & 10 &  9 & 0.9 & 10 \\
    \bottomrule
  \end{tabular}

  \vspace{0.5cm}
  \raggedright
  MAPE gives 9.0 for overestimation and 0.9 for underestimation.\\
  MALE treats both as a factor of 10.
\end{frame}

% ---------------------------------------------------------------------------
\subsection{Probabilistic classification}
% ---------------------------------------------------------------------------

\begin{frame}{Probabilistic classification}
  Regressor output $f_R(\vec{x}) \in [0, 1]$ converted to classifier:
  \[
    f_C(\vec{x}; \tau) = \begin{cases}
      1 & \text{if } f_R(\vec{x}) \geq \tau \\
      0 & \text{otherwise}
    \end{cases}
  \]

  \vspace{0.3cm}
  \begin{itemize}
    \item Low $\tau \approx 0$: high recall, low specificity
    \item High $\tau \approx 1$: high specificity, low recall
    \item Default: $\tau = 0.5$
  \end{itemize}
\end{frame}

\begin{frame}{Example: probability regressor output}
  \begin{columns}[T]
    \begin{column}{0.35\textwidth}
      \centering\small
      \rowcolors{2}{black!10!white}{}
      \begin{tabular}{rr}
        \toprule
        Expected & Predicted \\
        \midrule
        0 & 0.1 \\
        0 & 0.5 \\
        0 & 0.2 \\
        0 & 0.6 \\
        1 & 0.4 \\
        1 & 0.9 \\
        1 & 0.7 \\
        1 & 0.8 \\
        1 & 0.9 \\
        \bottomrule
      \end{tabular}
    \end{column}
    \begin{column}{0.6\textwidth}
      \centering\small
      Sort by predicted value, then compute TPR and FPR for each threshold:

      \vspace{0.2cm}
      \rowcolors{2}{black!10!white}{}
      \begin{tabular}{rrrr}
        \toprule
        Exp. & Threshold & TPR & FPR \\
        \midrule
        ---  & $\infty$ & $0/5$ & $0/4$ \\
        1 & $0.9$  & $2/5$ & $0/4$ \\
        1 & $0.8$  & $3/5$ & $0/4$ \\
        1 & $0.7$  & $4/5$ & $0/4$ \\
        0 & $0.6$  & $4/5$ & $1/4$ \\
        0 & $0.5$  & $4/5$ & $2/4$ \\
        1 & $0.4$  & $5/5$ & $2/4$ \\
        0 & $0.2$  & $5/5$ & $3/4$ \\
        0 & $0.1$  & $5/5$ & $4/4$ \\
        \bottomrule
      \end{tabular}
    \end{column}
  \end{columns}
\end{frame}

% ---- Figure: ROC curve ----

\begin{frame}{ROC curve}
  \centering
  \begin{tikzpicture}
    \begin{axis}[
        axis lines=left,
        xlabel={FPR},
        ylabel={TPR},
        xmin=0, xmax=1.05,
        ymin=0, ymax=1.05,
        xtick={0, 0.25, 0.5, 0.75, 1.0},
        ytick={0, 0.2, 0.4, 0.6, 0.8, 1.0},
        grid=both,
        width=0.55\textwidth,
        height=0.55\textwidth,
      ]

      % ROC curve
      \addplot[thick, mark=*, mark size=1.5pt] coordinates {
        (0.0, 0.0) (0.0, 0.2) (0.0, 0.4)
        (0.0, 0.6) (0.0, 0.8) (0.25, 0.8)
        (0.50, 0.8) (0.50, 1.0) (0.75, 1.0) (1.0, 1.0)
      };

      % Diagonal (random classifier)
      \addplot[dashed, gray] coordinates {(0, 0) (1, 1)};
    \end{axis}
  \end{tikzpicture}

  \small
  Points above diagonal $=$ better than random.
  \quad AUC $= 0.9$ in this example.
\end{frame}

\begin{frame}{Area under the ROC curve (AUC)}
  \begin{itemize}
    \item Summarizes the performance of all possible thresholds
    \item Ranges from 0 to 1 (1 = perfect)
    \item \textbf{Scale invariant}: measures how well predictions are ranked
    \item \textbf{Robust to class imbalance}: considers both recall and
      specificity
  \end{itemize}
\end{frame}

% ===========================================================================
\section{An experimental plan for data science}
% ===========================================================================

\begin{frame}{Why experimental planning?}
  \begin{itemize}
    \item Data science is \textbf{experimental} --- we lack theoretical models
      predicting algorithm performance
    \item Data and learning are \textbf{stochastic}
    \item Robust planning ensures reliable, decision-worthy results
  \end{itemize}

  \vspace{0.3cm}
  Key elements:
  \begin{itemize}
    \item \textbf{Hypothesis}: what do we want to validate?
    \item \textbf{Data}: representative dataset
    \item \textbf{Solution search}: preprocessing + learning
    \item \textbf{Performance metric}: how do we measure success?
  \end{itemize}
\end{frame}

% ---------------------------------------------------------------------------
\subsection{Sampling strategy}
% ---------------------------------------------------------------------------

\begin{frame}{Performance as a random variable}
  \begin{itemize}
    \item Training and testing on the same data $\Rightarrow$ \textbf{optimistic}
    \item Single hold-out $\Rightarrow$ depends on the random split
    \item Solution: treat performance as a \textbf{random variable}
    \item Generate multiple training/test splits to study the distribution
  \end{itemize}
\end{frame}

% ---- Figure: Cross-validation ----

\begin{frame}{Cross-validation}
  \centering
  \begin{tikzpicture}[scale=0.9, transform shape]
    \foreach \i in {1, 2, 3, 4} {
      \node at (2.2 * \i, 0) {\small Fold \i};
      \foreach \j in {1, 2, 3, 4} {
        \ifnum\i=\j
          \node [smallblock, minimum width=16mm, minimum height=6mm]
            (fold\i\j) at (2.2 * \i, -\j) {\small Test};
        \else
          \node [smalldarkblock, minimum width=16mm, minimum height=6mm]
            (fold\i\j) at (2.2 * \i, -\j) {\small Training};
        \fi
      }
    }
    \foreach \j in {1, 2, 3, 4} {
      \node [draw, dashed, fit={(fold1\j) (fold4\j)}] {};
    }
  \end{tikzpicture}

  \vspace{0.3cm}
  \small
  $r$ folds; each fold is the test set once, training set $r{-}1$ times.\\
  Prefer: \textbf{repeated} and \textbf{stratified} cross-validation.
\end{frame}

% ---------------------------------------------------------------------------
\subsection{Collecting evidence}
% ---------------------------------------------------------------------------

% ---- Figure: Experimental plan ----

\begin{frame}{Experimental plan}
  \centering
  \resizebox{0.4\textwidth}{!}{
  \begin{tikzpicture}
    \node [darkcircle] (data) at (0, 0) {Data};
    \node [block] (sampling) at (0, -2) {Sampling strategy};
    \path [line] (data) -- (sampling);

    \foreach \i in {1, 2, 4, 5} {
      \draw [dashed] (-7 + 2 * \i, -4.5) rectangle (-5.1 + 2 * \i, -3.5);
      \path [line] (sampling) -- (-6.1 + 2 * \i, -3.5);

      \node [smalldarkblock] (train\i) at (-6.4 + 2 * \i, -4) {Training};
      \node [smallblock] (test\i) at (-5.6 + 2 * \i, -4) {Test};

      \path [line] (-6.1 + 2 * \i, -4.5) -- (-6.1 + 2 * \i, -5.5);
    }
    \node [anchor=center] at (0, -4) {\dots};

    \draw [dashed] (-5, -5.5) rectangle (4.9, -10.5);

    \node [smalldarkblock, font=\small, inner sep=4pt] (train) at (-4, -7) {Training};
    \node [smallblock, inner sep=4pt] (test) at (-4, -9) {Test (no target)};

    \draw [dashed] (-3, -6) rectangle (3, -8);
    \node [anchor=south] at (0, -6.1) {Solution search algorithm};

    \node [block] (handling) at (-1.5, -7) {Preprocessing};
    \node [block] (learning) at (1.5, -7) {Machine learning};
    \node (model) at (4, -7) {%
      $\left[
      \begin{array}{c}
        \phi \\
        \theta \\
      \end{array}
      \right]$
    };

    \path [line] (train) -- (handling);
    \path [line] (handling) -- (learning);
    \path [line, dashed] (3, -7) -- (model);

    \node [block] (preprocess) at (-1.5, -9) {Preprocessor};
    \node [block] (prediction) at (1.5, -9) {Model};

    \path [line, dashed] (handling) -- (preprocess);
    \path [line, dashed] (learning) -- (prediction);

    \path [line] (test) -- (preprocess);
    \path [line] (preprocess) -- (prediction);

    \node [smallblock, inner sep=4pt] (predicted) at (4, -9) {predictions};
    \node (performance) at (4, -10) {$p$};
    \path [line] (prediction) -- (predicted);

    \node [smallblock, inner sep=4pt] (labels) at (-4, -10) {Test (target)};
    \path [line] (labels) -- (performance);
    \path [line] (predicted) -- (performance);

    \node (perfs) at (-4.2, -12) {%
      $\left[
        \begin{array}{c}
          p_1 \\
          p_2 \\
          \vdots \\
        \end{array}
      \right]$
    };

    \node [block] (hypothesis) at (-1, -12) {Hypothesis test};

    \path [line, dashed] (-4.2, -10.5) -- (perfs);
    \path [line] (perfs) -- (hypothesis);
  \end{tikzpicture}
  }
\end{frame}

\begin{frame}{Collecting evidence --- summary}
  For each sampling $D_k$:
  \begin{enumerate}
    \item Fit preprocessor $\phi_k$ and model $\theta_k$ on $D_{k,\text{train}}$
    \item Predict $\hat{y}_i = f_{\phi_k, \theta_k}(x_i)$ on $D_{k,\text{test}}$
    \item Compute performance $p_k = R(\vec{y}, \hat{\vec{y}})$
  \end{enumerate}

  \vspace{0.3cm}
  By definition, $p_k$ is free of \textbf{leakage}: parameters found without
  test data, predictions use only $x_i$ (no target $y_i$).
\end{frame}

\begin{frame}{Evaluation vs.\ validation}
  \begin{itemize}
    \item \textbf{Evaluation}: assessing performance using a test set
    \item \textbf{Validation}: interpreting the evaluation results to determine
      how well the solution generalizes to unseen data
  \end{itemize}

  \vspace{0.3cm}
  The sampled performance values $p_1, p_2, \ldots$ can be analyzed
  statistically to prove (or disprove) the hypothesis.
\end{frame}

% ---------------------------------------------------------------------------
\subsection{Estimating expected performance}
% ---------------------------------------------------------------------------

\begin{frame}{Estimating expected performance}
  Goal: reach performance threshold $p^*$.

  \vspace{0.3cm}
  Simple approach:
  \begin{itemize}
    \item Run $10 \times 10$-fold cross-validation
    \item Compute $\bar{p}$ and $\sigma$
    \item If $\bar{p} - \sigma \gg p^*$, the solution is likely good enough
  \end{itemize}

  \vspace{0.3cm}
  More sophisticated: \textbf{Bayesian analysis} to estimate the
  probability distribution of performance.
\end{frame}

\begin{frame}{Bayesian correlated $t$-test}
  Let $z_k = p_k - p^*$ (performance gain over goal).

  \vspace{0.2cm}
  Generative model (Benavoli et al., 2017):
  \[
    \vec{z} = \vec{1}\mu + \vec{v}, \quad
    \vec{v} \sim \operatorname{MVN}(0, \Sigma)
  \]
  with $\Sigma_{ii} = \sigma^2$ and $\Sigma_{ij} = \sigma^2 \rho$ for $i \neq j$.

  \vspace{0.2cm}
  Posterior of $\mu$ is a Student distribution:
  \[
    \mu \mid \vec{z} \;\sim\; \operatorname{St}\!\left(n{-}1,\; \bar{z},\;
      \left(\tfrac{1}{n} + \tfrac{\rho}{1-\rho}\right) s^2 \right)
  \]

  \vspace{0.2cm}
  Validate: $\Prob(\mu > 0 \mid \vec{z}) > \gamma$ (confidence level).
\end{frame}

% ---------------------------------------------------------------------------
\subsection{Comparing strategies}
% ---------------------------------------------------------------------------

\begin{frame}{Comparing strategies}
  Baseline $A(\lambda_0)$ vs.\ candidate $A(\lambda)$:
  \begin{itemize}
    \item Use the \textbf{same samplings} (paired performance samples)
    \item Let $\vec{z} = \vec{p}(\lambda) - \vec{p}(\lambda_0)$
    \item Validate: $\Prob(\mu > 0 \mid \vec{z}) > \gamma$
    \item $\mu > 0$: candidate is better
  \end{itemize}

  \vspace{0.3cm}
  Iterate: compare $A(\lambda_1)$ vs.\ best so far, keep the winner.

  \vspace{0.2cm}
  \small
  If confidence is not reached but $\mu > 0$: consider interpretability,
  computational cost, or ease of implementation. Always check if
  $\Prob(\mu < 0 \mid \vec{z})$ is acceptable.
\end{frame}

% ---------------------------------------------------------------------------
\subsection{Nesting experiments}
% ---------------------------------------------------------------------------

\begin{frame}{Nesting experiments}
  \begin{itemize}
    \item Hyperparameters $\lambda$ can be optimized with a \textbf{nested}
      experimental plan (e.g., grid search)
    \item Inner loop finds best $\lambda$; outer loop estimates performance
    \item Never use inner choices for production decisions
  \end{itemize}

  \vspace{0.3cm}
  Trade-offs:
  \begin{itemize}
    \item Computationally expensive (combinations multiply)
    \item Inner dataset is smaller $\Rightarrow$ less reliable estimates
    \item Alternative: unnest by comparing options two by two
  \end{itemize}
\end{frame}

% ---------------------------------------------------------------------------

\begin{frame}{Final remarks}
  \begin{itemize}
    \item Framework assumes data is \textbf{i.i.d.}
    \item Time-series, spatial data $\Rightarrow$ different sampling strategies
    \item Changing the sampling strategy also changes the validation method
  \end{itemize}
\end{frame}

% ---- Takeaways ----

\begin{frame}{Takeaways}
  \begin{itemize}
    \item Evaluation metrics should be chosen according to project goals
    \item The experimental plan gathers performance data systematically
    \item A hypothesis test validates the results
    \item Performance is a \textbf{random variable} --- study its distribution,
      not a single value
    \item Same samplings must be used when comparing algorithms
  \end{itemize}
\end{frame}

% ---- End ----

\begin{frame}[standout]
  Questions?
\end{frame}

\end{document}
