% slides/preprocessing.tex - Chapter 7: Data Preprocessing

% slides/preamble.tex - Shared preamble for all slide decks
% Metropolis theme with grayscale style matching the book

\documentclass[aspectratio=169]{beamer}

\usetheme{metropolis}

% ---------- Colors (grayscale) ----------
\definecolor{bookdark}{gray}{0.2}
\definecolor{bookgray}{gray}{0.5}
\definecolor{booklight}{gray}{0.92}

\setbeamercolor{normal text}{fg=bookdark, bg=white}
\setbeamercolor{alerted text}{fg=bookdark}
\setbeamercolor{frametitle}{fg=white, bg=bookdark}
\setbeamercolor{title separator}{fg=bookgray}
\setbeamercolor{progress bar}{fg=bookgray, bg=booklight}
\setbeamercolor{block title}{fg=white, bg=bookgray}
\setbeamercolor{block body}{fg=bookdark, bg=booklight}
\setbeamercolor{block title alerted}{fg=white, bg=bookdark}
\setbeamercolor{block body alerted}{fg=bookdark, bg=booklight}
\setbeamercolor{block title example}{fg=white, bg=bookgray}
\setbeamercolor{block body example}{fg=bookdark, bg=booklight}

\setbeamertemplate{frame numbering}[fraction]

% ---------- Fonts (matching the book) ----------
\usepackage[T1]{fontenc}
\usepackage{fontspec}
\usepackage[warnings-off={mathtools-colon,mathtools-overbracket}]{unicode-math}

\setmathfont{STIXTwoMath}[
  Extension={.otf},
  Path={./fonts/},
  Scale=1]

\setsansfont{STIXTwoText}[
  Extension={.otf},
  Path={./fonts/},
  UprightFont={*-Regular},
  BoldFont={*-Bold},
  ItalicFont={*-Italic},
  BoldItalicFont={*-BoldItalic}]

\setmainfont{STIXTwoText}[
  Extension={.otf},
  Path={./fonts/},
  UprightFont={*-Regular},
  BoldFont={*-Bold},
  ItalicFont={*-Italic},
  BoldItalicFont={*-BoldItalic}]

\setmonofont{CourierPrime}[
  Extension={.ttf},
  Path={./fonts/},
  UprightFont={*-Regular},
  BoldFont={*-Bold},
  ItalicFont={*-Italic},
  BoldItalicFont={*-BoldItalic},
  Scale=0.9]

% ---------- Packages ----------
\usepackage{amsmath}
\usepackage{mathtools}
\usepackage{graphicx}
\usepackage{booktabs}

% ---------- TikZ ----------
\usepackage{tikz}
\usetikzlibrary{shapes, arrows.meta, positioning, shapes.geometric, fit}

\tikzset{%
  decision/.style={draw, diamond, text centered, minimum height=0.5cm, minimum width=1cm},
  block/.style={rectangle, draw, text width=6em, text centered, rounded corners, minimum height=3em},
  mediumblock/.style={rectangle, draw, text width=3em, text centered, rounded corners, minimum height=2em},
  darkblock/.style={block, fill=gray, text=white},
  smallblock/.style={rectangle, rounded corners, draw, font=\tiny, minimum height=1em, inner sep=2pt},
  smalldarkblock/.style={smallblock, fill=gray, text=white},
  darkcircle/.style={draw, circle, fill=gray, text centered, text=white},
  smallcircle/.style={draw, circle, text centered, font=\tiny},
  smalldarkcircle/.style={smallcircle, fill=gray, text=white},
  line/.style={draw, -latex},
  dline/.style={draw, latex-latex},
  bigarrow/.style={draw, -latex, line width=3pt, gray},
}

% ---------- Math operators ----------
\DeclareMathOperator*{\argmax}{arg\,max}
\DeclareMathOperator*{\argmin}{arg\,min}
\DeclareMathOperator{\Prob}{P}
\DeclareMathOperator{\E}{E}
\DeclareMathOperator{\Var}{Var}
\DeclareMathOperator{\sign}{sign}
\DeclareMathOperator{\clamp}{clamp}

% ---------- Hyperref ----------
\usepackage{hyperref}
\hypersetup{colorlinks, urlcolor=bookgray, linkcolor=bookdark}

% ---------- Commands ----------
\renewcommand{\vec}[1]{\mathbf{#1}}
\newcommand{\code}[1]{\colorbox{black!10!white}{\texttt{#1}}}

% ---------- Book info, license, and disclaimer ----------
\newcommand{\bookframe}{%
  \begin{frame}{About these slides}
    These slides are companion material for the book

    \vspace{0.3cm}
    \begin{center}
      \textbf{Data Science Project: An Inductive Learning Approach}\\[2pt]
      Prof.~Dr.~Filipe A. N. Verri\\[4pt]
      \url{https://leanpub.com/dsp}
    \end{center}

    \vspace{0.3cm}
    All intellectual content comes from the book and is not AI-generated.
    Slides were produced with the assistance of
    \href{https://claude.ai/code}{Claude Code}.

    \vspace{0.3cm}
    {\small
    Licensed under
    \href{https://creativecommons.org/licenses/by-nc/4.0/}{CC BY-NC 4.0}.
    You are free to modify and redistribute this work as long as you give
    proper credit and do not use it for commercial purposes.}
  \end{frame}%
}

% ---------- TikZ color presets (used by book figures) ----------
\colorlet{circle edge}{black!50}
\colorlet{circle area}{black!20}
\tikzset{
  filled/.style={fill=circle area, draw=circle edge, thick},
  outline/.style={draw=circle edge, thick},
}


\title{Data Preprocessing}
\subtitle{Data Science Project: An Inductive Learning Approach}
\author{Prof.~Dr.~Filipe A. N. Verri}
\date{}

\begin{document}

\maketitle
\bookframe

% ---- Epigraph ----

\begin{frame}{}
  \vfill
  \begin{quote}
    I find your lack of faith disturbing.
    \begin{flushright}
      --- Darth Vader, \textit{Star Wars: Episode IV} (1977)
    \end{flushright}
  \end{quote}
  \vfill
\end{frame}

% ---- Overview ----

\begin{frame}{Overview}
  \begin{columns}[T]
    \begin{column}{0.48\textwidth}
      \textbf{Contents}
      \begin{itemize}
        \item Introduction
        \item Data cleaning
        \item Data sampling
        \item Data transformation
      \end{itemize}
    \end{column}
    \begin{column}{0.48\textwidth}
      \textbf{Objectives}
      \begin{itemize}
        \item Understand the main data preprocessing tasks and techniques
        \item Learn the behavior of the preprocessing chain (fitting, adjustment,
          application)
      \end{itemize}
    \end{column}
  \end{columns}
\end{frame}

% ===========================================================================
\section{Introduction}
% ===========================================================================

\begin{frame}{Why preprocess?}
  \begin{itemize}
    \item Tidy data is not necessarily suitable for modeling
    \item Example: perceptron requires \textbf{numerical} inputs
    \item Preprocessing adjusts data for the chosen learning machine
    \item Operations are \textbf{dependent on the learning method}
  \end{itemize}
\end{frame}

\begin{frame}{Three steps of a preprocessing technique}
  \begin{enumerate}
    \item \textbf{Fitting}: parameters adjusted to the training data
    \item \textbf{Adjustment}: training data transformed according to
      fitted parameters (may change sample size/distribution)
    \item \textbf{Applying}: operation applied to new data, sample by sample
  \end{enumerate}

  \vspace{0.3cm}
  Understanding these steps is crucial to avoid \textbf{data leakage}.
\end{frame}

\begin{frame}{Formal definition}
  Strategy $F$ takes a table $T = (K, H, c)$ and returns:
  \begin{itemize}
    \item Adjusted table $T' = (K', H', c')$
    \item Fitted preprocessor $f_\phi(z)$
  \end{itemize}

  \vspace{0.3cm}
  A chain of operations $F_1, \ldots, F_m$:
  \[
    f(z; \phi) = \left(f_{\phi_1} \circ \cdots \circ f_{\phi_m}\right)(z)
  \]
  Each operation depends on the result of the previous ones.
\end{frame}

\begin{frame}{Degeneration}
  The preprocessor \textbf{degenerates} over tuple $z$ if $f_\phi(z) = (?, \ldots, ?)$.

  \vspace{0.3cm}
  \begin{itemize}
    \item Unexpected values, incomplete information, \ldots
    \item If any step $f_{\phi_i}$ degenerates, the whole chain degenerates
    \item Developer must define a \textbf{default behavior}:
      \begin{itemize}
        \item Return a default value
        \item Redirect to a different model
        \item Raise an error/warning
      \end{itemize}
  \end{itemize}
\end{frame}

\begin{frame}{Preprocessing task categories}
  \begin{enumerate}
    \item \textbf{Data cleaning} --- remove errors and inconsistencies
    \item \textbf{Data sampling} --- select or create variations of the training set
    \item \textbf{Data transformation} --- adjust types and variables for modeling
  \end{enumerate}

  \vspace{0.3cm}
  Presented in typical application order (not fixed).
\end{frame}

% ===========================================================================
\section{Data cleaning}
% ===========================================================================

% ---------------------------------------------------------------------------
\subsection{Treating inconsistent data}
% ---------------------------------------------------------------------------

\begin{frame}{Treating inconsistent data}
  Three common tasks (parameters \textbf{not fitted} from data):
  \begin{itemize}
    \item \textbf{Unit conversion} --- ensure same units across columns
    \item \textbf{Range check} --- validate values within expected bounds
    \item \textbf{Category standardization} --- unify different representations
  \end{itemize}

  \vspace{0.3cm}
  Could be done in data handling, but having them in the preprocessor
  ensures consistent treatment of new data in production.
\end{frame}

\begin{frame}{Unit conversion}
  \centering\small
  \rowcolors{2}{black!10!white}{}
  \begin{tabular}{lp{8cm}}
    \toprule
    \multicolumn{2}{c}{\textbf{Unit conversion}} \\
    \midrule
    \textbf{Goal} &
      Convert physical quantities into the same unit of measurement. \\
    \textbf{Fitting} &
      None. User declares units and conversion factors. \\
    \textbf{Adjustment} &
      Sample by sample, independently. \\
    \textbf{Applying} &
      Converts values and drops the unit column. \\
    \bottomrule
  \end{tabular}
\end{frame}

\begin{frame}{Range check}
  \centering\small
  \rowcolors{2}{black!10!white}{}
  \begin{tabular}{lp{8cm}}
    \toprule
    \multicolumn{2}{c}{\textbf{Range check}} \\
    \midrule
    \textbf{Goal} &
      Check whether values are within the expected range. \\
    \textbf{Fitting} &
      None. User declares the valid range $[a, b]$. \\
    \textbf{Adjustment} &
      Sample by sample; degenerated samples may be removed. \\
    \textbf{Applying} &
      If $x \notin [a, b]$: replace with $?$, clamp to $[a, b]$, or degenerate. \\
    \bottomrule
  \end{tabular}
\end{frame}

\begin{frame}{Category standardization}
  \centering\small
  \rowcolors{2}{black!10!white}{}
  \begin{tabular}{lp{8cm}}
    \toprule
    \multicolumn{2}{c}{\textbf{Category standardization}} \\
    \midrule
    \textbf{Goal} &
      Map different names to a single canonical form. \\
    \textbf{Fitting} &
      None. User declares the mapping. \\
    \textbf{Adjustment} &
      Sample by sample, independently. \\
    \textbf{Applying} &
      Case standardization, special character removal, dictionary/fuzzy matching. \\
    \bottomrule
  \end{tabular}
\end{frame}

% ---------------------------------------------------------------------------
\subsection{Outlier detection}
% ---------------------------------------------------------------------------

\begin{frame}{Outlier detection}
  \begin{itemize}
    \item Observations significantly different from the rest
    \item Caused by errors or mixed phenomena
    \item Standard approach: remove outliers from the dataset
    \item Per-variable: replace outlier values with missing data
  \end{itemize}

  \vspace{0.3cm}
  \textbf{IQR heuristic:} Given $Q_1$, $Q_3$, and $\text{IQR} = Q_3 - Q_1$,\\
  a value is an outlier if $x < Q_1 - 1.5\,\text{IQR}$ or
  $x > Q_3 + 1.5\,\text{IQR}$.
\end{frame}

\begin{frame}{Outlier detection --- IQR}
  \centering\small
  \rowcolors{2}{black!10!white}{}
  \begin{tabular}{lp{8cm}}
    \toprule
    \multicolumn{2}{c}{\textbf{Outlier detection using the IQR}} \\
    \midrule
    \textbf{Goal} &
      Detect outliers using the IQR. \\
    \textbf{Fitting} &
      Store $Q_1$ and $Q_3$ for each variable. \\
    \textbf{Adjustment} &
      Sample by sample, independently. \\
    \textbf{Applying} &
      Replaces outlier values with missing data. \\
    \bottomrule
  \end{tabular}

  \vspace{0.5cm}
  \raggedright\small
  More advanced: One-Class SVM\footfullcite{Scholkopf2001} for generalizable outlier classification.
\end{frame}

\begin{frame}{Outlier removal}
  \centering\small
  \rowcolors{2}{black!10!white}{}
  \begin{tabular}{lp{8cm}}
    \toprule
    \multicolumn{2}{c}{\textbf{Outlier removal}} \\
    \midrule
    \textbf{Goal} &
      Remove observations that are outliers. \\
    \textbf{Fitting} &
      Parameters of the outlier classifier. \\
    \textbf{Adjustment} &
      Sample by sample; degenerated samples removed. \\
    \textbf{Applying} &
      Degenerates if classified as outlier; pass-through otherwise. \\
    \bottomrule
  \end{tabular}

  \vspace{0.3cm}
  \raggedright\small
  Developer must specify default behavior when an outlier is detected in production.
\end{frame}

% ---------------------------------------------------------------------------
\subsection{Treating missing data}
% ---------------------------------------------------------------------------

\begin{frame}{Treating missing data}
  Most models cannot handle missing data. Four strategies:
  \begin{enumerate}
    \item Remove \textbf{rows} with missing data
    \item Remove \textbf{columns} with missing data
    \item \textbf{Impute} the missing values
    \item \textbf{Indicator variable} + imputation
  \end{enumerate}

  \vspace{0.3cm}
  Removing rows ``on demand'' can change the data distribution,
  especially if data is not missing at random.
\end{frame}

\begin{frame}{Row removal (missing data)}
  \centering\small
  \rowcolors{2}{black!10!white}{}
  \begin{tabular}{lp{8cm}}
    \toprule
    \multicolumn{2}{c}{\textbf{Row removal based on missing data}} \\
    \midrule
    \textbf{Goal} &
      Remove observations with missing data in specified variables. \\
    \textbf{Fitting} &
      None. Variables to check are declared beforehand. \\
    \textbf{Adjustment} &
      Sample by sample; degenerated samples removed. \\
    \textbf{Applying} &
      Degenerates over rows with missing data in specified variables. \\
    \bottomrule
  \end{tabular}
\end{frame}

\begin{frame}{Column removal (missing data)}
  \centering\small
  \rowcolors{2}{black!10!white}{}
  \begin{tabular}{lp{8cm}}
    \toprule
    \multicolumn{2}{c}{\textbf{Column removal based on missing data}} \\
    \midrule
    \textbf{Goal} &
      Remove variables with missing data. \\
    \textbf{Fitting} &
      Mark all variables with missing data in the training set. \\
    \textbf{Adjustment} &
      Marked columns are dropped. \\
    \textbf{Applying} &
      Drops the same columns chosen during fitting. \\
    \bottomrule
  \end{tabular}

  \vspace{0.3cm}
  \raggedright\small
  Valuable information may be lost when removing columns for all samples.
\end{frame}

\begin{frame}{Imputation}
  \centering\small
  \rowcolors{2}{black!10!white}{}
  \begin{tabular}{lp{8cm}}
    \toprule
    \multicolumn{2}{c}{\textbf{Imputation of missing data}} \\
    \midrule
    \textbf{Goal} &
      Replace missing data with a statistic (mean, median, mode). \\
    \textbf{Fitting} &
      Statistic computed from available training data. \\
    \textbf{Adjustment} &
      Sample by sample, independently. \\
    \textbf{Applying} &
      Replaces missing values; optionally creates an indicator variable. \\
    \bottomrule
  \end{tabular}

  \vspace{0.3cm}
  \raggedright\small
  Indicator variable: useful when missingness itself is informative\\
  (e.g., ``days since last pregnancy'' is missing if male or zero children).
\end{frame}

% ===========================================================================
\section{Data sampling}
% ===========================================================================

\begin{frame}{Data sampling}
  After cleaning, select or create variations of the training set:
  \begin{itemize}
    \item \textbf{Random sampling} --- reduce dataset size
    \item \textbf{Scope filtering} --- reduce the modeled phenomenon's scope
    \item \textbf{Class balancing} --- equalize class representation
  \end{itemize}
\end{frame}

% ---------------------------------------------------------------------------
\subsection{Random sampling}
% ---------------------------------------------------------------------------

\begin{frame}{Random sampling}
  \centering\small
  \rowcolors{2}{black!10!white}{}
  \begin{tabular}{lp{8cm}}
    \toprule
    \multicolumn{2}{c}{\textbf{Random sampling}} \\
    \midrule
    \textbf{Goal} &
      Select a random subset of the training data. \\
    \textbf{Fitting} &
      None. User declares the sample size. \\
    \textbf{Adjustment} &
      Rows randomly chosen. \\
    \textbf{Applying} &
      \textbf{Pass-through}: does nothing with new data. \\
    \bottomrule
  \end{tabular}
\end{frame}

% ---------------------------------------------------------------------------
\subsection{Scope filtering}
% ---------------------------------------------------------------------------

\begin{frame}{Scope filtering}
  \centering\small
  \rowcolors{2}{black!10!white}{}
  \begin{tabular}{lp{8cm}}
    \toprule
    \multicolumn{2}{c}{\textbf{Scope filtering}} \\
    \midrule
    \textbf{Goal} &
      Remove observations that do not satisfy a predefined rule. \\
    \textbf{Fitting} &
      None. User declares the rule. \\
    \textbf{Adjustment} &
      Sample by sample; degenerated samples removed. \\
    \textbf{Applying} &
      Degenerates over samples that violate the rule. \\
    \bottomrule
  \end{tabular}

  \vspace{0.3cm}
  \raggedright\small
  Variation: \textbf{model trees} --- shallow decision trees that branch into
  different models at each leaf.
\end{frame}

% ---------------------------------------------------------------------------
\subsection{Class balancing}
% ---------------------------------------------------------------------------

\begin{frame}{Class balancing}
  \centering\small
  \rowcolors{2}{black!10!white}{}
  \begin{tabular}{lp{8cm}}
    \toprule
    \multicolumn{2}{c}{\textbf{Class balancing}} \\
    \midrule
    \textbf{Goal} &
      Balance the number of observations in each class. \\
    \textbf{Fitting} &
      User declares or calculates target class sizes. \\
    \textbf{Adjustment} &
      Undersample (random removal) or oversample (resampling). \\
    \textbf{Applying} &
      \textbf{Pass-through}: does nothing with new data. \\
    \bottomrule
  \end{tabular}

  \vspace{0.3cm}
  \raggedright\small
  Advanced: SMOTE\footfullcite{chawla2002smote} creates synthetic minority samples without repetition.
\end{frame}

% ===========================================================================
\section{Data transformation}
% ===========================================================================

\begin{frame}{Data transformation}
  Data is now clean and well-sampled. Transform columns to suit the model:
  \begin{itemize}
    \item \textbf{Type conversion} --- categorical $\leftrightarrow$ numerical
    \item \textbf{Normalization} --- scale values to expected ranges
    \item \textbf{Dimensionality reduction} --- reduce number of variables
    \item \textbf{Data enhancement} --- add external information
  \end{itemize}
\end{frame}

% ---------------------------------------------------------------------------
\subsection{Type conversion}
% ---------------------------------------------------------------------------

\begin{frame}{Categorical to numerical}
  \textbf{Label encoding:}
  \begin{itemize}
    \item Replace $x \in \{a, b, c\}$ with $x' \in \{1, 2, 3\}$
    \item Suitable when there is a natural order $a < b < c$
  \end{itemize}

  \vspace{0.3cm}
  \textbf{One-hot encoding:}
  \begin{itemize}
    \item Create a new column for each category
    \item Column $= 1$ if present, $0$ otherwise
    \item Group rare categories into an \emph{other} column
  \end{itemize}
\end{frame}

\begin{frame}{One-hot encoding}
  \centering\small
  \rowcolors{2}{black!10!white}{}
  \begin{tabular}{lp{8cm}}
    \toprule
    \multicolumn{2}{c}{\textbf{One-hot encoding}} \\
    \midrule
    \textbf{Goal} &
      Create a new column for each category value. \\
    \textbf{Fitting} &
      Store the unique values; optionally mark an \emph{other} category. \\
    \textbf{Adjustment} &
      Sample by sample, independently. \\
    \textbf{Applying} &
      New columns filled with $1$ or $0$; unknown values assigned to \emph{other}. \\
    \bottomrule
  \end{tabular}
\end{frame}

\begin{frame}{Numerical to categorical (binning)}
  \centering\small
  \rowcolors{2}{black!10!white}{}
  \begin{tabular}{lp{8cm}}
    \toprule
    \multicolumn{2}{c}{\textbf{Binning numerical values}} \\
    \midrule
    \textbf{Goal} &
      Create a categorical column from a numerical one. \\
    \textbf{Fitting} &
      Store the range of each bin (by frequency or by range). \\
    \textbf{Adjustment} &
      Sample by sample, independently. \\
    \textbf{Applying} &
      Assigns each value to the corresponding bin. \\
    \bottomrule
  \end{tabular}

  \vspace{0.3cm}
  \raggedright\small
  Also common: converting dates/intervals to numerical differences
  (e.g., birth date $\to$ age).
\end{frame}

% ---------------------------------------------------------------------------
\subsection{Normalization}
% ---------------------------------------------------------------------------

\begin{frame}{Standardization}
  \[
    x' = \frac{x - \mu}{\sigma}
  \]

  \centering\small
  \rowcolors{2}{black!10!white}{}
  \begin{tabular}{lp{8cm}}
    \toprule
    \multicolumn{2}{c}{\textbf{Standardization}} \\
    \midrule
    \textbf{Goal} &
      Scale values in a column (zero mean, unit variance). \\
    \textbf{Fitting} &
      Store $\mu$ and $\sigma$ from the training set. \\
    \textbf{Adjustment} &
      Sample by sample, independently. \\
    \textbf{Applying} &
      Scales values using the fitted $\mu$ and $\sigma$. \\
    \bottomrule
  \end{tabular}
\end{frame}

\begin{frame}{Rescaling}
  \[
    x' = a + (b - a) \, \frac{x - x_\text{min}}{x_\text{max} - x_\text{min}}
  \]

  \centering\small
  \rowcolors{2}{black!10!white}{}
  \begin{tabular}{lp{8cm}}
    \toprule
    \multicolumn{2}{c}{\textbf{Rescaling}} \\
    \midrule
    \textbf{Goal} &
      Rescale values to a target range $[a, b]$. \\
    \textbf{Fitting} &
      Store $x_\text{min}$ and $x_\text{max}$ from the training set. \\
    \textbf{Adjustment} &
      Sample by sample, independently. \\
    \textbf{Applying} &
      Rescales and clamps: $\max(a, \min(b, x'))$. \\
    \bottomrule
  \end{tabular}
\end{frame}

% ---------------------------------------------------------------------------
\subsection{Dimensionality reduction}
% ---------------------------------------------------------------------------

\begin{frame}{Dimensionality reduction}
  \textbf{Feature selection:}
  \begin{itemize}
    \item Select a subset of existing variables
    \item Example: rank by mutual information with target, keep top $k$
  \end{itemize}

  \vspace{0.3cm}
  \textbf{Feature extraction:}
  \begin{itemize}
    \item Create new variables as combinations of original ones
    \item Linear: PCA
    \item Non-linear: autoencoders
    \item Drawback: new variables are hard to interpret
  \end{itemize}
\end{frame}

% ---------------------------------------------------------------------------
\subsection{Data enhancement}
% ---------------------------------------------------------------------------

\begin{frame}{Data enhancement}
  \centering\small
  \rowcolors{2}{black!10!white}{}
  \begin{tabular}{lp{8cm}}
    \toprule
    \multicolumn{2}{c}{\textbf{Data enhancement}} \\
    \midrule
    \textbf{Goal} &
      Enrich the dataset with external information. \\
    \textbf{Fitting} &
      Store the external dataset and the join column. \\
    \textbf{Adjustment} &
      Left join with external dataset (same number of rows). \\
    \textbf{Applying} &
      Enhances each new observation with external information. \\
    \bottomrule
  \end{tabular}

  \vspace{0.3cm}
  \raggedright\small
  Example: join zip codes with socioeconomic data.
\end{frame}

% ---------------------------------------------------------------------------
\subsection{Unstructured data}
% ---------------------------------------------------------------------------

\begin{frame}{Comments on unstructured data}
  \begin{itemize}
    \item Any unstructured data can be transformed into structured data
    \item Bag of words, word embeddings, signal/image processing
    \item Modern methods (CNNs) learn preprocessing and model jointly
      \begin{itemize}
        \item Convolutional layers $=$ learned feature extraction
      \end{itemize}
    \item Unstructured data is a vast field, out of scope of this book
  \end{itemize}
\end{frame}

% ---- Takeaways ----

\begin{frame}{Takeaways}
  \begin{itemize}
    \item Each learning method requires specific preprocessing tasks
    \item Fitting the preprocessor is crucial to avoid leakage
    \item Default behavior when the chain degenerates must be specified
    \item Three categories: cleaning, sampling, transformation
    \item Preprocessing parameters are fitted, not fixed
  \end{itemize}
\end{frame}

% ---- End ----

\begin{frame}[standout]
  Questions?
\end{frame}

\end{document}
