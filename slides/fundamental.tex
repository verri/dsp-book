% slides/fundamental.tex - Chapter 2: Fundamental Concepts

% slides/preamble.tex - Shared preamble for all slide decks
% Metropolis theme with grayscale style matching the book

\documentclass[aspectratio=169]{beamer}

\usetheme{metropolis}

% ---------- Colors (grayscale) ----------
\definecolor{bookdark}{gray}{0.2}
\definecolor{bookgray}{gray}{0.5}
\definecolor{booklight}{gray}{0.92}

\setbeamercolor{normal text}{fg=bookdark, bg=white}
\setbeamercolor{alerted text}{fg=bookdark}
\setbeamercolor{frametitle}{fg=white, bg=bookdark}
\setbeamercolor{title separator}{fg=bookgray}
\setbeamercolor{progress bar}{fg=bookgray, bg=booklight}
\setbeamercolor{block title}{fg=white, bg=bookgray}
\setbeamercolor{block body}{fg=bookdark, bg=booklight}
\setbeamercolor{block title alerted}{fg=white, bg=bookdark}
\setbeamercolor{block body alerted}{fg=bookdark, bg=booklight}
\setbeamercolor{block title example}{fg=white, bg=bookgray}
\setbeamercolor{block body example}{fg=bookdark, bg=booklight}

\setbeamertemplate{frame numbering}[fraction]

% ---------- Fonts (matching the book) ----------
\usepackage[T1]{fontenc}
\usepackage{fontspec}
\usepackage[warnings-off={mathtools-colon,mathtools-overbracket}]{unicode-math}

\setmathfont{STIXTwoMath}[
  Extension={.otf},
  Path={./fonts/},
  Scale=1]

\setsansfont{STIXTwoText}[
  Extension={.otf},
  Path={./fonts/},
  UprightFont={*-Regular},
  BoldFont={*-Bold},
  ItalicFont={*-Italic},
  BoldItalicFont={*-BoldItalic}]

\setmainfont{STIXTwoText}[
  Extension={.otf},
  Path={./fonts/},
  UprightFont={*-Regular},
  BoldFont={*-Bold},
  ItalicFont={*-Italic},
  BoldItalicFont={*-BoldItalic}]

\setmonofont{CourierPrime}[
  Extension={.ttf},
  Path={./fonts/},
  UprightFont={*-Regular},
  BoldFont={*-Bold},
  ItalicFont={*-Italic},
  BoldItalicFont={*-BoldItalic},
  Scale=0.9]

% ---------- Packages ----------
\usepackage{amsmath}
\usepackage{mathtools}
\usepackage{graphicx}
\usepackage{booktabs}

% ---------- TikZ ----------
\usepackage{tikz}
\usetikzlibrary{shapes, arrows.meta, positioning, shapes.geometric, fit}

\tikzset{%
  decision/.style={draw, diamond, text centered, minimum height=0.5cm, minimum width=1cm},
  block/.style={rectangle, draw, text width=6em, text centered, rounded corners, minimum height=3em},
  mediumblock/.style={rectangle, draw, text width=3em, text centered, rounded corners, minimum height=2em},
  darkblock/.style={block, fill=gray, text=white},
  smallblock/.style={rectangle, rounded corners, draw, font=\tiny, minimum height=1em, inner sep=2pt},
  smalldarkblock/.style={smallblock, fill=gray, text=white},
  darkcircle/.style={draw, circle, fill=gray, text centered, text=white},
  smallcircle/.style={draw, circle, text centered, font=\tiny},
  smalldarkcircle/.style={smallcircle, fill=gray, text=white},
  line/.style={draw, -latex},
  dline/.style={draw, latex-latex},
  bigarrow/.style={draw, -latex, line width=3pt, gray},
}

% ---------- Math operators ----------
\DeclareMathOperator*{\argmax}{arg\,max}
\DeclareMathOperator*{\argmin}{arg\,min}
\DeclareMathOperator{\Prob}{P}
\DeclareMathOperator{\E}{E}
\DeclareMathOperator{\Var}{Var}
\DeclareMathOperator{\sign}{sign}
\DeclareMathOperator{\clamp}{clamp}

% ---------- Hyperref ----------
\usepackage{hyperref}
\hypersetup{colorlinks, urlcolor=bookgray, linkcolor=bookdark}

% ---------- Commands ----------
\renewcommand{\vec}[1]{\mathbf{#1}}
\newcommand{\code}[1]{\colorbox{black!10!white}{\texttt{#1}}}

% ---------- Book info, license, and disclaimer ----------
\newcommand{\bookframe}{%
  \begin{frame}{About these slides}
    These slides are companion material for the book

    \vspace{0.3cm}
    \begin{center}
      \textbf{Data Science Project: An Inductive Learning Approach}\\[2pt]
      Prof.~Dr.~Filipe A. N. Verri\\[4pt]
      \url{https://leanpub.com/dsp}
    \end{center}

    \vspace{0.3cm}
    All intellectual content comes from the book and is not AI-generated.
    Slides were produced with the assistance of
    \href{https://claude.ai/code}{Claude Code}.

    \vspace{0.3cm}
    {\small
    Licensed under
    \href{https://creativecommons.org/licenses/by-nc/4.0/}{CC BY-NC 4.0}.
    You are free to modify and redistribute this work as long as you give
    proper credit and do not use it for commercial purposes.}
  \end{frame}%
}

% ---------- TikZ color presets (used by book figures) ----------
\colorlet{circle edge}{black!50}
\colorlet{circle area}{black!20}
\tikzset{
  filled/.style={fill=circle area, draw=circle edge, thick},
  outline/.style={draw=circle edge, thick},
}


\title{Fundamental Concepts}
\subtitle{Data Science Project: An Inductive Learning Approach}
\author{Prof.~Dr.~Filipe A. N. Verri}
\date{}

\begin{document}

\maketitle
\bookframe

% ---- Epigraph ----

\begin{frame}{}
  \vfill
  \begin{quote}
    The simple believes everything,\\
    but the prudent gives thought to his steps.
    \begin{flushright}
      --- Proverbs 14:15 (ESV)
    \end{flushright}
  \end{quote}
  \vfill
\end{frame}

% ---- Overview ----

\begin{frame}{Overview}
  \begin{columns}[T]
    \begin{column}{0.48\textwidth}
      \textbf{Contents}
      \begin{itemize}
        \item Data science definition
        \item The data science continuum
        \item Fundamental data theory
      \end{itemize}
    \end{column}
    \begin{column}{0.48\textwidth}
      \textbf{Objectives}
      \begin{itemize}
        \item Define data science
        \item Present the main concepts about data theory
      \end{itemize}
    \end{column}
  \end{columns}
\end{frame}

% ===========================================================================
\section{Data science definition}
% ===========================================================================

\begin{frame}{Definitions in the literature}
  \begin{itemize}
    \item \textbf{Zumel \& Mount:} cross-disciplinary practice drawing on data
      engineering, statistics, data mining, ML, and predictive analytics
    \item \textbf{Wickham \& Grolemund:} discipline that transforms raw data into
      understanding, insight, and knowledge
    \item \textbf{Hayashi:} not only unifies statistics and data analysis, but
      intends to analyze and understand actual phenomena with data
  \end{itemize}
\end{frame}

\begin{frame}{Our definition}
  \begin{block}{Definition: Data science}
    Data science is the study of knowledge extraction from measurable phenomena
    using computational methods.
  \end{block}

  \vspace{0.5cm}
  Key terms:
  \begin{itemize}
    \item \textbf{Computational methods} --- use computers to handle data
    \item \textbf{Knowledge} --- information humans can understand and apply
    \item \textbf{Measurable phenomena} --- events we can quantify
    \item \textbf{Raw data} --- collected directly, not yet transformed
  \end{itemize}
\end{frame}

\begin{frame}{Data science vs conventional sciences}
  \begin{itemize}
    \item Conventional: named after object of study (biology $\to$ life)
    \item Data science: studies \textit{how} to extract knowledge from data
    \item Similar to ``computer science'' --- not the study of computers
    \item Conventional paradigm: model-driven (observe, hypothesize, validate)
    \item Data science paradigm: data-driven (extract knowledge from data)
    \item We give data the opportunity to \textbf{surprise us}
  \end{itemize}
\end{frame}

% ---- Figure: My view of data science ----

\def\verrids{(0,0) circle (20mm)}
\def\verrist{(-2.5,0) circle (15mm)}
\def\verride{(2.5,0) circle (15mm)}
\def\verrics{(0,-2.5) circle (15mm)}

\begin{frame}{My view of data science}
  \centering
  \begin{tikzpicture}
    \begin{scope}
      \clip \verrids;
      \fill[filled] \verrist;
      \fill[filled] \verride;
      \fill[filled] \verrics;
    \end{scope}
    \draw[outline] \verrids node(ds) {};
    \draw[outline] \verrist node {Statistics};
    \draw[outline, text width=27mm, text centered] \verride node {Philosophy / domain expertise};
    \draw[outline] \verrics node {Computer science};
    \node[anchor=north,above] at (0, 1) {Data science};
  \end{tikzpicture}

  \vspace{0.3cm}
  \small An entirely new science --- its basis comes from other sciences,\\
  but its object of study is particular enough to raise new questions.
\end{frame}

% ===========================================================================
\section{The data science continuum}
% ===========================================================================

\begin{frame}{From borrowed methods to a new science}
  \begin{itemize}
    \item Object of study in data science is not new
    \item Key factor: social demand and importance of data
    \item ``Data is the new oil''
    \item Methods develop $\to$ experiments validate $\to$ credibility grows
    \item Academic recognition: new courses and programs
    \item Unique questions emerge, distinct from parent disciplines
  \end{itemize}
\end{frame}

% ---- Figure: The data science continuum ----

\begin{frame}{The data science continuum}
  \centering
  \begin{tikzpicture}[node distance=4mm and 5mm, scale=0.88, transform shape]
    % Left: Established Sciences
    \node (stats) [mediumblock, text width=1.8cm] {Statistics};
    \node (cs) [mediumblock, text width=1.8cm, below=of stats] {Computer science};
    \node (ds) [mediumblock, text width=1.8cm, below=of cs] {Philosophy and others};
    \node (basebox) [draw, dashed, inner sep=0.3cm, fit={(stats) (ds)},
      label=above:{\small Established sciences}] {};
    % Flow to the right
    \node (principles) [darkblock, right=of basebox, text width=2cm,
      minimum height=2em] {Emergence of principles};
    \node (methods) [block, right=of principles, text width=2cm,
      minimum height=2em] {Unique methods};
    \node (validation) [darkblock, right=of methods, text width=2cm,
      minimum height=2em] {Validation and new challenges};
    \node (science) [block, right=of validation, text width=2cm,
      minimum height=2em] {Data science};
    % Arrows
    \draw[-{Stealth}] (basebox) -- (principles);
    \draw[-{Stealth}] (principles) -- (methods);
    \draw[-{Stealth}] (methods) -- (validation);
    \draw[-{Stealth}] (validation) -- (science);
  \end{tikzpicture}
\end{frame}

\begin{frame}{New questions in data science}
  \begin{itemize}
    \item How can we guarantee that the data is reliable?
    \item How can we collect data without biasing conclusions?
    \item How can we guarantee that data usage is ethical?
    \item How can we present results to non-experts?
  \end{itemize}
\end{frame}

% ===========================================================================
\section{Fundamental data theory}
% ===========================================================================

% ---------------------------------------------------------------------------
\subsection{Phenomena}
% ---------------------------------------------------------------------------

\begin{frame}{Phenomena and philosophy}
  \begin{itemize}
    \item Phenomenon: any observable event or process
    \item \textbf{Ontology} --- study of being, existence, categories
    \item \textbf{Epistemology} --- study of knowledge and justification
    \item \textbf{Logic} --- study of reasoning and inference
  \end{itemize}

  \vspace{0.5cm}
  Understanding phenomena requires both general philosophical
  knowledge and domain expertise.
\end{frame}

\begin{frame}{Aristotle's categories}
  \begin{itemize}
    \item Aristotle (384--322 BC) --- first systematic classification
    \item Ten categories: substance, quantity, quality, relation, place, time, \ldots
    \item Practical view: focus on the world we can perceive
    \item Foundation of logical reasoning and scientific classification
    \item Still used in computer and data systems
  \end{itemize}
\end{frame}

\begin{frame}{Why ontology matters for data science}
  \begin{itemize}
    \item Describing = reducing complexity to simple pieces
    \item Phenomena $\approx$ substance; data $\approx$ properties, relations, states
    \item Identifying entities and their properties $\to$ better data collection
    \item Understanding logical limitations avoids errors
    \item Common mistake: assuming columns with the same name have the same meaning
  \end{itemize}
\end{frame}

% ---------------------------------------------------------------------------
\subsection{Measurements}
% ---------------------------------------------------------------------------

\begin{frame}{Measurements}
  \begin{itemize}
    \item Data science focuses on \textbf{measurable phenomena}
    \item Data collection: systematic gathering of data on a phenomenon
    \item Systematic $=$ planned, with understood consequences
    \item \textbf{Sampling bias} --- influence of the collection method on conclusions
    \item Data storage: digitally storing collected data
  \end{itemize}
\end{frame}

\begin{frame}{Data types}
  \begin{itemize}
    \item Data types must correctly reflect the source phenomenon
    \item Data types restrict the operations we can perform
    \item Foundations from computer science:
      \begin{itemize}
        \item Algorithms and data structures
        \item Databases
      \end{itemize}
    \item Concepts are independent of programming language or RDBMS
  \end{itemize}
\end{frame}

% ---------------------------------------------------------------------------
\subsection{Knowledge extraction}
% ---------------------------------------------------------------------------

\begin{frame}{Reasoning}
  \begin{itemize}
    \item \textbf{Deductive reasoning} --- from general rules to specific conclusions
    \item \textbf{Inductive reasoning} --- from specific observations to general rules
    \item Data science relies on \textbf{inductive reasoning}
    \item Descartes: algebra to mechanize reasoning
    \item Leibniz: universal algebraic language for logical methods
  \end{itemize}
\end{frame}

\begin{frame}{Knowledge extraction methods}
  \begin{itemize}
    \item \textbf{Statistics} --- collection, organization, analysis, interpretation
    \item \textbf{Machine learning} --- algorithms that learn from data automatically
    \item \textbf{Operations research} --- computational methods to optimize decisions
    \item Domain-specific methods: bioinformatics, geoinformatics, \ldots
    \item Each method has its own assumptions and limitations
  \end{itemize}
\end{frame}

\begin{frame}{Data preprocessing}
  \begin{itemize}
    \item When data do not match the method's requirements
    \item \textbf{Data cleaning} --- detecting and correcting corrupt data
    \item \textbf{Data transformation} --- converting formats or types
    \item \textbf{Data enhancement} --- integrating data from different sources
    \item Methods are usually robust to imperfections, but preprocessing helps
  \end{itemize}
\end{frame}

% ---- Takeaways ----

\begin{frame}{Takeaways}
  \begin{itemize}
    \item Data science is a new science that studies knowledge extraction
      from measurable phenomena using computational methods
    \item The data science continuum: from borrowed methods to a distinct discipline
    \item Understanding phenomena, measurements, and knowledge extraction
      is fundamental
  \end{itemize}
\end{frame}

% ---- End ----

\begin{frame}[standout]
  Questions?
\end{frame}

\end{document}
