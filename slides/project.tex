% slides/project.tex - Chapter 3: Data Science Project

% slides/preamble.tex - Shared preamble for all slide decks
% Metropolis theme with grayscale style matching the book

\documentclass[aspectratio=169]{beamer}

\usetheme{metropolis}

% ---------- Colors (grayscale) ----------
\definecolor{bookdark}{gray}{0.2}
\definecolor{bookgray}{gray}{0.5}
\definecolor{booklight}{gray}{0.92}

\setbeamercolor{normal text}{fg=bookdark, bg=white}
\setbeamercolor{alerted text}{fg=bookdark}
\setbeamercolor{frametitle}{fg=white, bg=bookdark}
\setbeamercolor{title separator}{fg=bookgray}
\setbeamercolor{progress bar}{fg=bookgray, bg=booklight}
\setbeamercolor{block title}{fg=white, bg=bookgray}
\setbeamercolor{block body}{fg=bookdark, bg=booklight}
\setbeamercolor{block title alerted}{fg=white, bg=bookdark}
\setbeamercolor{block body alerted}{fg=bookdark, bg=booklight}
\setbeamercolor{block title example}{fg=white, bg=bookgray}
\setbeamercolor{block body example}{fg=bookdark, bg=booklight}

\setbeamertemplate{frame numbering}[fraction]

% ---------- Fonts (matching the book) ----------
\usepackage[T1]{fontenc}
\usepackage{fontspec}
\usepackage[warnings-off={mathtools-colon,mathtools-overbracket}]{unicode-math}

\setmathfont{STIXTwoMath}[
  Extension={.otf},
  Path={./fonts/},
  Scale=1]

\setsansfont{STIXTwoText}[
  Extension={.otf},
  Path={./fonts/},
  UprightFont={*-Regular},
  BoldFont={*-Bold},
  ItalicFont={*-Italic},
  BoldItalicFont={*-BoldItalic}]

\setmainfont{STIXTwoText}[
  Extension={.otf},
  Path={./fonts/},
  UprightFont={*-Regular},
  BoldFont={*-Bold},
  ItalicFont={*-Italic},
  BoldItalicFont={*-BoldItalic}]

\setmonofont{CourierPrime}[
  Extension={.ttf},
  Path={./fonts/},
  UprightFont={*-Regular},
  BoldFont={*-Bold},
  ItalicFont={*-Italic},
  BoldItalicFont={*-BoldItalic},
  Scale=0.9]

% ---------- Packages ----------
\usepackage{amsmath}
\usepackage{mathtools}
\usepackage{graphicx}
\usepackage{booktabs}

% ---------- TikZ ----------
\usepackage{tikz}
\usetikzlibrary{shapes, arrows.meta, positioning, shapes.geometric, fit}

\tikzset{%
  decision/.style={draw, diamond, text centered, minimum height=0.5cm, minimum width=1cm},
  block/.style={rectangle, draw, text width=6em, text centered, rounded corners, minimum height=3em},
  mediumblock/.style={rectangle, draw, text width=3em, text centered, rounded corners, minimum height=2em},
  darkblock/.style={block, fill=gray, text=white},
  smallblock/.style={rectangle, rounded corners, draw, font=\tiny, minimum height=1em, inner sep=2pt},
  smalldarkblock/.style={smallblock, fill=gray, text=white},
  darkcircle/.style={draw, circle, fill=gray, text centered, text=white},
  smallcircle/.style={draw, circle, text centered, font=\tiny},
  smalldarkcircle/.style={smallcircle, fill=gray, text=white},
  line/.style={draw, -latex},
  dline/.style={draw, latex-latex},
  bigarrow/.style={draw, -latex, line width=3pt, gray},
}

% ---------- Math operators ----------
\DeclareMathOperator*{\argmax}{arg\,max}
\DeclareMathOperator*{\argmin}{arg\,min}
\DeclareMathOperator{\Prob}{P}
\DeclareMathOperator{\E}{E}
\DeclareMathOperator{\Var}{Var}
\DeclareMathOperator{\sign}{sign}
\DeclareMathOperator{\clamp}{clamp}

% ---------- Hyperref ----------
\usepackage{hyperref}
\hypersetup{colorlinks, urlcolor=bookgray, linkcolor=bookdark}

% ---------- Commands ----------
\renewcommand{\vec}[1]{\mathbf{#1}}
\newcommand{\code}[1]{\colorbox{black!10!white}{\texttt{#1}}}

% ---------- Book info, license, and disclaimer ----------
\newcommand{\bookframe}{%
  \begin{frame}{About these slides}
    These slides are companion material for the book

    \vspace{0.3cm}
    \begin{center}
      \textbf{Data Science Project: An Inductive Learning Approach}\\[2pt]
      Prof.~Dr.~Filipe A. N. Verri\\[4pt]
      \url{https://leanpub.com/dsp}
    \end{center}

    \vspace{0.3cm}
    All intellectual content comes from the book and is not AI-generated.
    Slides were produced with the assistance of
    \href{https://claude.ai/code}{Claude Code}.

    \vspace{0.3cm}
    {\small
    Licensed under
    \href{https://creativecommons.org/licenses/by-nc/4.0/}{CC BY-NC 4.0}.
    You are free to modify and redistribute this work as long as you give
    proper credit and do not use it for commercial purposes.}
  \end{frame}%
}

% ---------- TikZ color presets (used by book figures) ----------
\colorlet{circle edge}{black!50}
\colorlet{circle area}{black!20}
\tikzset{
  filled/.style={fill=circle area, draw=circle edge, thick},
  outline/.style={draw=circle edge, thick},
}


\title{Data Science Project}
\subtitle{Data Science Project: An Inductive Learning Approach}
\author{Prof.~Dr.~Filipe A. N. Verri\\{\small with contributions from Prof.~Dr.~Johnny C. Marques}}
\date{}

\begin{document}

\maketitle
\bookframe

% ---- Epigraph ----

\begin{frame}{}
  \vfill
  \begin{quote}
    Figured I could throw myself a pity party or go back to school and learn
    the computers.
    \begin{flushright}
      --- Don Carlton, \textit{Monsters University} (2013)
    \end{flushright}
  \end{quote}
  \vfill
\end{frame}

% ---- Overview ----

\begin{frame}{Overview}
  \begin{columns}[T]
    \begin{column}{0.48\textwidth}
      \textbf{Contents}
      \begin{itemize}
        \item What is a project?
        \item CRISP-DM
        \item ZM approach
        \item Agile methodology
        \item Scrum framework
        \item Our approach
      \end{itemize}
    \end{column}
    \begin{column}{0.48\textwidth}
      \textbf{Objectives}
      \begin{itemize}
        \item Explore common methodologies
        \item Understand agile and Scrum
        \item Propose a Scrum extension for data science
      \end{itemize}
    \end{column}
  \end{columns}
\end{frame}

\begin{frame}{A data science project is a software project}
  \begin{itemize}
    \item Some components are constructed \textbf{from data}
    \item Part of the solution is not designed by a domain expert
    \item Example: spam filter --- learned from labeled emails
    \item Traditional testing (unit tests) is not sufficient
    \item Stochastic nature of data requires proper validation
  \end{itemize}
\end{frame}

% ===========================================================================
\section{What is a project?}
% ===========================================================================

\begin{frame}{What is a project?}
  According to the PMBOK\footfullcite{PMI2025}:

  \vspace{0.3cm}
  \begin{block}{Definition}
    A project is \textbf{a temporary endeavor} aimed at creating a \textbf{unique result},
    such as a product or service.
  \end{block}

  \vspace{0.3cm}
  Key characteristics:
  \begin{itemize}
    \item Well-defined \textbf{beginning and end} (temporary)
    \item Clear, \textbf{measurable}, and achievable objectives
    \item Requires strong \textbf{collaboration} between teams
    \item Distinct from \textbf{operations} (DevOps, MLOps, DataOps)
  \end{itemize}
\end{frame}

% ===========================================================================
\section{CRISP-DM}
% ===========================================================================

\begin{frame}{CRISP-DM}
  Cross Industry Standard Process for Data Mining (IBM, 1990s)

  \vspace{0.3cm}
  Cyclic process with six phases:
  \begin{enumerate}
    \item Business understanding
    \item Data understanding
    \item Data preparation
    \item Modeling
    \item Evaluation
    \item Deployment
  \end{enumerate}
\end{frame}

% ---- Figure: CRISP-DM ----

\begin{frame}{Diagram of the CRISP-DM process}
  \centering
  \begin{tikzpicture}[scale=0.85, transform shape]
    \node (1) at (0, 0) {};
    \node (2) at (4, -4) {};
    \node (3) at (0, -8) {};
    \node (4) at (-4, -4) {};

    \node [block] (bu) at (-1.5, -2) {Business understanding};
    \node [block] (du) at (1.5, -2) {Data understanding};
    \path [line] (bu) -- (du);
    \path [line] (du) -- (bu);

    \node [block] (dp) at (2.5, -3.5) {Data preparation};
    \node [block] (m) at (2.5, -5) {Modeling};
    \path [line] (dp) -- (m);
    \path [line] (m) -- (dp);

    \path [line] (du) -- (dp);

    \node [block] (e) at (0, -6.5) {Evaluation};
    \node [block] (d) at (-2.5, -4.25) {Deployment};

    \path [line] (m) -- (e);
    \path [line] (e) -- (d);
    \path [line] (e) -- (bu);

    \node [draw, circle, fill=gray, text centered, text=white] at (0, -4.25) {Data};

    \path [bigarrow] (1.east) to[out=0, in=90] (2.60);
    \path [bigarrow] (2.-60) to[out=-90, in=0] (3.east);
    \path [bigarrow] (3.west) to[out=180, in=-90] (4.-120);
    \path [bigarrow] (4.120) to[out=90, in=180] (1.west);
  \end{tikzpicture}
\end{frame}

\begin{frame}{CRISP-DM limitations}
  \begin{itemize}
    \item Completely focused on data
    \item Does not address software development aspects
    \item Product = models and findings, not the full software solution
    \item User interface, communication, integration are not addressed
    \item Good starting point, but should not be followed strictly
  \end{itemize}
\end{frame}

% ===========================================================================
\section{ZM approach}
% ===========================================================================

\begin{frame}{ZM approach --- Roles}
  \begin{itemize}
    \item \textbf{Project sponsor} --- main stakeholder, business interests
    \item \textbf{Client} --- domain expert, represents end users
    \item \textbf{Data scientist} --- sets analytic strategy, connects all roles
    \item \textbf{Data architect} --- manages data and data storage
    \item \textbf{Operations} --- manages infrastructure, deploys results
  \end{itemize}

  \vspace{0.3cm}
  Data science projects are always \textbf{collaborative}.
\end{frame}

\begin{frame}{ZM approach --- Processes}
  \begin{itemize}
    \item Define the goal --- what problem are we solving?
    \item Collect and manage data --- what information do we need?
    \item Build the model --- what patterns may solve the problem?
    \item Evaluate the model --- is it good enough?
    \item Present results and document --- how did we solve it?
    \item Deploy the model --- how to use the solution?
  \end{itemize}

  \vspace{0.3cm}
  Back-and-forth is possible at any stage.
\end{frame}

% ---- Figure: ZM approach ----

\begin{frame}{Diagram of the ZM data science process}
  \centering
  \begin{tikzpicture}[scale=0.85, transform shape]
    \node (1) at (0, 0) {};
    \node (2) at (1, -1) {};
    \node (3) at (0, -2) {};
    \node (4) at (-1, -1) {};
    \path [bigarrow] (1.east) to[out=0, in=90] (2.60);
    \path [bigarrow] (2.-60) to[out=-90, in=0] (3.east);
    \path [bigarrow] (3.west) to[out=180, in=-90] (4.-120);
    \path [bigarrow] (4.120) to[out=90, in=180] (1.west);

    \node [block] (dg) at (0, 1) {Define the goal};
    \node [block] (cm) at (3, -0.25) {Collect and manage data};
    \node [block] (bm) at (3, -1.75) {Build the model};
    \node [block] (em) at (0, -3) {Evaluate the model};
    \node [block] (pd) at (-3, -1.75) {Present results};
    \node [block] (dm) at (-3, -0.25) {Deploy the model};

    \path [dline] (dg) to[out=0, in=90] (cm.north);
    \path [dline] (cm) -- (bm);
    \path [dline] (bm.south) to[out=-90, in=0] (em.east);
    \path [dline] (em.west) to[out=180, in=-90] (pd.south);
    \path [dline] (pd) -- (dm);
    \path [dline] (dm.north) to[in=180, out=90] (dg.west);
  \end{tikzpicture}
\end{frame}

\begin{frame}{ZM approach --- Limitations}
  \begin{itemize}
    \item Suited for consulting or strongly hierarchical organizations
    \item Maintenance and monitoring delegated to operations
    \item Does not address software development aspects
    \item Unclear boundary between project end and operations
    \item Assumes a pilot/demo --- delegates final software to another group
    \item Modern orgs expect data scientists to build \textbf{production-ready} software
  \end{itemize}
\end{frame}

% ===========================================================================
\section{Agile methodology}
% ===========================================================================

\begin{frame}{Agile methodology}
  Alternative to the waterfall (sequential) methodology.

  \vspace{0.3cm}
  \textbf{Four values of the Agile Manifesto:}
  \begin{itemize}
    \item Individuals and interactions over processes and tools
    \item Working software over comprehensive documentation
    \item Customer collaboration over contract negotiation
    \item Responding to change over following a plan
  \end{itemize}

  \vspace{0.3cm}
  Items on the right are not discarded, but items on the left are valued more.
\end{frame}

% ===========================================================================
\section{Scrum framework}
% ===========================================================================

\begin{frame}{Scrum framework}
  \begin{itemize}
    \item Iterative, incremental process
    \item Three pillars: \textbf{transparency}, \textbf{inspection}, \textbf{adaptation}
    \item Organized in cycles called \textbf{sprints}
    \item Defined roles, ceremonies, and artifacts
  \end{itemize}
\end{frame}

\begin{frame}{Scrum roles}
  \begin{itemize}
    \item \textbf{Product owner}
      \begin{itemize}
        \item Defines product vision, manages product backlog
        \item Balances business requirements and technical capabilities
      \end{itemize}
    \item \textbf{Scrum master}
      \begin{itemize}
        \item Facilitator and coach, not a project manager
        \item Removes impediments, fosters self-organization
      \end{itemize}
    \item \textbf{Development team}
      \begin{itemize}
        \item Cross-functional, self-managing
        \item Delivers shippable increments each sprint
      \end{itemize}
  \end{itemize}
\end{frame}

% ---- Figure: Scrum framework ----

\begin{frame}{Scrum framework overview}
  \centering
  \begin{tikzpicture}[%
      font=\LARGE,
      every label/.style={font=\footnotesize},
      scale=0.88, transform shape,
    ]
    \node[align=center, label=below:product owner] (po) at (0.9, 0) {\faUserTie};
    \node[align=center, label=below:product backlog] (backlog) at (0.9, -1.5) {\faListOl};

    \node[align=center, label=below:dev team] (dev) at (3.2, 0) {\faUsers};
    \node[align=center, label=below:sprint backlog] (sbacklog) at (3.2, -1.5) {\faClipboardList};

    \draw[-Stealth] (backlog) -- (sbacklog) node[midway, above, font=\footnotesize, yshift=2mm] {sprint planning};

    \draw[-Stealth, very thick] (sbacklog.east)
      to[out=0, in=180] (5.6, -1.5)
      to[out=0, in=0] (5.6, 0.5)
      to[out=0, in=0] (5, 0.5)
      to[out=180, in=130] (4.7, -1.4);

    \node[font=\footnotesize] at (5.2, -0.5) {sprint};

    \node[align=center, label=above:Scrum master] (sm) at (4.2, 0.8) {\faUser};

    \draw[-Stealth, very thick] (6, -1.5) -- (7, -1.5);

    \draw[-Stealth] (5.7, 0.6)
      to[out=120, in=200] (5.8, 1)
      to[out=20, in=160] (6.2, 1)
      to[out=-20, in=20] (6, 0.3);

    \node[draw, rectangle, dashed, align=center, label=below:daily scrum] (daily) at (7, 2) {\faUser \faUsers};
    \draw[-] (daily.south west) -- (6, 0.8);

    \node[draw, rectangle, dashed, align=center, label=above:sprint review] (review) at (7.8, 0) {\faUser~\faUsers};
    \node[draw, rectangle, dashed, align=center, label=above:incremental version] (version) at (7.8, -1.3) {\faBox~\faUserTie};
    \node[draw, rectangle, dashed, align=center, label=above:sprint retrospective] (retrospective) at (7.8, -2.6) {\faUserTie~\faUser~\faUsers};
  \end{tikzpicture}
\end{frame}

\begin{frame}{Sprints and backlog}
  \begin{itemize}
    \item \textbf{Sprint} --- time-boxed iteration (1--4 weeks)
    \item \textbf{Product backlog} --- prioritized list of all desired work
    \item \textbf{Sprint backlog} --- subset committed for the sprint
    \item \textbf{Daily scrum} --- brief daily progress meeting
    \item \textbf{Burn down chart} --- visual progress tracking
    \item \textbf{Sprint review} --- demonstrate work, gather feedback
    \item \textbf{Sprint retrospective} --- reflect and improve
  \end{itemize}
\end{frame}

\begin{frame}{Scrum for data science?}
  \begin{itemize}
    \item Some consider Scrum inadequate for data science
    \item Argument: Scrum assumes known requirements
    \item Exploratory phases are not well supported
    \item Counter: Scrum is a \textbf{framework}, designed to be adapted
    \item Need a compromise between autonomy and a detailed plan
  \end{itemize}
\end{frame}

% ===========================================================================
\section{Our approach}
% ===========================================================================

\begin{frame}{Limitations of existing methodologies}
  \begin{itemize}
    \item \textbf{CRISP-DM} --- only data mining stages, no UI or data collection
    \item \textbf{ZM approach} --- addresses presentation but delegates software dev
    \item \textbf{Scrum} --- good for software dev, lacks exploratory phases
    \item None fully addresses a data science \textit{software} project
  \end{itemize}
\end{frame}

\begin{frame}{Two additional values}
  Beyond the Agile Manifesto:

  \vspace{0.5cm}
  \begin{itemize}
    \item \textbf{Confidence and understanding} of the model over performance
    \item \textbf{Code version control} over interactive environments
  \end{itemize}

  \vspace{0.5cm}
  Items on the right are not discarded, but items on the left are more important.
\end{frame}

\begin{frame}{Roles}
  \centering
  \small
  \rowcolors{2}{black!10!white}{}
  \begin{tabular}{lll}
    \toprule
    \textbf{Our approach} & \textbf{Scrum} & \textbf{ZM approach} \\
    \midrule
    Business spokesman  & Stakeholders  & Sponsor and client \\
    Lead data scientist & Product owner & Data scientist \\
    Scrum master        & Scrum master  & -- \\
    Data science team   & Dev team      & Data architect \& operations \\
    \bottomrule
  \end{tabular}
\end{frame}

\begin{frame}{Principles (1/2)}
  \begin{itemize}
    \item \textbf{Modularize the solution} --- front-end, back-end, dataset,
      solution search
    \item \textbf{Version control everything} --- code, data, documentation
    \item \textbf{CI/CD} --- automated testing and deployment
    \item \textbf{Reports as deliverables} --- version controlled, reproducible
    \item \textbf{Setup quantitative goals} --- avoid forever improving the model
  \end{itemize}
\end{frame}

\begin{frame}{Principles (2/2)}
  \begin{itemize}
    \item \textbf{Measure exactly what you want} --- custom metrics, beware of
      common pitfalls
    \item \textbf{Report model stability} --- understanding $>$ performance
    \item \textbf{Mask DS terminology in UI} --- use domain-specific terms
    \item \textbf{Monitor in production} --- concept drift, retraining
    \item \textbf{Use appropriate infrastructure} --- start simple, scale as needed
  \end{itemize}
\end{frame}

\begin{frame}{Three types of sprints}
  \begin{itemize}
    \item \textbf{Data sprint} --- collection, integration, tidying, exploration
    \item \textbf{Solution sprint} --- preprocessing, machine learning, validation
    \item \textbf{Application sprint} --- UI, communication, monitoring
  \end{itemize}

  \vspace{0.3cm}
  Sprints are sequential; no mixed sprints.\\
  Back-and-forth between sprint types is possible.
\end{frame}

% ---- Figure: Sprint tasks and results (horizontal layout) ----

\begin{frame}{Tasks and results for each sprint type}
  \centering
  \begin{tikzpicture}[
      every node/.style={font=\footnotesize, inner sep=3pt},
      taskbox/.style={rectangle, rounded corners, minimum height=1.4em,
        minimum width=1.4cm, align=center},
      resultbox/.style={rectangle, draw, rounded corners, text width=2.8cm,
        align=center, minimum height=1.4em},
      darkresultbox/.style={resultbox, fill=gray, text=white},
      sprintlabel/.style={font=\scriptsize\bfseries, anchor=north west},
    ]
    % === Data sprint ===
    \node[sprintlabel] at (0, 1.8) {Data sprint};
    \node[taskbox] (collect) at (0, 0.8) {Collect};
    \node[taskbox] (integrate) at (0, 0) {Integrate};
    \node[taskbox] (tidy) at (0, -0.8) {Tidy};
    \node[taskbox] (explore) at (0, -1.6) {Explore};
    \draw[dotted] (-0.95, 1.3) rectangle (0.95, -2.1);
    \node[resultbox] (dreport) at (0, -2.9) {Exploratory analysis};
    \node[darkresultbox] (data) at (0, -3.7) {Data};
    \draw[dashed, rounded corners] (-1.7, 1.7) rectangle (1.7, -4.3);

    % === Solution sprint ===
    \node[sprintlabel] at (3.8, 1.8) {Solution sprint};
    \node[taskbox] (preprocess) at (4.3, 0.4) {Preprocess};
    \node[taskbox] (ml) at (4.3, -0.4) {ML};
    \node[taskbox] (validation) at (4.3, -1.2) {Validation};
    \draw[dotted] (3.35, 0.9) rectangle (5.25, -1.7);
    \node[resultbox] (vreport) at (4.3, -2.9) {Validation report};
    \node[darkresultbox] (solution) at (4.3, -3.7) {Preprocessor \& model};
    \draw[dashed, rounded corners] (2.6, 1.7) rectangle (6, -4.3);

    % === Application sprint ===
    \node[sprintlabel] at (7.7, 1.8) {Application sprint};
    \node[taskbox] (ui) at (8.5, 0.4) {UI};
    \node[taskbox] (comm) at (8.5, -0.4) {Communication};
    \node[taskbox] (monitor) at (8.5, -1.2) {Monitoring};
    \draw[dotted] (7.35, 0.9) rectangle (9.65, -1.7);
    \node[darkresultbox] (app) at (8.5, -3.3) {Application};
    \draw[dashed, rounded corners] (6.7, 1.7) rectangle (10.3, -4.3);

    % === Legend ===
    \node[font=\scriptsize, anchor=west, text width=10cm] at (-1.7, -5)
      {\textbf{Dark nodes} are products fed into the next sprint.
       \textbf{Light nodes} are presented in the sprint review.};
  \end{tikzpicture}
\end{frame}

% ---- Figure: Sprint sequence ----

\begin{frame}{Example of sprints of a data science project}
  \centering
  \begin{tikzpicture}
    \node [anchor=west] at (0, 0) (s1) {data};
    \path [line] (s1.west) arc (-160:160:1);
    \path [line] (1.5, -0.5) -- (2.5, -0.5);

    \node [anchor=west] at (2.5, 0) (s2) {data};
    \path [line] (s2.west) arc (-160:160:1);
    \path [line] (4, -0.5) -- (5, -0.5);

    \node [anchor=west] at (5, 0) (s3) {solution};
    \path [line] (s3.west) arc (-160:160:1);
    \path [line] (6.5, -0.5) -- (7.5, -0.5) node [anchor=west] {\dots};

    \path [line] (0.5, -3.5) node [anchor=east] {\dots} -- (1, -3.5);
    \node [anchor=west] at (1, -3) (s5) {data};
    \path [line] (s5.west) arc (-160:160:1);
    \path [line] (2.5, -3.5) -- (3.5, -3.5);

    \node [anchor=west] at (3.5, -3) (s6) {solution};
    \path [line] (s6.west) arc (-160:160:1);
    \path [line] (5, -3.5) -- (6, -3.5);

    \node [anchor=west] at (6, -3) (s7) {application};
    \path [line] (s7.west) arc (-160:160:1);
    \path [line] (7.5, -3.5) -- (8.5, -3.5) node [anchor=west] {\dots};
  \end{tikzpicture}
\end{frame}

\begin{frame}{Relationship with other methodologies}
  \centering
  \small
  \rowcolors{2}{black!10!white}{}
  \begin{tabular}{lll}
    \toprule
    \textbf{CRISP-DM} & \textbf{ZM approach} & \textbf{Our approach} \\
    \midrule
    Bus.\ understanding & Define the goal & Product backlog \\
    Data understanding & Collect/manage data & Data sprint \\
    Data preparation & Collect/manage data & Data/solution sprint \\
    Modeling & Build the model & Solution sprint \\
    Evaluation & Evaluate the model & Solution sprint \\
    & Present results & Sprint reviews \\
    Deployment & Deploy the model & Application sprint \\
    \bottomrule
  \end{tabular}
\end{frame}

% ---- Takeaways ----

\begin{frame}{Takeaways}
  \begin{itemize}
    \item A data science project is a software project
    \item Modern methodologies should address software development
    \item Scrum can be adapted for data science with three sprint types:
      data, solution, and application
    \item Confidence and version control are key values
    \item The end result must be a \textbf{complete, production-ready} software product
  \end{itemize}
\end{frame}

% ---- End ----

\begin{frame}[standout]
  Questions?
\end{frame}

\end{document}
