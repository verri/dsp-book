% slides/learning.tex - Chapter 6: Learning from data

% slides/preamble.tex - Shared preamble for all slide decks
% Metropolis theme with grayscale style matching the book

\documentclass[aspectratio=169]{beamer}

\usetheme{metropolis}

% ---------- Colors (grayscale) ----------
\definecolor{bookdark}{gray}{0.2}
\definecolor{bookgray}{gray}{0.5}
\definecolor{booklight}{gray}{0.92}

\setbeamercolor{normal text}{fg=bookdark, bg=white}
\setbeamercolor{alerted text}{fg=bookdark}
\setbeamercolor{frametitle}{fg=white, bg=bookdark}
\setbeamercolor{title separator}{fg=bookgray}
\setbeamercolor{progress bar}{fg=bookgray, bg=booklight}
\setbeamercolor{block title}{fg=white, bg=bookgray}
\setbeamercolor{block body}{fg=bookdark, bg=booklight}
\setbeamercolor{block title alerted}{fg=white, bg=bookdark}
\setbeamercolor{block body alerted}{fg=bookdark, bg=booklight}
\setbeamercolor{block title example}{fg=white, bg=bookgray}
\setbeamercolor{block body example}{fg=bookdark, bg=booklight}

\setbeamertemplate{frame numbering}[fraction]

% ---------- Fonts (matching the book) ----------
\usepackage[T1]{fontenc}
\usepackage{fontspec}
\usepackage[warnings-off={mathtools-colon,mathtools-overbracket}]{unicode-math}

\setmathfont{STIXTwoMath}[
  Extension={.otf},
  Path={./fonts/},
  Scale=1]

\setsansfont{STIXTwoText}[
  Extension={.otf},
  Path={./fonts/},
  UprightFont={*-Regular},
  BoldFont={*-Bold},
  ItalicFont={*-Italic},
  BoldItalicFont={*-BoldItalic}]

\setmainfont{STIXTwoText}[
  Extension={.otf},
  Path={./fonts/},
  UprightFont={*-Regular},
  BoldFont={*-Bold},
  ItalicFont={*-Italic},
  BoldItalicFont={*-BoldItalic}]

\setmonofont{CourierPrime}[
  Extension={.ttf},
  Path={./fonts/},
  UprightFont={*-Regular},
  BoldFont={*-Bold},
  ItalicFont={*-Italic},
  BoldItalicFont={*-BoldItalic},
  Scale=0.9]

% ---------- Packages ----------
\usepackage{amsmath}
\usepackage{mathtools}
\usepackage{graphicx}
\usepackage{booktabs}

% ---------- TikZ ----------
\usepackage{tikz}
\usetikzlibrary{shapes, arrows.meta, positioning, shapes.geometric, fit}

\tikzset{%
  decision/.style={draw, diamond, text centered, minimum height=0.5cm, minimum width=1cm},
  block/.style={rectangle, draw, text width=6em, text centered, rounded corners, minimum height=3em},
  mediumblock/.style={rectangle, draw, text width=3em, text centered, rounded corners, minimum height=2em},
  darkblock/.style={block, fill=gray, text=white},
  smallblock/.style={rectangle, rounded corners, draw, font=\tiny, minimum height=1em, inner sep=2pt},
  smalldarkblock/.style={smallblock, fill=gray, text=white},
  darkcircle/.style={draw, circle, fill=gray, text centered, text=white},
  smallcircle/.style={draw, circle, text centered, font=\tiny},
  smalldarkcircle/.style={smallcircle, fill=gray, text=white},
  line/.style={draw, -latex},
  dline/.style={draw, latex-latex},
  bigarrow/.style={draw, -latex, line width=3pt, gray},
}

% ---------- Math operators ----------
\DeclareMathOperator*{\argmax}{arg\,max}
\DeclareMathOperator*{\argmin}{arg\,min}
\DeclareMathOperator{\Prob}{P}
\DeclareMathOperator{\E}{E}
\DeclareMathOperator{\Var}{Var}
\DeclareMathOperator{\sign}{sign}
\DeclareMathOperator{\clamp}{clamp}

% ---------- Hyperref ----------
\usepackage{hyperref}
\hypersetup{colorlinks, urlcolor=bookgray, linkcolor=bookdark}

% ---------- Commands ----------
\renewcommand{\vec}[1]{\mathbf{#1}}
\newcommand{\code}[1]{\colorbox{black!10!white}{\texttt{#1}}}

% ---------- Book info, license, and disclaimer ----------
\newcommand{\bookframe}{%
  \begin{frame}{About these slides}
    These slides are companion material for the book

    \vspace{0.3cm}
    \begin{center}
      \textbf{Data Science Project: An Inductive Learning Approach}\\[2pt]
      Prof.~Dr.~Filipe A. N. Verri\\[4pt]
      \url{https://leanpub.com/dsp}
    \end{center}

    \vspace{0.3cm}
    All intellectual content comes from the book and is not AI-generated.
    Slides were produced with the assistance of
    \href{https://claude.ai/code}{Claude Code}.

    \vspace{0.3cm}
    {\small
    Licensed under
    \href{https://creativecommons.org/licenses/by-nc/4.0/}{CC BY-NC 4.0}.
    You are free to modify and redistribute this work as long as you give
    proper credit and do not use it for commercial purposes.}
  \end{frame}%
}

% ---------- TikZ color presets (used by book figures) ----------
\colorlet{circle edge}{black!50}
\colorlet{circle area}{black!20}
\tikzset{
  filled/.style={fill=circle area, draw=circle edge, thick},
  outline/.style={draw=circle edge, thick},
}


\title{Learning from Data}
\subtitle{Data Science Project: An Inductive Learning Approach}
\author{Prof.~Dr.~Filipe A. N. Verri}
\date{}

\begin{document}

\maketitle
\bookframe

% ---- Epigraph ----

\begin{frame}{}
  \vfill
  \begin{quote}
    To understand God's thoughts we must study statistics, for these are
    the measures of His purpose.
    \begin{flushright}
      --- Florence Nightingale, her diary
    \end{flushright}
  \end{quote}
  \vfill
\end{frame}

% ---- Overview ----

\begin{frame}{Overview}
  \begin{columns}[T]
    \begin{column}{0.48\textwidth}
      \textbf{Contents}
      \begin{itemize}
        \item Introduction
        \item The learning problem
        \item Optimal solutions
        \item ERM inductive principle
        \item SRM inductive principle
        \item Linear problems
      \end{itemize}
    \end{column}
    \begin{column}{0.48\textwidth}
      \textbf{Objectives}
      \begin{itemize}
        \item Define the learning problem and the common predictive tasks
        \item Understand the main principles that guide the learning process
      \end{itemize}
    \end{column}
  \end{columns}
\end{frame}

% ===========================================================================
\section{Introduction}
% ===========================================================================

\begin{frame}{From AI to inductive learning}
  \begin{itemize}
    \item \textbf{Artificial intelligence} --- algorithms that exhibit intelligent behavior
    \item \textbf{Machine learning} --- algorithms that learn from experience/data
    \item \textbf{Predictive learning} --- making predictions about outcomes
    \item \textbf{Inductive learning} --- deriving general rules from specific observations
  \end{itemize}

  \vspace{0.3cm}
  The inferred general rule can make predictions about \emph{any} new instance.
\end{frame}

% ---- Figure: Learning field hierarchy ----

\begin{frame}{Learning field hierarchy}
  \centering
  \begin{tikzpicture}
    \draw[outline] (0,0) circle (30mm);
    \node[below] at (0, 2.6) {artificial intelligence};
    \draw[outline] (0,-0.5) circle (25mm);
    \node[below] at (0, 1.6) {machine learning};
    \draw[outline] (0,-1) circle (20mm);
    \node[below] at (0, 0.5) {predictive learning};
    \draw[outline] (0,-1.5) circle (15mm);
    \node[below] at (0, -1.0) {inductive learning};
  \end{tikzpicture}
\end{frame}

\begin{frame}{Statistical Learning Theory}
  \begin{itemize}
    \item General framework for predictive learning
    \item Formalizes the learning problem
    \item Provides bounds on generalization ability
    \item Guides the design of learning machines
  \end{itemize}
\end{frame}

% ===========================================================================
\section{The learning problem}
% ===========================================================================

\begin{frame}{Setup}
  Training set:
  \[
    \big\{(\vec{x}_i, y_i) : i = 1, \dots, n \big\}
  \]

  \begin{itemize}
    \item $\vec{x}_i \in \mathcal{X}$ --- feature vector
    \item $y_i \in \mathcal{Y}$ --- target variable
    \item Samples are i.i.d.\ drawn from $\Prob(x, y) = \Prob(y \mid x)\, \Prob(x)$
    \item Both $\Prob(x)$ and $\Prob(y \mid x)$ are \textbf{fixed but unknown}
  \end{itemize}
\end{frame}

\begin{frame}{Learning machine and risk}
  A \textbf{learning machine} generates models $f_\theta : \mathcal{X} \to \mathcal{Y}$
  for $\theta \in \Theta$.

  \vspace{0.3cm}
  Given a \textbf{loss} $\mathcal{L}(y, f_\theta(x))$, the \textbf{risk} is
  \[
    R(\theta) = \int \mathcal{L}(y, f_\theta(x))\, d\!\Prob(x, y)
  \]

  \vspace{0.3cm}
  Goal: find $f_\theta$ that minimizes $R(\theta)$ using only the training set.
\end{frame}

% ---------------------------------------------------------------------------
\subsection{Learning tasks}
% ---------------------------------------------------------------------------

\begin{frame}{Binary data classification}
  \begin{itemize}
    \item Output $y \in \{0, 1\}$ (negative and positive class)
    \item Loss: indicator function
      \[
        \mathcal{L}(y, f_\theta(x)) = \begin{cases}
          0 & \text{if } y = f_\theta(x) \\
          1 & \text{if } y \neq f_\theta(x)
        \end{cases}
      \]
    \item Risk $=$ probability of classification error
    \item $f_\theta$ is called a \textbf{classifier}
  \end{itemize}
\end{frame}

\begin{frame}{Regression estimation}
  \begin{itemize}
    \item Output $y \in \mathbb{R}$
    \item Loss: squared error
      \[
        \mathcal{L}(y, f_\theta(x)) = \big(y - f_\theta(x)\big)^2
      \]
    \item $f_\theta$ is called a \textbf{regressor}
  \end{itemize}
\end{frame}

% ---------------------------------------------------------------------------
\subsection{Remarks}
% ---------------------------------------------------------------------------

\begin{frame}{A few remarks}
  \textbf{Supervised vs.\ semisupervised:}
  \begin{itemize}
    \item Supervised: target known for all training samples
    \item Semisupervised: only a small subset labeled
  \end{itemize}

  \vspace{0.3cm}
  \textbf{Generative vs.\ discriminative:}
  \begin{itemize}
    \item Generative: models joint $\Prob(x, y)$ (more complex, more information)
    \item Discriminative: models $\Prob(y \mid x)$ directly (prefer if only predicting)
  \end{itemize}
\end{frame}

\begin{frame}{More remarks}
  \textbf{Multiclass classification:}
  \begin{itemize}
    \item $y$ takes more than two values
    \item One-versus-all: $l$ binary classifiers
    \item One-versus-one: $l(l-1)/2$ binary classifiers
  \end{itemize}

  \vspace{0.3cm}
  \textbf{Number of inputs and outputs:}
  \begin{itemize}
    \item Input/output can be scalar, vector, matrix, or tensor
    \item The learning machine must handle the structure of the data
  \end{itemize}
\end{frame}

% ===========================================================================
\section{Optimal solutions}
% ===========================================================================

\begin{frame}{Why study optimal solutions?}
  \begin{itemize}
    \item Optimal solutions assume $\Prob(y \mid x)$ is known
    \item Unrealistic but useful to understand how good a solution can be
    \item They depend only on $\Prob(y \mid x)$ (discriminative)
    \item Establish the \textbf{irreducible error} for each task
  \end{itemize}
\end{frame}

% ---------------------------------------------------------------------------
\subsection{Bayes classifier}
% ---------------------------------------------------------------------------

\begin{frame}{Bayes classifier}
  Optimal solution for binary classification:
  \[
    f_\text{Bayes}(x) = \argmax_{y \in \mathcal{Y}} \Prob(y \mid x)
  \]

  \vspace{0.3cm}
  \textbf{Bayes error rate} (irreducible error):
  \[
    R_\text{Bayes} = \int \min\!\big\{ b(x),\; 1 - b(x) \big\}\, d\!\Prob(x)
  \]
  where $b(x) = \Prob(y = 1 \mid x)$.
\end{frame}

% ---- Figure: Bayes classifier ----

\begin{frame}{Bayes classifier --- illustration}
  \centering
  \begin{tikzpicture}
    \begin{axis}[
        ticks=none,
        axis x line=bottom,
        axis y line=left,
        xlabel={$x$},
        ymax=0.42,
        xmin=-2.2, xmax=1.4,
        width=0.75\textwidth,
        height=0.6\textheight,
      ]
      \addplot+[fill=gray, draw=black, opacity=0.2, smooth, mark=none] coordinates {
        (-2, 0.1) (-1.5, 0.2) (-1, 0.35) (-0.5, 0.2) (0, 0.1)
      };
      \node at (axis cs:-1, 0.2) {$\Prob(x \mid y = 0)$};
      \addplot+[fill=gray, draw=black, opacity=0.4, smooth, mark=none] coordinates {
        (-0.8, 0.1) (-0.3, 0.2) (0.2, 0.35) (0.7, 0.2) (1.2, 0.1)
      };
      \node at (axis cs:0.2, 0.2) {$\Prob(x \mid y = 1)$};
      \draw[dashed, gray] (axis cs:-0.4, 0) -- (axis cs:-0.4, 0.4);
    \end{axis}
  \end{tikzpicture}

  \small
  The dashed line is the Bayes decision boundary.\\
  The darker intersection area causes the irreducible error.
\end{frame}

% ---------------------------------------------------------------------------
\subsection{Regression function}
% ---------------------------------------------------------------------------

\begin{frame}{Regression function}
  Optimal solution for regression estimation:
  \[
    r(x) = \int y\, d\!\Prob(y \mid x) = \E[y \mid x]
  \]

  \vspace{0.3cm}
  Irreducible error (law of total variance):
  \[
    R(r) = \E\!\big[ \Var(y \mid x) \big]
    \quad \text{(unexplained variance)}
  \]
\end{frame}

% ---- Figure: Explained vs unexplained variance ----

\begin{frame}{Explained vs.\ unexplained variance}
  \centering
  \begin{tikzpicture}
    \begin{axis}[
        axis lines=middle,
        xlabel={$x$},
        ylabel={$y$},
        xmin=-0.2, xmax=1.5,
        ymin=-0.5, ymax=1.8,
        xtick={0, 0.5, 1},
        ytick={0, 0.5, 1},
        domain=0:1,
        width=0.6\textwidth,
        height=0.55\textheight,
      ]

      \addplot[thick] {x} node[right] {$r(x) = x$};

      \addplot [draw=none, fill=gray, opacity=0.2] {x + 1} \closedcycle;
      \addplot [draw=none, fill=gray, opacity=0.2] {x - 1} \closedcycle;
      \draw[Stealth-Stealth, gray] (axis cs:0.5,0.5) -- (axis cs:0.5, 1.5)
        node[midway, right] {$\sigma = 1$};
      \node at (axis cs:0.5,1.3) [anchor=west] {\small Unexplained};

      \addplot[samples=2, style={dashed}] {0.5}
        node[midway, below, anchor=north west] {\small Explained};
    \end{axis}
  \end{tikzpicture}

  \small $\Prob(y \mid x) = \mathcal{N}(x, 1)$, \quad $\Prob(x) = \mathcal{U}(0, 1)$
\end{frame}

% ===========================================================================
\section{ERM inductive principle}
% ===========================================================================

\begin{frame}{Empirical Risk Minimization}
  Since $\Prob(z)$ is unknown, replace the risk by the \textbf{empirical risk}:
  \[
    R_n(\theta) = \frac{1}{n} \sum_{i=1}^n L(z_i, \theta)
  \]

  \vspace{0.3cm}
  \begin{itemize}
    \item $z_i = (x_i, y_i)$ --- training samples
    \item Traditional methods (least squares, maximum likelihood) are realizations of ERM
    \item Question: does $R_n(\theta) \to R(\theta)$?
  \end{itemize}
\end{frame}

% ---------------------------------------------------------------------------
\subsection{Consistency}
% ---------------------------------------------------------------------------

\begin{frame}{Consistency of the learning process}
  ERM is consistent if the empirical risk converges \textbf{uniformly}:
  \[
    \lim_{n \to \infty} \Prob\!\left(
      \sup_{\theta \in \Theta} \Big| R_n(\theta) - R(\theta) \Big| > \epsilon
    \right) = 0
  \]

  \vspace{0.3cm}
  \textbf{Fast rate of convergence} if, for $n > n_0$:
  \[
    \Prob\!\big(R(\theta_n) - R(\theta) > \epsilon\big) < \exp\!\left( - c\, n\, \epsilon^2 \right)
  \]
\end{frame}

% ---------------------------------------------------------------------------
\subsection{VC entropy}
% ---------------------------------------------------------------------------

\begin{frame}{VC entropy}
  For bounded loss functions $|L(z, \theta)| < M$:
  \begin{itemize}
    \item $N(z_1, \ldots, z_n; \Theta, \epsilon)$ --- size of minimal $\epsilon$-net
    \item \textbf{VC entropy:}
      \[
        H(n; \Theta, \epsilon) = \E\!\left[ \ln N(z_1, \ldots, z_n; \Theta, \epsilon) \right]
      \]
  \end{itemize}

  \vspace{0.3cm}
  Necessary and sufficient condition for uniform convergence:
  \[
    \lim_{n \to \infty} \frac{H(n; \Theta, \epsilon)}{n} = 0
  \]
\end{frame}

% ---------------------------------------------------------------------------
\subsection{VC dimension}
% ---------------------------------------------------------------------------

\begin{frame}{Growth function and VC dimension}
  \textbf{Growth function} (distribution-independent):
  \[
    G(n; \Theta) = \ln \sup_{z_1, \ldots, z_n} N(z_1, \ldots, z_n; \Theta)
  \]

  \vspace{0.3cm}
  Vapnik \& Chervonenkis (1968): $G(n; \Theta)$ is either
  \begin{itemize}
    \item $n \ln 2$ (VC dimension is \textbf{infinite}), or
    \item bounded by $h\!\left(\ln \frac{n}{h} + 1\right)$, where $h$ is the
      \textbf{VC dimension}
  \end{itemize}

  \vspace{0.3cm}
  Finite VC dimension $\Rightarrow$ consistency + fast convergence.
\end{frame}

\begin{frame}{Shattering}
  \begin{itemize}
    \item VC dimension $=$ max number of points that can be \textbf{shattered}
    \item $h$ points are shattered if they can be separated into two classes in
      all $2^h$ possible ways
    \item VC dimension measures \textbf{complexity of the hypothesis space},
      not the number of parameters
  \end{itemize}
\end{frame}

% ---- Figure: VC dimension of lines in the plane ----

\begin{frame}{VC dimension --- lines in the plane}
  \centering
  \begin{tikzpicture}
    \begin{axis}[
        axis lines=middle,
        xmin=-1, xmax=1,
        ymin=-1, ymax=1,
        xtick={0},
        ytick={0},
        domain=-1:1,
        width=0.55\textwidth,
        height=0.6\textheight,
      ]

      \addplot[only marks, mark=*, mark size=2pt] coordinates {
        (-0.5, 0.3) (-0.3, -0.5) (0.7, -0.5)
      };

      \addplot[dashed] {x} node[right] {$\theta_1$};
      \addplot[dashed] {-x} node[right] {$\theta_2$};
      \addplot[dashed] {-0.3} node[right] {$\theta_3$};
    \end{axis}
  \end{tikzpicture}

  \small
  VC dimension of lines in the plane $= 3$.\\
  A line can shatter 3 points in all $2^3 = 8$ ways, but not 4 points.
\end{frame}

% ---- Figure: VC dimension of sine waves ----

\begin{frame}{Infinite VC dimension --- sine functions}
  \centering
  \begin{tikzpicture}
    \begin{axis}[
        axis lines=middle,
        xlabel={$x$},
        ylabel={$y$},
        xmin=-1.2, xmax=1.2,
        ymin=-1.2, ymax=1.2,
        xtick={0},
        ytick={0},
        domain=-1:1,
        samples=100,
        width=0.55\textwidth,
        height=0.6\textheight,
      ]

      \addplot[smooth] {sin(deg(24 * x))};
      \addplot[dashed, smooth] {sin(deg(4 * x))};

      \addplot[only marks, mark=*, mark size=2pt] coordinates {
        (-0.942, 0.586) (-0.562, -0.780) (-0.337, -0.976)
        (0.313, 0.950) (0.562, 0.780) (0.942, -0.586)
      };
    \end{axis}
  \end{tikzpicture}

  \small
  $f(z; \theta) = \mathbb{1}_{\sin \theta x > 0}$ has infinite VC dimension\\
  with only \textbf{one} parameter $\theta$.
\end{frame}

% ===========================================================================
\section{SRM inductive principle}
% ===========================================================================

\begin{frame}{Generalization bound}
  \[
    R(\theta_n) \leq R_n(\theta_n) + \frac{B \mathcal{E}}{2} \left(
      1 + \sqrt{1 + \frac{4 R_n(\theta_n)}{B \mathcal{E}}}
    \right)
  \]
  with
  \[
    \mathcal{E} = 4 \frac{
      h \left( \ln \frac{2 n}{h} + 1 \right) - \ln \frac{\eta}{4}
    }{n}
  \]

  \vspace{0.3cm}
  \begin{itemize}
    \item $B$ --- upper bound of the loss, $h$ --- VC dimension
    \item Small $\nicefrac{n}{h}$ $\Rightarrow$ small empirical risk
      does \textbf{not} guarantee small true risk
  \end{itemize}
\end{frame}

% ---------------------------------------------------------------------------
\subsection{Overfitting and underfitting}
% ---------------------------------------------------------------------------

\begin{frame}{Overfitting and underfitting}
  \centering
  \rowcolors{2}{black!10!white}{}
  \begin{tabular}{lll}
    \toprule
    \textbf{Problem} & \textbf{Empirical risk} & \textbf{Confidence interval} \\
    \midrule
    Underfitting & High & Low \\
    Overfitting & Low & High \\
    \bottomrule
  \end{tabular}

  \vspace{0.5cm}
  \raggedright
  \begin{itemize}
    \item \textbf{Underfitting}: model too simple (low VC dim.)
    \item \textbf{Overfitting}: model too complex (high VC dim.)
    \item Must balance both terms to generalize well
  \end{itemize}
\end{frame}

% ---------------------------------------------------------------------------
\subsection{SRM principle}
% ---------------------------------------------------------------------------

\begin{frame}{Structural Risk Minimization}
  \begin{itemize}
    \item Build an \textbf{admissible structure}:
      \[
        S_1 \subset S_2 \subset \cdots \subset S_n \subset \cdots
        \quad \text{with} \quad h_1 \leq h_2 \leq \cdots
      \]
    \item Choose the subset $S_k$ that minimizes the \textbf{guaranteed risk}
      (empirical risk + confidence interval)
  \end{itemize}
\end{frame}

% ---- Figure: SRM trade-off ----

\begin{frame}{SRM trade-off}
  \centering
  \begin{tikzpicture}
    \begin{axis}[
        axis lines=middle,
        xlabel={$k$},
        ylabel={Risk},
        ytick={0},
        yticklabels={},
        xtick={0.1, 0.5, 1},
        xticklabels={$h_1$, $h^*$, $h_n$},
        grid=both,
        xmin=0, xmax=1.5,
        ymin=0, ymax=1.5,
        domain=0.1:1,
        width=0.65\textwidth,
        height=0.55\textheight,
      ]

      \addplot[smooth] {(x - 1)^2 + 0.1}
        node[right, text width=2.5cm, font=\small] {Empirical risk};
      \addplot[smooth] {x^2 + 0.1}
        node[right, text width=2.5cm, font=\small] {Confidence interval};
      \addplot[smooth, thick] {x^2 + (x - 1)^2 + 0.2}
        node[left, text width=2.5cm, font=\small] {Risk upper bound};
    \end{axis}
  \end{tikzpicture}

  \small
  The smallest guaranteed risk is found at $h^*$ in the admissible structure.
\end{frame}

% ---------------------------------------------------------------------------
\subsection{Bias-variance trade-off}
% ---------------------------------------------------------------------------

\begin{frame}{Bias-variance decomposition}
  For regression with noise $\epsilon$ ($\E[\epsilon] = 0$, $\Var(\epsilon) = \sigma^2$):

  \vspace{0.3cm}
  \[
    \E_D\!\left[ \big( y - \hat{f}(x; D) \big)^2 \right]
    = \underbrace{\sigma^2}_{\text{irreducible}}
    + \underbrace{\big( f(x) - \E[\hat{f}] \big)^2}_{\text{bias}^2}
    + \underbrace{\Var_D\!\big( \hat{f}(x; D) \big)}_{\text{variance}}
  \]

  \vspace{0.3cm}
  \begin{itemize}
    \item \textbf{Bias}: failure to capture relevant relationships
    \item \textbf{Variance}: modeling the noise in training data
    \item This is the general case of the SRM trade-off
  \end{itemize}
\end{frame}

% ---------------------------------------------------------------------------
\subsection{Regularization}
% ---------------------------------------------------------------------------

\begin{frame}{Regularization}
  Modify the loss to penalize model complexity:
  \[
    R_n(\theta) + \lambda\, \Omega(\theta)
  \]

  \begin{itemize}
    \item $\Omega(\theta)$ --- complexity penalty (e.g., $\|\theta\|^2$)
    \item $\lambda$ --- hyperparameter controlling the trade-off
    \item Acts as a proxy for the confidence interval in SRM
    \item Implicit regularization: early stopping, dropout, ensembles, pruning
  \end{itemize}
\end{frame}

% ===========================================================================
\section{Linear problems}
% ===========================================================================

\begin{frame}{Linear classification}
  \begin{itemize}
    \item Realize SRM concepts in practice
    \item Two learning machines:
      \begin{enumerate}
        \item \textbf{Perceptron} --- fix complexity, minimize empirical risk
        \item \textbf{Maximal margin classifier} --- fix empirical risk (zero),
          minimize confidence interval
      \end{enumerate}
  \end{itemize}
\end{frame}

\begin{frame}{AND and XOR datasets}
  \centering
  \begin{columns}[T]
    \begin{column}{0.45\textwidth}
      \centering
      \rowcolors{2}{black!10!white}{}
      \begin{tabular}{ccc}
        \toprule
        $x_1$ & $x_2$ & $y = x_1 \land x_2$ \\
        \midrule
        0 & 0 & 0 \\
        0 & 1 & 0 \\
        1 & 0 & 0 \\
        1 & 1 & 1 \\
        \bottomrule
      \end{tabular}
    \end{column}
    \begin{column}{0.45\textwidth}
      \centering
      \rowcolors{2}{black!10!white}{}
      \begin{tabular}{ccc}
        \toprule
        $x_1$ & $x_2$ & $y = x_1 \oplus x_2$ \\
        \midrule
        0 & 0 & 0 \\
        0 & 1 & 1 \\
        1 & 0 & 1 \\
        1 & 1 & 0 \\
        \bottomrule
      \end{tabular}
    \end{column}
  \end{columns}

  \vspace{0.3cm}
  AND is linearly separable. XOR is not.
\end{frame}

% ---------------------------------------------------------------------------
\subsection{Perceptron}
% ---------------------------------------------------------------------------

\begin{frame}{Perceptron}
  Linear classifier with model:
  \[
    f(x_1, x_2; \vec{w}) = u(w_0 + w_1 x_1 + w_2 x_2)
  \]
  where $u$ is the Heaviside step function:
  \[
    u(x) = \begin{cases}
      1 & \text{if } x > 0 \\
      0 & \text{otherwise}
    \end{cases}
  \]

  \vspace{0.3cm}
  The equation $\vec{w} \cdot \vec{x} = 0$ defines a \textbf{hyperplane}
  ($\vec{x} = [1, x_1, x_2]$).
\end{frame}

% ---- Figure: Perceptron AND ----

\begin{frame}{Perceptron --- AND dataset}
  \centering
  \begin{tikzpicture}
    \begin{axis}[
        axis x line=bottom,
        axis y line=left,
        xlabel={$x_1$},
        ylabel={$x_2$},
        width=0.5\textwidth,
        height=0.5\textwidth,
        xtick={0, 1},
        ytick={0, 1},
        xmin=-0.5, xmax=1.5,
        ymin=-0.5, ymax=1.5,
      ]
      \addplot+[only marks, mark=-, color=black, mark size=3pt] coordinates {
        (0, 0) (0, 1) (1, 0)
      };
      \addplot+[only marks, mark=+, color=black, mark size=3pt] coordinates {
        (1, 1)
      };
      \addplot+[domain=0:1.5, mark=none, black, thick] {1.1 - 0.6 * x};
    \end{axis}
  \end{tikzpicture}

  \small $\vec{w} = [-1.1,\; 0.6,\; 1]$
  \quad --- correctly classifies all samples.
\end{frame}

\begin{frame}{Perceptron AND --- truth table}
  \centering
  \rowcolors{2}{black!10!white}{}
  \begin{tabular}{ccc|cc}
    \toprule
    $x_1$ & $x_2$ & $y$ & $-1.1 + 0.6\,x_1 + x_2$ & $\hat{y}$ \\
    \midrule
    0 & 0 & 0 & $-1.1$ & 0 \\
    0 & 1 & 0 & $-0.1$ & 0 \\
    1 & 0 & 0 & $-0.5$ & 0 \\
    1 & 1 & 1 & $0.5$ & 1 \\
    \bottomrule
  \end{tabular}
\end{frame}

% ---- Figure: Perceptron XOR ----

\begin{frame}{Perceptron --- XOR dataset}
  \centering
  \begin{tikzpicture}
    \begin{axis}[
        axis x line=bottom,
        axis y line=left,
        xlabel={$x_1$},
        ylabel={$x_2$},
        width=0.5\textwidth,
        height=0.5\textwidth,
        xtick={0, 1},
        ytick={0, 1},
        xmin=-0.5, xmax=1.5,
        ymin=-0.5, ymax=1.5,
      ]
      \addplot+[only marks, mark=-, color=black, mark size=3pt] coordinates {
        (0, 0) (1, 1)
      };
      \addplot+[only marks, mark=+, color=black, mark size=3pt] coordinates {
        (0, 1) (1, 0)
      };
      \addplot+[domain=0:1.5, mark=none, black, thick] {-0.5 + x};
    \end{axis}
  \end{tikzpicture}

  \small $\vec{w} = [-0.5,\; 1,\; -1]$
  \quad --- no perceptron can solve XOR.
\end{frame}

\begin{frame}{Perceptron XOR --- truth table}
  \centering
  \rowcolors{2}{black!10!white}{}
  \begin{tabular}{ccc|cc}
    \toprule
    $x_1$ & $x_2$ & $y$ & $-0.5 + x_1 - x_2$ & $\hat{y}$ \\
    \midrule
    0 & 0 & 0 & $-0.5$ & 0 \\
    0 & 1 & 1 & $-1.5$ & 0 \\
    1 & 0 & 1 & $0.5$ & 1 \\
    1 & 1 & 0 & $-0.5$ & 0 \\
    \bottomrule
  \end{tabular}

  \vspace{0.3cm}
  \small The perceptron fails to classify $(0, 1)$ correctly.
\end{frame}

% ---- Perceptron training ----

\begin{frame}{Perceptron training}
  For a misclassified sample, update the weights:
  \[
    \vec{w}' = \vec{w} + \eta\, e\, \vec{x}
  \]
  where $e = y - u(\vec{w} \cdot \vec{x})$ and $\eta > 0$ (step size).

  \vspace{0.3cm}
  \begin{itemize}
    \item Converges if $\eta$ is small enough and data is linearly separable
    \item Makes no effort to reduce the confidence interval
    \item Simplest artificial neural network
  \end{itemize}
\end{frame}

% ---- Figure: Weight update (positive output) ----

\begin{frame}{Weight update --- positive output error}
  \centering
  \begin{tikzpicture}[scale=1.3]
    \draw[-Stealth] (0, 0) -- (2, 0) node[right] {$\vec{x}$};
    \draw[-Stealth] (0, 0) -- (1, 1.8) node[right] {$\vec{w}$};
    \draw[-Stealth, dashed] (1, 1.8) -- (-0.4, 1.8) node[above] {$-\eta\vec{x}$};
    \draw[-Stealth, thick, gray] (0, 0) -- (-0.4, 1.8) node[left] {$\vec{w}'$};
    \draw (0.4, 0) arc (0:59:0.4) node[right] {$\alpha$};
  \end{tikzpicture}

  \vspace{0.3cm}
  \small
  $\vec{w} \cdot \vec{x} > 0 \implies \alpha < 90^\circ$.\\
  Subtract $\eta\vec{x}$ to increase the angle.
\end{frame}

% ---- Figure: Weight update (negative output) ----

\begin{frame}{Weight update --- negative output error}
  \centering
  \begin{tikzpicture}[scale=1.3]
    \draw[-Stealth] (0, 0) -- (2, 0) node[right] {$\vec{x}$};
    \draw[-Stealth, thick, gray] (0, 0) -- (1, 1.8) node[right] {$\vec{w}'$};
    \draw[-Stealth, dashed] (-0.4, 1.8) -- (1, 1.8) node[above] {$\eta\vec{x}$};
    \draw[-Stealth] (0, 0) -- (-0.4, 1.8) node[left] {$\vec{w}$};
    \draw (0.4, 0) arc (0:101:0.4) node[above] {\phantom{a }$\alpha$};
  \end{tikzpicture}

  \vspace{0.3cm}
  \small
  $\vec{w} \cdot \vec{x} < 0 \implies \alpha > 90^\circ$.\\
  Add $\eta\vec{x}$ to decrease the angle.
\end{frame}

% ---------------------------------------------------------------------------
\subsection{Maximal margin classifier}
% ---------------------------------------------------------------------------

\begin{frame}{Maximal margin classifier}
  \begin{itemize}
    \item Fix empirical risk to \textbf{zero} (assume linearly separable)
    \item Minimize the \textbf{confidence interval} by maximizing the margin
    \item $\Delta$-margin hyperplane: $(\vec{w} \cdot \vec{x}) - b = 0$,
      $\|\vec{w}\| = 1$
    \item VC dimension:
      \[
        h \leq \min\!\left(\left\lfloor \frac{R^2}{\Delta^2} \right\rfloor, d\right) + 1
      \]
    \item Larger margin $\Delta$ $\Rightarrow$ lower VC dimension $\Rightarrow$
      better generalization
  \end{itemize}
\end{frame}

% ---- Figure: Maximal margin AND ----

\begin{frame}{Maximal margin --- AND dataset}
  \centering
  \begin{tikzpicture}
    \begin{axis}[
        axis x line=bottom,
        axis y line=left,
        xlabel={$x_1$},
        ylabel={$x_2$},
        width=0.5\textwidth,
        height=0.5\textwidth,
        xtick={0, 1},
        ytick={0, 1},
        xmin=-0.5, xmax=1.5,
        ymin=-0.5, ymax=1.5,
        domain=-0.5:1.5,
      ]
      \addplot+[only marks, mark=-, color=black, mark size=3pt] coordinates {
        (0, 0) (0, 1) (1, 0)
      };
      \addplot+[only marks, mark=+, color=black, mark size=3pt] coordinates {
        (1, 1)
      };
      \addplot+[mark=none, black, thick] {1.5 - x}
        node[above, pos=0.6, rotate=-45, font=\small] {optimal};
      \addplot+[mark=none, black, dashed] {1 - x}
        node[above, pos=0.6, rotate=-45, font=\small] {margin};
      \addplot+[mark=none, black, dashed] {2 - x};
    \end{axis}
  \end{tikzpicture}

  \small Support vectors: $(1, 0)$, $(0, 1)$, and $(1, 1)$.
\end{frame}

\begin{frame}{Maximal margin classifier --- properties}
  The classifier is built from \textbf{support vectors} only:
  \[
    f(x) = \sign\!\left(\sum_{i=1}^{n} y_i\, a_i\, (\vec{x}_i \cdot x) - b\right)
  \]
  where $a_i > 0$ for support vectors, $a_i = 0$ otherwise.

  \vspace{0.3cm}
  \begin{itemize}
    \item Number of parameters depends on data $\Rightarrow$ \textbf{nonparametric}
    \item Soft margin: allows $R_\text{emp} > 0$ for non-separable data
    \item Kernel trick: handle nonlinear problems
  \end{itemize}
\end{frame}

% ---- Takeaways ----

\begin{frame}{Takeaways}
  \begin{itemize}
    \item Optimal solutions establish how good a solution can possibly be
    \item Reducing error is not enough to guarantee a good solution
    \item Controlling model complexity is crucial for generalization
    \item The perceptron minimizes empirical risk (fixed complexity)
    \item The maximal margin classifier minimizes complexity (fixed risk)
  \end{itemize}
\end{frame}

% ---- End ----

\begin{frame}[standout]
  Questions?
\end{frame}

\end{document}
