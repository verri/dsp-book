% slides/data-handling.tex - Chapter 5: Data Handling

% slides/preamble.tex - Shared preamble for all slide decks
% Metropolis theme with grayscale style matching the book

\documentclass[aspectratio=169]{beamer}

\usetheme{metropolis}

% ---------- Colors (grayscale) ----------
\definecolor{bookdark}{gray}{0.2}
\definecolor{bookgray}{gray}{0.5}
\definecolor{booklight}{gray}{0.92}

\setbeamercolor{normal text}{fg=bookdark, bg=white}
\setbeamercolor{alerted text}{fg=bookdark}
\setbeamercolor{frametitle}{fg=white, bg=bookdark}
\setbeamercolor{title separator}{fg=bookgray}
\setbeamercolor{progress bar}{fg=bookgray, bg=booklight}
\setbeamercolor{block title}{fg=white, bg=bookgray}
\setbeamercolor{block body}{fg=bookdark, bg=booklight}
\setbeamercolor{block title alerted}{fg=white, bg=bookdark}
\setbeamercolor{block body alerted}{fg=bookdark, bg=booklight}
\setbeamercolor{block title example}{fg=white, bg=bookgray}
\setbeamercolor{block body example}{fg=bookdark, bg=booklight}

\setbeamertemplate{frame numbering}[fraction]

% ---------- Fonts (matching the book) ----------
\usepackage[T1]{fontenc}
\usepackage{fontspec}
\usepackage[warnings-off={mathtools-colon,mathtools-overbracket}]{unicode-math}

\setmathfont{STIXTwoMath}[
  Extension={.otf},
  Path={./fonts/},
  Scale=1]

\setsansfont{STIXTwoText}[
  Extension={.otf},
  Path={./fonts/},
  UprightFont={*-Regular},
  BoldFont={*-Bold},
  ItalicFont={*-Italic},
  BoldItalicFont={*-BoldItalic}]

\setmainfont{STIXTwoText}[
  Extension={.otf},
  Path={./fonts/},
  UprightFont={*-Regular},
  BoldFont={*-Bold},
  ItalicFont={*-Italic},
  BoldItalicFont={*-BoldItalic}]

\setmonofont{CourierPrime}[
  Extension={.ttf},
  Path={./fonts/},
  UprightFont={*-Regular},
  BoldFont={*-Bold},
  ItalicFont={*-Italic},
  BoldItalicFont={*-BoldItalic},
  Scale=0.9]

% ---------- Packages ----------
\usepackage{amsmath}
\usepackage{mathtools}
\usepackage{graphicx}
\usepackage{booktabs}

% ---------- TikZ ----------
\usepackage{tikz}
\usetikzlibrary{shapes, arrows.meta, positioning, shapes.geometric, fit}

\tikzset{%
  decision/.style={draw, diamond, text centered, minimum height=0.5cm, minimum width=1cm},
  block/.style={rectangle, draw, text width=6em, text centered, rounded corners, minimum height=3em},
  mediumblock/.style={rectangle, draw, text width=3em, text centered, rounded corners, minimum height=2em},
  darkblock/.style={block, fill=gray, text=white},
  smallblock/.style={rectangle, rounded corners, draw, font=\tiny, minimum height=1em, inner sep=2pt},
  smalldarkblock/.style={smallblock, fill=gray, text=white},
  darkcircle/.style={draw, circle, fill=gray, text centered, text=white},
  smallcircle/.style={draw, circle, text centered, font=\tiny},
  smalldarkcircle/.style={smallcircle, fill=gray, text=white},
  line/.style={draw, -latex},
  dline/.style={draw, latex-latex},
  bigarrow/.style={draw, -latex, line width=3pt, gray},
}

% ---------- Math operators ----------
\DeclareMathOperator*{\argmax}{arg\,max}
\DeclareMathOperator*{\argmin}{arg\,min}
\DeclareMathOperator{\Prob}{P}
\DeclareMathOperator{\E}{E}
\DeclareMathOperator{\Var}{Var}
\DeclareMathOperator{\sign}{sign}
\DeclareMathOperator{\clamp}{clamp}

% ---------- Hyperref ----------
\usepackage{hyperref}
\hypersetup{colorlinks, urlcolor=bookgray, linkcolor=bookdark}

% ---------- Commands ----------
\renewcommand{\vec}[1]{\mathbf{#1}}
\newcommand{\code}[1]{\colorbox{black!10!white}{\texttt{#1}}}

% ---------- Book info, license, and disclaimer ----------
\newcommand{\bookframe}{%
  \begin{frame}{About these slides}
    These slides are companion material for the book

    \vspace{0.3cm}
    \begin{center}
      \textbf{Data Science Project: An Inductive Learning Approach}\\[2pt]
      Prof.~Dr.~Filipe A. N. Verri\\[4pt]
      \url{https://leanpub.com/dsp}
    \end{center}

    \vspace{0.3cm}
    All intellectual content comes from the book and is not AI-generated.
    Slides were produced with the assistance of
    \href{https://claude.ai/code}{Claude Code}.

    \vspace{0.3cm}
    {\small
    Licensed under
    \href{https://creativecommons.org/licenses/by-nc/4.0/}{CC BY-NC 4.0}.
    You are free to modify and redistribute this work as long as you give
    proper credit and do not use it for commercial purposes.}
  \end{frame}%
}

% ---------- TikZ color presets (used by book figures) ----------
\colorlet{circle edge}{black!50}
\colorlet{circle area}{black!20}
\tikzset{
  filled/.style={fill=circle area, draw=circle edge, thick},
  outline/.style={draw=circle edge, thick},
}


\title{Data Handling}
\subtitle{Data Science Project: An Inductive Learning Approach}
\author{Prof.~Dr.~Filipe A. N. Verri}
\date{}

\begin{document}

\maketitle
\bookframe

% ---- Epigraph ----

\begin{frame}{}
  \vfill
  \begin{quote}
    $\dagger$ It's dangerous to go alone! Take this.
    \begin{flushright}
      --- Unnamed Old Man, \textit{The Legend of Zelda}
    \end{flushright}
  \end{quote}
  \vfill
\end{frame}

% ---- Overview ----

\begin{frame}{Overview}
  \begin{columns}[T]
    \begin{column}{0.48\textwidth}
      \textbf{Contents}
      \begin{itemize}
        \item Formal structured data
        \item Data handling pipelines
        \item Split-invariant operations
        \item Other operations
        \item An algebra for data handling
      \end{itemize}
    \end{column}
    \begin{column}{0.48\textwidth}
      \textbf{Objectives}
      \begin{itemize}
        \item Define a formal structure for tables
        \item Define operations on tables
        \item Understand split invariance
      \end{itemize}
    \end{column}
  \end{columns}
\end{frame}

% ===========================================================================
\section{Formal structured data}
% ===========================================================================

\begin{frame}{Indexed table}
  \begin{block}{Definition}
    An indexed table $T = (K, H, c)$:
    \begin{itemize}
      \item $K = \{K_1, \ldots, K_k\}$ --- set of index columns
      \item $H$ --- set of non-index columns
      \item $c$ --- cell function mapping (row, column) to values
    \end{itemize}
  \end{block}

  \vspace{0.3cm}
  \begin{itemize}
    \item Rows indexed by tuples $r = (k_1, \ldots, k_k)$
    \item Each row has a cardinality (repetitions of the entity)
    \item Values may be \textbf{missing} (?)
  \end{itemize}
\end{frame}

\begin{frame}{Nested rows and value matrices}
  \begin{itemize}
    \item \textbf{Nested row}: associates different columns with the same
      repetition of the entity
    \item \textbf{Value matrix} $V(r)$: stacking nested rows into a
      $\operatorname{card}(r) \times |H|$ matrix
    \item Value matrices are assumed minimal (no all-missing nested rows)
  \end{itemize}
\end{frame}

% ---------------------------------------------------------------------------
\subsection{Splitting and binding}
% ---------------------------------------------------------------------------

\begin{frame}{Split and bind}
  \begin{block}{Split}
    Given indicator $s(r) \in \{0, 1\}$, creates two disjoint tables:
    \[
      \operatorname{split}(T, s) = (T_0, T_1)
    \]
    A split \textbf{never breaks a row} --- indices define indivisible entities.
  \end{block}

  \begin{block}{Bind}
    Inverse of split --- combines two \textbf{disjoint} tables:
    \[
      \operatorname{bind}(T_0, T_1) = T
    \]
  \end{block}
\end{frame}

\begin{frame}{Premises in real-world applications}
  \begin{itemize}
    \item Entities represented by rows are indivisible
    \item Binding only for tables that do not share entities
    \item Data collection $\approx$ split from the universe of entities
    \item Fewer index columns $\to$ more information per entity required
    \item More index columns $\to$ restrictions on allowed operations
    \item Neglecting these issues leads to statistical biases
  \end{itemize}
\end{frame}

% ---------------------------------------------------------------------------
\subsection{Split invariance}
% ---------------------------------------------------------------------------

\begin{frame}{Split invariance}
  \begin{block}{Definition}
    Operation $f$ is \textbf{split-invariant} if for any table $T$ and split $s$:
    \[
      f(T_0 + T_1) = f(T_0) + f(T_1)
    \]
    where $T_0, T_1 = \operatorname{split}(T, s)$.
  \end{block}

  \vspace{0.3cm}
  \begin{itemize}
    \item Ensures the operation does not depend on how data was collected
    \item Avoids \textbf{data leakage} and sampling bias
    \item Sufficient (not necessary) condition for preventing leakage
  \end{itemize}
\end{frame}

% ---------------------------------------------------------------------------
\subsection{Illustrative example}
% ---------------------------------------------------------------------------

\begin{frame}{Illustrative example --- Student grades}
  \centering\small
  \rowcolors{2}{black!10!white}{}
  \begin{tabular}{llcc}
    \toprule
    \textbf{student} & \textbf{subject} & \textbf{year} & \textbf{grade} \\
    \midrule
    Alice & Chemistry & 2020 & 6 \\
    Alice & Math & 2019 & 8 \\
    Alice & Physics & 2019 & 7 \\
    Bob & Chemistry & 2018 & ? \\
    Bob & Chemistry & 2019 & 7 \\
    Bob & Math & 2019 & 9 \\
    Bob & Physics & 2019 & 4 \\
    Bob & Physics & 2020 & 8 \\
    Carol & Biology & 2020 & 8 \\
    Carol & Chemistry & 2020 & 3 \\
    Carol & Math & 2020 & 10 \\
    \bottomrule
  \end{tabular}
\end{frame}

\begin{frame}{Impact of index choice}
  \textbf{Index = (student, subject):}
  \begin{itemize}
    \item Cannot know if Alice took Biology
    \item Can confirm Bob passed Physics (second attempt)
    \item Cannot guarantee Carol only took classes in 2020
  \end{itemize}

  \vspace{0.5cm}
  \textbf{Index = (student):}
  \begin{itemize}
    \item Can confirm Alice never took Biology
    \item No information about Bob (not in our split)
    \item Can confirm Carol only took classes in 2020
  \end{itemize}

  \vspace{0.3cm}
  Fewer indices $\to$ more information about present entities.
\end{frame}

% ===========================================================================
\section{Data handling pipelines}
% ===========================================================================

% ---- Figure: Pipeline ----

\begin{frame}{Data handling pipeline}
  \centering
  \begin{tikzpicture}[every node/.style={font=\small, inner sep=4pt}]
    \node (s1) [darkcircle] at (0, 0) {Source 1};
    \node (s2) [darkcircle] at (0, -2) {Source 2};
    \node (f1) [mediumblock] at (2, 0) {$f_1$};
    \node (f2) [mediumblock] at (4, 0) {$f_2$};
    \node (f3) [mediumblock] at (2, -2) {$f_3$};
    \node (f4) [mediumblock] at (4, -2) {$f_4$};
    \node (f5) [mediumblock] at (6, -1) {$f_5$};
    \node (data) [darkcircle, minimum width=15mm] (data) at (8, -1) {Data};

    \path [line] (s1) -- (f1);
    \path [line] (f1) -- (f2);
    \path [line] (f1.east) -- (f4);
    \path [line] (s2) -- (f3);
    \path [line] (f3) -- (f4);
    \path [line] (f2) -- (f5);
    \path [line] (f4) -- (f5);
    \path [line] (f5) -- (data);
  \end{tikzpicture}

  \vspace{0.3cm}
  \small
  Forks and merges allow complex processing.\\
  Declarative approach enables optimization and parallel processing.
\end{frame}

% ===========================================================================
\section{Split-invariant operations}
% ===========================================================================

\begin{frame}{Split-invariant operations}
  Operations safe to use during data integration and tidying:
  \begin{itemize}
    \item Tagged splitting and binding
    \item Pivoting (long-to-wide and wide-to-long)
    \item Joining (left join with fixed table)
    \item Selecting columns
    \item Filtering rows
    \item Mutating (creating new columns)
    \item Aggregating (reducing row cardinality)
    \item Ungrouping (increasing index columns)
  \end{itemize}
\end{frame}

% ---------------------------------------------------------------------------
\subsection{Tagged splitting and binding}
% ---------------------------------------------------------------------------

\begin{frame}{Tagged bind and split}
  \begin{itemize}
    \item \textbf{Tagged bind}: combine disjoint tables + add source column $s$
    \item Solves: single observational unit in multiple tables
    \item Pay attention to hidden semantics (e.g., units of measurement)
    \item \textbf{Tagged split}: reverse --- split by values of column $s$
    \item Split-invariant by definition (each row has unique $s$ value)
  \end{itemize}
\end{frame}

% ---------------------------------------------------------------------------
\subsection{Pivoting}
% ---------------------------------------------------------------------------

\begin{frame}{Pivoting}
  \begin{columns}[T]
    \begin{column}{0.42\textwidth}
      \textbf{Long format}\\[0.2cm]
      \centering\small
      \rowcolors{2}{black!10!white}{}
      \begin{tabular}{lcc}
        \toprule
        \textbf{city} & \textbf{year} & \textbf{qty.} \\
        \midrule
        A & 2019 & 1 \\
        A & 2020 & 2 \\
        A & 2021 & 3 \\
        B & 2019 & 4 \\
        B & 2020 & 5 \\
        B & 2021 & 6 \\
        \bottomrule
      \end{tabular}
    \end{column}
    \begin{column}{0.05\textwidth}
      \centering\vspace{1.5cm}$\leftrightarrow$
    \end{column}
    \begin{column}{0.48\textwidth}
      \textbf{Wide format}\\[0.2cm]
      \centering\small
      \rowcolors{2}{black!10!white}{}
      \begin{tabular}{lccc}
        \toprule
        \textbf{city} & \textbf{2019} & \textbf{2020} & \textbf{2021} \\
        \midrule
        A & 1 & 2 & 3 \\
        B & 4 & 5 & 6 \\
        \bottomrule
      \end{tabular}
    \end{column}
  \end{columns}

  \vspace{0.5cm}
  \begin{itemize}
    \item \textbf{Long-to-wide}: name column values become new columns
    \item \textbf{Wide-to-long}: column names become values
    \item Both operations are split-invariant
    \item Solves: headers as values, variables in rows and columns
  \end{itemize}
\end{frame}

% ---------------------------------------------------------------------------
\subsection{Joining}
% ---------------------------------------------------------------------------

\begin{frame}{Joining}
  \begin{itemize}
    \item Combines two tables based on common index columns
    \item \textbf{Inner join}: only matching rows kept
      \begin{itemize}
        \item Can ``erase'' rows --- potential source of bias
      \end{itemize}
    \item \textbf{Left join} (with fixed table): all rows from first table kept
      \begin{itemize}
        \item Missing matches $\to$ missing values for second table columns
        \item \textbf{Split-invariant} when second table is fixed
      \end{itemize}
    \item Cardinality of joined rows must be constant (aggregate first if needed)
  \end{itemize}
\end{frame}

% ---------------------------------------------------------------------------
\subsection{Selecting and filtering}
% ---------------------------------------------------------------------------

\begin{frame}{Selecting and filtering}
  \textbf{Selecting} (columns):
  \begin{itemize}
    \item Choose a subset of non-index columns; rows unchanged
    \item Split-invariant if predicate does not depend on values
  \end{itemize}

  \vspace{0.5cm}
  \textbf{Filtering} (rows):
  \begin{itemize}
    \item Choose rows satisfying a predicate; columns unchanged
    \item Predicates combined with logical disjunction (or)
    \item Split-invariant because rows are treated independently
  \end{itemize}
\end{frame}

% ---------------------------------------------------------------------------
\subsection{Mutating and aggregating}
% ---------------------------------------------------------------------------

\begin{frame}{Mutating and aggregating}
  \textbf{Mutating}:
  \begin{itemize}
    \item Create new columns from a function of existing values
    \item Example: \code{y = x + 1}, \code{y = ifelse(x > 0, 1, 0)}
    \item Split-invariant (rows treated independently)
    \item Solves: multiple variables stored in one column
  \end{itemize}

  \vspace{0.5cm}
  \textbf{Aggregating}:
  \begin{itemize}
    \item Ensure all rows have cardinality 1
    \item Function maps value matrix to a single tuple
    \item Split-invariant (rows treated independently)
  \end{itemize}
\end{frame}

% ---------------------------------------------------------------------------
\subsection{Ungrouping}
% ---------------------------------------------------------------------------

\begin{frame}{Ungrouping}
  Transforms a non-index column into an index column.

  \vspace{0.3cm}
  \centering\small
  \rowcolors{2}{black!10!white}{}
  \begin{tabular}{l|cc}
    \toprule
    \textbf{city} & \textbf{year} & \textbf{qty.} \\
    \midrule
    A & (2019, 2020, 2020, 2021) & (1, 2, 3, 4) \\
    B & (2019, 2020, 2021) & (5, 6, 7) \\
    \bottomrule
  \end{tabular}

  \vspace{0.3cm}
  $\downarrow$ ungroup by \emph{year} $\downarrow$

  \vspace{0.3cm}
  \rowcolors{2}{black!10!white}{}
  \begin{tabular}{ll|c}
    \toprule
    \textbf{city} & \textbf{year} & \textbf{qty.} \\
    \midrule
    A & 2019 & 1 \\
    A & 2020 & (2, 3) \\
    A & 2021 & 4 \\
    B & 2019 & 5 \\
    B & 2020 & 6 \\
    B & 2021 & 7 \\
    \bottomrule
  \end{tabular}

  \vspace{0.3cm}
  \raggedright Split-invariant.
\end{frame}

% ===========================================================================
\section{Other operations}
% ===========================================================================

% ---------------------------------------------------------------------------
\subsection{Projecting or grouping}
% ---------------------------------------------------------------------------

\begin{frame}{Projection (grouping)}
  \begin{itemize}
    \item Reduces the number of index columns
    \item Changes the observational unit
    \item Rows are grouped by the remaining indices
    \item \textbf{Not split-invariant}
      \begin{itemize}
        \item Splitting may place parts of a group in different tables
        \item Binding after projection violates the disjoint property
      \end{itemize}
    \item Poor data schema + projection $\to$ incorrect conclusions
  \end{itemize}
\end{frame}

% ---------------------------------------------------------------------------
\subsection{Grouped and arranged operations}
% ---------------------------------------------------------------------------

\begin{frame}{Grouped and arranged operations}
  When more flexibility is needed (e.g., preprocessing):
  \begin{itemize}
    \item \textbf{Aggregation function}: returns single value from tuple
      (sum, mean, count, first)
    \item \textbf{Window function}: returns tuple of same size
      (cumsum, lag, rank)
    \item Require specifying \textbf{groups} and/or \textbf{order}
    \item \textbf{Not split-invariant} (grouping is always required)
  \end{itemize}
\end{frame}

% ---- Figure: Mutating with groups and order ----

\begin{frame}{Mutating with groups and order}
  \centering
  \begin{tikzpicture}[every node/.style={font=\small, inner sep=4pt}]
    \node (source) [darkcircle] at (0, -2) {Source};
    \node (group) [mediumblock] at (0, 0) {Group};
    \node (mutate) [mediumblock] at (2, 0) {Mutate};
    \node (ungroup) [mediumblock, text width=3.5em] at (4, 0) {Ungroup};
    \node (select) [mediumblock] at (6, 0) {Select};
    \node (temp) [darkcircle] at (6, -2) {};
    \node (join) [mediumblock] at (3, -2) {Join};
    \node (result) [darkcircle] at (3, -4) {Result};

    \path [line] (source) -- (group);
    \path [line] (group) -- (mutate);
    \path [line] (mutate) -- (ungroup);
    \path [line] (ungroup) -- (select);
    \path [line] (select) -- (temp);
    \path [line] (temp) -- (join);
    \path [line] (source) -- (join);
    \path [line] (join) -- (result);
  \end{tikzpicture}

  \vspace{0.3cm}
  \small Example: \code{y = cumsum(x) group by category sort by date}
\end{frame}

% ===========================================================================
\section{An algebra for data handling}
% ===========================================================================

\begin{frame}{Towards an algebra for data handling}
  \begin{itemize}
    \item Formal algebra for data transformations is an open problem
    \item Statistical data handling $\neq$ relational algebra
      (transformation vs retrieval)
    \item Notable efforts: SDTA\footfullcite{Song2021},
      LaraDB\footfullcite{hutchison2017laradb} (3 operators unifying RA + LA),
      Modin\footfullcite{petersohn2020towards} (14 operators from pandas)
    \item Desirable properties: completeness, composability
    \item Split invariance is a novel concept proposed in this book
    \item Future: a ``Turing complete'' data handling language
  \end{itemize}
\end{frame}

% ---- Takeaways ----

\begin{frame}{Takeaways}
  \begin{itemize}
    \item Split-invariant operations avoid sampling bias
    \item One must understand the properties and premises of the operations
    \item Index choice determines what conclusions are safe
    \item Projection/grouping is \textbf{not} split-invariant --- use carefully
  \end{itemize}
\end{frame}

% ---- End ----

\begin{frame}[standout]
  Questions?
\end{frame}

\end{document}
