% slides/mathematical-foundations.tex - Appendix: Mathematical Foundations

% slides/preamble.tex - Shared preamble for all slide decks
% Metropolis theme with grayscale style matching the book

\documentclass[aspectratio=169]{beamer}

\usetheme{metropolis}

% ---------- Colors (grayscale) ----------
\definecolor{bookdark}{gray}{0.2}
\definecolor{bookgray}{gray}{0.5}
\definecolor{booklight}{gray}{0.92}

\setbeamercolor{normal text}{fg=bookdark, bg=white}
\setbeamercolor{alerted text}{fg=bookdark}
\setbeamercolor{frametitle}{fg=white, bg=bookdark}
\setbeamercolor{title separator}{fg=bookgray}
\setbeamercolor{progress bar}{fg=bookgray, bg=booklight}
\setbeamercolor{block title}{fg=white, bg=bookgray}
\setbeamercolor{block body}{fg=bookdark, bg=booklight}
\setbeamercolor{block title alerted}{fg=white, bg=bookdark}
\setbeamercolor{block body alerted}{fg=bookdark, bg=booklight}
\setbeamercolor{block title example}{fg=white, bg=bookgray}
\setbeamercolor{block body example}{fg=bookdark, bg=booklight}

\setbeamertemplate{frame numbering}[fraction]

% ---------- Fonts (matching the book) ----------
\usepackage[T1]{fontenc}
\usepackage{fontspec}
\usepackage[warnings-off={mathtools-colon,mathtools-overbracket}]{unicode-math}

\setmathfont{STIXTwoMath}[
  Extension={.otf},
  Path={./fonts/},
  Scale=1]

\setsansfont{STIXTwoText}[
  Extension={.otf},
  Path={./fonts/},
  UprightFont={*-Regular},
  BoldFont={*-Bold},
  ItalicFont={*-Italic},
  BoldItalicFont={*-BoldItalic}]

\setmainfont{STIXTwoText}[
  Extension={.otf},
  Path={./fonts/},
  UprightFont={*-Regular},
  BoldFont={*-Bold},
  ItalicFont={*-Italic},
  BoldItalicFont={*-BoldItalic}]

\setmonofont{CourierPrime}[
  Extension={.ttf},
  Path={./fonts/},
  UprightFont={*-Regular},
  BoldFont={*-Bold},
  ItalicFont={*-Italic},
  BoldItalicFont={*-BoldItalic},
  Scale=0.9]

% ---------- Packages ----------
\usepackage{amsmath}
\usepackage{mathtools}
\usepackage{graphicx}
\usepackage{booktabs}

% ---------- TikZ ----------
\usepackage{tikz}
\usetikzlibrary{shapes, arrows.meta, positioning, shapes.geometric, fit}

\tikzset{%
  decision/.style={draw, diamond, text centered, minimum height=0.5cm, minimum width=1cm},
  block/.style={rectangle, draw, text width=6em, text centered, rounded corners, minimum height=3em},
  mediumblock/.style={rectangle, draw, text width=3em, text centered, rounded corners, minimum height=2em},
  darkblock/.style={block, fill=gray, text=white},
  smallblock/.style={rectangle, rounded corners, draw, font=\tiny, minimum height=1em, inner sep=2pt},
  smalldarkblock/.style={smallblock, fill=gray, text=white},
  darkcircle/.style={draw, circle, fill=gray, text centered, text=white},
  smallcircle/.style={draw, circle, text centered, font=\tiny},
  smalldarkcircle/.style={smallcircle, fill=gray, text=white},
  line/.style={draw, -latex},
  dline/.style={draw, latex-latex},
  bigarrow/.style={draw, -latex, line width=3pt, gray},
}

% ---------- Math operators ----------
\DeclareMathOperator*{\argmax}{arg\,max}
\DeclareMathOperator*{\argmin}{arg\,min}
\DeclareMathOperator{\Prob}{P}
\DeclareMathOperator{\E}{E}
\DeclareMathOperator{\Var}{Var}
\DeclareMathOperator{\sign}{sign}
\DeclareMathOperator{\clamp}{clamp}

% ---------- Hyperref ----------
\usepackage{hyperref}
\hypersetup{colorlinks, urlcolor=bookgray, linkcolor=bookdark}

% ---------- Commands ----------
\renewcommand{\vec}[1]{\mathbf{#1}}
\newcommand{\code}[1]{\colorbox{black!10!white}{\texttt{#1}}}

% ---------- Book info, license, and disclaimer ----------
\newcommand{\bookframe}{%
  \begin{frame}{About these slides}
    These slides are companion material for the book

    \vspace{0.3cm}
    \begin{center}
      \textbf{Data Science Project: An Inductive Learning Approach}\\[2pt]
      Prof.~Dr.~Filipe A. N. Verri\\[4pt]
      \url{https://leanpub.com/dsp}
    \end{center}

    \vspace{0.3cm}
    All intellectual content comes from the book and is not AI-generated.
    Slides were produced with the assistance of
    \href{https://claude.ai/code}{Claude Code}.

    \vspace{0.3cm}
    {\small
    Licensed under
    \href{https://creativecommons.org/licenses/by-nc/4.0/}{CC BY-NC 4.0}.
    You are free to modify and redistribute this work as long as you give
    proper credit and do not use it for commercial purposes.}
  \end{frame}%
}

% ---------- TikZ color presets (used by book figures) ----------
\colorlet{circle edge}{black!50}
\colorlet{circle area}{black!20}
\tikzset{
  filled/.style={fill=circle area, draw=circle edge, thick},
  outline/.style={draw=circle edge, thick},
}


\title{Mathematical Foundations}
\subtitle{Data Science Project: An Inductive Learning Approach}
\author{Prof.~Dr.~Filipe A. N. Verri}
\date{}

\begin{document}

\maketitle
\bookframe

% ---- Epigraph ----

\begin{frame}{}
  \vfill
  \begin{quote}
    Maar ik maak steeds wat ik nog niet kan om het te leeren kunnen.
    \begin{flushright}
      --- Vincent van Gogh, The Complete Letters, Volume Three
    \end{flushright}
  \end{quote}
  \vfill
\end{frame}

% ---- Overview ----

\begin{frame}{Overview}
  \begin{columns}[T]
    \begin{column}{0.48\textwidth}
      \textbf{Contents}
      \begin{itemize}
        \item Algorithms and data structures
        \item Set theory
        \item Linear algebra
        \item Probability
      \end{itemize}
    \end{column}
    \begin{column}{0.48\textwidth}
      \textbf{Objectives}
      \begin{itemize}
        \item Consolidate notations and definitions used throughout the book
        \item Remind the reader of the main computational, mathematical, and
          statistical concepts
      \end{itemize}
    \end{column}
  \end{columns}
\end{frame}

% ===========================================================================
\section{Algorithms and data structures}
% ===========================================================================

% ---------------------------------------------------------------------------
\subsection{Computational complexity}
% ---------------------------------------------------------------------------

\begin{frame}{Computational complexity}
  \begin{itemize}
    \item Total amount of resources (time, space) as a function of input size
    \item We focus on \textbf{asymptotic} complexity --- behavior as input grows
  \end{itemize}

  \vspace{0.3cm}
  \textbf{Big-O notation:} $f$ is $O(g)$ if $\exists\, c > 0$ such that
  $f(n) \leq c\, g(n)$ for all $n \geq n_0$.

  \vspace{0.3cm}
  Common complexity classes:
  \[
    O(1) < O(\log n) < O(n) < O(n \log n) < O(n^2) < O(2^n) < O(n!)
  \]
\end{frame}

\begin{frame}{Big-O properties}
  \begin{itemize}
    \item Worst-case analysis: upper bound on resources for any input of
      size $n$
    \item An $O(n)$ algorithm is not always faster than an $O(n^2)$ algorithm
      --- only for large enough $n$
    \item Sequential composition:
      \[
        O(f) + O(g) = O(\max(f, g))
      \]
  \end{itemize}
\end{frame}

% ---------------------------------------------------------------------------
\subsection{Algorithmic paradigms}
% ---------------------------------------------------------------------------

\begin{frame}{Divide and conquer}
  \begin{enumerate}
    \item Divide the problem into smaller subproblems
    \item Solve each subproblem recursively
    \item Combine the solutions
  \end{enumerate}

  \vspace{0.3cm}
  Examples: merge sort, quick sort, binary search.
\end{frame}

\begin{frame}{Binary search}
  Given a sorted array $\vec{a} = [a_1, a_2, \ldots, a_n]$ and key $x$:

  \vspace{0.2cm}
  \begin{enumerate}
    \item Set $l \gets 1$, $r \gets n$
    \item While $l \leq r$:
      \begin{itemize}
        \item $m \gets \lfloor (l + r) / 2 \rfloor$
        \item If $x = a_m$: return \textit{true}
        \item If $x < a_m$: $r \gets m - 1$; else $l \gets m + 1$
      \end{itemize}
    \item Return \textit{false}
  \end{enumerate}

  \vspace{0.2cm}
  Search space halved at each step:
  $\displaystyle\frac{n}{2^{i-1}} = 1 \implies i = 1 + \log n$

  \vspace{0.2cm}
  Time complexity: $O(\log n)$.
\end{frame}

\begin{frame}{Greedy algorithms}
  \begin{itemize}
    \item Solved with incremental, locally optimal steps
    \item Overall solution is \textbf{not} guaranteed to be optimal
    \item Examples: Dijkstra's algorithm, Prim's algorithm
  \end{itemize}

  \vspace{0.3cm}
  \textbf{Knapsack problem:}
  \[
    \text{maximize } \sum_{i=1}^n v_i x_i \quad
    \text{subject to } \sum_{i=1}^n w_i x_i \leq W
  \]
  Greedy heuristic: sort by value, add if it fits.\\
  Time complexity: $O(n \log n)$ (dominated by sorting).
\end{frame}

\begin{frame}{Brute force and backtracking}
  \textbf{Brute force:}
  \begin{itemize}
    \item Try all possible solutions
    \item Guaranteed optimal but usually exponential: $O(2^n)$ for knapsack
  \end{itemize}

  \vspace{0.3cm}
  \textbf{Backtracking:}
  \begin{itemize}
    \item Solve incrementally; undo when a constraint is violated
    \item Special case of brute force, often exponential
    \item Example: Sudoku --- $n^m$ possible fillings for $m$ empty cells
  \end{itemize}
\end{frame}

% ---- Figure: Sudoku backtracking ----

\begin{frame}{Backtracking --- Sudoku example}
  \centering
  \resizebox{0.65\textwidth}{!}{
  \begin{tikzpicture}
    \node (root) at (1, 0) {$\begin{array}{|c|c|c|c|}
      \hline
      ? &   &   & 1 \\
      \hline
      1 & 2 & 4 &   \\
      \hline
        & 4 & 1 & 2 \\
      \hline
      2 &   &   &   \\
      \hline
      \end{array}$};
    \node (n1) at (-3.6, -3) {$\begin{array}{|c|c|c|c|}
      \hline
      \color{gray} 1 &   &   & \color{gray} 1 \\
      \hline
      \color{gray} 1 & 2 & 4 &   \\
      \hline
        & 4 & 1 & 2 \\
      \hline
      2 &   &   &   \\
      \hline
      \end{array}$};
    \node (n2) at (-1.2, -3) {$\begin{array}{|c|c|c|c|}
      \hline
      \color{gray} 2 &   &   & 1 \\
      \hline
      1 & \color{gray} 2 & 4 &   \\
      \hline
        & 4 & 1 & 2 \\
      \hline
      \color{gray} 2 &   &   &   \\
      \hline
      \end{array}$};
    \node (n3) at (1.2, -5) {$\begin{array}{|c|c|c|c|}
      \hline
      3 & ? &   & 1 \\
      \hline
      1 & 2 & 4 &   \\
      \hline
        & 4 & 1 & 2 \\
      \hline
      2 &   &   &   \\
      \hline
      \end{array}$};
    \node (n4) at (3.6, -3) {$\begin{array}{|c|c|c|c|}
      \hline
      4 & ? &   & 1 \\
      \hline
      1 & 2 & 4 &   \\
      \hline
        & 4 & 1 & 2 \\
      \hline
      2 &   &   &   \\
      \hline
      \end{array}$};
    \node (dots) at (3.6, -5) {\dots};
    \draw[-Stealth] (root.west) -- (n1.north);
    \draw[-Stealth] (n1.north) -- (root.south west);
    \draw[-Stealth] (root.south west) -- (n2.north);
    \draw[-Stealth] (n2.north) -- (root.south);
    \draw[-Stealth] (root.south) -- (n3.north west);
    \draw[-Stealth] (n3.north east) -- (root.south east);
    \draw[-Stealth] (root.south east) -- (n4.north);
    \draw[-Stealth] (n4.south) -- (dots);
  \end{tikzpicture}
  }

  \small
  Constraint violations shown in gray cause backtracking.
\end{frame}

% ---------------------------------------------------------------------------
\subsection{Data structures}
% ---------------------------------------------------------------------------

\begin{frame}{Data structures}
  \begin{itemize}
    \item \textbf{Array}: homogeneous collection accessed by index;
      $\vec{a} = [a_1, a_2, \ldots, a_n]$
    \item \textbf{Stack}: LIFO; push and pop from top only
    \item \textbf{Queue}: FIFO; enqueue at back, dequeue from front
    \item \textbf{Tree}: nodes with children, no cycles; root, leaves
    \item \textbf{Graph}: nodes with edges; directed or undirected
  \end{itemize}
\end{frame}

\begin{frame}{Binary tree}
  Recursive definition:
  \[
    T = \begin{cases}
      \emptyset & \text{if empty} \\
      (v, T_l, T_r) & \text{value } v \text{ with left } T_l \text{ and right } T_r
    \end{cases}
  \]

  \vspace{0.3cm}
  Parentheses notation:
  $(1, (2, \emptyset, \emptyset), (3, \emptyset, \emptyset))$\\
  is a tree with root $1$, left child $2$, right child $3$.
\end{frame}

\begin{frame}{Graphs}
  A graph $G = (V, E)$ where $V$ = vertices, $E \subseteq V \times V$ = edges.

  \begin{columns}[T]
    \begin{column}{0.4\textwidth}
      \centering
      \begin{tikzpicture}
        \node[draw, circle] (v1) at (0, 0) {1};
        \node[draw, circle] (v2) at (2, 0) {2};
        \node[draw, circle] (v3) at (2, 2) {3};
        \node[draw, circle] (v4) at (0, 2) {4};
        \draw[->] (v1) -- (v2);
        \draw[->] (v2) -- (v3);
        \draw[->] (v3) -- (v4);
        \draw[->] (v4) -- (v1);
        \draw[->] (v1) -- (v3);
      \end{tikzpicture}
    \end{column}
    \begin{column}{0.55\textwidth}
      Adjacency matrix:
      \[
        A = \begin{pmatrix}
          0 & 1 & 1 & 0 \\
          0 & 0 & 1 & 0 \\
          0 & 0 & 0 & 1 \\
          1 & 0 & 0 & 0
        \end{pmatrix}
      \]
    \end{column}
  \end{columns}

  \vspace{0.3cm}
  \small
  Weighted graphs: $w : E \to \mathbb{R}$ assigns a weight to each edge.
\end{frame}

% ===========================================================================
\section{Set theory}
% ===========================================================================

\begin{frame}{Sets}
  A set is an unordered collection of unique elements: $\{1, 2, 3\}$.

  \vspace{0.3cm}
  Special sets:
  \begin{itemize}
    \item \textbf{Universe set} $\Omega$: all elements in a given context
    \item \textbf{Empty set} $\emptyset$: no elements
  \end{itemize}
\end{frame}

% ---------------------------------------------------------------------------
\subsection{Set operations}
% ---------------------------------------------------------------------------

\begin{frame}{Set operations}
  \begin{itemize}
    \item \textbf{Union}: $A \cup B$ --- elements in $A$ or $B$
    \item \textbf{Intersection}: $A \cap B$ --- elements in both $A$ and $B$
    \item \textbf{Difference}: $A \setminus B$ --- elements in $A$ but not $B$
    \item \textbf{Complement}: $A^c = \Omega \setminus A$
    \item \textbf{Inclusion}: $A \subseteq B$ --- all elements of $A$ are in $B$
  \end{itemize}
\end{frame}

\begin{frame}{Set operations properties}
  Given sets $A$, $B$, $C$:
  \begin{itemize}
    \item \textit{Commutativity:} $A \cup B = B \cup A$,\quad $A \cap B = B \cap A$
    \item \textit{Associativity:}
      $(A \cup B) \cup C = A \cup (B \cup C)$
    \item \textit{Distributivity:}
      $A \cup (B \cap C) = (A \cup B) \cap (A \cup C)$
  \end{itemize}

  \vspace{0.3cm}
  \textbf{De Morgan's laws:}
  \[
    (A \cup B)^c = A^c \cap B^c
    \qquad
    (A \cap B)^c = A^c \cup B^c
  \]

  \vspace{0.2cm}
  Difference in terms of complement:
  $A \setminus B = A \cap B^c$
\end{frame}

\begin{frame}{Inclusion properties}
  \begin{itemize}
    \item \textit{Reflexivity:} $A \subseteq A$
    \item \textit{Antisymmetry:} $A \subseteq B$ and $B \subseteq A$ iff $A = B$
    \item \textit{Transitivity:} $A \subseteq B$ and $B \subseteq C$ implies
      $A \subseteq C$
  \end{itemize}
\end{frame}

% ---------------------------------------------------------------------------
\subsection{Boolean algebra}
% ---------------------------------------------------------------------------

\begin{frame}{Relation to Boolean algebra}
  \centering
  \rowcolors{2}{black!10!white}{}
  \begin{tabular}{lcc}
    \toprule
    \textbf{Operation} & \textbf{Set} & \textbf{Boolean} \\
    \midrule
    Union / OR & $A \cup B$ & $A \lor B$ \\
    Intersection / AND & $A \cap B$ & $A \land B$ \\
    Complement / NOT & $A^c$ & $\lnot A$ \\
    \bottomrule
  \end{tabular}

  \vspace{0.5cm}
  \raggedright
  De Morgan's laws also hold:
  $\lnot(A \lor B) = \lnot A \land \lnot B$

  \vspace{0.2cm}
  Boolean algebra is the foundation of digital electronics and
  programming control flow.
\end{frame}

% ===========================================================================
\section{Linear algebra}
% ===========================================================================

\begin{frame}{Basic objects}
  \begin{itemize}
    \item \textbf{Vector}: ordered collection of numbers;
      $\vec{v} = [v_i]_{i=1,\ldots,n}$
    \item \textbf{Matrix}: rectangular array;
      $A = (a_{ij})_{i=1,\ldots,n;\; j=1,\ldots,m}$
    \item \textbf{Tensor}: generalization to $k$ indices (rank $k$)
      \begin{itemize}
        \item Scalar: rank 0; Vector: rank 1; Matrix: rank 2
      \end{itemize}
  \end{itemize}
\end{frame}

% ---------------------------------------------------------------------------
\subsection{Operations}
% ---------------------------------------------------------------------------

\begin{frame}{Addition and scalar multiplication}
  \textbf{Addition:}
  \begin{itemize}
    \item Vectors: $(\vec{v} + \vec{w})_i = v_i + w_i$
    \item Matrices: $(A + B)_{ij} = a_{ij} + b_{ij}$
  \end{itemize}

  \vspace{0.3cm}
  \textbf{Scalar multiplication:}
  \begin{itemize}
    \item $(\alpha\vec{v})_i = \alpha v_i$
    \item $(\alpha A)_{ij} = \alpha a_{ij}$
  \end{itemize}
\end{frame}

\begin{frame}{Dot product and matrix multiplication}
  \textbf{Dot product} (inner product):
  \[
    \vec{v} \cdot \vec{w} = \sum_{i=1}^n v_i w_i
  \]

  \vspace{0.3cm}
  \textbf{Matrix multiplication}: $C = AB$, where
  \[
    c_{ij} = \sum_{k=1}^n a_{ik}\, b_{kj}
  \]
  Columns of $A$ must equal rows of $B$.\\
  Vectors are column matrices unless stated otherwise.
\end{frame}

\begin{frame}{Transpose, determinant, inverse}
  \textbf{Transpose}: $(A^T)_{ij} = a_{ji}$

  \vspace{0.3cm}
  \textbf{Determinant}: $\det(A)$ --- measure of signed volume.
  \[
    \begin{vmatrix} a & b \\ c & d \end{vmatrix} = ad - bc
  \]
  $\det(A) \neq 0$ iff $A$ is invertible. $\det(AB) = \det(A)\det(B)$.

  \vspace{0.3cm}
  \textbf{Inverse}: $A A^{-1} = A^{-1} A = I_n$
  \[
    \begin{pmatrix} a & b \\ c & d \end{pmatrix}^{-1}
    = \frac{1}{ad - bc}
    \begin{pmatrix} d & -b \\ -c & a \end{pmatrix}
  \]
\end{frame}

\begin{frame}{Inverse --- general formula}
  \[
    A^{-1} = \frac{1}{\det(A)} \operatorname{adj}(A)
  \]
  where $\operatorname{adj}(A)$ is the transpose of the cofactor matrix.

  \vspace{0.3cm}
  Cofactor of entry $a_{ij}$: determinant of $A$ with row $i$ and
  column $j$ removed, times $(-1)^{i+j}$.
\end{frame}

% ---------------------------------------------------------------------------
\subsection{Systems and eigenvalues}
% ---------------------------------------------------------------------------

\begin{frame}{Systems of linear equations}
  \[
    A\vec{x} = \vec{b}
  \]
  \begin{itemize}
    \item $A$: matrix of constants
    \item $\vec{x}$: vector of unknowns
    \item $\vec{b}$: vector of constants
  \end{itemize}

  \vspace{0.3cm}
  Unique solution iff $A$ is invertible: $\vec{x} = A^{-1}\vec{b}$.
\end{frame}

\begin{frame}{Eigenvalues and eigenvectors}
  An eigenvalue $\lambda$ of $A$ satisfies:
  \[
    A\vec{v} = \lambda\vec{v}
  \]
  for some non-zero eigenvector $\vec{v}$.

  \vspace{0.3cm}
  Eigenvalues are roots of the \textbf{characteristic polynomial}:
  \[
    \det(A - \lambda I_n) = 0
  \]
\end{frame}

% ===========================================================================
\section{Probability}
% ===========================================================================

% ---------------------------------------------------------------------------
\subsection{Axioms and main concepts}
% ---------------------------------------------------------------------------

\begin{frame}{Kolmogorov axioms}
  \begin{enumerate}
    \item $\Prob(A) \geq 0$ for any event $A$
    \item $\Prob(\Omega) = 1$ where $\Omega$ is the sample space
    \item If $A \cap B = \emptyset$:
      $\Prob(A \cup B) = \Prob(A) + \Prob(B)$
  \end{enumerate}

  \vspace{0.3cm}
  \textbf{Sum rule} (non-disjoint):
  \[
    \Prob(A \cup B) = \Prob(A) + \Prob(B) - \Prob(A \cap B)
  \]
\end{frame}

\begin{frame}{Joint, conditional, independence}
  \textbf{Joint probability}:
  $\Prob(A, B) = \Prob(A \cap B)$

  \vspace{0.3cm}
  \textbf{Law of total probability}: if $B_1, \ldots, B_n$ partition
  $\Omega$:
  \[
    \Prob(A) = \sum_{i=1}^n \Prob(A, B_i)
  \]

  \vspace{0.3cm}
  \textbf{Conditional probability}: $\Prob(A \mid B)$

  \vspace{0.3cm}
  \textbf{Independence}: $\Prob(A \mid B) = \Prob(A)$, equivalently
  $\Prob(A, B) = \Prob(A) \cdot \Prob(B)$
\end{frame}

\begin{frame}{Bayes' rule}
  \[
    \Prob(A \mid B) = \frac{\Prob(B \mid A) \cdot \Prob(A)}{\Prob(B)}
  \]

  \vspace{0.3cm}
  One of the most important formulas in probability theory.\\
  Foundation of Bayesian statistics and machine learning.
\end{frame}

% ---------------------------------------------------------------------------
\subsection{Random variables}
% ---------------------------------------------------------------------------

\begin{frame}{Random variables}
  A random variable $X : \Omega \to E$ maps outcomes to values.
  \[
    \Prob(X \in A) = \Prob(\{\omega \in \Omega : X(\omega) \in A\})
  \]

  \vspace{0.3cm}
  \begin{itemize}
    \item $E = \mathbb{R}$: continuous random variable
    \item $E = \mathbb{Z}$: discrete random variable
    \item $X \sim P$: $X$ follows distribution $P$
  \end{itemize}
\end{frame}

\begin{frame}{PMF, PDF, CDF}
  \textbf{Probability mass function} (discrete):
  \[
    p_X(x) = \Prob(X = x)
  \]

  \vspace{0.2cm}
  \textbf{Probability density function} (continuous):
  \[
    \Prob(a \leq X \leq b) = \int_a^b f_X(x)\, dx
  \]

  \vspace{0.2cm}
  \textbf{Cumulative distribution function}:
  \[
    F_X(x) = \Prob(X \leq x)
  \]
\end{frame}

% ---------------------------------------------------------------------------
\subsection{Expectation and moments}
% ---------------------------------------------------------------------------

\begin{frame}{Expectation}
  \[
    \E[X] = \sum_x x \cdot p_X(x)
    \qquad \text{or} \qquad
    \E[X] = \int_{-\infty}^{\infty} x \cdot f_X(x)\, dx
  \]

  \vspace{0.3cm}
  \textbf{Properties} (linearity):
  \begin{align*}
    \E[cX] &= c\,\E[X] \\
    \E[X + c] &= \E[X] + c \\
    \E[X + Y] &= \E[X] + \E[Y]
  \end{align*}

  \vspace{0.2cm}
  General: $\E[g(X)] = \sum_x g(x) \cdot p_X(x)$
  or $\int g(x) \cdot f_X(x)\, dx$.
\end{frame}

\begin{frame}{Variance}
  \[
    \Var(X) = \E\!\left[(X - \E[X])^2\right]
  \]

  \vspace{0.3cm}
  Equivalent form:
  \[
    \Var(X) = \E[X^2] - \E[X]^2
  \]

  \vspace{0.2cm}
  Proof:
  \begin{align*}
    \Var(X) &= \E\!\left[X^2 - 2X\E[X] + \E[X]^2\right] \\
            &= \E[X^2] - 2\E[X]\E[X] + \E[X]^2 \\
            &= \E[X^2] - \E[X]^2
  \end{align*}
\end{frame}

\begin{frame}{Sample statistics}
  \textbf{Sample mean}: $\displaystyle\bar{X} = \frac{1}{n}\sum_{i=1}^n X_i$

  \vspace{0.3cm}
  \textbf{Law of large numbers}:
  $\displaystyle\lim_{n \to \infty} \frac{1}{n}\sum_{i=1}^n X_i = \E[X]$
  \quad (for i.i.d.\ $X_i \sim X$)

  \vspace{0.3cm}
  \textbf{Sample variance}:
  $\displaystyle S^2 = \frac{1}{n-1}\sum_{i=1}^n (X_i - \bar{X})^2$

  \vspace{0.2cm}
  \small Denominator $n-1$ corrects for bias (Bessel's correction).
\end{frame}

\begin{frame}{Higher moments}
  \textbf{$k$-th moment}: $\E[X^k]$

  \vspace{0.3cm}
  \textbf{Sample skewness} (3rd moment):
  \[
    \text{Skewness} = \frac{\frac{1}{n}\sum_{i=1}^n (X_i - \bar{X})^3}{S^3}
  \]
  Zero for symmetric; positive = right-skewed; negative = left-skewed.

  \vspace{0.3cm}
  \textbf{Sample kurtosis} (4th moment):
  \[
    \text{Kurtosis} = \frac{\frac{1}{n}\sum_{i=1}^n (X_i - \bar{X})^4}{S^4} - 3
  \]
  Positive = heavier tails than normal; negative = lighter tails.
\end{frame}

% ---------------------------------------------------------------------------
\subsection{Common distributions}
% ---------------------------------------------------------------------------

\begin{frame}{Bernoulli distribution}
  $X \sim \text{Bern}(p)$, two outcomes (success/failure).

  \vspace{0.3cm}
  \begin{itemize}
    \item $\E[X] = p$
    \item $\Var(X) = p(1 - p)$
  \end{itemize}
\end{frame}

\begin{frame}{Poisson distribution}
  $X \sim \text{Poisson}(\lambda)$, number of events in a fixed interval.

  \vspace{0.3cm}
  PMF:
  \[
    p_X(x) = \frac{e^{-\lambda}\, \lambda^x}{x!}
  \]

  \vspace{0.3cm}
  \begin{itemize}
    \item $\E[X] = \lambda$
    \item $\Var(X) = \lambda$
  \end{itemize}
\end{frame}

\begin{frame}{Normal distribution}
  $X \sim \mathcal{N}(\mu, \sigma^2)$, bell-shaped density.

  \vspace{0.3cm}
  PDF:
  \[
    f_X(x) = \frac{1}{\sqrt{2\pi\sigma^2}}
      \exp\!\left(-\frac{(x - \mu)^2}{2\sigma^2}\right)
  \]

  \vspace{0.3cm}
  \begin{itemize}
    \item $\E[X] = \mu$,\quad $\Var(X) = \sigma^2$
    \item Standard normal: $\mathcal{N}(0, 1)$
  \end{itemize}
\end{frame}

\begin{frame}{Central limit theorem}
  Given $X_1, \ldots, X_n$ i.i.d.\ with mean $\mu$ and finite variance
  $\sigma^2$:
  \[
    \sqrt{n}\,(\bar{X} - \mu) \;\sim\; \mathcal{N}(0, \sigma^2)
    \quad \text{as } n \to \infty
  \]

  \vspace{0.3cm}
  The sample mean is approximately normal for large $n$,
  with mean $\mu$ and variance $\sigma^2 / n$.

  \vspace{0.3cm}
  One of the most important results in probability and statistics.
\end{frame}

\begin{frame}{T distribution}
  $X \sim \mathcal{T}(\nu)$, bell-shaped, heavier tails than normal.

  \vspace{0.3cm}
  \begin{itemize}
    \item $\nu > 0$: degrees of freedom
    \item Location-scale generalization:
      $\mu + \sigma X \sim \mathcal{T}(\mu, \sigma^2, \nu)$
    \item Converges to normal:
      $\displaystyle\lim_{\nu \to \infty} \mathcal{T}(\nu) = \mathcal{N}(0, 1)$
  \end{itemize}
\end{frame}

\begin{frame}{Gamma distribution}
  $X \sim \text{Gamma}(\alpha, \beta)$, right-skewed density.

  \vspace{0.3cm}
  PDF:
  \[
    f_X(x) = \frac{\beta^\alpha\, x^{\alpha - 1}\, e^{-\beta x}}
      {\Gamma(\alpha)}
  \]
  where $\displaystyle\Gamma(\alpha) = \int_0^\infty t^{\alpha-1}\, e^{-t}\, dt$.

  \vspace{0.3cm}
  Commonly used as a \textbf{conjugate prior} in Bayesian analysis.
\end{frame}

% ---------------------------------------------------------------------------
\subsection{Combinatorics}
% ---------------------------------------------------------------------------

\begin{frame}{Permutations and combinations}
  \textbf{Factorial}:
  $n! = n \cdot (n-1) \cdots 2 \cdot 1$,\quad $0! = 1$

  \vspace{0.3cm}
  \textbf{Permutation}: number of arrangements of $n$ elements $= n!$

  \vspace{0.3cm}
  \textbf{Combination}: number of ways to choose $k$ from $n$:
  \[
    \binom{n}{k} = \frac{n!}{k!\,(n - k)!}
  \]
\end{frame}

% ---- End ----

\begin{frame}[standout]
  Questions?
\end{frame}

\end{document}
