% slides/history.tex - Chapter 1: A Brief History of Data Science

% slides/preamble.tex - Shared preamble for all slide decks
% Metropolis theme with grayscale style matching the book

\documentclass[aspectratio=169]{beamer}

\usetheme{metropolis}

% ---------- Colors (grayscale) ----------
\definecolor{bookdark}{gray}{0.2}
\definecolor{bookgray}{gray}{0.5}
\definecolor{booklight}{gray}{0.92}

\setbeamercolor{normal text}{fg=bookdark, bg=white}
\setbeamercolor{alerted text}{fg=bookdark}
\setbeamercolor{frametitle}{fg=white, bg=bookdark}
\setbeamercolor{title separator}{fg=bookgray}
\setbeamercolor{progress bar}{fg=bookgray, bg=booklight}
\setbeamercolor{block title}{fg=white, bg=bookgray}
\setbeamercolor{block body}{fg=bookdark, bg=booklight}
\setbeamercolor{block title alerted}{fg=white, bg=bookdark}
\setbeamercolor{block body alerted}{fg=bookdark, bg=booklight}
\setbeamercolor{block title example}{fg=white, bg=bookgray}
\setbeamercolor{block body example}{fg=bookdark, bg=booklight}

\setbeamertemplate{frame numbering}[fraction]

% ---------- Fonts (matching the book) ----------
\usepackage[T1]{fontenc}
\usepackage{fontspec}
\usepackage[warnings-off={mathtools-colon,mathtools-overbracket}]{unicode-math}

\setmathfont{STIXTwoMath}[
  Extension={.otf},
  Path={./fonts/},
  Scale=1]

\setsansfont{STIXTwoText}[
  Extension={.otf},
  Path={./fonts/},
  UprightFont={*-Regular},
  BoldFont={*-Bold},
  ItalicFont={*-Italic},
  BoldItalicFont={*-BoldItalic}]

\setmainfont{STIXTwoText}[
  Extension={.otf},
  Path={./fonts/},
  UprightFont={*-Regular},
  BoldFont={*-Bold},
  ItalicFont={*-Italic},
  BoldItalicFont={*-BoldItalic}]

\setmonofont{CourierPrime}[
  Extension={.ttf},
  Path={./fonts/},
  UprightFont={*-Regular},
  BoldFont={*-Bold},
  ItalicFont={*-Italic},
  BoldItalicFont={*-BoldItalic},
  Scale=0.9]

% ---------- Packages ----------
\usepackage{amsmath}
\usepackage{mathtools}
\usepackage{graphicx}
\usepackage{booktabs}

% ---------- TikZ ----------
\usepackage{tikz}
\usetikzlibrary{shapes, arrows.meta, positioning, shapes.geometric, fit}

\tikzset{%
  decision/.style={draw, diamond, text centered, minimum height=0.5cm, minimum width=1cm},
  block/.style={rectangle, draw, text width=6em, text centered, rounded corners, minimum height=3em},
  mediumblock/.style={rectangle, draw, text width=3em, text centered, rounded corners, minimum height=2em},
  darkblock/.style={block, fill=gray, text=white},
  smallblock/.style={rectangle, rounded corners, draw, font=\tiny, minimum height=1em, inner sep=2pt},
  smalldarkblock/.style={smallblock, fill=gray, text=white},
  darkcircle/.style={draw, circle, fill=gray, text centered, text=white},
  smallcircle/.style={draw, circle, text centered, font=\tiny},
  smalldarkcircle/.style={smallcircle, fill=gray, text=white},
  line/.style={draw, -latex},
  dline/.style={draw, latex-latex},
  bigarrow/.style={draw, -latex, line width=3pt, gray},
}

% ---------- Math operators ----------
\DeclareMathOperator*{\argmax}{arg\,max}
\DeclareMathOperator*{\argmin}{arg\,min}
\DeclareMathOperator{\Prob}{P}
\DeclareMathOperator{\E}{E}
\DeclareMathOperator{\Var}{Var}
\DeclareMathOperator{\sign}{sign}
\DeclareMathOperator{\clamp}{clamp}

% ---------- Hyperref ----------
\usepackage{hyperref}
\hypersetup{colorlinks, urlcolor=bookgray, linkcolor=bookdark}

% ---------- Commands ----------
\renewcommand{\vec}[1]{\mathbf{#1}}
\newcommand{\code}[1]{\colorbox{black!10!white}{\texttt{#1}}}

% ---------- Book info, license, and disclaimer ----------
\newcommand{\bookframe}{%
  \begin{frame}{About these slides}
    These slides are companion material for the book

    \vspace{0.3cm}
    \begin{center}
      \textbf{Data Science Project: An Inductive Learning Approach}\\[2pt]
      Prof.~Dr.~Filipe A. N. Verri\\[4pt]
      \url{https://leanpub.com/dsp}
    \end{center}

    \vspace{0.3cm}
    All intellectual content comes from the book and is not AI-generated.
    Slides were produced with the assistance of
    \href{https://claude.ai/code}{Claude Code}.

    \vspace{0.3cm}
    {\small
    Licensed under
    \href{https://creativecommons.org/licenses/by-nc/4.0/}{CC BY-NC 4.0}.
    You are free to modify and redistribute this work as long as you give
    proper credit and do not use it for commercial purposes.}
  \end{frame}%
}

% ---------- TikZ color presets (used by book figures) ----------
\colorlet{circle edge}{black!50}
\colorlet{circle area}{black!20}
\tikzset{
  filled/.style={fill=circle area, draw=circle edge, thick},
  outline/.style={draw=circle edge, thick},
}


\title{A Brief History of Data Science}
\subtitle{Data Science Project: An Inductive Learning Approach}
\author{Prof.~Dr.~Filipe A. N. Verri}
\date{}

\begin{document}

\maketitle
\bookframe

% ---- Epigraph ----

\begin{frame}{}
  \vfill
  \begin{quote}
    ``Begin at the beginning,'' the King said gravely, ``and go on till you
    come to the end: then stop.''
    \begin{flushright}
      --- Lewis Carroll, \textit{Alice in Wonderland}
    \end{flushright}
  \end{quote}
  \vfill
\end{frame}

% ---- Overview ----

\begin{frame}{Overview}
  \begin{columns}[T]
    \begin{column}{0.48\textwidth}
      \textbf{Contents}
      \begin{itemize}
        \item The term ``data science''
        \item Timeline and historical markers
      \end{itemize}
    \end{column}
    \begin{column}{0.48\textwidth}
      \textbf{Objectives}
      \begin{itemize}
        \item Understand the history of the term
        \item Recognize major milestones
        \item Identify important figures
      \end{itemize}
    \end{column}
  \end{columns}
\end{frame}

% ===========================================================================
\section{The term ``data science''}
% ===========================================================================

% ---- Peter Naur ----

\begin{frame}{Peter Naur (1928--2016)}
  \begin{itemize}
    \item Danish computer scientist and mathematician
    \item Coined ``data science'' in the 1960s\footfullcite{Naur1974}
    \item Suggested replacing ``computer science'' with \textbf{datalogy}
    \item Emphasis on data as a fundamental component
    \item Data science deals with data; meaning is delegated to other fields
  \end{itemize}
\end{frame}

% ---- Figure: Naur's view ----

\def\naurds{(0,0) circle (20mm)}
\def\naurcs{(0:5mm) circle (15mm)}
\def\naurde{(0:40mm) circle (15mm)}

\begin{frame}{Naur's view of data science}
  \centering
  \begin{tikzpicture}
    \begin{scope}
      \clip \naurds;
      \fill[filled] \naurcs;
    \end{scope}
    \draw[outline] \naurds node(ds) {};
    \draw[outline] \naurcs node {Computer science};
    \draw[outline] \naurde node {Domain expertise};
    \node[anchor=north,above] at (0,2) {Data science};
  \end{tikzpicture}

  \vspace{0.5cm}
  \small Data science studies techniques to deal with data,\\
  but delegates the meaning of data to other fields.
\end{frame}

% ---- William Cleveland ----

\begin{frame}{William Cleveland (born 1943)}
  \begin{itemize}
    \item American statistician
    \item Used ``data science'' in 2001 to name a new discipline\footfullcite{Cleveland2001}
    \item Plan to enlarge the major areas of statistics
    \item Data science = ``modern'' statistics + computing + domain expertise
    \item Credited with defining data science as used today
  \end{itemize}
\end{frame}

% ---- Figure: Cleveland's view ----

\def\clevelandds{(0,0) circle (20mm)}
\def\clevelandst{(0:-5mm) circle (15mm)}
\def\clevelandde {(2,1) circle (15mm)}
\def\clevelandcs {(2,-1) circle (15mm)}

\begin{frame}{Cleveland's view of data science}
  \centering
  \begin{tikzpicture}
    \begin{scope}
      \clip \clevelandds;
      \fill[filled] \clevelandst;
      \fill[filled] \clevelandde;
      \fill[filled] \clevelandcs;
    \end{scope}
    \draw[outline] \clevelandds node(ds) {};
    \draw[outline] \clevelandst node {Statistics};
    \draw[outline] \clevelandde node {Domain expertise};
    \draw[outline] \clevelandcs node {Computer science};
    \node[anchor=north,above] at (0,2) {Data science};
  \end{tikzpicture}

  \vspace{0.5cm}
  \small Statistics enlarged by computer science and domain expertise.
\end{frame}

% ---- Buzzword or a new science? ----

\begin{frame}{Buzzword or a new science?}
  \begin{itemize}
    \item No consensus on the definition of data science
    \item Most usages: rough reference to data-driven techniques or marketing
    \item Naur and Cleveland: enlarged scope of an existing science
    \item Counter-argument: social and economic demand supports a distinct science
    \item Many ``data scientist'' roles and degree programs emerging
  \end{itemize}
\end{frame}

% ===========================================================================
\section{Timeline and historical markers}
% ===========================================================================

\begin{frame}{Two timelines}
  \begin{itemize}
    \item \textbf{Data handling} --- sources, collection, organization, storage, transformation
    \item \textbf{Data analysis} --- knowledge extraction and learning from data
  \end{itemize}
\end{frame}

% ---------------------------------------------------------------------------
\subsection{Timeline of data handling}
% ---------------------------------------------------------------------------

% ---- Figure: Timeline of data handling ----

\begin{frame}{Timeline of the ages of data handling}
  \centering
  \begin{tikzpicture}[scale=1.3]
    \draw (0,0) -- (8,0);
    \foreach \x in {0,1,...,8} {
      \draw (0.9 * \x,-0.1) -- (0.9 * \x,0.1);
    }
    \foreach \x/\y/\z in {%
        0/Pre-digital/{3,800 BC -- 18th c.},
        2/Digital/{1890 -- 1960},
        4/Formal/{1970s},
        6/Integrated/{1980 -- 1990},
        8/Ubiquitous/{2000 --}} {
      \node[anchor=north] at (0.9 * \x,-0.1) {\footnotesize\y};
      \node[anchor=south] at (0.9 * \x,0.1) {\footnotesize\z};
    }
  \end{tikzpicture}
\end{frame}

% ---- Pre-digital age ----

\begin{frame}{Pre-digital age}
  \begin{itemize}
    \item Lebombo bone ($\sim$40,000 years old) --- earliest tally stick\footfullcite{Beaumont2013}
    \item First known census: 3,800 BC, Babylonian Empire\footfullcite{Grajalez2013}
    \item Sumerian archaic writing ($\sim$3,500 BC) --- recording transactions\footfullcite{Ifrah1998}
    \item Storage evolution: clay tablets, papyrus, codex, paper, printing press
    \item \textbf{Florence Nightingale} (1820--1910)
      \begin{itemize}
        \item Used statistics to influence public opinion
        \item Developed the polar area diagram
        \item Standardized healthcare data collection
      \end{itemize}
  \end{itemize}
\end{frame}

% ---- Digital age ----

\begin{frame}{Digital age}
  \begin{itemize}
    \item Punched cards --- earliest famous usage by Bouchon (1725)
    \item 1890 US Census --- first machine-readable punched cards
    \item \textbf{Herman Hollerith} (1860--1929) --- tabulating machine, later IBM
    \item ENIAC (1945) --- first electronic general-purpose computer
    \item UNIVAC I --- used for the 1950 US Census
    \item Digital computers: exponential growth in capture and storage
  \end{itemize}
\end{frame}

% ---- Formal age ----

\begin{frame}{Formal age}
  \begin{itemize}
    \item \textbf{Edgar F. Codd} (1923--2003) --- relational model (1970)
    \item Data organized in tables (relations)\footfullcite{Codd1970}; rows = records, columns = attributes
    \item Minimizes redundancy, improves integrity (normalization)
    \item Led to SQL (1974) --- standard language for relational databases
    \item New challenge: aggregating data from different sources
  \end{itemize}
\end{frame}

% ---- Integrated age ----

\begin{frame}{Integrated age}
  \begin{itemize}
    \item ETL (Extract, Transform, Load) process
    \item Data warehousing concept (late 1980s, IBM)
    \item \textbf{Ralph Kimball} vs \textbf{Bill Inmon} --- bottom-up vs top-down
    \item Both agree: data warehouses are the foundation for BI and analytics
    \item Walmart case study (early 1990s) --- single source of truth
  \end{itemize}
\end{frame}

% ---- Ubiquitous age ----

\begin{frame}{Ubiquitous age --- Key figures}
  \begin{itemize}
    \item \textbf{Cerf \& Kahn} --- TCP/IP protocols
    \item \textbf{Tim Berners-Lee} --- World Wide Web
    \item \textbf{Jobs, Wozniak, Gates} --- personal computers
    \item \textbf{Page \& Brin} --- Google
    \item \textbf{Zuckerberg} --- Facebook and social media
  \end{itemize}
\end{frame}

\begin{frame}{Ubiquitous age --- Technologies}
  \begin{itemize}
    \item NoSQL databases --- unstructured, semi-structured, structured data
    \item Five V's of big data: Volume, Velocity, Variety, Veracity, Value
    \item \textbf{Cutting \& Cafarella} --- Apache Hadoop (2006), HDFS, MapReduce\footfullcite{Dean2008}
    \item Distributed computing frameworks (Hadoop, Spark)
    \item IoT --- proliferation of data sources
  \end{itemize}
\end{frame}

% ---------------------------------------------------------------------------
\subsection{Timeline of data analysis}
% ---------------------------------------------------------------------------

% ---- Figure: Timeline of data analysis ----

\begin{frame}{Timeline of the ages of data analysis}
  \centering
  \begin{tikzpicture}[scale=1.3]
    \draw (0,0) -- (8,0);
    \foreach \x in {0,1,...,8} {
      \draw (\x,-0.1) -- (\x,0.1);
    }
    \foreach \x/\y/\z in {%
        0/Summary statistics/{3,800 BC -- 16th c.},
        4/Probability advent/{17th c. -- 19th c.},
        7/Learning from data/{20th c. -- present}} {
      \node[anchor=north] at (\x,-0.1) {\footnotesize\y};
      \node[anchor=south] at (\x,0.1) {\footnotesize\z};
    }
  \end{tikzpicture}
\end{frame}

% ---- Summary statistics ----

\begin{frame}{Summary statistics}
  \begin{itemize}
    \item Earliest form of statistical analysis
    \item Term ``statistics'' refers to analysis of data \textit{about the state}
    \item Central tendencies (e.g., arithmetic mean)
    \item Variability (e.g., range)
  \end{itemize}
\end{frame}

% ---- Probability advent ----

\begin{frame}{Probability advent}
  Foundations of modern probability theory (17th century):
  \begin{itemize}
    \item Blaise Pascal (1623--1662)
    \item Pierre de Fermat (1601--1665)
    \item Christiaan Huygens (1629--1695)
    \item Jacob Bernoulli (1655--1705)
  \end{itemize}

  \vspace{0.5cm}
  Led to the field of \textbf{statistical inference}.
\end{frame}

\begin{frame}{Bayes' rule}
  \begin{itemize}
    \item \textbf{Thomas Bayes} (1701--1761)
    \item Calculates conditional probabilities from evidence
    \item Foundation of \textbf{learning from evidence}\footfullcite{Bayes1763}
    \item Na\"ive Bayes classifiers likely used since the 18th century
  \end{itemize}
\end{frame}

\begin{frame}{Gauss' method of least squares}
  \begin{itemize}
    \item \textbf{Carl Friedrich Gauss} (1777--1855)
    \item Developed circa 1794 for calculating the orbit of Ceres\footfullcite{Gauss1809}
    \item Beginning of \textbf{regression analysis}
    \item Shift: solving overdetermined systems using data
  \end{itemize}
\end{frame}

\begin{frame}{Playfair's data visualization}
  \begin{itemize}
    \item \textbf{William Playfair} (1759--1823)
    \item Invented line, area, bar chart, pie chart, circle graph\footfullcite{Playfair1786}
    \item Graphical representation of economic data
    \item Changed how we communicate data insights
  \end{itemize}
\end{frame}

% ---- Learning from data ----

\begin{frame}{Learning from data}
  \begin{itemize}
    \item 20th century onward
    \item Development of \textbf{learning machines}
    \item Shift from fitting theoretical models to extracting knowledge from data
    \item Algorithms that learn with minimal human intervention
    \item Fueled by advances in computation and data storage
  \end{itemize}
\end{frame}

\begin{frame}{Fisher's discriminant analysis (1930s)}
  \begin{itemize}
    \item \textbf{Sir Ronald A. Fisher} (1890--1962)
    \item Linear functions to separate classes of objects\footfullcite{Fisher1936}
    \item Foundation of classification and dimensionality reduction
    \item Highlighted the importance of \textbf{feature selection}
  \end{itemize}
\end{frame}

\begin{frame}{Shannon's information theory (1940s)}
  \begin{itemize}
    \item \textbf{Claude Shannon} (1916--2001)
    \item Quantification, storage, and communication of information\footfullcite{Shannon1948}
    \item Key concepts: entropy, mutual information, information gain
    \item Data as a sequence of symbols that can be compressed
    \item Foundation of several machine learning algorithms
  \end{itemize}
\end{frame}

\begin{frame}{K-Nearest Neighbors (1951)}
  \begin{itemize}
    \item \textbf{Evelyn Fix} (1904--1965) and \textbf{Joseph Hodges Jr.} (1922--2000)
    \item Non-parametric method for classification and regression\footfullcite{Fix1951}
    \item Shift from parametric to \textbf{non-parametric methods}
    \item Intuitive: similar objects likely belong to the same class
  \end{itemize}
\end{frame}

\begin{frame}{Rosenblatt's perceptron (1960s)}
  \begin{itemize}
    \item \textbf{Frank Rosenblatt} (1928--1971), psychologist
    \item First model of a \textbf{learning machine}\footfullcite{Rosenblatt1958}
    \item Could learn simple tasks (AND, OR)
    \item Foundation of artificial neural networks
    \item Minsky \& Papert (1969): limited to linearly separable problems
    \item Contributed to the first AI winter
  \end{itemize}
\end{frame}

\begin{frame}{Hunt's inducing trees (1966)}
  \begin{itemize}
    \item \textbf{Earl Hunt} --- inducing decision trees from data
    \item Based on information entropy
    \item Precursor of Quinlan's ID3 algorithm\footfullcite{Hunt1966,Quinlan1986}
    \item Intuitive, interpretable, symbolic rules
  \end{itemize}
\end{frame}

\begin{frame}{Tukey's exploratory data analysis (1977)}
  \begin{itemize}
    \item \textbf{John W. Tukey} (1915--2000)
    \item Countered dominance of confirmatory statistical methods
    \item Advocated informal study of data before formal modeling
    \item Introduced box plot, stem-and-leaf display
    \item Philosophy: data analysis should include open-ended exploration\footfullcite{Tukey1977}
  \end{itemize}
\end{frame}

\begin{frame}{Empirical risk minimization (1960s--1980s)}
  \begin{itemize}
    \item \textbf{Vladimir Vapnik} (born 1936) and \textbf{Alexey Chervonenkis} (1938--2014)
    \item VC entropy and VC dimension (1968)\footfullcite{Vapnik1968}
    \item ERM principle --- foundation of statistical learning theory
    \item Theoretical framework for understanding learning from data
  \end{itemize}
\end{frame}

\begin{frame}{Resurgence of neural networks (1986)}
  \begin{itemize}
    \item \textbf{Backpropagation} developed independently by several researchers\footfullcite{LeCun1986,Rumelhart1986}
    \item Training of multi-layer neural networks
    \item Solves nonlinearly separable problems
    \item Fueled by availability of data and computational power
    \item Preference for simple algorithms and intuitive models
  \end{itemize}
\end{frame}

\begin{frame}{Ensembles (1990s)}
  \begin{itemize}
    \item Combine multiple learning machines to improve performance
    \item \textbf{Boosting} --- sequential, correcting previous errors\footfullcite{Schapire1990}
    \item \textbf{Bagging} --- parallel, trained with small data variations\footfullcite{Breiman1996}
    \item Random forests (Ho, 1995) --- bagging\footfullcite{Ho1995}
    \item XGBoost (Friedman, 2001) --- gradient boosting\footfullcite{Friedman2001}
  \end{itemize}
\end{frame}

\begin{frame}{Support vector machines (1995)}
  \begin{itemize}
    \item \textbf{Cortes \& Vapnik}\footfullcite{Cortes1995} --- based on VC theory and ERM principle
    \item Finds optimal separating hyperplane with maximum margins
    \item Uses Cover's theorem\footfullcite{Cover1965} (high-dimensional mapping)
    \item Practical and efficient learning machines
  \end{itemize}
\end{frame}

\begin{frame}{Deep learning (late 2000s)}
  \begin{itemize}
    \item Neural networks with multiple layers
    \item State-of-the-art in computer vision and NLP
    \item \textbf{Bengio, Hinton, LeCun} --- 2018 Turing Award\footfullcite{Goodfellow2016}
  \end{itemize}
\end{frame}

\begin{frame}{LUSI learning theory (2010s)}
  \begin{itemize}
    \item \textbf{Vapnik}\footfullcite{Vapnik2015} --- Learning Using Statistical Invariants
    \item Extension of statistical learning theory
    \item Based on properties preserved under transformations
    \item ``Complete statistical theory of learning''
  \end{itemize}
\end{frame}

\begin{frame}{Large Language Models (2017--)}
  \begin{itemize}
    \item \textbf{Vaswani et al.}\footfullcite{Vaswani2017} (2017) --- transformer architecture
    \item Attention mechanisms replace recurrence and convolution
    \item Enabled development of LLMs
    \item OpenAI's ChatGPT (2022) --- wide public availability
    \item Capable of classification and reasoning without task-specific training
  \end{itemize}
\end{frame}

% ---- Takeaways ----

\begin{frame}{Takeaways}
  \begin{itemize}
    \item We have evolved both in theory and application of data-driven sciences
    \item No consensus on the definition of data science
    \item Sufficient evidence to support data science as a distinct science
  \end{itemize}
\end{frame}

% ---- End ----

\begin{frame}[standout]
  Questions?
\end{frame}

\end{document}
