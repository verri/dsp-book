% slides/structured-data.tex - Chapter 4: Structured Data

% slides/preamble.tex - Shared preamble for all slide decks
% Metropolis theme with grayscale style matching the book

\documentclass[aspectratio=169]{beamer}

\usetheme{metropolis}

% ---------- Colors (grayscale) ----------
\definecolor{bookdark}{gray}{0.2}
\definecolor{bookgray}{gray}{0.5}
\definecolor{booklight}{gray}{0.92}

\setbeamercolor{normal text}{fg=bookdark, bg=white}
\setbeamercolor{alerted text}{fg=bookdark}
\setbeamercolor{frametitle}{fg=white, bg=bookdark}
\setbeamercolor{title separator}{fg=bookgray}
\setbeamercolor{progress bar}{fg=bookgray, bg=booklight}
\setbeamercolor{block title}{fg=white, bg=bookgray}
\setbeamercolor{block body}{fg=bookdark, bg=booklight}
\setbeamercolor{block title alerted}{fg=white, bg=bookdark}
\setbeamercolor{block body alerted}{fg=bookdark, bg=booklight}
\setbeamercolor{block title example}{fg=white, bg=bookgray}
\setbeamercolor{block body example}{fg=bookdark, bg=booklight}

\setbeamertemplate{frame numbering}[fraction]

% ---------- Fonts (matching the book) ----------
\usepackage[T1]{fontenc}
\usepackage{fontspec}
\usepackage[warnings-off={mathtools-colon,mathtools-overbracket}]{unicode-math}

\setmathfont{STIXTwoMath}[
  Extension={.otf},
  Path={./fonts/},
  Scale=1]

\setsansfont{STIXTwoText}[
  Extension={.otf},
  Path={./fonts/},
  UprightFont={*-Regular},
  BoldFont={*-Bold},
  ItalicFont={*-Italic},
  BoldItalicFont={*-BoldItalic}]

\setmainfont{STIXTwoText}[
  Extension={.otf},
  Path={./fonts/},
  UprightFont={*-Regular},
  BoldFont={*-Bold},
  ItalicFont={*-Italic},
  BoldItalicFont={*-BoldItalic}]

\setmonofont{CourierPrime}[
  Extension={.ttf},
  Path={./fonts/},
  UprightFont={*-Regular},
  BoldFont={*-Bold},
  ItalicFont={*-Italic},
  BoldItalicFont={*-BoldItalic},
  Scale=0.9]

% ---------- Packages ----------
\usepackage{amsmath}
\usepackage{mathtools}
\usepackage{graphicx}
\usepackage{booktabs}

% ---------- TikZ ----------
\usepackage{tikz}
\usetikzlibrary{shapes, arrows.meta, positioning, shapes.geometric, fit}

\tikzset{%
  decision/.style={draw, diamond, text centered, minimum height=0.5cm, minimum width=1cm},
  block/.style={rectangle, draw, text width=6em, text centered, rounded corners, minimum height=3em},
  mediumblock/.style={rectangle, draw, text width=3em, text centered, rounded corners, minimum height=2em},
  darkblock/.style={block, fill=gray, text=white},
  smallblock/.style={rectangle, rounded corners, draw, font=\tiny, minimum height=1em, inner sep=2pt},
  smalldarkblock/.style={smallblock, fill=gray, text=white},
  darkcircle/.style={draw, circle, fill=gray, text centered, text=white},
  smallcircle/.style={draw, circle, text centered, font=\tiny},
  smalldarkcircle/.style={smallcircle, fill=gray, text=white},
  line/.style={draw, -latex},
  dline/.style={draw, latex-latex},
  bigarrow/.style={draw, -latex, line width=3pt, gray},
}

% ---------- Math operators ----------
\DeclareMathOperator*{\argmax}{arg\,max}
\DeclareMathOperator*{\argmin}{arg\,min}
\DeclareMathOperator{\Prob}{P}
\DeclareMathOperator{\E}{E}
\DeclareMathOperator{\Var}{Var}
\DeclareMathOperator{\sign}{sign}
\DeclareMathOperator{\clamp}{clamp}

% ---------- Hyperref ----------
\usepackage{hyperref}
\hypersetup{colorlinks, urlcolor=bookgray, linkcolor=bookdark}

% ---------- Commands ----------
\renewcommand{\vec}[1]{\mathbf{#1}}
\newcommand{\code}[1]{\colorbox{black!10!white}{\texttt{#1}}}

% ---------- Book info, license, and disclaimer ----------
\newcommand{\bookframe}{%
  \begin{frame}{About these slides}
    These slides are companion material for the book

    \vspace{0.3cm}
    \begin{center}
      \textbf{Data Science Project: An Inductive Learning Approach}\\[2pt]
      Prof.~Dr.~Filipe A. N. Verri\\[4pt]
      \url{https://leanpub.com/dsp}
    \end{center}

    \vspace{0.3cm}
    All intellectual content comes from the book and is not AI-generated.
    Slides were produced with the assistance of
    \href{https://claude.ai/code}{Claude Code}.

    \vspace{0.3cm}
    {\small
    Licensed under
    \href{https://creativecommons.org/licenses/by-nc/4.0/}{CC BY-NC 4.0}.
    You are free to modify and redistribute this work as long as you give
    proper credit and do not use it for commercial purposes.}
  \end{frame}%
}

% ---------- TikZ color presets (used by book figures) ----------
\colorlet{circle edge}{black!50}
\colorlet{circle area}{black!20}
\tikzset{
  filled/.style={fill=circle area, draw=circle edge, thick},
  outline/.style={draw=circle edge, thick},
}


\title{Structured Data}
\subtitle{Data Science Project: An Inductive Learning Approach}
\author{Prof.~Dr.~Filipe A. N. Verri}
\date{}

\begin{document}

\maketitle
\bookframe

% ---- Epigraph ----

\begin{frame}{}
  \vfill
  \begin{quote}
    Like families, tidy datasets are all alike, but every messy dataset is
    messy in its own way.
    \begin{flushright}
      --- Hadley Wickham, \textit{Tidy Data}
    \end{flushright}
  \end{quote}
  \vfill
\end{frame}

% ---- Overview ----

\begin{frame}{Overview}
  \begin{columns}[T]
    \begin{column}{0.48\textwidth}
      \textbf{Contents}
      \begin{itemize}
        \item Data types
        \item Database normalization
        \item Tidy data
        \item Bridging normalization and tidiness
        \item Data semantics and interpretation
      \end{itemize}
    \end{column}
    \begin{column}{0.48\textwidth}
      \textbf{Objectives}
      \begin{itemize}
        \item Understand common data types and formats
        \item Associate data format and semantics
        \item Enable the reader to perform data tasks well
      \end{itemize}
    \end{column}
  \end{columns}
\end{frame}

% ===========================================================================
\section{Data types}
% ===========================================================================

\begin{frame}{Stevens' types}
  \centering
  \rowcolors{2}{black!10!white}{}
  \begin{tabular}{ll}
    \toprule
    \textbf{Data type} & \textbf{Allowed operations} \\
    \midrule
    Nominal  & $=$ \\
    Ordinal  & $=, <$ \\
    Interval & $=, <, +, -$ \\
    Ratio    & $=, <, +, -, \times, \div$ \\
    \bottomrule
  \end{tabular}

  \vspace{0.5cm}
  \small
  \begin{tabular}{ll}
    Nominal:  & colors \\
    Ordinal:  & small $<$ medium $<$ large \\
    Interval: & temperature in Celsius \\
    Ratio:    & weight in kilograms \\
  \end{tabular}
\end{frame}

\begin{frame}{Limitations of Stevens' types}
  \begin{itemize}
    \item Do not exhaust all possibilities (e.g., probabilities)
    \item Data types are not always evident from the data alone
    \item Same data can be interpreted differently depending on context
    \item Data scientists must be aware of types and their consequences
    \item Good assumptions and hypotheses are a key part of the methodology
  \end{itemize}
\end{frame}

% ===========================================================================
\section{Database normalization}
% ===========================================================================

% ---------------------------------------------------------------------------
\subsection{Relational algebra}
% ---------------------------------------------------------------------------

\begin{frame}{Relational algebra --- Concepts}
  \begin{itemize}
    \item \textbf{Relation} --- table with rows (tuples) and columns (attributes)
    \item \textbf{Projection} $\pi_{A,C}(X)$ --- select only specified columns
    \item \textbf{Join} $S \bowtie T$ --- combine relations on common attributes
    \item \textbf{Functional dependency} $U \to V$ --- if tuples agree on $U$,
      they agree on $V$
    \item \textbf{Multi-valued dependency} $U \twoheadrightarrow V$ ---
      $R = R[UV] \bowtie R[UW]$
    \item \textbf{Join dependency} $*\{X_1, \ldots, X_n\}$ ---
      $R = \bowtie\{R[X_1], \ldots, R[X_n]\}$
  \end{itemize}
\end{frame}

% ---------------------------------------------------------------------------
\subsection{Normal forms}
% ---------------------------------------------------------------------------

\begin{frame}{Normal forms}
  Progressive conditions to reduce redundancy:
  \begin{itemize}
    \item \textbf{1NF} --- all attributes are atomic
    \item \textbf{2NF} --- 1NF + non-prime attributes fully depend on primary key
    \item \textbf{3NF} --- 2NF + no transitive dependencies
    \item \textbf{BCNF} --- every functional dependency is a result of keys
    \item \textbf{4NF} --- every multi-valued dependency is a result of keys
    \item \textbf{PJNF} --- key dependencies imply all join dependencies
  \end{itemize}
\end{frame}

\begin{frame}{Example: 2NF vs 3NF}
  2NF (redundant ``course credits''):

  \centering
  \small
  \rowcolors{2}{black!10!white}{}
  \begin{tabular}{cccc}
    \toprule
    \textbf{Student} & \textbf{Course} & \textbf{Credits} & \textbf{Grade} \\
    \midrule
    Alice & Math & 4 & A \\
    Alice & Physics & 3 & B \\
    Bob & Math & 4 & B \\
    Bob & Physics & 3 & A \\
    \bottomrule
  \end{tabular}

  \vspace{0.4cm}
  \raggedright
  3NF (separate tables):

  \centering
  \rowcolors{2}{black!10!white}{}
  \begin{tabular}{cc}
    \toprule
    \textbf{Course} & \textbf{Credits} \\
    \midrule
    Math & 4 \\
    Physics & 3 \\
    \bottomrule
  \end{tabular}
  \quad
  \rowcolors{2}{black!10!white}{}
  \begin{tabular}{ccc}
    \toprule
    \textbf{Student} & \textbf{Course} & \textbf{Grade} \\
    \midrule
    Alice & Math & A \\
    Alice & Physics & B \\
    Bob & Math & B \\
    Bob & Physics & A \\
    \bottomrule
  \end{tabular}
\end{frame}

\begin{frame}{PJNF and invalid joins}
  \begin{itemize}
    \item Relation $R[ABC]$ with primary key $ABC$ (no non-trivial FDs)
    \item Constraint: if $(a,b,c')$, $(a,b',c)$, $(a',b,c) \in R$
      then $(a,b,c) \in R$
    \item Join dependency $*\{AB, AC, BC\}$ not implied by key
    \item Must split into $R_1[AB]$, $R_2[AC]$, $R_3[BC]$
    \item Careless joins may produce invalid tuples
  \end{itemize}
\end{frame}

% ===========================================================================
\section{Tidy data}
% ===========================================================================

\begin{frame}{Tidy data}
  A standardized way to organize data values (Wickham, 2014):
  \begin{itemize}
    \item Each \textbf{value} belongs to a variable and an observation
    \item Each \textbf{variable} (column) = same attribute across units
    \item Each \textbf{observation} (row) = same unit across attributes
    \item \textbf{Observational unit} = individual entity being measured
  \end{itemize}

  \vspace{0.3cm}
  \centering
  \small
  \rowcolors{2}{black!10!white}{}
  \begin{tabular}{cccc}
    \toprule
    \textbf{Concept} & \textbf{Structure} & \textbf{Contains} & \textbf{Across} \\
    \midrule
    Variable & Column & Same attribute & Units \\
    Observation & Row & Same unit & Attributes \\
    \bottomrule
  \end{tabular}
\end{frame}

\begin{frame}{Messy vs tidy}
  \begin{columns}[T]
    \begin{column}{0.48\textwidth}
      \textbf{Messy}\\[0.2cm]
      \centering\small
      \rowcolors{2}{black!10!white}{}
      \begin{tabular}{lcc}
        \toprule
         & \textbf{2019} & \textbf{2020} \\
        \midrule
        Brazil & 100 & 200 \\
        USA &  & 400 \\
        \bottomrule
      \end{tabular}
    \end{column}
    \begin{column}{0.48\textwidth}
      \textbf{Tidy}\\[0.2cm]
      \centering\small
      \rowcolors{2}{black!10!white}{}
      \begin{tabular}{lcc}
        \toprule
        \textbf{Country} & \textbf{Year} & \textbf{Cases} \\
        \midrule
        Brazil & 2019 & 100 \\
        Brazil & 2020 & 200 \\
        USA & 2019 & \\
        USA & 2020 & 400 \\
        \bottomrule
      \end{tabular}
    \end{column}
  \end{columns}

  \vspace{0.5cm}
  Tidy data: variables and observations are clear from the table itself.
\end{frame}

% ---------------------------------------------------------------------------
\subsection{Common messy datasets}
% ---------------------------------------------------------------------------

\begin{frame}{Problem: headers are values}
  \begin{columns}[T]
    \begin{column}{0.48\textwidth}
      \textbf{Messy} (Pew Forum)\\[0.2cm]
      \centering\scriptsize
      \rowcolors{2}{black!10!white}{}
      \begin{tabular}{lrrr}
        \toprule
        \textbf{Religion} & \textbf{<\$10k} & \textbf{\$10-20k} & \textbf{\dots} \\
        \midrule
        Agnostic & 27 & 34 & \dots \\
        Atheist & 12 & 27 & \dots \\
        Buddhist & 27 & 21 & \dots \\
        \bottomrule
      \end{tabular}
    \end{column}
    \begin{column}{0.48\textwidth}
      \textbf{Tidy}\\[0.2cm]
      \centering\scriptsize
      \rowcolors{2}{black!10!white}{}
      \begin{tabular}{llr}
        \toprule
        \textbf{Religion} & \textbf{Income} & \textbf{Freq.} \\
        \midrule
        Agnostic & <\$10k & 27 \\
        Agnostic & \$10-20k & 34 \\
        \dots & \dots & \dots \\
        Atheist & <\$10k & 12 \\
        Atheist & \$10-20k & 27 \\
        \dots & \dots & \dots \\
        \bottomrule
      \end{tabular}
    \end{column}
  \end{columns}

  \vspace{0.3cm}
  Table becomes longer but narrower.
\end{frame}

\begin{frame}{Problem: multiple variables in one column}
  \begin{columns}[T]
    \begin{column}{0.48\textwidth}
      \textbf{Messy} (TB dataset)\\[0.2cm]
      \centering\scriptsize
      \rowcolors{2}{black!10!white}{}
      \begin{tabular}{lllr}
        \toprule
        \textbf{country} & \textbf{year} & \textbf{column} & \textbf{cases} \\
        \midrule
        AD & 2000 & m014 & 0 \\
        AD & 2000 & m1524 & 0 \\
        AD & 2000 & m2534 & 1 \\
        \dots & \dots & \dots & \dots \\
        \bottomrule
      \end{tabular}
    \end{column}
    \begin{column}{0.48\textwidth}
      \textbf{Tidy}\\[0.2cm]
      \centering\scriptsize
      \rowcolors{2}{black!10!white}{}
      \begin{tabular}{lllllr}
        \toprule
        \textbf{country} & \textbf{year} & \textbf{sex} & \textbf{age} & \textbf{cases} \\
        \midrule
        AD & 2000 & m & 0--14 & 0 \\
        AD & 2000 & m & 15--24 & 0 \\
        AD & 2000 & m & 25--34 & 1 \\
        \dots & \dots & \dots & \dots & \dots \\
        \bottomrule
      \end{tabular}
    \end{column}
  \end{columns}

  \vspace{0.3cm}
  Same number of rows, but wider.
\end{frame}

\begin{frame}{Problem: variables in both rows and columns}
  \begin{itemize}
    \item Most complicated case of messy data
    \item One column contains variable names (e.g., ``element'' = tmax/tmin)
    \item Day columns (d1, d2, \ldots) are values, not variable names
    \item Fix: lengthen first, then widen by variable names
  \end{itemize}
\end{frame}

\begin{frame}{Problem: multiple observational units in one table}
  \begin{itemize}
    \item Common during data collection
    \item Example: billboard data with track info repeated for each week
    \item Fix: separate into one table per observational unit
    \item Create unique identifiers to link the tables
  \end{itemize}
\end{frame}

\begin{frame}{Problem: single unit in multiple tables}
  \begin{itemize}
    \item Data split across files (e.g., one file per year)
    \item Table/file itself represents a variable value
    \item Fix: make columns compatible, combine, add origin column
  \end{itemize}
\end{frame}

% ===========================================================================
\section{Bridging normalization, tidiness, and data theory}
% ===========================================================================

\begin{frame}{Bridge between concepts}
  \centering
  \rowcolors{2}{black!10!white}{}
  \begin{tabular}{lll}
    \toprule
    \textbf{Relations} & \textbf{Tidy data} & \textbf{Philosophy} \\
    \midrule
    Entities & Observational units & Substance \\
    Tuple & Observation & Primary substance \\
    Primary key & Fixed variables & Univocal name \\
    Non-prime attr. & Measured variable & Predicate \\
    \bottomrule
  \end{tabular}

  \vspace{0.5cm}
  \raggedright
  The ontological understanding of data influences how it is organized.
\end{frame}

% ---------------------------------------------------------------------------
\subsection{Tidy or not tidy?}
% ---------------------------------------------------------------------------

\begin{frame}{Tidy or not tidy?}
  Temperature measured by 3 sensors, 3 times a day:

  \vspace{0.3cm}
  \begin{columns}[T]
    \begin{column}{0.48\textwidth}
      \textbf{Unit = measurement event}\\[0.2cm]
      \centering\scriptsize
      \rowcolors{2}{black!10!white}{}
      \begin{tabular}{llcc}
        \toprule
        \textbf{date} & \textbf{time} & \textbf{sensor} & \textbf{temp} \\
        \midrule
        01-01 & 00:00 & 1 & 20 \\
        01-01 & 00:00 & 2 & 21 \\
        01-01 & 00:00 & 3 & 22 \\
        \dots & \dots & \dots & \dots \\
        \bottomrule
      \end{tabular}
    \end{column}
    \begin{column}{0.48\textwidth}
      \textbf{Unit = time instant}\\[0.2cm]
      \centering\scriptsize
      \rowcolors{2}{black!10!white}{}
      \begin{tabular}{llccc}
        \toprule
        \textbf{date} & \textbf{time} & \textbf{t1} & \textbf{t2} & \textbf{t3} \\
        \midrule
        01-01 & 00:00 & 20 & 21 & 22 \\
        01-01 & 08:00 & 21 & 22 & 23 \\
        \dots & \dots & \dots & \dots & \dots \\
        \bottomrule
      \end{tabular}
    \end{column}
  \end{columns}

  \vspace{0.3cm}
  Both are tidy. Tidiness is a matter of \textbf{perspective}.
\end{frame}

% ---------------------------------------------------------------------------
\subsection{Change of observational unit}
% ---------------------------------------------------------------------------

\begin{frame}{Decomposition trees}
  $R[ABCDE]$ with $A \to D$, $B \to E$, $AB \to C$

  \vspace{0.3cm}
  \centering
  Valid decompositions to 3NF:

  \vspace{0.3cm}
  \begin{tikzpicture}
    \node (root1) at (0, 0) {ABCDE}
      child {node {AD}}
      child {node {ABCE}
        child {node {BE}}
        child {node {ABC}}};
    \node (root2) at (5, 0) {ABCDE}
      child {node {BE}}
      child {node {ABCD}
        child {node {AD}}
        child {node {ABC}}};
  \end{tikzpicture}
\end{frame}

\begin{frame}{Invalid decomposition trees}
  \centering
  \begin{tikzpicture}
    \node (root1) at (0, 0) {ABCDE}
      child {node {ABC}}
      child {node {ABDE}
        child {node {AD}}
        child {node {ABE}
          child {node {BE}}
          child {node[gray] {AB}}}};
    \node (root2) at (5, 0) {ABCDE}
      child {node {ABC}}
      child {node {ABDE}
        child {node {BE}}
        child {node {ABD}
          child {node {AD}}
          child {node[gray] {AB}}}};
  \end{tikzpicture}

  \vspace{0.3cm}
  \small $R[AB]$ is not a consequence of a functional dependency.
\end{frame}

\begin{frame}{Change of observational unit}
  \begin{itemize}
    \item Traverse decomposition tree from bottom to top with joins
    \item After each join, perform summarization on new observational unit
    \item Example: student enrollment $\to$ student summary (avg.\ grade, total load)
    \item Order of joins and summarization is crucial
    \item Not trivial to calculate all possible decomposition trees
  \end{itemize}
\end{frame}

% ===========================================================================
\section{Data semantics and interpretation}
% ===========================================================================

\begin{frame}{Data semantics and interpretation}
  \begin{itemize}
    \item Beyond functional dependencies: \textbf{statistical dependencies}
    \item Attributes may exist in an unknown $P(A, B)$
    \item Important to understand relationships between observations:
      \begin{itemize}
        \item Independent? Identically distributed? Selection bias?
        \item Temporal dependence? Hidden variables?
      \end{itemize}
    \item Wrong assumptions $\to$ wrong conclusions
  \end{itemize}
\end{frame}

% ===========================================================================
\section{Unstructured data}
% ===========================================================================

\begin{frame}{Unstructured data}
  \begin{itemize}
    \item No predefined data model (text, images, videos)
    \item Can be converted to structured data (e.g., bag-of-words)
    \item Conversion is not always straightforward or lossless
    \item Out of scope of this book
  \end{itemize}
\end{frame}

% ---- Takeaways ----

\begin{frame}{Takeaways}
  \begin{itemize}
    \item The choice of observational unit is not always straightforward
    \item Format and types must reflect what the solution will ``see'' in
      production
    \item Normalization (storage) and tidy data (analysis) are complementary
    \item Tidiness is a matter of perspective
  \end{itemize}
\end{frame}

% ---- End ----

\begin{frame}[standout]
  Questions?
\end{frame}

\end{document}
