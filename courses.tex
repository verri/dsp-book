\chapter{Course plan}

In the following, I present the course plan for PO-235 and CMC-16.

Any questions about the classes should be sent via Google Classroom.  If your question is
of general interest, please use the main stream.  If your question is personal and about
a specific assignment or grade, please use the private stream.

\newpage
\thispagestyle{empty}
\section*{PO-235 Data science project}

\emph{Course plan (\the\year{})}

Prof. Filipe A. N. Verri

\paragraph{Important:} Only graduate students are allowed to take this course.

\paragraph{Number of students:} Approx. 20

\paragraph{Course load:} 3--0--0--4

\paragraph{Requirements:} Advanced programming skills, strong statistics background, and
beginner level machine learning skills.

\paragraph{Course program:}
Brief history of data science.  Fundamental data concepts. Stages in a Data Science
project.  Data Infrastructure. Data integration from multiple sources. Data engineering
and shaping.  Inductive learning and principles of statistical learning theory.
Application of machine learning models in real-world problems.  Experimental planning for
data science. Model evaluation and Bayesian analysis.  Documentation and deployment.
Ethical and legal issues in data science.  Privacy-preserving computational approaches.

\paragraph{Goals:}
Providing the theoretical background and the practical concepts to develop an end-to-end
data science project for an inductive task.

\paragraph{Teaching methodology:}
Expository classes in common classroom, using whiteboard, slide presentations, coding
examples, books and scientific papers. Supplementary didactic materials will be available
in Google Classroom. The development of the case study will happen during home study
hours, including programming and scientific paper writing.  All classes will be given in
English.  Students are encouraged to ask questions in English, but Portuguese is also
allowed. All written and oral assignments must be in English.

\paragraph{Grading:} Two individual written tests in the \nth{1} ($T_1$ and $T_2$) and
another in the \nth{2} quarter ($T_3$).  Also, a group activity that includes writting a
scientific paper, developing a data science product, and a 30 minutes presentation ($L$).

Final grades will be calculated as
\begin{equation*}
  \text{\nth{1} Q} = \sqrt{T_1 T_2}\text{,} \qquad
  \text{\nth{2} Q} = \sqrt{T_3 L}\text{,} \qquad
  \text{Exam} = L\text{.}
\end{equation*}

\paragraph{Case study:} Exactly 6 groups will be formed.  Each group will be responsible for
a case study.  Students must choose a real-world problem and develop a data science
project, including data collection, data transformation, inductive learning, validation,
documentation, and deployment.  The results must be presented in a scientific paper
format and a 30 minutes presentation.  The trained models must be incorporated in a
data science product, such as a web application, a mobile application, or a web service.

\paragraph{Bibliography:}
\begin{itemize}
  \itemsep 0pt
  \item \fullcite{Zumel2019}.
  \item \fullcite{Wickham2023}.
  \item \fullcite{Kelleher2018}.
\end{itemize}

The first two books (\citeauthor{Zumel2019,Wickham2023}) are available online for free.

% \paragraph{Must read:}
% \begin{itemize}
%   \itemsep 0pt
%   \item In-progress textbook at \href{https://comp.ita.br/~verri/dsp-book}{comp.ita.br/~verri/dsp-book}.
%   \item \fullcite{Vapnik1999}.
%   \item \fullcite{Benavoli2017}.
% \end{itemize}

Any required extra material will be made available in Google Classroom.
\thispagestyle{empty}

\newpage
\paragraph{Calendar:} The expected schedule is presented below.
\thispagestyle{empty}

\begin{center}
  \begin{tabular}{ll}
    \toprule
    \multicolumn{2}{c}{\bfseries \nth{1} Quarter} \\
    \midrule
    Week & Topics \\
    \midrule
    \multirow{2}{*}{1} & Brief history of data science \pcref{chap:history} \\
      & Preliminaries \pcref{chap:preliminaries} \\
    \midrule
    2 & \bfseries Written test \\
    \midrule
    \multirow{2}{*}{3} & Fundamental data concepts \pcref{chap:data} \\
      & Stages in a data science project \\
    \midrule
    4 & \multirow{2}{*}{Inductive learning and statistical learning theory} \\
    5 &  \\
    \midrule
    6 & Data infrastructure and data integration from multiple sources \\
    \midrule
    7 & Data engineering and shaping \\
    \midrule
    8 & \bfseries Written test \\
    \bottomrule
  \end{tabular}
\end{center}

\begin{center}
  \begin{tabular}{ll}
    \toprule
    \multicolumn{2}{c}{\bfseries \nth{2} Quarter} \\
    \midrule
    Week & Topics \\
    \midrule
    1 & \multirow{2}{*}{Application of machine learning models in real-world problems} \\
    2 &  \\
    \midrule
    3 & \multirow{2}{*}{Experimental planning for data science} \\
    4 & \\
    \midrule
    5 & \multirow{2}{*}{Model evaluation and Bayesian analysis} \\
    6 & \\
    \midrule
    7 & \bfseries Written test \\
    \midrule
    \multirow{3}{*}{8} & Documentation and deployment \\
      & Ethical and legal issues in data science \\
      & Privacy-preserving computational approaches \\
    \bottomrule
  \end{tabular}
\end{center}

Case studies will be presented during exam weeks.  At most 3 case studies will be
presented per day, with 30 minutes for each presentation and 20 minutes for questions.

\thispagestyle{empty}

\newpage
\thispagestyle{empty}
\section*{CMC-16 Data science practices}

\emph{Course plan (\the\year{})}

Prof. Filipe A. N. Verri

\paragraph{Important:} Only ITA's undergraduate students are allowed to take this course.

\paragraph{Number of students:} Approx. 20 (no more than 40 students)

\paragraph{Course load:} 2--0--1--5

\paragraph{Requirements:} CMC-13 or CMC-15

\paragraph{Course program:}
Brief history of Data Science. Stages in a Data Science project. Tidy Data. Data
integration from multiple sources. Data engineering and shaping. Inductive learning and
statistical learning theory. Experimental planning for Data Science. Model evaluation and
Bayesian Analysis. Documentation and deployment. Privacy-preserving computational
approaches.

\paragraph{Goals:}
Further studying the practical aspects of Data Science (in relation to CMC-13) and providing
the mathematical foundations to ensure the correct usage of Data Science techniques.

The specific goals are:
\begin{itemize}
  \item Understanding the steps and people involved in Data Science projects;
  \item Developing an end-to-end case study, including data collection, data transformation,
    inductive learning, validation, documentation, and deployment; and
  \item Critically evaluate the results and implications of the case study.
\end{itemize}

\paragraph{Teaching methodology:}
Expository classes in common classroom, using whiteboard, slide presentations, coding
examples, books and scientific papers. Supplementary didactic materials will be available
in Google Classroom. The development of the case study will happen during laboratory
classes and home study hours, including programming and writing essays.

\paragraph{Grading:} One individual written test in the \nth{1} and another in the \nth{2} quarter.
Essay and oral presentation about the case study (in groups) for the final exam.

\paragraph{Case study:} Exactly 6 groups will be formed.  Each group will be responsible for
a case study.  Students must choose a real-world problem and develop a data science
project, including data collection, data transformation, inductive learning, validation,
documentation, and deployment.  The results must be presented in a short essay (max. 3
pages) and a 30 minutes presentation.  The trained models must be incorporated in a data
science product, such as a web application, a mobile application, or a web service.

\thispagestyle{empty}
\paragraph{Bibliography:}
\begin{itemize}
  \item Nina Zumel and John Mount. Practical Data Science with R. Manning, 2nd Edition, 2019.
  \item Hadley Wickham and Garret Grolemund, R for Data Science: Import, Tidy, Transform, Visualize, and Model Data. O’Reilly Media, 2017.
  \item John D. Kelleher and Brendan Tierney. Data Science, MIT Press, 2018.
\end{itemize}

The first two books (Zumel and Mount, and Wickham and Grolemund) are available online for free.

\thispagestyle{empty}
\paragraph{Recommended reading:}
\begin{itemize}
  \item In-progress textbook at \href{https://comp.ita.br/~verri/dsp-book}{comp.ita.br/\textasciitilde{}verri/dsp-book}.
  \item \fullcite{Vapnik1999}.
  \item \fullcite{Benavoli2017}.
\end{itemize}
Any extra material will be made available in Google Classroom.

\newpage
\paragraph{Calendar:} The expected schedule is presented below.
\thispagestyle{empty}

\begin{center}
  \begin{tabular}{ll}
    \toprule
    \multicolumn{2}{c}{\bfseries \nth{1} Quarter} \\
    \midrule
    Week & Topics \\
    \midrule
    1 & Brief history of Data Science and CMC-13 review \\
    2 & Stages in a Data Science project \\
    3 & Tidy Data and data integration from multiple sources \\
    4 & Data engineering and shaping \\
    5 & \multirow{2}{*}{Inductive learning and statistical learning theory} \\
    6 &  \\
    7 & Case study discussion and definitions \\
    8 & \bfseries Written test \\
    \bottomrule
  \end{tabular}
\end{center}

\begin{center}
  \begin{tabular}{ll}
    \toprule
    \multicolumn{2}{c}{\bfseries \nth{2} Quarter} \\
    \midrule
    Week & Topics \\
    \midrule
    1 & Experimental planning for Data Science \\
    2 & Model evaluation \\
    3 & Bayesian Analysis \\
    4 & Documentation and deployment \\
    5 & Privacy-preserving computational approaches \\
    6 & \bfseries Written test \\
    7 & \multirow{2}{*}{\bfseries Presentations and discussions} \\
    8 & \\
    \bottomrule
  \end{tabular}
\end{center}

\thispagestyle{empty}
