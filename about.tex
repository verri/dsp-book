\chapter{Preface}

\noindent Dear reader, \vspace{1em}

This book is based on the lecture notes from my course PO-235 Data Science Project, which
I teach to graduate students at both the Aeronautics Institute of Technology (ITA) and the
Federal University of São Paulo (UNIFESP) in Brazil.  I have been teaching this subject
since 2021, and I have continually updated the material each year.

Also, I was the coordinator of the Data Science Specialization Program (CEDS) at ITA.
That experience, which included a great deal of administrative work, as well as teaching and
supervising professionals in the course, has helped me to understand the needs of the
market and the students.

Moreover, parts of the project development methodology presented here came from my
experience as a lead data scientist in R\&D projects for the Brazilian Air Force,
which included projects in areas such as image processing, natural language processing,
and spatio-temporal data analysis.

Literature provides us a wide range of excellent theoretical books on machine learning and
statistics, and highly regarded practical books on data science tools.  However, I missed
something that could provide a solid foundation on data science, covering all steps in a
data science project, including its software engineering aspects.

My goal is to provide a book that serve as a textbook for a course on data science
projects or as a reference for professionals working in the field.  I strive to maintain a
formal tone while preserving the practical aspects of the subject.  I do not focus on
a specific tool or programming language, but rather seek to explain the semantics of data
science tasks that can be implemented in any programming language.

Also, instead of teaching specific machine learning algorithms, I try to explain why
machine learning works, thereby increasing awareness of its pitfalls and limitations.
For this purpose, I assume you have a strong mathematical and statistical foundation.

One important artificial constraint I have imposed in the material (for the sake of the
course) is that I only consider predictive methods, more specifically inductive ones. I do
not address topics such as clustering, association rules mining, transductive learning,
anomaly detection, time series forecasting, reinforced learning, etc.

I expect my approach on the subject to provide understanding of all steps in a data
science project, including a deeper focus on correct evaluation and validation of data
science solutions.

Note that, in this book, I openly express my opinions and beliefs. On several occasions it
may sound controversial.  I am not trying to be rude or to demean any researcher or
practitioner in the field; rather, I aim to be honest and transparent.

\vspace{1em}
\emph{I'd rather be bold and straightforward than cower about my beliefs.}
\vspace{1em}

Hope you enjoy reading.

\newpage

\begin{parwithqr}{https://github.com/verri/dsp-book}
  I intend to make this book forever free and open-source. You can find the source code at
  \href{\aurl}{github.com/verri/dsp-book}. Derivatives are not allowed, but you can
  contribute to the book. Contributors will be acknowledged here.
\end{parwithqr}

\vspace{3em}

\begin{lparwithqr}{https://leanpub.com/dsp}
  However, if you like this book, consider purchasing the e-book version at
  \href{\aurl}{leanpub.com/dsp}. Any amount you pay will help me keep this book updated
  and write new books.
\end{lparwithqr}

\vspace{3em}

\begin{parwithqr}{https://github.com/verri/dsp-book/discussions}
  If you have suggestions or questions, please open or join a discussion at
  \href{\aurl}{github.com/verri/dsp-book/discussions}. Feel free to ask anything.
  Theoretical discussions and practical advice are also welcome.
\end{parwithqr}

\vspace{3em}

\begin{lparwithqr}{https://www.buymeacoffee.com/verri}
  If you want to support me, you can \emph{purchase a coffee} at
  \href{\aurl}{buymeacoffee.com/verri}. I will greatly appreciate it.  I have a number of
  ideas for new books and courses, and financial support will enable me to make them a
  reality.
\end{lparwithqr}

\newpage

\section*{Contributors}

I would like to thank the following contributors for their help in improving this book:

\begin{itemize}
  \itemsep0em
  \item Johnny C. Marques
  \item Manoel V. Machado (aka \emph{ryukinix})
  \item Vitor V. Curtis
\end{itemize}

All contributors freely waived their rights to the content they contributed to this book.
