\chapter{Preface}

\noindent Dear reader, \vspace{1em}

This book has started from the lectures notes of the course PO-235 Data Science Project
that I teach for graduate students at the Aeronautics Institute of Technology (ITA) in
Brazil.  I have been teaching this course since 2023, and I have been improving the
material every year.

Before that, I was the coordinator of the Data Science Specialization Program (CEDS) at
ITA.  That experience, which included lots of administrative work, but also teaching and
supervising professionals in the course, has helped me to understand the needs of the
market and the students.

Literature provides us a lot of great theoretical books on machine learning and
statistics, and excellent practical books on data science tools.  However, I missed
something that could provide a solid foundation on data science, covering all steps in a
data science project, including its software engineering aspects.

My goal is to provide a book that can be used as a textbook for a course on data science
projects or as a reference for professionals working in the field.  I try to be as
formal as possible, without losing the practical aspects of the subject.  Also, I do not
focus on a specific tool or programming language, but I try to explain the semantics of
data science tasks that can be implemented in any programming language.

Also, instead of teaching specific machine learning algorithms, I try to explain why
machine learning works, increasing awareness of its pitfalls and limitations.
For that, I assume you have a strong mathematical and statistics background.

One important artificial constraint I have imposed in the material (for the sake of the
course) is that I only consider \emph{predictive methods}, more specifically inductive
ones.  I do not address topics such as clustering, association-rules mining, transductive
learning, anomaly detection, time series forecasting, reinforced learning, etc.

I expect my approach on the subject provide awareness of all steps in a data science
project including a deeper focus on correct evaluation and validation of data science
solutions.

Note that, in this book, I openly express my opinions and beliefs. Many times it might sound
controvertial.  I am not trying to be rude or to demean any researcher or practitioner in the
field.  I am just trying to be honest and transparent.

\vspace{1em}
\emph{I'd rather be bold and straightforward than cower about my beliefs.}
\vspace{1em}

Hope you enjoy the reading.

\newpage

\begin{parwithqr}{https://github.com/verri/dsp-book}
  I intend to make this book forever free and open-source.  You can find the source code at
  \href{\aurl}{github.com/verri/dsp-book}.  Derivatives are not allowed, but you can
  contribute to the book.  Contributors will be acknowledged here.
\end{parwithqr}

\vfill

\begin{parwithqr}{https://leanpub.com/dsp}
  However, if you like this book, consider buying the e-book version at
  \href{\aurl}{leanpub.com/dsp}.  Any amount you pay will help me to keep this book
  updated and to write new books.
\end{parwithqr}

\vfill

\begin{parwithqr}{https://github.com/verri/dsp-book/discussions}
  If you have suggestions or questions, please open or join a discussion at
  \href{\aurl}{github.com/verri/dsp-book/discussions}.  Feel free to ask anything.
  Theoretical discussions and practical advices are also welcome.
\end{parwithqr}

\vfill

\begin{parwithqr}{https://comp.ita.br/\~verri/dsp-book-print}
  Students can find a printable version (A4 paper, double-sided, short-edge spiral
  binding) of this book at \href{\aurl}{comp.ita.br/\textasciitilde{}verri/dsp-book-print}.
\end{parwithqr}

\vfill

\begin{parwithqr}{https://www.buymeacoffee.com/verri}
  If you want to support me, you can \emph{buy me a coffee} at
  \href{\aurl}{buymeacoffee.com/verri}.  I will appreciate it a lot.
  I have a lot of ideas for new books and courses, and financial support will help me to
  make them real.
\end{parwithqr}

\newpage

\section*{Contributors}

I would like to thank the following contributors for their help in improving this book:

\begin{itemize}
  \itemsep0em
  \item \texttt{ryukinix}, Manoel V. Machado (aka lerax)
  \item Johnny C. Marques
\end{itemize}
