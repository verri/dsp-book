\chapter{Preface}

\begin{parwithqr}{https://github.com/verri/dsp-book}
  I intend to make this book forever free and open-source.  You can find the source code at
  \href{\aurl}{github.com/verri/dsp-book}.  Derivatives are not allowed, but you can
  contribute to the book.  Contributors will be acknowledged in the book.
\end{parwithqr}

\vfill

\begin{parwithqr}{https://leanpub.com/dsp}
  However, if you like this book, consider buying the e-book version at
  \href{\aurl}{leanpub.com/dsp}.  Any amount you pay will help me to keep this book
  updated and to write new books.
\end{parwithqr}

\vfill

\begin{parwithqr}{https://github.com/verri/dsp-book/discussions}
  If you have suggestions or questions, please open a discussion at
  \href{\aurl}{github.com/verri/dsp-book/discussions}.  Feel free to ask anything.
  Theoretical discussions and practical advices are also welcome.
\end{parwithqr}

\vfill

\begin{parwithqr}{https://comp.ita.br/\~verri/dsp-book-print}
  Students can found a printable version (A4 paper, double-sided, short-edge spiral
  binding) of this book at \href{\aurl}{comp.ita.br/\textasciitilde{}verri/dsp-book-print}.
\end{parwithqr}

\newpage
This book comprises the lectures notes of the course PO-235 Data Science Project.
I hope someday it becomes an actual book. For now, beware many typos, grammar errors, ugly
typesetting, disconnected material, etc.

Also, it is important to highlight that:
\begin{itemize}
  \item This is not a Machine Learning book, and I do not intend to explain how specific
    ML algorithms work.
  \item This contains some kind of introductory material on data science.  Although I
    introduce the fundamental concepts, I expect you have strong mathematical and
    statistics background.
  \item An artificial constraint I have imposed in the material (for the sake of the
    course) is that I only consider \emph{predictive methods}, more specifically
    inductive ones. I do not address topics such as clustering, association-rules
    mining, transductive learning, anomaly detection, time series forecasting, reinforced
    learning, etc.
\end{itemize}

I have decided to work on this material because the books I like on data science are
either
\begin{itemize}
  \item too broad and too shallow, in the sense they hide many mathematical foundations
    and focus on just explaining what data science is and where it is applied;
  \item too tool-centric, in the sense that they focus only on a specific toolbox or
    programming language; or
  \item too machine-learning-y, exposing the functioning of some machine learning
    algorithms and missing the foundations of learning and/or practical aspects.
\end{itemize}

So\dots I expect my approach on the subject provide:
\begin{itemize}
  \item awareness of all steps in a data science project;
  \item deeper focus (than most books) on data handling, describing the semantics of dataset
    operators instead of restraining ourselves with a specific tool;
  \item deeper focus (than most books) on \emph{why} machine learning works, increasing awareness of its pitfalls and
    limitations;
  \item deeper focus (than most books) on correct evaluation and validation
    (pre-deployment) of machine learning models.
\end{itemize}

This book will probably cover the following material:
\begin{itemize}
  \item Brief history of data science.
  \item Fundamental concepts.
  \item Data science project.
  \item Data handling.
  \item Data preprocessing.
  \item Learning from data.
  \item Evaluation and validation.
\end{itemize}

Note that, in this book, I openly express my opinions and beliefs. Many times it might sound
controvertial.  I am not trying to be rude or to demean any researcher or practitioner in the
field.  I am just trying to be honest and transparent.

\emph{I'd rather be bold and straightforward than cower about my beliefs.}

\section*{Contributors}

\begin{itemize}
    \itemsep0em
    \item \texttt{ryukinix}, Manoel V. Machado (aka lerax)
    \item Johnny C. Marques
\end{itemize}

