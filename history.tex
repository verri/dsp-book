\chapter{Brief history of Data Science}

There are many points-of-view about the beginning of data science.  For the sake of
contextualization, I separate the topic in three approaches: the history of the term,
important figures and other historical markers, and a broad timeline of data-driven
sciences.

\section{The term ``Data Science''}

The term ``data science'' itself was coined in the 1960s by Peter Naur (/naʊə/). Naur was
a Danish computer scientist and mathematician who made significant contributions to the
field of computer science, including his work on the development of programming languages.
His ideas and concepts laid the groundwork for the way we think about programming and data
processing today.

Naur disliked the term computer science and suggested it be called datalogy or data
science.  In the 1960s, the subject was practised in Denmark under Peter
Naur's term datalogy, which means the science of data and data processes.

He coined this term to emphasize the importance of data as a fundamental component of
computer science and to encourage a broader perspective on the field that included
data-related aspects. At that time, the field was primarily centered on programming
techniques, but Naur's concept broadened the scope to recognize the intrinsic role of data
in computation.

In his book\footnote{Peter Naur: Concise Survey of Computer Methods, 397 p.
Studentlitteratur, Lund, Sweden, ISBN 91-44-07881-1, 1974.
\url{http://www.naur.com/Conc.Surv.html}}, ``Concise Survey of Computer Methods'', he
parts from the concept that \emph{data} is ``a representation of facts or ideas in a
formalised manner capable of being communicated or manipulated by some
process.''\footnote{I. H. Gould (ed.): ‘IFIP guide to concepts and terms in data
processing’, North-Holland Publ. Co., Amsterdam, 1971.} Note however that his view of the
science only ``deals with data [\dots] while the relation of data to what they represent
is delegated to other fields and sciences.''

TODO:
In the late 1980s and early 1990s, William S. Cleveland, a prominent statistician, used
the term ``data science" in his work to describe the emerging discipline that combined
elements of statistics, computer science, and domain expertise. His efforts were aimed at
developing methodologies for exploring and analyzing complex datasets, which laid the
foundation for what we now recognize as data science.

\section{Historical markers}

\section{Timeline}

\subsection{Timeline of data collection}

\begin{itemize}
  \item Pre-digital Age
  \item Digital Age
  \item Formal Age
  \item Integrated Age
  \item Ubiquitous Age
\end{itemize}

\subsection{Timeline of data analysis}
